% Options for packages loaded elsewhere
\PassOptionsToPackage{unicode}{hyperref}
\PassOptionsToPackage{hyphens}{url}
%
\documentclass[
  9pt,
  letterpaper,
]{scrbook}

\usepackage{amsmath,amssymb}
\usepackage{iftex}
\ifPDFTeX
  \usepackage[T1]{fontenc}
  \usepackage[utf8]{inputenc}
  \usepackage{textcomp} % provide euro and other symbols
\else % if luatex or xetex
  \usepackage{unicode-math}
  \defaultfontfeatures{Scale=MatchLowercase}
  \defaultfontfeatures[\rmfamily]{Ligatures=TeX,Scale=1}
\fi
\usepackage{lmodern}
\ifPDFTeX\else  
    % xetex/luatex font selection
\fi
% Use upquote if available, for straight quotes in verbatim environments
\IfFileExists{upquote.sty}{\usepackage{upquote}}{}
\IfFileExists{microtype.sty}{% use microtype if available
  \usepackage[]{microtype}
  \UseMicrotypeSet[protrusion]{basicmath} % disable protrusion for tt fonts
}{}
\makeatletter
\@ifundefined{KOMAClassName}{% if non-KOMA class
  \IfFileExists{parskip.sty}{%
    \usepackage{parskip}
  }{% else
    \setlength{\parindent}{0pt}
    \setlength{\parskip}{6pt plus 2pt minus 1pt}}
}{% if KOMA class
  \KOMAoptions{parskip=half}}
\makeatother
\usepackage{xcolor}
\setlength{\emergencystretch}{3em} % prevent overfull lines
\setcounter{secnumdepth}{6}
% Make \paragraph and \subparagraph free-standing
\ifx\paragraph\undefined\else
  \let\oldparagraph\paragraph
  \renewcommand{\paragraph}[1]{\oldparagraph{#1}\mbox{}}
\fi
\ifx\subparagraph\undefined\else
  \let\oldsubparagraph\subparagraph
  \renewcommand{\subparagraph}[1]{\oldsubparagraph{#1}\mbox{}}
\fi


\providecommand{\tightlist}{%
  \setlength{\itemsep}{0pt}\setlength{\parskip}{0pt}}\usepackage{longtable,booktabs,array}
\usepackage{calc} % for calculating minipage widths
% Correct order of tables after \paragraph or \subparagraph
\usepackage{etoolbox}
\makeatletter
\patchcmd\longtable{\par}{\if@noskipsec\mbox{}\fi\par}{}{}
\makeatother
% Allow footnotes in longtable head/foot
\IfFileExists{footnotehyper.sty}{\usepackage{footnotehyper}}{\usepackage{footnote}}
\makesavenoteenv{longtable}
\usepackage{graphicx}
\makeatletter
\def\maxwidth{\ifdim\Gin@nat@width>\linewidth\linewidth\else\Gin@nat@width\fi}
\def\maxheight{\ifdim\Gin@nat@height>\textheight\textheight\else\Gin@nat@height\fi}
\makeatother
% Scale images if necessary, so that they will not overflow the page
% margins by default, and it is still possible to overwrite the defaults
% using explicit options in \includegraphics[width, height, ...]{}
\setkeys{Gin}{width=\maxwidth,height=\maxheight,keepaspectratio}
% Set default figure placement to htbp
\makeatletter
\def\fps@figure{htbp}
\makeatother

\makeatletter
\@ifpackageloaded{bookmark}{}{\usepackage{bookmark}}
\makeatother
\makeatletter
\@ifpackageloaded{caption}{}{\usepackage{caption}}
\AtBeginDocument{%
\ifdefined\contentsname
  \renewcommand*\contentsname{Table of contents}
\else
  \newcommand\contentsname{Table of contents}
\fi
\ifdefined\listfigurename
  \renewcommand*\listfigurename{List of Figures}
\else
  \newcommand\listfigurename{List of Figures}
\fi
\ifdefined\listtablename
  \renewcommand*\listtablename{List of Tables}
\else
  \newcommand\listtablename{List of Tables}
\fi
\ifdefined\figurename
  \renewcommand*\figurename{Figure}
\else
  \newcommand\figurename{Figure}
\fi
\ifdefined\tablename
  \renewcommand*\tablename{Table}
\else
  \newcommand\tablename{Table}
\fi
}
\@ifpackageloaded{float}{}{\usepackage{float}}
\floatstyle{ruled}
\@ifundefined{c@chapter}{\newfloat{codelisting}{h}{lop}}{\newfloat{codelisting}{h}{lop}[chapter]}
\floatname{codelisting}{Listing}
\newcommand*\listoflistings{\listof{codelisting}{List of Listings}}
\makeatother
\makeatletter
\makeatother
\makeatletter
\@ifpackageloaded{caption}{}{\usepackage{caption}}
\@ifpackageloaded{subcaption}{}{\usepackage{subcaption}}
\makeatother
\ifLuaTeX
  \usepackage{selnolig}  % disable illegal ligatures
\fi
\usepackage{bookmark}

\IfFileExists{xurl.sty}{\usepackage{xurl}}{} % add URL line breaks if available
\urlstyle{same} % disable monospaced font for URLs
\hypersetup{
  pdftitle={A Compilation of the Bab's Writings},
  pdfauthor={The Bab},
  hidelinks,
  pdfcreator={LaTeX via pandoc}}

\title{A Compilation of the Bab's Writings}
\usepackage{etoolbox}
\makeatletter
\providecommand{\subtitle}[1]{% add subtitle to \maketitle
  \apptocmd{\@title}{\par {\large #1 \par}}{}{}
}
\makeatother
\subtitle{Translated Using GPT-4}
\author{The Bab}
\date{}

\begin{document}
\frontmatter
\maketitle

\renewcommand*\contentsname{Table of contents}
{
\setcounter{tocdepth}{5}
\tableofcontents
}
\mainmatter
\bookmarksetup{startatroot}

\chapter*{Welcome!}\label{sec-welcome}
\addcontentsline{toc}{chapter}{Welcome!}

\markboth{Welcome!}{Welcome!}

\newpage{}

This book is a provisional translation of the Bab's Writings. All items
are taken from the original Arabic and Persian from various sources and
translated using the latest version of OpenAI's GPT.

There are thousands of works which have not been translated by people
yet. Some of which had been translated, by authorized sources of the
Baha'i Faith, others by academics and those interested in the revelation
of the Bab.

We acknowledge GPT is imperfect, but we can acknowledge it is unbiased
in that it was trained from many data points, some of which may favor a
perspective, but the totality of all data points should bring the
translation as close to a neutral voice as possible. We hope you find
this book useful. You are welcome to share it as you see fit.

The Writings of the Bab are organized by date where possible, or by the
location He was in when the Writing was revealed. Books highlighted in
bold are those which were revealed unsolicited, and books in a normal
font were revealed as responses to questions solicited directly to the
Bab. The Bayan was revealed for everyone unsolicited. The unsolicited
writings will be more general and broad in scope, and revealed to be in
a context for everyone (or particular audiences when specified). The
solicited ones may regularly seem to repeat themes elsewhere, but be
placed in a particular context relevant to whom originally asked the Bab
a question. Each are equally important.

If you would like to help with translations, or to help refine the
translations made, visit the
\href{https://www.hgworld.org/ctw/index.php?title=Main_Page}{Baha'i
Collaborative Translation Wiki.}

Discussion can be had on Reddit at the
\href{https://www.reddit.com/r/BabandBahaullah/}{Bab and Baha'u'llah}
community.

\newpage{}

\part{Writings Revealed in the Year 1844}

\chapter*{Ahsan al-Qasas (The Best of
Stories)}\label{sec-ahsan-al-qasas}
\addcontentsline{toc}{chapter}{Ahsan al-Qasas (The Best of Stories)}

\markboth{Ahsan al-Qasas (The Best of Stories)}{Ahsan al-Qasas (The Best
of Stories)}

\section*{Introduction}\label{introduction}
\addcontentsline{toc}{section}{Introduction}

\markright{Introduction}

This is the Book of the Supreme Heaven, entitled the Best of Stories, an
explanation of the Sura of Joseph.

``Aḥsan al-Qaṣaṣ'' ``Qayyūm al-Asmā''' from the blessed works of His
Holiness the Most High (His Holiness the Báb).

\section*{The Surah of Sovereignty (Surah
1)}\label{sec-the-sura-of-sovereignty}
\addcontentsline{toc}{section}{The Surah of Sovereignty (Surah 1)}

\markright{The Surah of Sovereignty (Surah 1)}

In the name of God, the Most Merciful, the Most Compassionate

Praise be to God Who has sent down the Book upon His servant with truth,
that it may be a light for the worlds, blazing and bright. This is
indeed the path of 'Ali, established in truth before your Lord, already
ordained in the Mother of the Book, straight and true. And indeed, in
the Mother of the Book, with Us, it is for 'Ali, and upon the greatest
truth, it was with the Merciful, wisely ordained. And it is the truth
from God, and upon the pure religion, it was inscribed in the Mother of
the Book, around the Mount, recorded. Indeed, this is the truth, the
path of God in the heavens and the earth. So whoever wishes may take a
path to God in truth. Indeed, this is the firm religion, and sufficient
is God, and those with Him, knowledgeable of the Book, as witness.
Indeed, this is the truth, upon the greatest word from God, the Ancient,
sent around the fire. This is indeed the secret in the heavens and the
earth, and upon the wondrous matter by the hands of God, the High, it
was truly written in the Mother of the Book. God has decreed to bring
forth that Book in the interpretation of the Best of Stories, from
Muhammad, son of Hasan, son of 'Ali, son of Muhammad, son of 'Ali, son
of Musa, son of Ja'far, from Muhammad, son of 'Ali, son of Husayn, son
of 'Ali, son of Abi Talib, upon His servant, to be God's proof from the
Reminder upon the worlds, compellingly. I bear witness to God as He
bears witness to Himself, that He is the truth, there is no god but He,
and the angels, guardians around the Reminder with justice, there is no
god but He, and He is God, knowledgeable of everything. The pure
religion is this Reminder, whole. So whoever desires Islam, let him
submit his affair, for God will write him in the book of the righteous,
submitted. And upon the pure religion, it was with truth, praiseworthy.
And whoever disbelieves in Islam, God will not accept from him any of
his deeds on the Day of Judgment, not even a little, on the truth, by
the truth, nothing. And it is right upon God to burn him with the fire
of God, the wondrous, by the decree of the Book, from the decree of the
Gate, destined. God, there is no god but He, and He is God, watchful
over the believers. God, there is no god but He, and He is God, a
witness over the believers. God, there is no god but He, and He is God,
knowledgeable of the believers. God, there is no god but He, and He is
God, encompassing the worlds. And God will not accept from anyone any
deed unless he comes to the Gate, prostrating through the Gate. The
Ancient God, praiseworthy around the Gate. God has authorized you upon
truth, so prostrate and draw near, for in a drop of water, the fire was,
God the truth was, prostrated upon the earth in truth, witnessed. O
assembly of kings and sons of kings, turn away from the kingdom of God,
all of you, upon the truth, beautifully. O king of the Muslims, then
support after the Book, our greatest Reminder, in truth, for God has
decreed for you and those barefoot around you, in valuation, upon the
path, standing in truth, questioned. O king, by the truth of God, if you
oppose with the Reminder, God will judge you among kings with fire on
the Day of Judgment, and today you will not find, apart from God, the
High, on the truth, by the truth, any helper. O king, purify the sacred
earth from the people who reject the Book before the day the Reminder
comes suddenly by the permission of God, the High, upon the strong
matter, severely. And God has written upon you to submit the Reminder
and its command, and to make the lands subservient in truth by His
permission, for you are in this world, upon the kingdom, merciful, and
in the hereafter, among the people of the garden of pleasure around the
holy, you were dwelling. O king, let not the kingdom deceive you, for
every soul will taste death, it was by the truth, upon the truth,
written from the judgment of God, and be content with the judgment of
God, the truth, for the kingdom in the Mother of the Book, concerning
the Reminder by the hands of God, was truly recorded. And support God
with yourselves and your swords in the shadow of this greatest Reminder
for this pure religion, strong in truth, upon the truth. O minister of
the king, ease off from God, there is no god but He, the just truth, and
remove yourself from the kingdom, for we shall inherit the earth and
those upon it, by the permission of God, the Wise, and indeed, it was
with truth upon you and upon the kingdom, witnessed. And we have
guaranteed for yourselves, by the permission of God, to obey the
Reminder with pure truth, that for you in valuation, in the garden of
Eden, is a great kingdom upon the truth. And your kingdom is vain, and
God has made the enjoyment of this world for the disbelievers, and with
God.

Your sovereignty, Hasan of the lineage, has indeed been in truth upon
truth from ancient times, and indeed, for us in the garden of eternity,
is a lofty kingdom. We grant it to whomsoever we will of our servants,
those who were in this cause with God and for His signs, supporters in
truth. O assembly of kings, convey our signs to the Turks and the land
of India swiftly in truth upon truth, and beyond its lands, from the
east of the earth to its west, strongly in truth upon truth. O servants
of the Merciful, indeed, God has not created you and has not provided
for you except for a matter that was with God in the Mother of the Book,
great in truth upon truth. And follow what God has revealed to us of the
decrees of the Gate in that Book, submitting to God and His command,
pleased upon the truth. And know that if you support God, He will
support you on the Day of Judgment with the greatest reminder, on the
path, a noble support. By God, if you do well, you do well for
yourselves; and if you disbelieve in God and His intentions, we are, by
God, beyond creation and sovereignty, rich in truth. O people of the
earth, whoever obeys the remembrance of God and this Book, has indeed
obeyed God and His allies in truth, and indeed, in the hereafter, he is
among the people of the Garden of Pleasure with God, written. And
indeed, we have set the mountains moving on the earth and the stars upon
the throne, around the fire, in the water's pole, from the presence of
the reminder with God, the truth. And none of you will forsake another,
and He is the Dominant over His servants, and He is God, knowledgeable
of everything.

\section*{The Surah of Scholars (Surah
2)}\label{the-surah-of-scholars-surah-2}
\addcontentsline{toc}{section}{The Surah of Scholars (Surah 2)}

\markright{The Surah of Scholars (Surah 2)}

\emph{Surah of Joseph Verse 1: ``Alif, Lām, Rā. These are the verses of
the clear Book.''}

In the name of God, the Most Merciful, the Most Compassionate,

Alif Lam Ra. These are the verses of the clear Book, Alif Lam Mim. That
Book, from God, the Truth, concerning the Reminder, was revealed in
truth around the fire. Indeed, we have made the verses in that Book
clear, a reminder and good news for the servants of the Merciful, for
those who are with God and His signs, faithful in truth, who do not
compel their parents and deter them with truth from deviance at the edge
of the Book's knowledge as a mandate. Just, who was with truth harsh
upon the kingdom, those who fear their Lord, for a day that was in truth
harshly written in the Mother of the Book, we fear from the Merciful on
a grim day, a name inscribed in the Mother of the Book, a day when no
soul controls anything for another soul, and the command then is for
God, the Most High, and God was over everything a Witness.

Indeed, we have made the verses an argument for Our word over you. Can
you produce a chapter like it? Bring your proof if you are truthful with
God, seeing. By God, if humans and jinn united to produce the like of a
sura from this Book, they could not, even if they supported each other.
O assembly of scholars, fear God in your opinions from this day, for the
Reminder among you from Us has been a ruling and witnessing truth. And
disregard what you take from other than the Book of God, the Truth, for
you stand accountable upon the Straight Path in valuation. And God has
made suspicion in all the tablets a clear sin. And perhaps God will
pardon you for what you have earned for yourselves before this day, for
He is Forgiving, Merciful to the repentant.

And God has forbidden you from ruling on anything other than the pure
knowledge from this Book, without truth and exertion. And indeed, We
have sent down to you this Book witnessed in truth, so you may learn
from what God has decreed in our favor, far from falsehood. And you have
in the valuation on the Straight Path a position accountable in truth,
truly. God asks His servants about what they do in the religion of God
without true judgment in that Book, concealed. Soon, God will show you
the signs of the Reminder suddenly on Earth in truth, imminent.

O assembly from the People of the Book, fear God and do not be deceived
by your knowledge, and follow the Book from the Reminder in truth, by
God, the truth, no soul that follows it but follows all the scriptures
sent down from the heavens from God, the Truth, and God is aware of what
you do. And no soul that denies the Book but denies the Oneness of the
Merciful and disbelieves in the prophets and the scriptures sent from
the heavens in truth, and their abode was the fire by truth, and God
decreed the judgment in the Mother of the Book, executed in truth. And
if God willed, He could have guided all people, and God has made our
signs clear to the worlds. And the Reminder is the truth from God in
truth, and God was over everything Powerful.

And when people disbelieved in our signs, they were ungrateful to God,
the Most High, and those who disbelieved will know their place in a
valley of the confined, which God named in the Mother of the Book as
Hell. And indeed, we by truth make you taste the heat of Hell and your
replacement from the toxic fire, decreed in the Book. This is indeed the
truth from God as a recompense for your stubbornness with our signs and
the remembrance of God, the Most High. This is the recompense like what
you were in misery with God and with our signs, and the command of your
Lord, the Merciful, is truth, and the promise of God was indeed enacted
in truth. Praise be to God who revealed the light from Himself upon our
servant to be a straight path in the worlds in truth, and indeed, We
reveal to you what God has revealed to us; you were indeed high in the
Mother of the Book with God. And what is in the heavens and the earth is
by truth, and God forgives whom He wills and punishes whom He wills, and
He is God, Powerful over everything. Fear the day you will return to
God, and there is no judgment but His; there every soul is recompensed
for what it has earned, and indeed, We do not wrong a thing.

Those who believe in the Reminder sent down to him from his Lord and the
believers, each believes in God and His signs and do not differentiate
between any of His signs, and the Muslims say by truth, ``Our Lord, we
have heard the call of God's Reminder and obeyed, so forgive us, for You
are the truth, and to You is the return by truth, O believers ! We
indeed do not burden you except with what you can bear, and the kingship
was God's alone, God wants to lighten your punishment and send His mercy
upon you, and no soul has earned anything except we have written it by
the judgment of the Book, preserved. Say, 'Our Lord, God, our truth,
there is no deity but He, so forgive us with Your mercy and have mercy
on us, for You are our Patron, and write for us the return to You in
truth by truth, God, there is no deity but He, created the letter Alif
strong on the command, God, there is no deity but He, has decreed the
letter Lam for His judgment on the judgment of the Book, God, there is
no deity but He, has made the letter Ra for the expansion of His command
as He wills in the Mother of the Book around the fire, decreed in
truth.''

\section*{The Surah of Faith (Surah
3)}\label{the-surah-of-faith-surah-3}
\addcontentsline{toc}{section}{The Surah of Faith (Surah 3)}

\markright{The Surah of Faith (Surah 3)}

\emph{Surah of Joseph Verse 2: Indeed, We have sent it down as an Arabic
Qur'ān that you might understand.}

We sent it down as an Arabic Qur'an so that you may understand. Ta-Ha.
God has indeed sent down the Qur'an to His servant so that he may teach
people that God has power over all things. He who sent down the
Criterion in Arabic, without any crookedness, upon His servant, purely
and truly as a revelation to show you His signs and the interpretation
of the narratives along the right and straight path. Indeed, this is the
path of 'Ali in the heavens and on earth, uniquely true from God, the
Most High. He is God, there is no deity but He, He has sent down to you
this Book with the greatest Reminder, affirming the messengers and what
God has revealed in the scriptures; there can be no alteration to the
words of God, the Truth, and it is the truth in the Mother of the Book,
inscribed around the fire.

This Book, if it had been from other than God, the Truth, they would
have found in it much discrepancy. And glorified is God, our Lord,
nothing is hidden from Him on the earth or in the heaven, and we have
recorded everything in this Book by God's permission, hidden and truly
inscribed from God. And those who disbelieve in the exalted door of God,
we have prepared for them a painful punishment by the true judgment of
God, and He is God, mighty and wise. Indeed, we have sent down to Our
servant this Book from God in truth, and made the verses therein precise
and unambiguous, and none knows their interpretation except God and
those of Our servants whom we choose, the devoted ones.

So ask the Reminder for its interpretation, for he has been, by God's
grace, knowledgeable of its verses by the decree of the Book. Our Lord,
we praise You after You have guided us, and grant us mercy from You, as
You were capable and giving by truth. Indeed, those who disbelieve in
the greatest Reminder of God, their wealth and their children will not
avail them against God in the least, and they have no power apart from
God; those are the companions of the fire, by the just decree of God,
abiding therein forever and ever. God has indeed supported with His help
whomever He wills among His servants, and we have indeed made the love
of women, sons, and wealth appear desirable to you, all of which are
barriers to death. And God has indeed made good provision for those who
support the remembrance of God, the Most High, with their hands, their
tongues, and their wealth, God the Self-Sufficient, and He is God,
mighty and praiseworthy.

When the People of the Book argue with you about something, say, ``I
have no knowledge except what my Lord has taught me; indeed, I have
submitted my face to God, the creator of the heavens and the earth.''
And whoever is arrogant in worshiping Him by turning away from My
remembrance, it is right for God to burn him with the greater fire,
justly deserved by truth upon truth. And you desire nothing except what
God wills for you; indeed, He has been knowledgeable and wise. And the
polytheists among you say, ``The fire will not touch us except for a few
days''; when we gather them on the Day of Judgment around the fire, they
will testify against themselves that the punishment of their Lord had
been inscribed in the Mother of the Book from the Door of old. And
sovereignty belongs to God; He grants sovereignty to whom He wills and
withdraws it from whom He wills, and He is God, powerful over all
things. And we indeed humiliate the disbelievers as they wish and honor
the believers as they wish, and God has indeed placed the good in your
hands by truth, and God has placed your hands in the Mother of the Book,
raised right. And indeed, in the creation of the night and day and their
intermingling, and the bringing forth of the living from the dead and
the dead from the living, are signs for the greater remembrance of God.
Thus, it has been inscribed in the preserved tablet with God, the Most
High.

O servants of the Merciful, do not take disbelievers as allies instead
of the believers who preceded you. And whoever meets God with disbelief
in the Book and this Reminder, then from God he will have nothing, and
the Truth Himself has warned you, and to God, the Truth, is the return
of all worlds if you fear your Lord, the Merciful, about what your hands
have earned from the work of Satan, then hasten to forgiveness from your
Lord before the day you find your deeds present before you. And God
knows what is in the heavens and on the earth by truth, and He is God,
rich over all worlds. O servants of God, the Merciful Himself warns you
not to say about God except the truth, and He knows what you conceal in
yourselves and what you declare, and He has indeed been kind and
merciful to His servants on the truth by truth. O people, if you believe
in God alone, then follow me in the greater remembrance of your Lord so
that God may forgive your sins, and God has indeed been forgiving and
merciful to the believers. And we indeed choose the messengers by Our
word and favor their descendants with the great remembrance of God, some
over others by the decree of the Book, hidden. And we indeed have given
you the judgment of the doors by the permission of God, the All-Hearing,
and He is God, a witness over all things. And we indeed sent Our spirit
to Mary and accepted the vow of the wife of Imran, dedicated by God, the
Most High, and He is God, knowledgeable of His believing servants. And
we indeed gave the prophet Zechariah the good news with Our name, John,
affirming the word of God, the greatest; this is from God, and we make
him by that a master and a keeper in the Mother of the Book. The
creation of the worlds is like Our command; when we want to say to it,
``Be,'' it becomes, in the Book of the praised God, existing around the
fire. And God has taught you the knowledge of the Book from the
Criterion, the Gospel, the Torah, the Psalms, and what is beyond them
from the scriptures, and you have indeed been at your Lord's door at the
point from the concealed `Ba,' stopped. And we indeed have inspired you
with news of the unseen and sent down to you this Book in truth, and we
have forbidden you the impurities and made lawful for you the good
things so that people may believe in your remembrance, hoping for the
honor of God's ancient religion by truth. And God was knowledgeable of
all things. And those who think that they can harm you with something
from knowledge have indeed fallen from the sky to a dead, uprooted
earth, and God was a witness over all things. And God has made your
being touched by our beings, and your existence a trinity of the light
of God's ancient essence, our Lord, and He is God, powerful over all
things. And the polytheists plotted against themselves in your
remembrance and will not harm except themselves, and God has indeed
fulfilled His promise, and I am purifying you and causing you to die and
raising you to God, the Truth, and you will judge by God's permission on
the Day of Judgment in what people differ in the remembrance of God, the
Most High, and God was a witness over all things. Some from the city
said, ``We are the supporters of God''; when the Reminder came to them
suddenly, then they turned away from our support, and God is my Lord and
your Lord, the Truth, so worship Him, and this is the straight path by
your Lord, so soon God will judge between people by truth, then they
will not find in themselves any distress from God's pure judgment, and
the matter was decreed in the Mother of the Book. And when the matter
reaches intensity, argue by God's permission with the polytheists, and
say, ``Come, let us call upon our Lord, who is no deity but He, by
ourselves and yourselves, and God is indeed the Truth, a witness over
us, and He is God, knowledgeable of all things. By your Lord, if you
confront the disbelievers, and God is indeed the Truth, a witness over
us, and He is God, knowledgeable of all things. By your Lord, if you
confront the disbelievers, people look to the edge of the sky, and we
have sent down upon them by God's permission a lightning bolt of
firestone, and if not for your prayer, the earth and some upon it would
have been burned, and God was powerful over all things. Say, O People of
the Book, believe in a word from God, equal between me and you, that you
worship none but Him, and do not associate anything in His worship, and
do not take from among you some as lords apart from God; He is one God,
like whom there is nothing, and He is God, a witness over all things.

\section*{The Surah of the City (Surah
4)}\label{the-surah-of-the-city-surah-4}
\addcontentsline{toc}{section}{The Surah of the City (Surah 4)}

\markright{The Surah of the City (Surah 4)}

\emph{The Surah of Joseph Verse 3: We relate to you, {[}O Muḥammad{]},
the best of stories in what We have revealed to you of this Qur'ān
although you were, before it, among the unaware.}

In the name of God, the Most Merciful, the Most Compassionate

We relate to you the best of stories through what We have revealed to
you of this Qur'an, though you were unaware of it before. Truly, We have
clarified the stories for those who desire us at the door, loving and
responding. And indeed, We have sent down the Book in truth upon truth
so that people might be in accordance with the Name of the Remembrance
written in the Book.

We have indeed sent down the Book to every nation in their language, and
We have indeed sent down this Book in the language of the Remembrance
truly and wonderfully. It is indeed the truth from God and in the Mother
of the Book, decreed among the most eloquent of Arabs. It is the most
eloquent among the most articulate, and it is the greatest talisman in
truth, and it was in the Mother of the Book marked as a talisman.

We have indeed made you a sun over the worlds and a luminous moon, a
chaste herald, and a strong pillar over the worlds, that perhaps the
people would be guardians of God's signs in truth upon truth. But the
people were not pleased with the true signs of God, and the people were
strangely against God and His decrees.

O servants of God, did not our servant show you the true promise
concerning what God intended for you in the proper religion? And indeed,
God has made the signs our signs, and concerning the matter of the
Remembrance, the decree of the Book was written in the Mother of the
Book. This is one of them for anyone who has a covenant from the
Merciful around their neck, truly and rightly, and has entered into that
gate by the permission of God, the Praiseworthy, pure and cleansed.

O people of the city, you are the polytheists against your Lord if you
believed in Muhammad, the Messenger of God, and the seal of the
prophets, and his book the Criterion, which falsehood cannot approach.
Indeed, We have sent down to Our servant this book by His permission,
just like it, if you do not believe in it, then your faith in Muhammad
and the book before was indeed a lie witnessed by God. And if you
disbelieve in it, then your disbelief in Muhammad and his book was
certainly known to yourselves truly and certainly.

O people of the city and those around it from the Bedouins, what is
wrong with you, how did you disbelieve in Muhammad openly and wrongly
after his death? Did not God and His prophet take from you a covenant
regarding the guardianship of His successor in many places on the earth
truly and abundantly? If you believed in God, who there is no deity
except He, then what is wrong with you, how do you judge for yourselves
with other than what God revealed in His true book preserved?

By your Lord, if you do not believe in our remembrance and this book,
then be certain that your abode is the fire wherein you will dwell
forever, and you will have no helper from God the Most High on the day
of judgment. Indeed, some souls among you have died in disbelief before,
and you did not believe in Muhammad nor those around him after his
ascension except you disbelieved in his successor. What is wrong with
you, do you not contemplate the Qur'an as a true revelation? God indeed
promises you Paradise and Satan invites you to your religion which leads
you to Hell.

So whoever among you wishes may believe and whoever wishes may
disbelieve, and indeed God is free of need from all the worlds. And
indeed, the power is with God, the Almighty of old. O people of the
city, fear God on a day when you will not be able to do anything for
yourselves, and indeed the decree from us was written in truth upon
truth. What is wrong with you, how did you disbelieve in God, your
creator, who there is no deity except He, who created you and provided
for you from His bounty, and indeed He has always been a witness over
you.

Will you not then contemplate the Qur'an as a revelation? Will they not
then contemplate the Criterion as an interpretation? Fear God from whom
we take on the truth severely, if you were in what you were and do not
return to the high remembrance of God truly and soon. Then God will show
you in the reckoning a fire that has encompassed yourselves; there you
will find no helper apart from God the Most High. Have you believed in
something other than the true God, and God has always been a witness
over all things.

Have you believed in something other than the true God? God has always
been a witness over all things. Have you believed in two souls other
than the true God, while God encompasses all things? O delight of the
eye, strike upon the people of the city a decisive blow. God has decreed
for one of them two elevated gardens around the door, each with trees.
One garden is watered from two basins, and the other drinks from two
cups. Both are by God's permission, surrounded by fire and stopped at
two waters. There are also two rivers in the land of the two Wests, with
whales in one of the bays.

Then he asked his companion among the first two, ``Which of you will
take charge in the other two?'' And in those two hours standing, a month
on the disbelief with certainty for the souls themselves and the two
souls thereafter, by the true God, then connect with the truth. Which of
the two souls in the two parties was praiseworthy around the fire?
Indeed, the truth was recognized in the Sacred Mosque, seeing the state
in the greater truth. Have you disbelieved in the one who created you
from dust, then from a sperm-drop, then fashioned you into a man? O
people of polytheism, why have you made your hardship with the door
another door? The true God has indeed set your seat by the fire. The
book rules, settled and released.

\section*{The Surah of Joseph (Surah
5)}\label{the-surah-of-joseph-surah-5}
\addcontentsline{toc}{section}{The Surah of Joseph (Surah 5)}

\markright{The Surah of Joseph (Surah 5)}

\emph{Surah of Joseph Verse 4: {[}Of these stories mention{]} when
Joseph said to his father,~``O my father, indeed I have seen {[}in a
dream{]} eleven stars and the sun and the moon; I saw them prostrating
to me.''}

In the name of God, the Most Merciful, the Most Compassionate

When Joseph said to his father, ``O my father, indeed I have seen eleven
stars, the sun, and the moon; I saw them prostrating to me.'' Indeed, We
showed him in the vision that exalted station. And indeed, We recount to
you from the news of the unseen from the Book of God, the Preserver,
witnessed. And indeed, with what We have found from our sincere
followers, God has clothed it as our beauty, a shining shadow. And
again, no, what God intended in the belly of the Book is nothing without
us, truly beautiful. And the Merciful has specifically mentioned Joseph,
the essence of the Messenger, in the fruit of the virgin Hussein, son of
Ali, son of Abi Talib, witnessed. God showed him above the throne with
the sensations of the heart, that the sun, the moon, and the stars were
indeed prostrating to him, God the Truth, witnessed.

When Hussein said to his father one day, ``Indeed, I have seen eleven
stars, and the sun and the moon, I saw them encompassing me in truth,
prostrating to the ancient God.'' God, who there is no deity but He,
speaks the most truthful speech in truth, wonderfully. Perhaps the
people were patient with the signs of their Lord in truth. Praise be to
God who has interpreted Hussein's vision in truth on the land of the
heart, around the truth, witnessed. And God has decreed his testimony by
the testimony of monotheism by himself about himself in truth, accepted
because God has witnessed him by Himself with the testimony of
monotheism from himself on the truth, witnessed.

And indeed, the Wise has informed about the secret of his vision in what
has been revealed in the Criterion to His beloved, hidden. Indeed, the
Qur'an of dawn was witnessed. And indeed, the stars of the throne
prostrated in the Book of God for the killing of Hussein in truth upon
truth, and their number was in the Mother of the Book, eleven. He is God
who has placed monotheism in the realities of things, from His rays, in
truth, willingly and unwillingly. And He is the one who has created the
letters by Himself in truth as an example. And He is the one who has
determined the letters of identity for His narrative in the greatest
truth, eleven. And He is the one who has made the Imams the word of
monotheism in the written line. And He is the one who has decreed the
prostration of the sun, the moon, and the stars in the Mother of the
Book by the decree of the Book, written. God, who there is no deity but
He, and indeed your patron the truth is the most truthful of speakers in
narrative.

And God intended by the sun Fatimah, and by the moon Muhammad, and by
the stars the Imams of truth in the Mother of the Book, known. They are
the ones who weep for Joseph by the permission of God, the truth,
prostrating and standing. And the people weep like the shade of shelter
for Hussein by the permission of God, the truth, prostrating equally.
And whoever prostrates other than the Merciful or associates anything
with God in worship, it is rightful upon God to admit him into the fire,
abiding therein forever. Indeed, this is the true recompense from your
Lord for what you were stubborn against our Lord's signs. God, who there
is no deity but He, the truth, and God was the Lord of the worlds,
worshipped in truth.

And indeed, the innovations of the signs, the hours, the souls, and the
horizons are signs for the first of understanding among you, who were
witnesses by the high remembrance of God. Fear God and do not say in the
greatest remembrance of God anything other than God, for indeed, We have
taken His covenant from every prophet and his nation by His remembrance,
and We did not send the messengers except with that valuable covenant.
And We do not rule by truth anything except after His covenant in that
greatest gate. Thus, God will unveil the cover from your sights at the
appointed time; there you will be intensely looking at the high
remembrance of God.

And the polytheists said, ``Indeed, we have wronged ourselves after what
God Himself warned us in His Book, and you were that day stopped in the
hub of the fire.'' There, you cannot escape, and when you ask for water
from the owner, he will surely make you taste the water in the best part
of the fire, hotter upon hot, and water from the best of the bitter
tree. When you drink a drop of it, your limbs will be severed from your
bodies, and you will wish for death, but God has not decreed that for
you as recompense for your association with God in this world. And
indeed, God was encompassing over everything.

And those who fulfill the covenant of God and do not purchase anything
from the signs with anything false, those are on guidance from the high
remembrance of God, and those are truly the companions of the garden in
the Book of God, and indeed, the decree was written in the Mother of the
Book. Those who believe in God and His signs in truth sincerely, God
will reward them on the Day of Resurrection with double the reward and a
good return in truth elevated. Indeed, this is the true recompense from
your Lord, a plentiful reward. And indeed, this is the paradise God has
decreed as recompense for your deeds, for what you have done in the
religion of God, praiseworthy in truth. Do you think that other than our
remembrance this is the truth from God, and it was not from God on the
truth, or do you say upon God the lie? How often do you disbelieve in
the praiseworthy God openly?

\section*{The Surah of Testimony (Surah
6)}\label{the-surah-of-testimony-surah-6}
\addcontentsline{toc}{section}{The Surah of Testimony (Surah 6)}

\markright{The Surah of Testimony (Surah 6)}

\emph{From the Surah of Joseph Verse 5: He said, ``O my son, do not
relate your vision to your brothers or they will contrive against you a
plan. Indeed Satan, to man, is a manifest enemy.}

In the name of God, the Most Merciful, the Most Compassionate

He said, ``O my sons, do not recount your visions to your brothers, lest
they plot against you. Indeed, Satan is a clear enemy to man.'' Remember
the mercy of your Lord, His servant Ali, and indeed, We have revealed
the Book to Our servant so that people may bear witness to the
remembrance of God the Most High in this matter. And God did not intend
anything from the servants on this day except by His true judgment as a
revelation. If you love God, then follow Him; God will love you, and
indeed the reward for remembering God the Great has been inscribed in
His book by the hands of truth.

Whoever does an atom's weight of good will see its full reward on the
Day of Judgment, truthfully and abundantly given. And whoever does an
atom's weight of evil shall taste the fire named by the ancient God as
bitter by the permission of God. And Ali said, ``O my sons, do not tell
of what God has shown you of your affairs to your brothers, out of mercy
and patience for themselves.'' God the Most High is just, and He is God,
mighty and praiseworthy. If you inform them of your affair in any part
of what God has decreed for you, they will plot against you to kill
themselves in the love of God, excluding your true self as a witness.

And God willed, as He pleased, that your face should be bloodied on the
ground, truly and deeply stained. And God willed, as He pleased, to see
your hair dyed with your blood and your self on the ground, slain
unjustly by the truth. And your body on the ground, naked, and God
willed, as He pleased, to see your daughters and your women in the hands
of the disbelievers, unjustly captives. And God willed, as He pleased,
to see the faces of your followers before you, reddened with the dye of
themselves and their bodies on the ground, wounded and unjustly cast
down.

So do not reveal anything of what God has decreed in your essence from
the secret of the hidden testimony to yourself, even a little of the
word. If you inform them of your secret affair in truth, even a little,
there they will sacrifice themselves for the love of God, longing for
God. And God was, with His servants, truly and truly compassionate. God,
who there is no deity but He, speaks the most truthful speech,
wonderfully. Perhaps people were with the signs of their Lord, the
Merciful, on the right path, in the ways of the door, praisefully
grateful.

And indeed, the brothers of Joseph knew from the secret of his affair a
letter on the secret, profoundly secret, and thus the traditions of the
prophets and the martyrs have proceeded to be killed in His cause, and
God was over everything truly a witness. And the devil's disbelief
proceeded after killing Joseph, and that matter was with the Merciful
and His chosen ones and in all the tablets, truly and truly cursed. And
after his disbelief, he intended greatly on the Greatest Word unjustly,
but God will soon remove his concern and cast him into the sea of
darkness, where some layers are above others, plotting.

And those who wronged us will know that they do not precede us in the
knowledge of the Book, not even a letter, and we were encompassing over
all worlds with God the Most High, truly and truly. And God has counted
you in the Mother of the Book among the five hidden ones, secretly
secret. And God has taught you knowledge that those before you could not
encompass. And indeed, we have taught you the knowledge of innovations
from the Merciful, innovatively, and what God did not will was not by
anything, and it was not in the knowledge of your Lord by anything.

And God has spoken truly and truly, ``Did you not pledge to you, O
servants of God, in our true covenant on the truth, a heavy covenant, do
not say about God the truth except the greatest truth, affirming and
peacefully?'' And indeed, we have taken from you in the sight of God a
firm covenant, out of love for our followers from the Wonderful, on the
truth, and the command was from God the Most High, great. And God
intended for you in this book a command on the greatest truth, imposed,
and people were in negligence and discord in this greatest gate from our
great command, unjustly, and He is God, High, Great.

God is independent of you, and He is God, encompassing everything. Has
not the remembrance of God reached you, O servants of God, repeatedly
from our great command? O people of the earth, fear God in that
cultivated leaf from the single tree, for indeed, by the truth, it is a
right as He is God and His allies on the right, truly, and God was over
everything a witness. And God has decreed among the servants that you
worship none but Him around the gate, and this is the pure religion on
the path of equity, beneficently.

\section*{The Surah of the Visit (Surah
7)}\label{the-surah-of-the-visit-surah-7}
\addcontentsline{toc}{section}{The Surah of the Visit (Surah 7)}

\markright{The Surah of the Visit (Surah 7)}

\emph{The Surah of Joseph Verse 6: And thus will your Lord choose you
and teach you the interpretation of narratives {[}i.e., events or
dreams{]} and complete His favor upon you and upon the family of Jacob,
as He completed it upon your fathers before, Abraham and Isaac. Indeed,
your Lord is Knowing and Wise.''}

In the name of God, the Most Merciful, the Most Compassionate

Thus, your Lord chooses you and teaches you the interpretation of
narratives and completes His favor upon you and upon the family of
Jacob, as He completed it upon your forefathers, Abraham and Isaac.
Indeed, your Lord is all-knowing, wise. God has revealed the Criterion
to our mention, that it may be a bearer of good news to all the worlds
along the equator and a warner. Indeed, We have completed Our favor upon
the inhabitants of the heavens and the earth with the bounty of the
remembrance truly and justly as a grace.

And God has placed the greatest of blessings on this day of yours, the
remembrance of God the Most High, and God was a witness over everything.
Thus, We have chosen you truly and taught you from the interpretation of
the Book, what is not befitting anyone but you. You have been foremost
in response to God the Most High, truly and rightfully mentioned. And
God has chosen Hussein from His servants and made him truly and justly
an Imam and a martyr. And he preceded his brothers in the knowledge of
the Merciful, a convincing letter on the secret, what was in the hidden
line of the secret of concealment.

And God has completed His favor on Hussein and his successors by making
their favor like His own favor over all the worlds. He is the one who
accepts from his visitor by visiting the truth for himself and has
called the martyrs to His throne in truth, there is no deity but He,
without likeness, truly. And God has not decreed for His secret on a
letter of the letters interpretatively. He is the one who promised the
visitors to meet Himself, and God's promise was truly enacted.

He is the one who decreed the square in the paths of the visit in the
visit truly and justly, and the matter was decreed in the Mother of the
Book around the fire. He is the one who chose for Joseph a letter from
the secret and for his parents before a letter from the line around the
secret, concealed. He is the one who was and nothing was with Him, truly
and rightfully mentioned. He is the existing one, who always was and
will be in His rank, nothing exists truly and justly.

He is the one who has taught you from the interpretation of the
narratives as you wished by what We wished, truly and justly from the
truth, wonderfully. He is the one who has raised the geometry from that
gate, honoring the gate, and God was over everything powerful. So follow
what God has revealed to you in the matter of the remembrance openly
with the greatest truth and in the greatest secret, for We do not wish
for your souls but the gardens of Eden around the pleasure, existing
truly.

We show you the truth; We do not want from you any reward for the
greatest gate and on the truth, gratefully, except patience and
helplessness. God the Most High, and He is God, mighty and praiseworthy.
And the lands have been filled from the bounty of your Lord, the
Merciful, generously. And We have extended to you as We began you, and
we only confirm it in the Book of God, your denial of Our signs from the
remembrance often. Why do you not remember for yourselves a reminder
from God the truth a little?

How do you disbelieve in God, and you were dead not knowing anything
from the knowledge of the Book, truly and justly? And He is the one who
created you, then provided for you, then will cause you to die, then
will revive you if He wills. Your possession and the greatest word from
God, do you not fear from a day when a patron will not benefit a patron
anything, and the kingdom on that day is truly for the Merciful, for the
family of God, and for their followers, as was written in the Mother of
the Book.

On the day when the Spirit and the angels stand around the remembrance
by the permission of God the truth, arrayed in a row, no one will enter
Paradise except he who has a strong covenant from God on his neck. And
those who disbelieved, who spoke the wrongful word unjustly, will not
enter Paradise except those who were Jews or Christians. By God, those
are their wishful thinking of polytheism, and the judgment was known in
the Mother of the Book. Thus, they will face from the Merciful a command
in truth, truly witnessed.

Those will not be able to and will not be able to benefit themselves
apart from the remembrance, neither harm nor benefit, except by the
permission of the Merciful who spoke truly in His Book from our servant,
a letter on the truth, repentantly. Do people think that we were distant
from the creation? No, on the day we unveil the leg above their legs,
that people may see the Merciful and His remembrance in the land of
gathering soon. They will say, ``Would that we had taken a path with the
door. Would that we had not taken any men besides the door unjustly,
alas! The remembrance has come to us from before us and from behind us
and from our left, and we were veiled from it.''

\section*{The Surah of the Secret of Oneness (Surah
8)}\label{the-surah-of-the-secret-of-oneness-surah-8}
\addcontentsline{toc}{section}{The Surah of the Secret of Oneness (Surah
8)}

\markright{The Surah of the Secret of Oneness (Surah 8)}

In the name of God, the Most Merciful, the Most Compassionate

\emph{Indeed, in Joseph and his brothers, there were signs for those who
ask.}

Praise be to God who has exalted His signs in His noble book, and He is
God, watchful over everything. Indeed, We have clarified the signs in
that book for the people of understanding, those who have been solitary
around the door. And God has exalted Joseph by our name, and God has
made him a guardian in the book, stationed around the fire. And God has
made Joseph and his brothers the word of monotheism and the Merciful has
measured them with twelve letters. This word was in the Mother of the
Book at your Lord, inscribed around the line, inscribed.

And God has specially attributed to Joseph the letter `Ha' as a reward
for his standing by the Merciful on the known day, truly and justly on
earth, solitary. And God has made monotheism for those who inquire at
our door and for those standing in the depths of uniqueness by our
permission, those who have been upright around the door. God, who there
is no deity but He, the truth, and He is God, worshiped in truth upon
truth.

Fear, O servants of God, the day you were around the fire by the
permission of God the Most High, questioned. Not a word do you utter but
with it is an angel from us, by the remembrance, watchful. And our
servant was never hidden on the day from the sincere servants of God.
And God has created for you the Day of Resurrection as an appointed time
of justice. The Day of Decision placed the balance by the permission of
God before us, justly and upon the correct path, balanced.

Soon He will inform you on this day about what you have been doing in
the religion of God, openly and secretly. O servants of God, have we not
entrusted to you in our word, truly and clearly, a covenant not to say
in our servant any of the letters deceitfully? So exalted is He who
there is no deity but He, and He is God, encompassing the worlds. God
did not intend from your creation but the prostration of the Merciful
intended in this door's path.

He is the one who created the believers from water that was in the
Mother of the Book, fresh, and He is the one who made the realities of
the disbelievers from salty water that was in the root of Hell, bitter.
And God has made the realities of the believers, the signs of our
servant, elevated for those who want God and His allies from before the
door, sincerely chosen. And we have made our word on earth a witness by
the believers. And we have raised him to the station of holiness,
visible. And we have brought him close to us and made him firm in truth.

God has stationed him on the straight path in pure truth, commanded. And
God has fixed him on the balance from the judgment of the Book, decreed
by the Book, a lesson for those who are insightful at God the Most High,
in truth. And a reminder for those who are around the fire by the wise
fire, patient. And God intended by the name of Joseph our high word,
which was witnessed around the fire. He is the one who sends upon you
His signs by the permission of God, fearful in truth, truly monastic.

A reminder for whoever wishes to remember or fears the Merciful in the
just judgment, which was truly decreed. And He is the one who has taught
you in His signs the origins of the hereafter, perhaps people were
content with God and His signs in truth. By the Lord of the heavens and
the earth, it is indeed the truth from us. And we have taken our
covenant with the permission of God from all the worlds, for rarely do
people say if God had sent to us a human in his absence, we would have
followed him, and we would have been guided by his judgment to the
truth, guiding.

Fear, O servants of God, the day the judgment of the book was decreed in
the Mother of the Book. Did you not think in the remembrance of God the
Greatest and our word a false thought? By your true Lord, you do not
know a letter of His knowledge, what God has taught in the Mother of the
Book, truly and justly from ancient times. So remember the remembrance
of your Lord, the Merciful, at both ends of the day and in part of the
night, as God has commanded you in His book before, and the judgment of
God was decreed in the Mother of the Book.

\section*{The Surah of the Witness (Surah
9)}\label{the-surah-of-the-witness-surah-9}
\addcontentsline{toc}{section}{The Surah of the Witness (Surah 9)}

\markright{The Surah of the Witness (Surah 9)}

In the name of God, the Most Merciful, the Most Compassionate

\emph{When they said to Joseph and his brother, ``Our father loves you
more than us, and we are a strong group. Truly, our father is in clear
error.''}

God has revealed the Book from Him, clarifying everything, a mercy and a
good news for Our servants who have knowledge of the high remembrance of
God, insightful of the Book. When they said the letters, ``There is no
god but God,'' and that Joseph is more beloved to our father than us by
what preceded of God's knowledge, a letter hidden in the secret, veiled
in the secret, absent in the line, elevated in the secret of the hidden,
beyond what is in our hands and the hands of all worlds.

And indeed, We are a faction in what God intended concerning the matter
of Joseph, the Prophet Muhammad the Arab, written around the line. And
God has favored our father with His own grace and measured the hidden
secret of His affair by what is in the hands of the worlds with clear
revelation. The Merciful is established on the throne, and He is God,
was over everything capable. And God has created things by His power
truly and justly, and He is the one who invented the heavens and the
earth and what is between them by His command, creatively around the
fire, to teach people that the matter of God was in the Mother of the
Book, existing around the fire truly and justly.

And He is God, was by His power watchful over everything, and He is God,
was encompassing everything. And He is God, intended in the hidden of
the secret on the secret of the line at the point of the door,
interpretatively. And He is the one who made the selection from the door
for the knowers truly and justly witnessed. O servants of the Merciful,
shake towards this date palm trunk by the permission of your true Lord,
God, who placed it in the Mother of the Book truly and justly from the
truth, high. And He is the one who drops from it to your souls, moist on
the truth truly, harvested.

For indeed, We have marked His remembrance by the Merciful in a day that
was in the Mother of the Book anciently, and you in that day were not
forgotten in the book, nor around the fire forgotten. And do not say how
does one speak about God who was in the age on the truth truly,
twenty-five. Listen, for by the Lord of the heavens and the earth, I am
a servant of God, clear signs have come to me from the remainder of God,
the Awaited Imam before you, this is my book, it was with God in the
Mother of the Book, written truly on the truth.

And God has made me blessed wherever I was and commanded me with prayer
and patience as long as I am alive among you on the earth. And those who
invoke from God some of the narratives concerning the door on the door
unjustly a little, do they think they can produce a book like this from
God, the truth, witnessed on the truth? The truth truly says, ``There is
no god but God alone, He has no partner, and there is nothing like Him,
and He is God, was by the truth on the truth anciently. If the humans
and the jinn were to gather to produce a book like this truly, they
would not be able, even if they were from the people of the earth and
like them helped by some.

By your true Lord, they would not be able with a fraction of its letter,
nor on its interpretations of some of the secret, at all. And God has
sent it down by His power from Him, and people cannot match a letter of
it by analogy at all, without comparison. These are the news of the
unseen inspired to you, you were by God, the Praiseworthy, around the
fire, and soon your Lord will come to you on the Day of Judgment, the
judgment of truth raised from Him. Enter whom you wish into the mercy of
God and turn away from the wrongdoers around Hell and let them in the
fire on the truth, prostrate.

Do you believe in some of the book and disbelieve in some? Has God
permitted this for you, or do you invent lies about God from where you
were by the knowledge of Satan unjustly on the wrong by the truth,
deceived? And indeed, We have sent down the remembrance, and God and His
angels were over you by the truth, watchful. Fear, O servants of God,
and be sincere in the religion of God, witnesses on the truth. Those who
fear their Lord in the unseen and were by the Merciful and His allies
around the door, chosen truly, so soon God will teach them their
judgments of what they need for themselves openly from the truth to the
truth soon.

And God has revealed to me if you love God, then follow me in this
religion by the truth on the truth from the truth to the creation,
uprightly. And your Lord, God, said by the truth, ``I am merciful on my
believing servants from the people of the door truly on the truth.'' And
exalted is God above what the wrongdoers say in the signs of the door,
greatly high. Say, ``The command of God has come, so do not hasten it,
for the command of God was on the truth by the truth near, and the
promise of God was truly enacted.''

\section*{The Surah of Blindness (Surah
10)}\label{the-surah-of-blindness-surah-10}
\addcontentsline{toc}{section}{The Surah of Blindness (Surah 10)}

\markright{The Surah of Blindness (Surah 10)}

In the name of God, the Most Merciful, the Most Compassionate

\emph{``Kill Joseph or cast him to the earth; it will free your father's
face for you, and you shall be a righteous people thereafter.''}

Mention the Greatest Remembrance in the remembrance of our Lord the Most
High, praiseworthy. So exalted is He who has revealed the Book as He
wished, with what He wished, and He is God, was over everything capable.
And I was not except as We willed, and God was a witness over
everything.

Indeed, We have favored Joseph and his brothers with signs from the
secret of that great door, making it clear. If they said, ``Conceal the
sign of Joseph which God has placed among you to be upon the truth on
earth, able men,'' or ``Cast him to the land of uniqueness so your
father's face may be freed for you, and you may thereafter be a people
on the kingdom, pure with the hidden secret.'' He is the one who has
revealed among you a soul from your souls, and indeed, We have made him
a human on the truth, noble.

And indeed, We have made the heavens and the earth signs for Our
servant, and God was a witness over everything. Soon God will show you
in the seat of truth His signs on the truth, truly great from Us. And
indeed, We have sent down the signs in the scriptures, and God was
knowledgeable of His signs and the remembrance truly and justly. And
indeed, We have shown you signs of God in the horizons and within
yourselves so that you may testify that He is the truth, and God has
made the way for those who are by their Lord at the door, questioned on
the truth.

And God has revealed this book to His servant to be a witness over the
worlds by Our permission, truly and justly. And indeed, We have
accounted in that book all that God has sent down to the prophets and
the truthful in all the tablets, truly and justly together. And We have
left from Our judgment in you what you have left from the Book of God, a
letter on the covered secret, hidden. And people have lied about what
they preceded us in obedience to the Merciful, somewhat, and God was a
witness over everything on the truth.

Soon our servant will appear to you in a blindness of light for those
who await the remembrance of God truly, morning and evening. Those are
the ones whom God has made insightful in His pure religion and straight
on the straight path. And from the people are those who say, ``We
believe in God,'' and mention the Greatest, and God was insightful over
His servants on the truth. They have disbelieved with their tongues
after their souls were assured, and God knew them in their souls that
they are disbelievers in God the Most High, newly.

Soon the fire and its stones will encompass them in the valley of Hell
by the permission of God the Most High, soon. This was for you a
recompense from God for what you were hypocrites in the religion of God,
the just, and God was a witness to what you do. O people, has not the
news of those before you come to you, and I warned you of the punishment
of the Merciful truly and severely? Did not God promise you a paradise
as wide as the heavens and the earth prepared for Our servants who were
believers in the high remembrance of God and were pious in truth on the
truth?

For you in it is what your souls desire by Our permission, truly and
justly, a fresh bounty. So exalted is He who there is no deity but He,
the Lord, and there is nothing like Him, and God was worshiped in truth
on the truth. For indeed, We have warned you by the permission of God of
a remembrance that you were deprived of without right. O people, aspire
to the remembrance of God the truth, asking about the door, and to God
the truth on the truth, devotedly.

For by the Lord of the heavens, God has indeed decreed your provision in
this sky, truly and justly, descended and divided. What is wrong with
you that you do not come out of your desires after the truth has come to
you from God the Most High, great? He is the one who has decreed the sun
as a light and the moon as a luminary that you may seek from His bounty
the expanse of the goods of life, praiseworthy on the truth truly. And
God has decreed the abode of the Hereafter for those who want God and
His servant on the paths of the upright line around the fire, defined.

Know, O servants of God, that your Lord, God the truth, has delivered
His argument to you for what you were with God and His signs on the
truth, watchful. And only the believers are those who want in themselves
nothing less than the Merciful and His remembrance on the truth,
beloved. And indeed, the great secret has come to you and your openly
from God for the great matter, wonderfully. Soon God will teach you of
His command what no one before Him could encompass on the truth truly,
expertly.

He is the one who has sent down from the sky blessed water on the earth,
purifying. And He is the one who has created from water humans, then
made them kinship and patient on the truth truly.

\section*{The Surah of The Line (Surah
11)}\label{the-surah-of-the-line-surah-11}
\addcontentsline{toc}{section}{The Surah of The Line (Surah 11)}

\markright{The Surah of The Line (Surah 11)}

In the name of God, the Most Merciful, the Most Compassionate

\emph{One of them said, ``Do not kill Joseph but throw him into the
depths of the well so that some travelers might pick him up if you are
to do something.''}

Praise be to God who created the heavens and the earth truly and justly,
layer upon layer, to show people that their Lord, the Merciful, is
indeed the truth and that He was over everything capable. And He is the
one who created you from dust, then from a sperm drop, then from a clot,
then a lump of flesh, then He brought you forth as another creation. So
blessed is God, the best of creators, wise.

And He is the one who keeps you in the darkness of the womb and brings
you out of your mothers' bellies while you know little of the book, only
a fraction. Return to your Creator and kill yourselves in the way of the
true God, humbly; God the Most High, and He is God, was mighty,
praiseworthy. And He is the one who has sent down these verses to you
from Him truly, as good news for all believers. He is God, worshipped;
there is no deity but God truly, and He is God, was knowledgeable about
everything.

And He is the one who gives you glad tidings by the name of His servant
truly and justly, and he was indeed in the Mother of the Book by God,
high and wise. And He is the one for whom God did not make the secret of
His name known before, truly and justly named. And He is God, gave him
peace on the day of his birth, the day of his mission, and the day he is
gathered on the earth of the heart among those around the fire, uniquely
and justly. That is the secret of secrets from the Wonderous one who
there is no deity but He, the Most High, and God was over everything
capable.

And God has managed the matter in the door by His power, and the
judgment in the Mother of the Book around the fire was decreed. And
indeed, we have made you a great sign by the permission of the true God,
and God was a witness over everything. And you are of the great prophets
truly, you were witnessed around the fire. And indeed, we have prevented
you from mentioning the door by the permission of your true Lord God,
from what you were unrightly, improperly sighing. The day we fold the
sky with our hands, the matter was decreed truly, and the oven fled by
the permission of the true God, and God was over everything capable.

There, you will recognize from our command what you were far from
unrightly. On that day, the believers will rejoice soon by the
permission of God the Most High. Soon the believers from the people of
the door will rejoice with what their Lord has brought them from the
spirit, and the spirit in the Mother of the Book around the water was
fragrant. And God has made for you this day on the land of the path on
the bridge of fire, a meeting time. When one of Joseph's brothers, and
he is Hasan ibn Ali in the Mother of the Book, was great by the fire, by
the ancient fire, he said, ``Do not kill Joseph but throw him in the
depths of the well of uniqueness around the fire, hidden.''

God intended from the well a blindness of the hidden in the air of the
hidden secret on the secret in the Mother of the Book around the line,
inscribed. People do not know what Joseph's brothers have done, God in
the right of Joseph the Most High was a witness. And God has decreed for
Joseph a journey from the door to the door by the judgment of the Book
around the fire, hidden. And God has made the visitors of Hussein a
journey to the secure sanctuary by the permission of God the Most High,
and He is God, was mighty, wise.

Indeed, those who travel from the door to God in the depth of uniqueness
truly and justly around the door, solitary, those have found Hussein on
the hidden edge truly, witnessed in the depths of the well. Those are
the travelers in the Mother of the Book by the mention of the secret
around the fire, inscribed truly and justly. And God has created Joseph
and his brothers in the worlds of holiness from the drop of creation
from that water, existing. When we found from Joseph a love towards the
greater remembrance, we clothed him by the permission of God with the
garments of prophecy by those who were his share in the Mother of the
Book around the fire, decreed.

And we were not far from the truth about the worlds, and we recount to
the servants the matter which people were deceitfully far from the truth
about. God intended in your day today the greater word, this speaker
about God truly at 25, wonderfully. While people did not believe in God
and His signs truly, we preserved him in the depths of the well around
the fire, hidden. And God has decreed that some travelers among you, who
were in the Mother of the Book truly and justly in response at the door
around the water, praiseworthy earlier, might pick him up. That was a
judgment from God truly and justly, and the judgment in the Mother of
the Book was decreed. And God has made you Muslims in His religion if
you were with God and His signs truly and justly at that door,
patiently.

\section*{The Surah of the Consultation (Surah
12)}\label{the-surah-of-the-consultation-surah-12}
\addcontentsline{toc}{section}{The Surah of the Consultation (Surah 12)}

\markright{The Surah of the Consultation (Surah 12)}

In the name of God, the Most Merciful, the Most Compassionate

\emph{``They said, `O our father, why do you not trust us with Joseph
when indeed we are to him sincere advisors?'}

As the mention of the Most High God, which has been spoken in the Mother
of the Book around the fire, witnessed. God has revealed to me that I am
God; there is no deity but I. I have decreed the merit of the
remembrance as My own over all the worlds. And indeed, We have destined
for the believers, by the permission of God, gardens among gardens in
the land of divine pleasure around the sanctified House. And indeed, We
have made Hell for the disbelievers from the judgment of the Book,
encompassing by the judgment of the Book. By your true Lord, we have
transformed the bodies of the disbelievers into other than their own in
the coffin at the bottom of the fire, by the permission of the ancient
God, a transformation as recompense for what they earned by their hands
and they were ungrateful towards God and His signs before the door.

Do you believe in yourselves apart from the soul of the Most High God,
truly and justly nothing? And God has made clear His signs for the
people of the earth and the heavens truly and completely from the door
entirely. And none believe in God and His signs truly except the early
believers from the people of the door, a few. And most people have been
polytheists with their true Lord, decreed by the Book by the permission
of the Merciful. And none believe in God and His remembrance purely
truly except a few from the first. And indeed, We have made you a
support for the servants and a barrier for the lands, truly and justly,
by the permission of the ancient God, intensely. By the true God, only
those among the believing men and women who have the covenant of God and
our covenant on the pure religion will follow you, purely on the truth.
God, who there is no deity but He, entrusted to God your heart, for He
is the truth, and God was over everything capable. And He is the truth,
there is no deity but He, the Possessor of the great resurrection,
intensely.

And God has offered our guardianship to the heavens, the earth, and the
mountains, they refused to bear it and feared from it, but man bore it;
this great mention of God was high, and thus it has been in the Book of
God, the Preserver, under the name of the Encompassing, wronged, and in
the hands of the people, among those who do not recognize it from the
judgment of the Book, ignorant. And God has revealed His secret between
the lines at the point of the fire, and it is only mentioned by those
who came to the remembrance before the door, desiring towards the
praised God, and God was over everything a witness. And He is the one
who innovated your true self, appearing for our greatness on the
judgment of the Book, decreed. And indeed, we have made you in the
Mother of the Book by God the Most High, wise. And God has made you an
example of Our likenesses truly without how from indication and
definition, and God was over everything capable. And God has kept the
secret of His servant in the pole of the fire from this Book, for what
God has decreed in the knowledge of the unseen from His hidden secret on
the line around the concealment, inscribed. Thus the Merciful will
benefit you greatly in keeping our matter truly and justly as a reward.

When Joseph's brothers said to their father in the sight of the
Greatest, `Why do you not teach us the knowledge of Joseph, and we are
witnesses of God, the Sustainer?' And God was over everything a witness.
And God has made us upon Hussein with the secret of uniqueness hidden
around the fire, witnessed. And God informed about that sign, the
judgment of the Most High and His sons on the judgment of the concealed
uniqueness in the blindness of the identity hidden in the secret of the
throne of eternity shining from the light of eternity, on the judgment
of the Eternal around the water, inscribed. Thus the people disbelieved
in God after Hussein stood on the land of Karbala, conveying about the
station of love for himself, uniquely apart from the ancient God, and
God was over everything knowledgeable. Thus the people turned away from
God and from the reward of the two paradises on the disbelief of Satan,
cursed, and they followed his partnership, the kingdom of Satan,
displayed apart from the true God, and He is God, was mighty,
praiseworthy. God, who there is no deity but He, truthful in the
statement on the tongue of the door truly and justly, appreciated. Thus
those who fought Hussein on the land of the Euphrates will be punished
with the severest punishment and the harshest torment truly and justly,
greatly. What is wrong with you, O troops of Satan, has not the truth
come to you on its horse on the day of Ashura after it has conveyed to
you the greatest command of God from Himself truly in the truth
intensely? Did he not seek water for himself truly and for his youngest
child, the infant, humbly on the earth at the source of the great
command, weakly? O people of polytheism, is there not among you a soul
that fears God for itself and conveys water to the water at the last
breath truly and justly, God the true, a little drop? God knows the
heart of Hussein and his heat from the great thirst and his patience in
the God, the One, the ancient, and God was over him a witness truly. By
the true God, we found his heart on that day hotter than a piece of iron
heated in the ancient fire, and God did not witness for Himself except
as He witnessed for Himself. So await the taking of God the Most High
truly and justly on the land of Hell intensely. God killed a people who
killed him at the height of pain to the extent of injustice. What is for
these polytheists and ourselves, so soon you will see them in the land
of the gathering, the power of the Greatest God in us truly. And we have
judged upon them with the greatest fire truly, forever and eternally.
And the truth by the truth said, `I will indeed fill Hell with them as a
recompense for what they earned, God the true, and we will not judge by
lifting the eternal punishment on them truly, by the true God, from
something to the smallest part of the flake of flake.'

\section*{The Surah of Paradise (Surah
13)}\label{the-surah-of-paradise-surah-13}
\addcontentsline{toc}{section}{The Surah of Paradise (Surah 13)}

\markright{The Surah of Paradise (Surah 13)}

In the name of God, the Most Merciful, the Most Compassionate

\emph{``Send him with us tomorrow to frolic and play, and we will surely
watch over him.''}

Blessed is He who there is no deity except Him, the one with the command
in His hand, and He is God, was over everything capable. And indeed, We
have decreed for every age, truly and justly, a challenge, and with
every hardship, there is ease, truly and justly, so that people may know
that the gate of God is the truth, and He is God, was a witness with the
believers. And He is the one who has made the sun and the moon by a
calculation from the fire, and We have ordained for them in the book of
decree, in the stations of the measure, truly and justly, in the pole of
the fire, an adjustment. The sun is not to overtake the moon in its
course, nor the moon to overtake the sun in its light, and for each, a
term is prescribed on the permission, and nothing exists except that its
decree is in the Mother of the Book from that gate, inscribed.

And He is the one who has sent down from the cloud of His power these
verses from an olive tree that was neither of the east nor of the west,
neither earthly nor heavenly, upon those lines in the line of the
equator, on the paths of equality, by the permission of the Merciful,
around the gate, a revelation. O servants of the Merciful, if you
believe in the true God, do not doubt the matter of the remembrance, for
God, your protector, the truth, was over everything capable. And indeed,
we have made clear the signs in that book truly for those of vision from
the people of the gate, who were in the book of God, the Mighty, around
the fire, entrusted. Those who believe in their Lord on the true value,
and they are on the high path in this valuation, straight. They are
those who do not set up another god with God and fulfill the covenant of
the true God on the line of justice, praiseworthy. Those are the people
of Paradise, by the permission of God, dwelling therein eternally,
seeing only spirit upon spirit from the mention of the Most High God,
and He is God, was mighty, great.

They are those who dwell on the red terraces around the gate, stationed,
seeing in it neither sun nor air except that they meet the angels with
the mention of the Most High God, on the name of the Living God,
witnessed. This is the day God promised to entertain His servants, and
they say from our side upon them, ``Peace, peace.'' O servants of the
Merciful, did we not make the earth flat for you truly, and We have
decreed the sky above you at the center of the throne, preserved? And
indeed, we have created you by the permission of the ancient God, with
import, truly in phases. Why do you not hope for the great God
respectfully? Did we not make the sea raging for you? Did we not make
the earth subservient that you may extract from it what you plant for
yourselves from the bounty of the gate, the true God, appreciated? Why
do you not believe in the mention of the Most High God truly and justly,
a little? Why do you not hope for the great God respectfully? How do you
deny, untruly, the signs of the praised God, derisively, untruly? Soon
the fire will encompass you at the bottom of the coffin by the
permission of God the Most High, and He is God, was anciently
knowledgeable.

And indeed, God has prepared for the polytheists among you on the line
of justice, a perpetual punishment truly and justly, painful. And when
they said, Joseph's brothers to their father truly, ``Send our brother
with us tomorrow on a secret from your hidden secret around the line,
hidden.'' And indeed, we intended to be with Joseph tomorrow truly and
justly, with the truth around the fire, witnessed. And it is the day
that God had decreed wherein we said to the greatest word on the squared
secret truly and justly, the ancient God who there is no deity except
Him, and God was over everything a witness. And when they said, ``O our
father, send him with us tomorrow truly so he may settle at the point of
ice from the cold mountain around the point of connection with something
from the point of fire from the mountain of justice around the water of
bounty truly and justly,'' and indeed, we were for him, by the praised
God, truly preservers. And our followers will ask about us about the
appearance of our matter on this greatest gate truly and justly, not
from a letter from the knowledge of the book much, and indeed, they say,
``Send him with us tomorrow to frolic and play, and indeed, we were on
the truth truly in his matter on the great gate by God, the Preserver,
preserved. God the Greatest from their patience at the gate to the gate
at a point from the gate, God the Most High, a little, and indeed, we
know what no one besides us knows, and the soul of the gate, and God was
over everything on the truth knowledgeable. Do you not think about the
creation of the heavens and yourselves thoughtfully on the truth truly
with the truth lightly? Indeed, God did not create the heavens and the
earth and what is between them truly and justly in vain, aimlessly. He
is God, the truth, who there is no deity except Him, intended from you
that you worship only Him in the way of this gate purely for Him on the
truth truly, and God was rich concerning the worlds. God, worshipped,
there is no deity except Him alone, no partner to Him, indeed, He is the
truth. By the truth, the god of all the worlds, if there were any beside
Him on the truth, god, they would need untruly something with the truth,
a third.''

\section*{The Surah of the Holy (Surah
14)}\label{the-surah-of-the-holy-surah-14}
\addcontentsline{toc}{section}{The Surah of the Holy (Surah 14)}

\markright{The Surah of the Holy (Surah 14)}

In the name of God, the Most Merciful, the Most Compassionate

\emph{``He said, `Indeed, it saddens me that you take him away, and I
fear that a wolf may eat him while you are heedless of him.'\,''}

Remember God in the greatest word, standing around the fire, witnessed.
So listen to what is revealed to you from your Lord; indeed, He is the
Truth, there is no deity except Him, and God was over everything
capable. God has sent down the book, in it is the clarification of
everything, and a mercy and good news for our first followers who were
with the remembrance of the Most High God in that gate, hidden around
the fire. Soon, God, your true patron, will show you His throne on the
Day of Recompense on the line of the equator, and you will indeed
witness that His power was equal over all worlds. There, people will
recognize the true matter of God regarding the gate truly and justly,
powerfully around the fire, great. And on that day, you will surely
testify to yourselves in servitude, and you will not find for yourselves
a bit of flake from some things truly and justly, an aid.

And what you wish, if you wished, but God has willed as He wished with
what you wished in the remembrance of God, the Wonderful, truly and
justly, from the point of fire, a revelation. And indeed, we have made
the stories of the book by the permission of God from the self of power,
who was in the Mother of the Book, knowledgeable. And God was over
everything capable. And indeed, the command has come to you from God
truly blessed. Do you conceal the truth and act with falsehood? What is
wrong with you, do not you believe in God, the Praised, your Creator,
and indeed, He was over what His servants did on the truth, truly great,
a witness. Soon the Merciful will taste you in the valuation from a fire
that was from the fire of Hell, blazing, a recompense for what you
disbelieve in the signs of the book from God the true, what was revealed
to our servant, and you were from it untruly, gasping and ungrateful.

O servants of God, has not the foretelling of the first come to you, and
the judgment from God has been known to you on the truth, truly
repeatedly, and He who is like nothing else and He is God, was mighty,
wise. Did we not guide you to the ways of those before you, and you do
not find in yourselves for our language on the truth, truly, a change?
The way of God which has passed before you, so by your Lord, people will
not find a change to the way of God, the Mighty. And indeed, we grant
favor to whom we will by the remembrance of God, the true, who was in
the response with the remembrance of the Most High God in the Mother of
the Book, on the truth truly around the gate, mentioned. And indeed, we
have elevated him from the holy place, a location on the truth truly,
witnessed, to be a sign by the truth for those before you and after you
by the permission of the ancient God, truly great, great. And He is the
witness over you by our permission, so soon will God show you the
greater remembrance in the valuation of what you earn for yourselves
from your secrets and openly, by the permission of God on what the book
counted truly and justly, preserved.

Be patient in God, for God was over the worlds, knowledgeable. Be
patient, O remembrance of God, patiently on the truth in the truth,
beautifully. And indeed, we have made you over the people by the
permission of God the Most High, an overseer and a reckoner. So soon
will God inform them on the distant earth by the permission of God the
Praised, truly and justly, near. So by your Lord, we are closer to you
than yourselves to yourselves. What is wrong with you, do not you see by
the remembrance of God, the Impregnable, in the truth, seeing? The day
He reveals the shin by the shin, people will call us by the mention
around the gate, humbly for the truth truly, intensely. And that day is
truly at your Lord's; say then whoever wills takes me to God, his Lord,
the Rich, truly and justly on that truth, a path.

So soon will the Merciful call you to prostrate to Himself, and you will
not be able to. So from among you, whoever trusts God on the truth as a
guardian, there the guardianship is God's truth, and you had no helper
apart from the Merciful on the truth truly. So soon will God show you
His signs in the robes of the sun by the decree of the book on the paths
of the gate truly and justly, near. The thunder glorifies His praise,
and the angels from His fear, and He is God, was over everything
encompassing. And there is nothing but glorifies with His praise, and
you do not know from the knowledge of the book from some letters on the
truth truly, anything. God, who there is no deity except Him, has
revealed to me that He is the truth from God, and God has not decreed
for our words in anything of the book on the point of the gate, a
change. And He is God, the truth, was over everything a witness. And
indeed, we have made the signs in that book by the permission of God the
Praised, securely on the truth truly, irrevocably. And we do not reveal
in it a letter of similarity for what God has decreed for the believers
in this book around the gate, witnessed. And indeed, it saddens me that
you go with him after what God has revealed on the truth truly in that
book from Him, elevated. And I fear that a wolf may eat him, and you are
untruly from the assumption of the devil, far. And if not for the fear
in his matter from what God knows in your hearts untruly, the matter
would have been like the sun at the point.

\section*{The Surah of the Will (Surah
15)}\label{the-surah-of-the-will-surah-15}
\addcontentsline{toc}{section}{The Surah of the Will (Surah 15)}

\markright{The Surah of the Will (Surah 15)}

In the name of God, the Most Merciful, the Most Compassionate.

\emph{They said, ``If a wolf devours him while we are a group, then we
would be losers indeed.''}

Taha. He is God; there is no god but He. There is nothing like unto Him.
To Him belongs creation and command, and He has knowledge of all things.
By the Lord of heaven and earth, this book is from God, the Truth. God
suffices as a witness over His servant, in truth and with truth. God is
the Self-Sufficient, and you are the needy ones before the Most
Merciful. What God has decreed in this gate has been ordained.

So follow what God has sent down to you in His book, the true and pure
truth, openly and strongly. Fear God concerning this gate with truth,
and be the best helpers in the way of the gate of the Most High, the
Praised One. God has decreed your matter in His book, in the secret of
the book with truth, in the cycle of water. So pardon the people and
forgive much. God is a witness to what you do. God loves those believers
in this gate whose attributes are in the book of God, the Most High,
with truth, in the cycle of fire, written. He is the One who has decreed
your sustenance in the sky, apportioned if you desire God and the
Hereafter.

By your Lord, the Most Merciful, God has not made for you besides the
Most High a servant with knowledge of the book, the truth, from the
truth, in the secret of the letter B, the All-Knowing. He is the One who
has decreed your affairs in His mighty book, in the ways of equality, in
the cycle of the gate, organized, so that you may seek His bounty and be
grateful morning and evening, with truth, following the judgment of the
book much. He is the One who has shown His signs to His servants with
the permission of God in the glorious book, with truth, the Praised One,
in the ways of equality, uniquely, so that you may learn from the pure
mention of God and what you need in the way of His love, with truth,
asking and being thankful.

God has intended in this book all goodness, in the name of the gate, in
the secret of fire, decreed in the mother of the book. He is the One who
has made our hearts the abode of His command and our breasts the vessels
of His secrets by His power. God is capable of all things. He is the One
who has confirmed our actions in His praiseworthy book to the discerning
believers, in a decisive verse of the Quran. ``And you did not throw
when you threw, but God threw,'' with truth, capable of all things.
Those who commit sins falsely after the truth has come to them from God,
the Most High, powerfully, we will burn them in the Hereafter in a great
fire as a recompense for their evil deed, justly with its like. I am not
unjust to the servants, with truth, even a tiny bit. Those who do
righteous deeds in our remembrance, God will give them of His bounty,
with truth, manifold and many times. God is capable of all things.

O people of the Criterion, if you are in truth among the people of the
Quran and this remembrance is the truth, the same book, then return to
it by the permission of God, your true Master. God has made the return
on the Day of Resurrection witnessed. God has decreed what He has
ordained, and the matter in the mother of the book concerning fire has
been fulfilled. When the brothers of Joseph said to their father, ``If a
wolf devours him, we are surely losers,'' on the earth around the fire,
they were mentioned in truth. We have revealed to you what God has
placed in the depths of this verse, with truth, so that people may
believe in God and His signs, with truth, powerfully.

When they said, ``The letters `There is no god but God' are on the day
of the beginning on the land of the heart around the fire, famous.''
``If a wolf devours him, we are surely a group for him,'' and God is
powerful over all things. We have not made it in truth in the secret of
Joseph on the land of the Euphrates around the fire, envied. We do not
want you to be other than His caller on the earth, with truth, a
witness. What do you think of us concerning the truth with God, the
Truth? Loss. We have made, by the permission of God, our signs upon the
worlds, with truth, around the fire, a guardian. God has permitted them
in revelation and interpretation as they will, by what they will, with
truth, praised. We have protected them from indication and from what you
were away from in untruth, forsaken. God has borne witness, and God
suffices as a witness with our signs, with truth. By God, the Truth,
they only will what God wills, their Lord, in their secret and their
openness, with truth. God is a witness over all things.

Know, servants of the Most Merciful, that we have made clear our signs
in this book, with truth, in the cycle of the gate, so that people may
be praiseworthy by God, the Praised One. Perhaps people may be thankful
to God, the Most High. Say, ``Call upon God and His name, the Most
Merciful, in the way of the gate, for God has the best names, from the
gate, veiled around the fire.''

\section*{The Surah of the Throne (Surah
16)}\label{the-surah-of-the-throne-surah-16}
\addcontentsline{toc}{section}{The Surah of the Throne (Surah 16)}

\markright{The Surah of the Throne (Surah 16)}

In the name of God, the Most Merciful, the Most Compassionate.

\emph{When they went away with him and decided to put him in the depths
of the well, We inspired him, ``You will surely inform them of this
affair of theirs while they do not perceive.''}

Truly, God has sent down the Book, wherein is an explanation of all
things, a mercy and good tidings for all the worlds. Praise be to God,
who sent down the Book in truth upon our word, so that people might be
mentioned in this gate around the fire in the Mother of the Book.

This is Our Book, speaking the truth against you. Yet most people do not
believe in God and His signs concerning this gate with truth. God is a
witness over all things. O servants of the Most Merciful, We have given
you glad tidings by God's permission in the Greater Reminder from the
ancient command, with truth upon truth, in this gate, a weighty matter.
What God wants from you this day is obedience to God, the Truth, to His
command, with truth, the upright truth from God, the Ancient, a reward.

Whoever obeys God in Our command, with truth, out of love for the
Greater Reminder of God, and compliance with it, will meet God above the
throne, pleased and well-pleasing. But whoever denies Our command
rebelliously, in untruth, will taste from the Merciful a tree of boiling
water emerging from the root of Sijjin, justly, a Zaqqum. He will have
no covenant from God in Our hands, truly and purely. Whoever desires
God's covenant fulfills Our covenant in this Reminder of Ours, openly
and strongly, with truth.

Fear, O servants of God, the day when the command comes to you from God,
the Truth, suddenly and closely, with truth upon truth. On that day,
people will possess nothing of Our kingdom except by God's permission,
the Truth, through Our servant. God is knowledgeable of all things.
Except those who have Our Book around their necks in truth, who were
responsive to the Reminder before, with truth and righteousness. The
Reminder has fulfilled its covenant with those who came with truth to
the righteous covenant.

Fear, O servants of God, the severe seizure of your Lord, the Most
Merciful, in truth. God has prepared shackles and chains and fruits with
a bitter taste and a great torment for those who disbelieve in the Book
and Our Reminder weakly, and who believe in it only mockingly, falsely
and vainly. These are the polytheists before God. They have no judgment
from God except the fire of God, the Most High, severe.

Shall I inform you, O servants of God, of Our seizure upon the people on
the Day of Decision, with truth upon truth, severe? Whoever turns away
from God's Reminder, adorned falsely, and in untruth, deceitfully, by
the Lord of heaven and earth, We will make them taste on the Day of
Resurrection by the judgment of the Book, with truth, the fire of the
fire from the tree of fire at the bottom of the coffin, as decreed in
the Mother of the Book.

When the brothers of Joseph went with him to the land of oneness,
secretly, around the inscribed line, they placed the letters of oneness,
the letter H, in the depths of the well, secretly from the heart, around
the fire, veiled. We inspired him, ``You will surely inform them of this
affair of theirs.'' God will show them this letter in the place of
secrecy around the gate, witnessed. The letters of oneness do not feel
anything except God's feeling in the place of the heart, in truth from
their Lord. God is a witness over them, with truth.

God has veiled that word with a veil of the secret around the throne,
hidden. Those who desire God and His signs in the ways of the gate, with
the letter H, these are in truth around the fire, witnessed. God created
Joseph and his brothers from the blessed tree of oneness, and people do
not know their secrets, what is with God, the Truth, in the Mother of
the Book, written around the fire. Do they not know that they do not act
except by God's permission, their true Lord? God is truly worshiped
after denial and affirmation. The Most Merciful will not ask you on the
Day of Resurrection about what they do for themselves; He will ask you
about what you do in your secrets and your openness. God is
knowledgeable of what you do.

God, there is no god but He, is the truest in speech, unique. God has
ordained the share of people in the secret of all signs, except in
weakness against the truth. God, the Truth, is truly submissive. God has
made His own friends praiseworthy in their deeds, with truth upon truth.
God is capable of all things.

Look, O servants of God, We have placed, by God's permission, Our
servant in the depths of the well, around the fire, veiled with truth.
God has placed in this gate a secret upon a secret, hidden. You will
soon know what you were ignorant of before, with truth upon truth, in
the garment of the sun, by God's permission, the Most High.

O believers, whatever the Messenger gives you, take it, and whatever he
forbids you, abstain from it. God has made him wise with truth upon
truth. God has made the Reminder encompassing over people,
knowledgeable. God is capable of all things.

\section*{The Surah of the Gate (Surah
17)}\label{the-surah-of-the-gate-surah-17}
\addcontentsline{toc}{section}{The Surah of the Gate (Surah 17)}

\markright{The Surah of the Gate (Surah 17)}

In the name of God, the Most Merciful, the Most Compassionate.

\emph{They came to their father in the evening, weeping.}

Truly, this is the Book in which there is no doubt, a guidance from God,
the Truth, for all the worlds. We have made it a guidance for Our
servants, those who are in this gate, upon the gate, in truth and with
truth, around water, witnessed. We have joined you with souls from Our
own soul, who are pure in truth and with truth, around fire.

When the truth comes to you, follow it. God has made it in the Book from
Us, in truth and with truth, accounted. We have made it with the
knowledge of the Book by God's permission, the Most High, from the point
of fire, knowledgeable. We have confirmed it from God, who is without
peer. God is capable of all things.

O servants of God, if you ask Him for something and He does not respond
to you in truth, do not grieve, for it is by God's command from Us, in
truth and with truth, serene and praiseworthy. We have shown you the
matter in your dream in truth, and if you had revealed it to them, they
would have disputed over the matter. God, your Lord, the Truth, knows
what is in the breasts. We do not change the condition of a people until
they change what is within themselves.

Taste the torment of the blazing fire for what you turned away from this
gate. Your Lord is strong and knowing over His servants. We will destroy
the wrongdoers with the justice of the people of Pharaoh, with severe
torment and great punishment. Do not grieve over those who deny in your
presence and rely on God, your Lord. He is the All-Hearing, All-Knowing,
in truth. God is capable of all things.

O people of the earth, whatever you spend in the way of God, the Truth,
you will find it in the hands of the Preserver, safeguarded in this
gate. O people of the earth, believe in the light that God has sent down
with me, purely and truly, and do not follow the footsteps of Satan, for
he commands you to associate partners with God, your Lord. God does not
forgive associating partners with Him, but He forgives whatever is less
than that for whom He wills. God is knowledgeable of all things.

God has decreed for the believing emigrants the forgiveness of the
Reminder and the greatest pleasure, according to the judgment of the
Book, in the judgment of the Book around the gate, decreed. We have
decreed that some kin have greater rights over others, as God has
decreed in the Mother of the Book, in the secret of the gate, inscribed.
O believers, God has not sent down a sign in the Book, nor in the
horizons, nor in the souls, except that people may know, in truth, that
the Reminder is the truth from God. God is knowledgeable of all things.

O people of the earth, by your Lord, who is without peer, God has left
nothing for Himself after the Reminder and this Book as a proof. Be
patient with the truth of God, the Praised One, in this gate. Now, with
the truth, the perishing ones perish clearly, and the believers live
clearly. God is capable of all things.

O people of the Throne, hear My call from around the fire. I am God;
there is no god but Me. Worship Me and establish prayer for the greatest
Reminder, purely, without people. Your Lord is God, the Truth, the True
One. Those you call upon besides Him are the companions of the fire,
justly. The Reminder is the true path, the straight line, around the
fire, upright.

O people of the earth, fear God and do not let Satan deceive you from
the truth. The Reminder is the truth, with truth. You and those you call
upon besides Him are the people of the fire, in the Mother of the Book,
written. O people of the earth, have you not considered the creation of
the heavens and the earth? If there were two gates from the Reminder,
they would have been corrupted. God has managed the kingdom by His gate,
the truth. God is knowledgeable of all things.

O believers, fear the Day of Truth. We have gathered you around the fire
and will question you about what you have done with the Reminder. In
truth, We will make the polytheists taste the heat of the fire with the
greatest torment. We will give the patient ones the best reward in the
land of saffron, by the judgment of the Book, from the judgment of the
gate, a resting place. We have sent it down to the people with truth.
People have no choice but to submit and accept. The Reminder clarifies
the truth from God, the Most High, uniquely.

The Reminder is from God to give you glad tidings of His promise and to
warn you of His vengeance. It is hidden in the Mother of the Book, in
the secret of the Book, around the point of fire, preserved. The
Reminder is before you, behind you, on your right and left, by God's
permission, with truth, so that people may be mentioned in the wonderful
signs of this Book, around the gate, in truth. This has been a station
for you on the path, clearly known. Fear God, O possessors of intellect,
in the secret of God, hidden in the line of this Book, as God has
decreed around the fire, veiled.

God has not asked you for any reward in this Book for the Reminder, nor
for truth upon truth, except affection for near kin, inscribed around
the gate in the Mother of the Book. We have intended to make the
believers witnesses on earth, by God's permission, the Most High. God is
capable of all things. God has made the Reminder better for you than
yourselves, for yourselves, to recite God's signs to you, to purify you,
and to bring you out of darkness into light. God is a witness over all
things.

This is from the news of the unseen, which We reveal to him, so that
people may be mentioned in the wonderful Reminder, with truth, in this
gate, around the fire, witnessed. Glory be to the One who is without
peer. God has given His servant what He has not given anyone from the
gates, with truth upon truth. You have been given knowledge from the
greatest Reminder, with truth upon truth, only a little from the letters
around the water.

O people of the earth, do not follow the greatest Reminder in what the
Ummiya did with Hussain, unjustly, in the sacred land. By God, the
Truth, it is the truth, and God is a witness over it. The believers came
to Us in the evening after the Book to weep for the greatest gate. Tell
them to settle in the abyss of oneness. God is knowledgeable of what you
do. Tell the believers not to follow Satan, for he has been a clear
enemy to the believers in the Book of God.

\section*{The Surah of the Path (Surah
18)}\label{the-surah-of-the-path-surah-18}
\addcontentsline{toc}{section}{The Surah of the Path (Surah 18)}

\markright{The Surah of the Path (Surah 18)}

In the name of God, the Most Merciful, the Most Compassionate.

\emph{They said, ``O our father, we went racing with one another and
left Joseph with our belongings, and a wolf devoured him. But you would
not believe us even if we were truthful.''}

Kaf Ha Ya Ain Sad. Praise be to God, who created the heavens and the
earth in truth. Their judgment is decreed from the point of the letter B
in the center of the fire around the gate.

We have sent down this book upon you in truth as a clarification, and we
have made your sign by God's permission in this book, determined by the
judgment of the book, decreed in truth and with truth. Even if all
people gathered, they could not produce a letter like it, even if some
of them supported others. God, there is no god but He, the living, the
worshipped, the ancient. He is knowledgeable of all things, and they
encompass nothing of the knowledge of the book except what the Reminder
wills. God is capable of all things.

God has created for you a path, this gate, extended, and has ordained
for you a mountain, raised in truth and with truth. God has intended to
bring you out of darkness into the book, witnessed in truth, so you may
seek His bounty from what you were in, away from untruth, forsaken. Seek
forgiveness from God, your Master, there is no god but He, the Truth.
You will find God oft-returning, merciful at the gate.

He is the one who sent down from the sky upon you, in truth, water from
the gate, abundantly. He brought forth for you from your land gardens of
joy and colors by His power. He is capable of all things. So you may
know that your Lord, the Most Merciful, is knowledgeable of all things.
There is no hidden thing in the heavens and the earth except that We
have made it in this book, around the gate, concealed. God will inform
you on the Day of Resurrection from the tongue of the Reminder about
what you used to do in your secrets and openly. He is encompassing of
all things in truth and with truth.

God has made for us a book, preserved in truth and with truth. God
erases what He wills and confirms what He wills, and He has made the
Reminder in the Mother of the Book, around the fire, concealed. Those
who act in the guardianship of the people of God, the Truth, will
witness their deeds from the Reminder, mentioned. Those who earn sin in
the path of the tyrant, God has not decreed for their deeds any weight
on the Day of Resurrection. God has prepared for them at the bottom of
the coffin a great fire. He will show them their deeds as regrets upon
themselves, scattered dust on the earth, like a mirage in a desert. The
thirsty one thinks it to be water, but when he comes to it, he finds
nothing. They will find their deeds in Hell, fire upon fire, iron. This
is the recompense for what they used to deny our signs in truth with
untruth.

O servants of God, know that the proof of God has been in the Mother of
the Book in that gate, eloquently. God has not made the command of the
Reminder but Our command, raised in truth and with truth. O people of
knowledge, fear God in the Reminder, and God will teach you from the
tongue of the Reminder the interpretation of the book, in truth and with
truth, uniquely. God has sent down the signs in that book so that people
may witness our signs in that gate.

We have sent down the command upon Our servant, in what has been in the
Mother of the Book, in truth and with truth, altogether. He has been the
servant of God, in truth, on the upright path with justice, around the
fire, straight. O people, the command has come to you from the Most
Merciful, in truth and with truth, around the fire, uniquely. God has
not willed on this day, except His servant, the beloved, in truth and
with truth.

Fear God from a day that has been new in the Mother of the Book. God has
informed the believers of the judgment of the brothers of Joseph after
returning from the land of oneness, what they did to Joseph in truth.
God is knowledgeable of all things. When they returned in truth, they
said, ``O our father, we went racing on the land of blindness and left
Joseph with the belongings of oneness from our Lord, who is without
peer. The watchers devoured him with an indication in the abyss of
innovation. They were mentioned in the Mother of the Book as the wolf.''

They said the letters ``There is no god but God'' upon the land of
blindness. ``You would not believe us in prostration even if we
witnessed in truth and with truth, praiseworthy.'' This judgment is the
truth from God, our Master, as God decreed the prostration of the stars
to Hussain upon the dust, judged by the gate around the fire. We have
been truthful in our testimony to the secret of oneness from Hussain, in
truth and with truth. God suffices as a witness over ourselves in truth
and with truth. God knows what is in the heavens and the earth. He is
the Most Merciful, upon the Throne, in truth, over all worlds equally.

\section*{The Surah of Sinai (Surah
19)}\label{the-surah-of-sinai-surah-19}
\addcontentsline{toc}{section}{The Surah of Sinai (Surah 19)}

\markright{The Surah of Sinai (Surah 19)}

In the name of God, the Most Merciful, the Most Compassionate.

\emph{They brought his shirt with false blood. He said, ``No, your souls
have enticed you to something. Patience is most fitting, and God is the
one sought for help against what you describe.''}

Alif Lam Mim Ra. Hear the call of your Lord on Mount Sinai: there is no
god but He, and I am the Exalted. What God has decreed in the Mother of
the Book has been concealed in truth. God has sent down the greatest
truth in His word so that people may know that God has preserved fire in
the point of ice.

We have made the Reminder from Us, raised in truth and with truth. We
have made it powerful with God, and exalted in truth. Fear God, O
servants of God. We have favored Our servant above many of those whom We
have created, so that you may know that God does what He wills and rules
as He pleases, witnessed in truth and with truth.

We bestowed favor upon you in your youth by the command of the Exalted
God, when the decree in the Mother of the Book was ordained. We have
sent down to you in your maturity this book, a book witnessed in truth
and with truth. God has ensured your true matter with Us, and the
promise in the Mother of the Book has been fulfilled. We have given you
a book in truth and with truth, witnessed in secret. We are the speakers
behind you, in truth and with truth, by God's permission, the Exalted,
the Powerful.

Convey to the servants Our command, which God has intended for you, in
truth and with truth, praiseworthy. Know, O servants of the Most
Merciful, if you are in this gate, in truth, fearing God, that God has
favored you above the worlds with Our command. Follow what has been
revealed to Our servant in this book of the world's judgments, all in
truth and with truth. God does not want from you today anything but His
remembrance, firmly in truth.

Did not the Most Merciful promise you His signs at this gate as a
reminder? Did We not say to you that We have a just measure in every
successor, in truth and with truth? Why do you not believe in Our signs
if you are sincerely with God, the Exalted, the Strong? God has willed
in this book Our secret command, in truth and with truth, so that people
may enter the houses of God through their doors, in truth and with
truth, humbly and submissively.

God has intended from the men of the house, Our first followers, who
were the keepers of the gate in truth, to establish them on the heights
around the gate, in truth and with truth, recognizing each by their
marks, by God's permission, the Exalted, the Pure, as noble steeds. O
evil servants, why do you not believe in this gate of God, the Exalted,
the Strong, openly? Did We not create you and mention you in the Mother
of the Book, around the gate, remembered? Did We not extend you and
guide you in the powerful way of truth? Did We not provide you
abundantly from the bounty of the Exalted God? God has made the command
of Our servant in truth and with truth, near and singular.

So await the victory of God for yourselves. When Our command comes, it
will come suddenly, greatly and mightily. This day is the truth from
your Lord. On that day, you will possess nothing of the knowledge of the
book, not even a letter, in truth, severed. The sovereignty that day is
the truth for the Most Merciful. You will not be able to speak, nor
whisper in truth. This is the straight path in the Mother of the Book,
in truth and with truth. God has decreed this path around the fire,
purely and straight.

They say, ``When is it?'' Say, ``It is with God, perhaps God's command
is near.'' When the brothers of Joseph came with God's shirt to their
father, with thin, reddish blood, witnessed around the fire, God
informed them that Joseph's blood was written as God's fire in the
Mother of the Book. God has made the unity of the gates in that greatest
gate false blood, thin, as written in the Mother of the Book.

We say, ``No, your souls have enticed you with God's command. Patience
and reliance on God is most fitting for what God has decreed on the Day
of Remembrance, witnessed.'' God is the one sought for help concerning
the matter of Joseph and the gate. God is capable of all things.

\section*{The Surah of Light (Surah
20)}\label{the-surah-of-light-surah-20}
\addcontentsline{toc}{section}{The Surah of Light (Surah 20)}

\markright{The Surah of Light (Surah 20)}

In the name of God, the Most Merciful, the Most Compassionate.

\emph{And there came a caravan, and they sent their water-drawer, and he
let down his bucket. He said, ``Good news! This is a boy!'' And they
concealed him as merchandise, but God is knowledgeable of what they do.}

Alif Lam Mim. Listen to what is revealed to you from your Lord. Indeed,
you are in the sacred valley, at the point of fire in the heart of ice,
around the truth. God has decreed what He has ordained, and the
beginning has been witnessed at the point of conclusion. God, there is
no god but He, and He is encompassing of all things.

We bestow sovereignty on whom We will among Our servants by God's
permission, the Truth, without cause. God's word does not change in
truth and with truth. God has made us powerful with the mighty name.
When Our command comes in truth suddenly, the hearts will boil over in
truth and with truth at that gate, in a great rush. On that day, the
believers will rejoice at meeting Us, in truth and with truth, with joy
and happiness.

The hands of the Jews and the Christians have become bound in what they
say against Us, falsely and deceitfully. God has made Our hands
extended, spending on whom We will among Our servants, in truth and with
truth, from that gate abundantly. We withhold from whom We will among
Our servants, justly and with praise. No one has the right to speak
about Us falsely and deceitfully.

Know, O servants of God, that the light has come to you from the Exalted
God, brightly, so that you may seek His bounty from what He has decreed
in His dominion, in truth and with truth. Do not stray far from your
journeys from what God has decreed in your journeys to that gate, in
truth and with truth, closely. God is the All-Knowing, and you do not
know anything of the knowledge of the book in that gate. We teach those
who enter that gate, in truth and with truth, whom We will among Our
servants, those who were written as pure in the Mother of the Book.

We have decreed for the wrongdoers the fire of Hell, in the wondrous
command of the Exalted God, decreed. O servants of the Most Merciful,
God has made among you a reminder and a bearer of good news like
yourselves. If you love God, follow him; God will love you. God's
promise is true in the Mother of the Book, fulfilled. We have sent down
this book from God, blessed upon Our servant, so that you may believe in
it and support him. On the day when God calls to you on the land of the
heart, praised.

By God, the Truth, if you disbelieve after the Reminder has come to you
with strong proof from your Lord, We will make you taste on the Day of
Resurrection from the greatest torment in the depths of Hell, greatly.
Know, O servants of God, that God has favored us over you with His own
favor, and we have favored Our servant over you with Our favor upon you.
If you are patient with His signs in truth, We have ordained for you in
your book from before, the ordinances of the greatest name of God, in
truth and with truth, greatly. Spend in the way of God from what God and
His Messenger love, in secret and openly, so that you may be gathered on
the Day of Resurrection among the lines of the believers.

Whoever turns away from Our Reminder, God will not accept any of his
deeds. Satan has partnered with him in his command in truth, and he is
rejected at that gate by God. We have placed protection in the hearts of
those who believe in Our signs and glorify their Creator morning and
evening from that exalted gate. God is Mighty and Praised. Fear God, O
servants of the Most Merciful, and do much good.

O servants of the Most Merciful, take your adornment at every mosque.
This is God's command to you at that gate, in truth and with truth, as
decreed in the Mother of the Book. We have sent the caravan of love by
God's permission to this love. The vision of the heart let down its
bucket. He said, ``Good news! This is the truth. This is a boy unlike
any the eyes have seen.''

O people of blindness, conceal him as merchandise, altered from unity,
so that you may be mentioned by the Exalted God around the fire. We have
intended from this boy the greatest word. This is an Arab boy on the
land of the heart, pure. God has preserved him in the depths of the well
around the fire on the mountain of cold, in truth and for truth.

O believers, fear God and do not sell him for a counted price, away from
the pure abyss of oneness. God is seeing of what you do. Know, O
servants of God, in truth and with truth, that you should view him not
with the eyes of your hearts. You have purchased him for a paltry price,
counted dirhams. What God decrees for you in the book, the Reminder
knows. You have not been brought forth by the secret of the heart. You
come to Us with his shirt stained with false blood, and We say, ``No,
your souls have enticed you.'' Patience is most fitting, and God is the
one sought for help against what you describe falsely against Our
servant.

If you recognize him by God's sight within you, you have been guided
like those before you. You bring me his shirt stained with false blood,
a phantom, thin and red. But God has accepted this Reminder from you in
favor, for you cannot do without it in truth and with truth. Keep God's
secret within you. God has made for you a station on the path, standing.
Do not give wisdom to the foolish, for they have believed in the Exalted
God and His secret and are in the abyss of weakness, weak.

O servants of the Most Merciful, fear God concerning Our trusts in you
and guard them as you guard yourselves, beautifully in truth and with
truth. If you cannot guard them, return them to the truth and cast them
behind the crimson sea in the world of blindness. Conceal them in the
axis of splendor on Mount Sinai, in truth. You will find all your deeds
with God at this gate in a sealed book, hidden and preserved. God, the
Truth, has created you by His command. Do you find today, besides the
Exalted God, any helper?

\section*{Surah 21}\label{surah-21}
\addcontentsline{toc}{section}{Surah 21}

\markright{Surah 21}

\newpage{}

\part{Writings Revealed in the Year 1848}

\chapter*{Persian Bayán}\label{sec--persian-bayuxe1n}
\addcontentsline{toc}{chapter}{Persian Bayán}

\markboth{Persian Bayán}{Persian Bayán}

This is what has been revealed by the Lord, the Exalted, the Most High:

\emph{In the name of God, the Most Immaculate, the Most Sacred.}

\section*{Introduction}\label{introduction-1}
\addcontentsline{toc}{section}{Introduction}

\markright{Introduction}

1 Glorification and sanctification befit the sanctified court of the
majesty of His might and the glory of His sovereignty, for He has
eternally existed and shall forever exist by His own essence and being.
He has eternally and shall forever remain exalted by His primordial
loftiness, beyond the comprehension of all things.

2 He has not manifested the sign of His recognition in anything except
through the inability of all things to comprehend Him. He has not
revealed Himself unto any thing except by His own essence, for He has
eternally been exalted above any association with things.

3 He has created all things in such a manner that all, through their
innate reality, confess to Him on the Day of Resurrection that there is
none like unto Him, nor any equal, nor any peer, nor any companion, nor
any similitude. Rather, He has ever been singular in the sovereignty of
His divinity and ever mighty in the majesty of His lordship.

4 No thing has truly known Him as He deserves to be known, nor is it
possible for anything to truly know Him as He deserves to be known. This
is because whatever is attributed to Him with the mention of
``thingness'' has been created by Him through the sovereignty of His
will. He has manifested Himself through Himself in the loftiness of His
station.

5 He has created the sign of His will in the innermost reality of all
things so that they may attain certitude that He is the First and the
Last, He is the Manifest and the Hidden. He is the Creator and the
Provider. He is the Omnipotent and the All-Knowing. He is the
All-Hearing. And He is the All-Seeing and the All-Hearing. He is the
Subduer and the Self-Sustaining. He is the Giver of life and the Cause
of death. He is the Omnipotent and the Inaccessible. He is the Exalted
and the Most High.

6 He does not and will not signify anything except His own exaltation in
glorification, His sublimity in sanctity, His inaccessibility in
oneness, and His loftiness in greatness. He has no primacy except
through His own primality, and He has no finality except through His own
finality. Every thing that has been measured within Him or will be
measured has become a ``thing'' through its ``thingness'' and has been
realized through its ``existence.'' By Him, God initiated the creation
of all things, and unto Him, the creation of all things returns. He is
the One to whom all the Most Beautiful Names belong and have always
belonged.

7 The essence of His being is sanctified beyond every name and
description, and His luminous reality transcends all loftiness and
sublimity. His pure essence is sanctified from every restriction and
elevation. He is the First, yet cannot be known through it. He is the
Last, yet cannot be described through it. He is the Manifest, yet cannot
be defined by it. He is the Hidden, yet cannot be comprehended by it.

8 He is the first to believe in \emph{Him Whom God shall make manifest},
and He is the first to believe in the one who has been manifested. He is
the unique reality through whom the creation of all things takes place.
Through His provision, all things are sustained. Through His death, the
death of all things is revealed. Through His life, the life of all
things is manifested. Through His resurrection, the resurrection of all
things is disclosed.

9 A Countenance has arisen, the like of which the eyes of existence have
never beheld---neither before nor after this. This is the visage of
divinity and the countenance of lordship, settled under the shadow of
the face of God's divinity, and signifying the sovereignty of oneness.
Had it been known that the love of this Countenance could be tasted by
all things, its mention would not have ceased. When it was not bowed to
the essence of its being was created as it is and upon what it stands.
Otherwise, nothing would experience the taste of its love.

10 Light upon light within light leading to light upon light: God guides
to His light whomsoever He wills, and He elevates to His light whomever
He desires. He is the Originator and the Restorer. He is the One God,
the Singular, whose own self-manifestation has brought forth eighteen
souls. These souls were created from His essence before all things, and
He has established their recognition within the innermost reality of all
things. This serves the purpose of enabling all beings to testify,
through their own essence, that He is the One, the First, and the
Ever-Living.

11 No decree has been issued for any contingent being except the
recognition of His essence. This recognition sanctifies His being from
all that is other than Him. At His command, all things are created, for
to Him belong the creation and the command, from before and after. He is
the Lord of all worlds.

12 It should not remain hidden from the observer of these words that God
caused the Qur'án to return on the Day of Resurrection with the
manifestation of His own essence within it. Thereafter, He brought all
things into being anew. It was as though all things were created for the
first time, for everything that has been created exists for the Day of
the Manifestation of God. He is the One to whom all things turn and the
One to whom all things ultimately return. When He appeared with the
manifestation of His signs of power, there was no doubt that all things
attained, to the utmost extent possible, the meeting with God.

13 Once again, God, exalted and glorious, created the Primal Will and,
through it, brought forth all things. The mention of the creation of all
things as a wondrous act serves as proof that His creation has eternally
existed and shall forever exist. For when no contingent being existed,
God was still God, and there was no creation to restore. Verily, God has
always been in the loftiness of His sanctity. Whatever is beneath Him
remains confined within the limits of its own proximity.

14 The first creation of all things occurred at this very moment, which
is described as the Day of Friday. It has been decreed as a day of
remembrance of God. The Lord of Might and Glory created this wondrous
creation by His command and established it under His shadow until He
returns it. There is no doubt that God originates this creation and then
brings it back, and verily God is powerful over all things. The
arrangement of the creation of all things has been made in accordance
with the number of all things, through commandments revealed from His
sanctified realm. These commandments have arisen as the dawning rays of
His bountiful sun so that all things, through the remembrance of all
things in all things, might attain perfection in preparation for the
appearance of the final resurrection.

15 On that Day, every thing shall be recompensed with the reward of all
things, whether it be through His justice for the self or through His
grace for confirmations. His knowledge of all things before all things
mirrors His knowledge of all things after all things, just as His power
over all things before creation matches His power over all things after
their creation. God has eternally been all-knowing of all things and
omnipotent over all things. To Him belong the Most Beautiful Names, from
before and after. All who are in the heavens, on the earth, and between
them glorify Him. There is no God but He, the Mighty, the Beloved.

16 With the eye of certainty, observe that the gates of the religion of
the Bayán are arranged according to the number of all things. Beneath
the shadow of each gate, the angels of the heavens, the earth, and what
lies between them are prostrate by the permission of God. They magnify,
sanctify, glorify, and revere Him. They perform their duties and exalt
Him. On the Day of the Manifestation of God, which is the appearance of
the Point of the Bayán in its final state, all will return unto Him.
Whenever all things return unto Him, even the souls of those who had
refused, the fruits of all things shall be made manifest before Him.
Blessed is the one who is gathered on the Day of Resurrection in the
presence of God and who turns toward His countenance.

17 God has determined that each gate of the gates of all things serves
as a reality unto itself. To it, all accounts return by virtue of the
Bayán, according to what has been enacted in that gate. Hasten, then, to
this, and hasten again, and yet again, and once more, and once more.
This is because God is the swiftest of all reckoners. There may arise a
situation where not all gates of all things present themselves before
Him. In such a case, He will decree the return of the creation of the
Bayán, and the heavens elevated in the Bayán will be folded in His
grasp, just as the heavens of the Qur'án were encompassed before Him.
Though the gates of the Qur'án were numerous and varied for the
believers, when God brought forth the creation of the Qur'án to His
light, they all became a single reality, a single gate from among the
gates of remembrance. Thus does God do as He wills and ordains what He
desires. He is not questioned about what He does, but all shall be
questioned concerning all things.

18 At the moment when the entirety of the Qur'án is returned, and the
beginning of the creation of all things within the Bayán takes place,
the Point---manifesting lordship---shall stand upon the land named Basṭ.
The heavens elevated in the Qur'án shall be folded entirely and return
to the first Point. None shall bear witness to this except God and those
near Him.

19 Although no matter was revealed in the Qur'án more momentous than the
Day of Resurrection, God, the Reckoner, accounted for the total number
of souls who adhered to the religion of the Qur'án. Upon their return,
one soul from among all these souls stood in the presence of God. This
soul became the return of all things, and the creation of all things in
another dispensation was elevated by God's command. Take heed, O people
of the Bayán, to guard yourselves so that you do not veil yourselves
from God, your Lord, while you claim by night and by day that you love
Him or sanctify Him.

\section*{Vahid 1 (Tawhid - Oneness and
Unity)}\label{vahid-1-tawhid---oneness-and-unity}
\addcontentsline{toc}{section}{Vahid 1 (Tawhid - Oneness and Unity)}

\markright{Vahid 1 (Tawhid - Oneness and Unity)}

\subsection*{Gate 1 (There is No God But
God)}\label{gate-1-there-is-no-god-but-god}
\addcontentsline{toc}{subsection}{Gate 1 (There is No God But God)}

In the first gate of the number of all things, God, exalted and
glorified, has decreed as an obligation the word: \emph{``There is no
God but God, truly, truly.''}

All of the Bayán will return to this word, and the spreading forth of
the final creation will arise from it. The recognition of this word is
dependent on the recognition of the Point of the Bayán, whom God has
made the essence of the Seven Letters within it.

Whoever attains certainty that it is the Point of the Qur'án in its
finality and the Point of the Bayán in its beginning, and that it is the
Primal Will, self-sustaining, and through which all things are created
by its command and sustained by it, has attained true faith.

Its essence bears witness to the oneness of its Lord. Whoever does not
believe in it shall be cast into the fire. And what fire is greater than
that for one who does not believe in it? Conversely, whoever believes in
it shall enter affirmation. And what paradise is loftier than that for
one who believes in it?

This is the word that has been exalted, magnified, glorified,
sanctified, and extolled by its Lord morning and evening. Regard this
word as you would regard the sun in the sky. Look upon one who believes
in it as you would look upon a mirror, for every believer in the essence
of the Seven Letters derives their being from one of the names of God,
exalted and glorified.

Externally, such a one is a leaf from the Tree of Affirmation. All
things return to this one reality, and all things are created through
this one reality. This singular reality, on the Day of Resurrection, is
none other than \emph{Him Whom God shall make manifest}, who declares in
every instance: \emph{``I am God. There is no God but Me, the Lord of
all things. All besides Me are My creation. O My creation, worship
Me.''}

He is the mirror of God through whom the mirror of dominion is
manifested, composed of the Living Letters. Nothing can be seen within
Him but God. Whoever in the Bayán utters the word \emph{``There is no
God but God''} is thereby directed toward God.

Just as creation begins with Him, so too does creation return to Him.
The purpose of this world is that, at the time of the appearance of
\emph{Him Whom God shall make manifest}, people do not claim, ``We say
There is no God but God and that is the essence of religion.'' This is
because what they say is but a reflection of His sun, which was
manifested during His first appearance. He is more deserving of this
word than all the essences of creation, through His very self.

The mirror declares, ``The sun is within me,'' but before the sun, it is
evident that the mirror reflects its light. He proclaims: \emph{``We
have known you, O people of the Bayán! The loftiness of your existence
lies in the word of your Lord. Do not veil yourselves from Him Whom God
shall make manifest on the Day of Resurrection, in truth. For what you
speak within your hearts is like His appearance in your innermost
beings, and what He speaks is what God has borne witness to of
Himself---that there is no God but Him, the Sovereign, the
Self-Subsisting.''}

Today, any soul that utters this word, the essence of all religion,
undoubtedly speaks in the voice of Muhammad, the Messenger of God---may
God's blessings and peace be upon Him and His family---of the past. The
sun of this word has been within their hearts, and what manifests from
them today is but a reflection of it. Thus, in His next appearance, He
will return to them, in the appearance of the Point of the Bayán, not in
His first manifestation.

During His first manifestation, the tree of oneness had not yet been
elevated within the essences of creation. Now, after one thousand two
hundred and seventy years, this tree has reached the stage of bearing
fruit. Whoever is within it is but a reflection of the sun of the Point
of the Criterion (\emph{furqán}), which is identical to the Point of the
Bayán. Before Him, this must necessarily become apparent.

An example is set forth in the supreme word, upon which the entirety of
religion is established. Through its utterance, the foundation of all
religion is confirmed. In the hour of death, all will speak this word
and return to Him.

The reflections within mirrors inevitably return to their origin. When
the mirrors reflect the image of the sun, they return to it, for their
existence began with it. The mirrors' purpose lies solely in their
capacity as mirrors, reflecting the sun from which they originated. The
exaltation of the Word of the \emph{Furqán} in the past, and the
exaltation of the Word of the Bayán in the future, reflect a similar
pattern relative to the Sun of Truth.

Consider the various aspects arising from this Word, such as the
knowledge of the names of God, the recognition of the Prophet, the
recognition of the Imáms of guidance, the gates of guidance, and the
innumerable branches of subsidiary matters.

Each soul that becomes veiled by one of these aspects is thereby
separated from the reality of its own existence, which originated from
God and will ultimately return to Him, provided it stems from the Tree
of Affirmation.

The sign of its oneness should testify to the Sun. If, God forbid, it
does not testify to the Sun, it is unworthy of mention. Consider the
souls that associate themselves with the Qur'án. How much of their
judgment contradicts what God has revealed, as evidenced by their
actions. These contradictions pertain to their essence, not the branches
arising from their essence.

That which branches into what is less than the truth reverts to its
essence. If its essence does not testify to God, it is not worthy of
mention before Him. However, that which branches from true essences will
ultimately return to those essences. If these essences are steadfast
signs, residing firmly within the mirrors of their hearts rather than
merely being transient, they will return to their proper stations both
in origin and in return.

As the sun has eternally shone forth, these mirrors eternally bear
witness. The bounty of God has never been withheld or exhausted in any
circumstance. Blessed is the one who declares: \emph{``God is my Lord,
and I associate no partner with Him in His Lordship.''}

Verily, the essence of the Seven Letters is the gate of God. I shall not
invoke any gate alongside Him, and whoever believes in \emph{Him Whom
God shall make manifest} attains thereby the first gate of the One, the
First. Blessed are those who succeed on this great day. This is the day
when all are presented before God, their Lord,

\subsection*{Gate 2 (Muhammad and His Manifestations
Returned)}\label{gate-2-muhammad-and-his-manifestations-returned}
\addcontentsline{toc}{subsection}{Gate 2 (Muhammad and His
Manifestations Returned)}

The essence of this gate is that Muhammad---may God's blessings and
peace be upon Him---and the manifestations of His own self returned to
this world.

They were the first of His servants to be present in the presence of God
on the Day of Resurrection, acknowledging His oneness and delivering the
signs of His gate to all. God fulfilled His promise, as stated in the
Qur'án: \emph{``And We desired to show favor to those who were oppressed
in the land and to make them leaders and to make them inheritors.''}

He made them Imáms. The same reasoning that establishes the prophethood
of Muhammad---may God's blessings and peace be upon Him---is evident in
their return to this world before God and the learned ones. This is
proven by the signs of God, which no one on earth can imitate, for they
are utterly incapable of producing anything comparable.

There is no doubt that the nobility of a servant lies in affirming the
oneness of God, recognizing Him, acknowledging His justice, obeying Him,
and being content with His will. Similarly, there is no doubt that these
sanctified souls attained the essence of all loftiness and glory before
any other soul.

Every living being that imagines honor finds it only through the good
pleasure of God. There is no doubt that they were the first lights to
bow before God, accepting the signs that He revealed through His gate
and conveying them.

There exists no loftier station in the realm of possibility than this:
that the heart of a servant bears witness to God. No one should remain
veiled from their Beloved to the extent of nineteen times nine. Every
soul, in whatever it undertakes during its life, wills nothing but the
good pleasure of God. This is the ultimate aim of all.

There is no doubt that the good pleasure of God is only made manifest
through the pleasure of one upon whom God has conferred His proof. It is
certain that these sanctified lights attained the good pleasure of God
before all things. This is the loftiest elevation above every loftiness
and the most sublime distinction above all other distinctions.

It is beyond doubt that their return in the final manifestation is
greater in the sight of God than their initial appearance. Just as
prophethood was established in the past, today guardianship
(\emph{wiláyah}) is established.

Even though the appearance of the Point of the Bayán is identical to the
appearance of Muhammad, their return is also the same. However, through
the manifestation of God, all names exist under His shadow and testify
to Him. He is the First and the Last, the Manifest and the Hidden. To
Him belong the Most Beautiful Names. God has set apart their names
during this cycle as the Living Letters (\emph{ḥurúf-i-ḥayy}).

This refers to the fourteen sanctified souls and the hidden and
preserved name, which is associated with the Four Gates, the Lights of
the Throne, or the Bearers of Creation, Sustenance, Death, and Life.
These collectively amount to the number of \emph{ḥayy} (18),
representing the closest names to God. Everything besides them is guided
through their guidance. God began the creation of the Bayán with them,
and He will cause the return of the creation of the Bayán through them.
They are lights that have eternally existed. They have eternally been
and will continue to be prostrate before the Throne of Truth.

In every manifestation, they have been mentioned by a particular name
among the people. With each appearance, their physical names may have
changed, yet their essential names, which signify God and manifest
within their hearts, remain constant. Without the nearness of their
essence, they could not have stood in the presence of God, who has
eternally existed and will forever exist.

God has names beyond limit, yet all are revealed through these names.
Just as the guidance of all depends on their guidance, in the hearts of
these names nothing is seen but God. Indeed, within the heart of any
believing soul, whether male or female, there is no vision but that of
the name upon which their heart depends, granted by God. Within them,
nothing is perceived except: \emph{``He is God, the One. To Him belong
creation and command, from before and after. There is no God but Him,
the Living, the Self-Subsisting.''}

Every soul that believed in Muhammad, peace be upon Him, or in anyone
before Him, has returned under His shadow. \emph{``And each shall be
recompensed for what they have earned, and God is witness over all
things.''}

\subsection*{Gate 3 (Ali Returned)}\label{gate-3-ali-returned}
\addcontentsline{toc}{subsection}{Gate 3 (Ali Returned)}

This pertains to 'Alí, upon Him be peace, who returned to this world
with those who believed in Him and those beneath them. He is the second
to believe in the Point after Sín (*the letter ``S'').

\subsection*{Gate 4 (Fatimah Returned)}\label{gate-4-fatimah-returned}
\addcontentsline{toc}{subsection}{Gate 4 (Fatimah Returned)}

This pertains to Fáṭimah, peace be upon her, who returned to the life of
this world with all who believed in her and those beneath her.

\subsection*{Gate 5 (Hasan Returned)}\label{gate-5-hasan-returned}
\addcontentsline{toc}{subsection}{Gate 5 (Hasan Returned)}

This pertains to Ḥasan, upon Him be peace, who returned to the life of
this world with all who believed in Him and those beneath them.

\subsection*{Gate 6 (Husayn Returned)}\label{gate-6-husayn-returned}
\addcontentsline{toc}{subsection}{Gate 6 (Husayn Returned)}

This pertains to Ḥusayn, upon Him be peace, who returned to the life of
this world with all who believed in Him and those beneath them.

\subsection*{Gate 7 (Ali ibn al-Husayn
Returned)}\label{gate-7-ali-ibn-al-husayn-returned}
\addcontentsline{toc}{subsection}{Gate 7 (Ali ibn al-Husayn Returned)}

This pertains to ʿAlí ibn al-Ḥusayn, peace be upon Him, who returned to
the life of this world with all who believed in Him and those beneath
them.

\subsection*{Gate 8 (Muhammad ibn Ali
Returned)}\label{gate-8-muhammad-ibn-ali-returned}
\addcontentsline{toc}{subsection}{Gate 8 (Muhammad ibn Ali Returned)}

This pertains to Muḥammad ibn ʿAlí, peace be upon Them, who returned to
the life of this world with all who believed in Him and those beneath
them.

\subsection*{\texorpdfstring{\textbf{Gate 9 (Ja'far ibn Muhammad
Returned)}}{Gate 9 (Ja'far ibn Muhammad Returned)}}\label{gate-9-jafar-ibn-muhammad-returned}
\addcontentsline{toc}{subsection}{\textbf{Gate 9 (Ja'far ibn Muhammad
Returned)}}

This pertains to Jaʿfar ibn Muḥammad, peace be upon Them, who returned
to the life of this world with all who believed in Him and those beneath
them.

\subsection*{Gate 10 (Musa ibn Ja'far
Returned)}\label{gate-10-musa-ibn-jafar-returned}
\addcontentsline{toc}{subsection}{Gate 10 (Musa ibn Ja'far Returned)}

This pertains to Músá ibn Jaʿfar, peace be upon Them, who returned to
this world with all who believed in Him and those beneath them.

\subsection*{Gate 11 (Ali ibn Musa
Returned)}\label{gate-11-ali-ibn-musa-returned}
\addcontentsline{toc}{subsection}{Gate 11 (Ali ibn Musa Returned)}

This pertains to ʿAlí ibn Músá, peace be upon Them, who returned to this
world with all who believed in Him and those beneath them.

\subsection*{Gate 12 (Muhammad ibn 'Ali
Returned)}\label{gate-12-muhammad-ibn-ali-returned}
\addcontentsline{toc}{subsection}{Gate 12 (Muhammad ibn 'Ali Returned)}

This pertains to Muḥammad ibn ʿAlí, peace be upon Them, who returned to
this world with all who believed in Him and those beneath them.

\subsection*{Gate 13 (Ali ibn Muhammad
Returned)}\label{gate-13-ali-ibn-muhammad-returned}
\addcontentsline{toc}{subsection}{Gate 13 (Ali ibn Muhammad Returned)}

This pertains to ʿAlí ibn Muḥammad, peace be upon Them, who returned to
this world with all who believed in Him and those beneath them.

\subsection*{Gate 14 (Hasan ibn Ali
Returned)}\label{gate-14-hasan-ibn-ali-returned}
\addcontentsline{toc}{subsection}{Gate 14 (Hasan ibn Ali Returned)}

This pertains to Ḥasan ibn ʿAlí, peace be upon Them, who returned to
this world with all who believed in Him and those beneath them.

\subsection*{Gate 15 (Appearance of the
Proof)}\label{gate-15-appearance-of-the-proof}
\addcontentsline{toc}{subsection}{Gate 15 (Appearance of the Proof)}

This pertains to the appearance of the Ḥujjat, peace be upon Him,
through the signs and clear evidences in the manifestation of the Point
of the Bayán, which is identical to the manifestation of the Furqán.
Although the Point of the Bayán appeared first and the Point of the
Furqán appeared second, the manifestation of the Hujjat occurred in the
Fifteenth Gate.

The wisdom behind this lies in the fact that the Point, in its state of
abstraction, represents the pure manifestation of God. In the station of
divinity (\emph{ulūhiyyat}), it is manifest in the first position
mentioned. In the station of determination, which is the primal will
(\emph{mashiyyat awliyyah}), it is mentioned in the second position. In
the station of upholding all souls, which is specific to the Fourteenth
Manifestation, it is mentioned in the Fifteenth Gate.

The Point, in its station of primality, has eternally existed and will
forever remain. It is most deserving of embodying all names from the
essence of the names themselves. For example, when the name of divinity
is mentioned, the name of lordship is also present, along with all other
names. Despite this, it is always manifest under the name of divinity in
the station of the Point.

The examples of all names, in their exalted realities, appear from Him:
\emph{``He is the First when He is the Last, and He is the Hidden when
He is the Manifest. He is the One who is mentioned by every name at the
time when He is not mentioned by any but the name: There is no God but
Him, the Self-Subsisting Sustainer.''}

\subsection*{Gate 16 (The First Gate
Returns)}\label{gate-16-the-first-gate-returns}
\addcontentsline{toc}{subsection}{Gate 16 (The First Gate Returns)}

This pertains to the return of the First Gate to this world with all who
believed in Him in truth, and those beneath them.

\subsection*{Gate 17 (The Second Gate
Returns)}\label{gate-17-the-second-gate-returns}
\addcontentsline{toc}{subsection}{Gate 17 (The Second Gate Returns)}

This pertains to the return of the Second Gate to this world with all
who believed in Him in truth, and those beneath them.

\subsection*{Gate 18 (The Third Gate
Returns)}\label{gate-18-the-third-gate-returns}
\addcontentsline{toc}{subsection}{Gate 18 (The Third Gate Returns)}

This pertains to the return of the Third Gate to this world with all who
believed in Him in truth, and those beneath them.

\subsection*{Gate 19 (The Fourth Gate
Returns)}\label{gate-19-the-fourth-gate-returns}
\addcontentsline{toc}{subsection}{Gate 19 (The Fourth Gate Returns)}

This pertains to the return of the Fourth Gate to this world with all
who believed in Him in truth, and those beneath them.

\section*{Vahid 2 (The Day of
Resurrection)}\label{vahid-2-the-day-of-resurrection}
\addcontentsline{toc}{section}{Vahid 2 (The Day of Resurrection)}

\markright{Vahid 2 (The Day of Resurrection)}

\subsection*{Gate 1 (Recognition of the Proof and the
Evidence)}\label{gate-1-recognition-of-the-proof-and-the-evidence}
\addcontentsline{toc}{subsection}{Gate 1 (Recognition of the Proof and
the Evidence)}

This explains the recognition of the Proof (\emph{ḥujjat}) and the
Evidence (\emph{dalīl}).

The summary of this gate is as follows: God, exalted be His glory, in
every cycle sends forth a proof corresponding to the highest station of
loftiness by which the people of that cycle take pride. For instance,
during the time of the Qur'án's revelation, the pride of all people lay
in the eloquence of speech. Therefore, God revealed the Qur'án with the
utmost loftiness of eloquence and established it as the miracle of the
Messenger of God, peace be upon Him. In the Qur'án, God affirmed the
truth of the Messenger of God and the religion of Islam through its
verses, which are the greatest of evidences.

The greatness of this proof lies in the fact that all speak through
alphabetical letters, yet God revealed the words of the Qur'án in such a
manner that if all who are upon the earth were to gather and attempt to
produce a single verse comparable to those of the Qur'án, they would be
unable. All would be rendered powerless.

The secret of this lies in the fact that God revealed the Qur'án from
the Tree of Will (\emph{mashiyyat}), which is the Muhammadan reality, in
the language of the Prophet Himself. That Tree, being inaccessible,
reveals no letter except that it draws forth the spirit of the thing
upon its descent.

For example, if it is revealed: \emph{``We have initiated this creation
as a command from Us, established over all things.''} When the term
``initiation'' is mentioned, it encompasses everything that is
associated with the name of all things. This is because none but God
encompasses all things, and only His word can transcend and dominate
over all things. Through His utterance, all creation begins.

Similarly, if God reveals: \emph{``And indeed, We worship this creation
as a promise from Us; verily, We have power over all things.''}

At the moment of the descent of this word, it takes hold of the spirits
of all things and causes their return within the manifestation of this
verse. This ensures that, on the Day of Resurrection, all things shall
be present before God, confirming the truth of their return.

None but God is capable of this, for whatever God speaks emanates from
the Tree of Reality itself. Through it, the essence of a thing is
created.

If something lies below the station of \emph{ʿIlliyyīn} (the exalted
ones), it is negated from the Letters of Affirmation. Conversely, if it
is from the Letters of \emph{ʿIlliyyīn}, it is affirmed as part of them.
This is because the word of God is truth, and in everything where it is
revealed, the essence of that thing becomes attached to it so that it
may testify to its truth.

Thus, it has already been revealed: \emph{``The fire is true, and
paradise is true.''} The act of creation and the spirit of the Word are
established as truth in their respective stations. Any soul that
contemplates this will certainly observe that the spirits of truth are
embodied through the manifestation of the primal Point in the signs of
God within the realities of souls and the horizons.

As God previously mentioned in the Qur'án: \emph{``We shall show them
Our signs in the horizons and within themselves until it becomes clear
to them that it is the truth.''}

Until one perceives the essence of all things, which is the spirit of
their heart, they cannot comprehend the realization that \emph{``The
word of God is truth.''} It is not merely through verbal mention that
the realization of truth occurs within the essence of a thing. This
reality is unique to God, exalted and glorified. None but Him is the
creator of things, the provider of things, the one who causes their
death, and the one who brings them to life.

Every word that is affirmed or negated in His dominion occurs within the
shadow of that which He has revealed through His signs. In reality,
these words themselves are nothing other than what manifests through the
appearances of the signs of God and His words.

For when God mentions a believer, the creation of that believer is
brought into being. Similarly, when He reveals something as being below
the Letters of \emph{ʿIlliyyīn}, the spirits of that entity are created
in accordance with it. This is the secret of why the signs of God are a
proof over all creation and the greatest of evidences.

The clear evidences (\emph{bayyinát}) and the most magnificent
manifestations are proofs of His power and knowledge. There is no doubt
that during the cycle of the Point of the Bayán, the pride of the people
of understanding (\emph{ulú'l-albáb}) was in the knowledge of oneness,
the subtleties of understanding, and the inaccessible realities
recognized by those with true insight.

For this reason, God placed the knowledge of His proof, like the proof
of the Messenger of God, within the very essence of His verses. He
caused words to flow from His tongue at the height of oneness and the
loftiness of abstraction, such that every soul possessing the spirit of
oneness humbly submitted to Him, except for those who failed to
comprehend what He spoke to His Beloved.

From Him emanated countless expressions of wisdom and knowledge, which
none but God has understood or recognized. Even though the manifestation
of the Sun of Truth itself inspires all contingent beings to comprehend
His loftiness, it is through the verses and words bestowed upon Him by
God that He has drawn all existences to the radiance of His signs.
\emph{``He has no equal by which He may be known, no peer by which He
may be described, no likeness by which He may be compared, no partner
with whom He may be associated, nor any similitude by which He may be
likened. Exalted is God far above all such comparisons, supremely
glorified!''}

In Him, nothing is seen but God, and indeed, all are devoted worshippers
before Him. During this cycle, God has bestowed upon the Point of the
Bayán His verses and clear evidences, making Him an impenetrable proof
against all things. Even if all who dwell upon the earth were to gather,
they could not produce a single verse comparable to those which God
caused to flow from His tongue.

Any soul that contemplates these verses with certainty observes that
they are beyond human capacity. These verses are exclusively from God,
the One, the Absolute. God has caused His words to flow from the tongue
of whomever He wills and will not cause them to flow except from the
Point of Will (\emph{mashiyyat}). He is the sender of all messengers and
the revealer of all scriptures.

If this matter had been within human capacity, someone should have been
able to produce verses comparable to the Qur'án during the 1,270 years
from its revelation to the revelation of the Bayán. However, despite the
loftiness of their power, all who attempted to extinguish the Word of
God failed utterly.

Even today, if anyone examines the time from the initial revelation of
the Bayán until now, they will surely observe that those who
acknowledged the proof of its verses and proclaimed them to all were
indeed the proofs (\emph{ḥujjaj}) of God. Even if their proof had not
been outwardly evident, their spiritual exaltation and insight were
undeniable.

The lowest of the disciples of the late, exalted Sayyid---the Exalted
One (\emph{Siyyid-i-A`lá})---surpassed in wisdom and understanding the
scholars and sages of the earth. Among those who have affirmed the proof
of the verses, whether from this community or others, there has never
been any doubt concerning their elevated piety.

Though this statement is made in recognition of the weakness of people's
comprehension, it remains true that the testimony of God outweighs that
of all who are upon the earth. There is no doubt that the testimony of
God is only made manifest through the testimony of one whom He has
appointed as His proof.

The self-evident nature of the verses is sufficient as testimony,
rendering all who dwell upon the earth powerless before them. These
verses are a lasting proof from God until the Day of Resurrection. If
anyone reflects on the manifestation of this Tree of Will, they will
undoubtedly affirm the loftiness of the Cause of God. Consider that this
revelation emanates from a soul only 24 years of age, who was devoid of
any formal training in the sciences learned by others. Despite this, the
verses are recited with such fluency and clarity, without contemplation
or hesitation.

In just five hours, he composes a thousand verses in supplication,
without the pen ceasing. Interpretations and profound scientific
insights emerge from him, demonstrating exalted stations of
understanding and oneness that surpass the grasp of all scholars and
sages. All who encounter such works admit their inability to comprehend
them fully.

There is no doubt that all of this originates from God. Scholars who
have labored their entire lives in study and effort struggle to compose
even a single line of eloquent Arabic, often concluding with words
unworthy of mention. All of this is a proof intended for creation, for
the Cause of God is too exalted and majestic to be known except through
Him.

Indeed, it is through Him that others are known, not the reverse. By the
essence of the one true God, whose existence is singular and eternal,
the signs of this revelation are brighter than the light of the sun at
midday. The effects of those who have been guided by Him, even if they
ascend to the highest levels of knowledge and understanding, resemble
the light of stars in the night compared to the brightness of His
guidance.

May God forgive such comparisons. How can the ocean of eternity be
grasped by the sea of contingency? How can the remembrance of the
primordial be likened to the remembrance of limitations?

\emph{Glorified and exalted is God above all that is mentioned by
earthly and heavenly allusions.}

All that has been described pertains to the limited realms of creation.
However, this is exactly how the argument unfolds in this Resurrection,
just as God previously questioned through His own tongue: \emph{``Whose
book is the Qur'án?''} All the believers responded: \emph{``It is the
Book of God.''}

God then asked: \emph{``Is there any difference between the Furqán and
the Bayán?''} The hearts replied: \emph{``No, by God! All is from our
Lord, and none will take heed except those endowed with insight.''}

God then revealed: \emph{``That was My word through the tongue of
Muhammad, the Messenger of God, peace be upon Him. This is My word
through the tongue of the Essence of the Seven Letters, the Gate of God,
peace be upon Him.''}

Whoever believes in it has no refuge if they wish to remain steadfast in
their faith except to believe in these verses. Otherwise, their essence
will be rendered void, and their deeds will become as though they were
nothing, unmentioned and forgotten.

God then revealed: *``O My creation! All of you, from the beginning of
your lives to their end, strive with utmost effort to earn My good
pleasure. If you act upon any secondary matter, it is because I have
revealed it in My Book. If you believe in the Imáms of guidance or seek
nearness to Me through visiting their graves, it is because their
celestial status has been alluded to in the Qur'án. If you acknowledge
the prophethood of Muhammad, the Messenger of God, it is because He is
My Messenger. If you circumambulate around the Kaʿbah, it is because I
have called it My house. If you revere the Qur'án, it is because it is
My word.''*

Indeed, upon every soul---even if it belongs to the nation of
Adam---whatever one does is based on their relationship to Me, as they
perceive it within themselves. However, they remain veiled, holding
false assumptions and failing to recognize the realities of subsequent
manifestations of My Cause. There is nothing whose judgment does not
return to this human form, which has been created by My command.

This form returns degree by degree until it reaches My Prophet. The
prophethood of My Prophet is not established except through the Book
revealed to Him and the proof granted to Him. Today, which is the Day of
My Manifestation, I have appeared in My very self. This utterance is
like the mention of the Kaʿbah, which I have called My House.

In reality, for My essence, there is neither a beginning nor an end,
neither manifestation nor concealment. Today, whatever returns to this
self, which recites My verses, returns to Me. Whatever does not return
to Him does not return to Me. This is My manifestation in My very self
and My concealment in My very essence, for this possible reality cannot
exist in possibility itself. No station higher than this can be
conceived in the Bayán.

How greatly veiled you are, O people, imagining that all your actions in
your respective positions are for My good pleasure. Yet the verse that
signifies Me and the signs of My power, whose treasures reside in His
innate being, are recited by My permission. Despite this, you have
confined Him unjustly to a mountain that none of its inhabitants is
worthy of mention. Before Him, who is with Me, there exists none but a
single soul, one of the Living Letters of My Book. Before Him, who
stands in My presence, My hand bears witness, and no single lamp of
light suffices as proof. However, in the positions reached through the
progression of degrees, numerous lamps shine brightly. All upon the
earth, created for Him, are sustained by His radiance, yet remain veiled
from Him to the extent of a single lamp.

On this day, I bear witness to My creation, and apart from My testimony,
nothing holds any value before Me. No paradise is greater for My
creation than being in the presence of My very self and believing in My
verses. Similarly, no fire is more severe than the veiling of these
souls from the Manifestation of My self and their rejection of My
verses.

You claim to act on My behalf, yet how does this benefit you? Do you not
see My verses confirming what you previously claimed in My Book? Even
now, you feel no shame, despite witnessing that My Book is firmly
established and all who believe in it believe in Me. Soon, you will see
that your pride will be in your faith in these verses. Yet today, what
benefits your souls is proclaiming belief in what does not benefit you
and veils you from My cause, bringing harm to yourselves.

No harm has reached or will reach the Manifestation of My self. Whatever
harm has occurred or will occur returns to your own souls. Show
compassion for yourselves, and do not ascend in the air of imagined
satisfaction of My will while veiling yourselves from the truth of My
good pleasure.

My good pleasure is manifest through the proof by which the religion of
all is established. Yet, you remain veiled from it, even as you
associate yourselves with the Qur'án. By My sanctified essence, no
paradise for this creation is loftier than the manifestation.

There is no paradise greater than union with Me and My verses, and no
fire more severe than being veiled from Me and My verses. If you claim
that your inability is not evident to you, then journey throughout the
east and west of the earth. Yet, this statement is meaningless, for
today the truth of all upon the earth is connected to the Dispensation
of Islam.

If the eloquent ones of this Dispensation are unable to produce the like
of these verses, it is evidence that all are powerless. If they claim
they are not powerless, why do they not produce a verse similar to My
verses, born of innate reality rather than acquired knowledge or theft?

In every age, before the truth, even those like the sorcerers of Moses'
time demonstrated what was within their capacity. Praise be to God that,
from the time of My manifestation to today, even such demonstrations
have not appeared from the scholars of this Dispensation. They falsely
claim to ascend toward the loftiness of God's satisfaction while
remaining veiled from the true Manifestation of His power.

This alone suffices as the abasement of the scholars of Islam. They
claim knowledge of Islam and promote its teachings while veiling
themselves from Him whose word establishes Islam. If they were merely
content with their own veiling and refrained from wronging others or
judging contrary to what was revealed in the Qur'án, their fate would
still be the fire of their own deeds.

However, they have worsened their condition and that of those who regard
them as scholars of Islam. Indeed, whoever is veiled from the
manifestation of God faces their own punishment. If they were to reflect
upon the verses of God, they would observe their own incapacity. In that
moment, neither rulership nor Islam and those under its shadow would not
have been content with veiling themselves from the truth. The pride of
all lies in following the truth. If their actions had not been ambiguous
to themselves, matters would not have reached this point. There is no
doubt that God will manifest His truth to all through His proof, as He
does even today.

If anyone who associates themselves with Islam---whether from among
those in positions of authority or those considered scholars---wishes to
affirm the proof of these verses, it would be achieved in less than the
blink of an eye. If they possess power, they could summon all the
scholars and say to them, ``Through your judgments, you have caused
veiling from the one who possesses the verses.''

The matter can only take one of two forms: either they themselves have
brought forth a book and its verses are present, or they have not. If
they have not, this verse, written here, is sufficient:

\emph{``Glorified are You, O God! You are the Sovereign of sovereigns.
You bestow sovereignty upon whomever You will and remove it from
whomever You will. You exalt whomever You will and abase whomever You
will. You grant victory to whomever You will and forsake whomever You
will. You enrich whomever You will and impoverish whomever You will. You
manifest whomever You will over whomever You will. In Your grasp is the
dominion of all things. You create what You will by Your command.
Verily, You are all-knowing, omnipotent, and powerful.''}

Speak in the manner He has spoken, naturally and innately. Write in the
manner He has written, without pause or hesitation. If you cannot, it is
proof that your deeds have been carried out without truth, and the
bearer of these verses is the Truth, sent from God, and there is no
doubt that God has revealed these verses upon him just as He revealed
upon the Messenger of God. Today, the amount of verses resembling these
totals over a hundred thousand verses, in addition to his scrolls of
supplications and his scientific and wisdom-filled compositions.

In just five hours, a thousand verses emanate from him, or as quickly as
a scribe is able to transcribe them. He recites the verses of God. This
makes it possible to estimate that, had his works been freely
disseminated from the beginning of the manifestation until today, an
unimaginable quantity of works would have been spread among the people.

If you claim that these verses are not sufficient proofs in themselves,
consider the Qur'án. Did God, in proving the prophethood of the
Messenger of God---may God's blessings and peace be upon Him and His
family---use anything other than His verses as proof? Reflect upon this.

God has revealed:

\emph{``None disputes concerning the signs of God except those who
disbelieve, so do not let their movement through the land deceive you.
Those before them, like the people of Noah, rejected the truth, and
every nation sought to seize their messenger. They disputed with
falsehood to invalidate the truth, but I seized them, and how severe was
My punishment! Thus, the decree of your Lord is fulfilled upon those who
disbelieve, that they are the companions of the Fire.''}

In affirming the sufficiency of the Book, God has revealed:

\emph{``Is it not enough for them that We have sent down to you the
Book, recited to them? Indeed, in that is a mercy and a reminder for a
people who believe.''}

When God Himself testifies to the sufficiency of the Book through its
own verses, how can anyone claim otherwise? One cannot deny the
sufficiency of the Book as proof. If someone repeats what the first
objectors said, their stance falls into one of two categories: either
their intent is to reject the truth outright---which brings them no
benefit, as God has revealed:

\emph{``Even if they see every sign, they will not believe in it.''}

And elsewhere:

\emph{``Indeed, those upon whom the decree of your Lord has been
fulfilled will not believe, even if every sign comes to them, until they
see the painful punishment.''}

Alternatively, their intent may be to act cautiously in matters of
religion. If this is the case, the understanding is straightforward:

\emph{``In what discourse, after God and His signs, will they
believe?''}

They can either present themselves and inquire about their concerns in
the manner of the verses, listening directly to what is recited---verses
free from contemplation, formulation, or synthesis---or they can send
someone to sit briefly in His presence, record what is recited from the
verses of God, and reflect on them. They would then realize that these
words are not the product of thought, combination, or contrivance.

If such a process were possible, it would have occurred with the Qur'án
from the beginning of Islam until today. Similarly, since the onset of
this Cause until now, no one has been able to produce anything
comparable through such means. If objections are raised about the Arabic
grammar or pronunciation within the recitation, such criticisms are
invalid. This is because grammatical rules are derived from the verses
themselves, not the other way around. There is no doubt that the bearer
of the verses disclaims these rules and the associated knowledge for
himself.

In truth, the lack of adherence to these rules, while presenting verses
of this nature and such words, holds no bearing for those of
understanding (\emph{ulú'l-albáb}). There is no greater accomplishment
than understanding the Book of God, for the purpose of knowledge is to
comprehend His Book. The Tree upon which the Book of God is revealed
does not require knowledge of these sciences. Rather, all grammatical
rules and expressions are derived from what God has revealed and are
established upon it.

Many individuals possess extensive knowledge of these sciences, yet
their faith in the verses of God is firmly established. This is because
the purpose of knowledge is to understand the commands of God and follow
His good pleasure, not to remain distant from them. If these sciences
themselves bore fruit, the learned ones of Arabia would have surpassed
the non-Arabs in their spiritual station. However, true honor lies not
in these sciences but in the good pleasure of God, knowledge of His
oneness, and abiding in the shadow of obedience to Him and satisfaction
with Him.

There is no doubt that most actions performed between individuals and
God are intended to seek His good pleasure. Yet, few truly comprehend
His good pleasure unless they discern the pleasure of His proof. Today,
the good pleasure of God is inseparable from the good pleasure of His
proof and those who dwell under His shadow. Even though others may
consider themselves rightly guided, only what God testifies to remains
enduring. The deeds of those who do not follow the command of God will
perish.

Just as the rejection of the Qur'án by certain figures---such as
Christian monks and the eloquent of the Arabs---was remembered in the
early days of Islam, so too will the veiling of people today be
remembered. Today, there is no action more beneficial for a servant than
to examine the verses of the Bayán with fairness, observing the truth of
the truth with absolute certainty. Through this, they may recognize the
manifestation of meeting with Him is the meeting with God, and His good
pleasure is the good pleasure of God. None should remain veiled, for all
were created for this purpose, as God has revealed:

\emph{``It is God who raised the heavens without pillars you see, then
established Himself upon the Throne. He subjected the sun and the moon,
each running for a specified term. He manages the affair, explains the
signs, so that you may attain certainty in the meeting with your
Lord.''}

There is no doubt that any mirror turning toward the sun becomes
illuminated by it. Without this, the sun itself rises and sets without
affecting the mirror. The honor of all creation lies in attaining the
fruit of their existence: the meeting with God and belief in His verses.
Without this, the very existence of a thing becomes void. This is the
same tree that planted the Qur'án within the hearts of the people for
the sake of today. Today, all take pride and exalt themselves in
relation to it and act accordingly.

This is the meaning of \emph{``There is no power and no strength except
in God''} in the realm of divine legislation. If they were to remove the
unfounded claims of association with Him, they would not possess even
the strength of a fly. This abasement alone is sufficient for those who
are veiled, as they claim association with Him yet act as they do.
Instead of supporting Him as the fruit of their existence and His
victory today, they neither support Him nor even accept being neutral.
Were they content with neutrality, this Tree would not have been
confined to this mountain.

God suffices as the ultimate judge over all His servants. He will judge
with justice, and His judgment is manifest in these very words. This
judgment will distinguish, until the Day of Resurrection, between those
who turn toward Him and those who do not. All matters of divine decree
He is the best of helpers, the best of protectors, the best of
reckoners, and the best of judges.

\subsection*{Gate 2 (Understanding and Interpreting the
Bayan)}\label{gate-2-understanding-and-interpreting-the-bayan}
\addcontentsline{toc}{subsection}{Gate 2 (Understanding and Interpreting
the Bayan)}

1 None can encompass the knowledge of what God has revealed in the Bayán
except those whom God wills.

2 The essence of this gate is that no one can fully comprehend what God
has revealed in the Bayán except \emph{Him Whom God shall make manifest}
or those to whom He grants knowledge.

3 This resembles the tree from which the Bayán has sprung. Even if all
the oceans of the heavens and the earth were turned into ink, all things
became pens, and all souls were made to record, they could not interpret
a single letter of the Bayán as it truly is.

4 God has not given any letter of the Bayán a beginning or an end.

5 None is permitted to interpret what God has revealed in the Bayán
except the Letters of \emph{ʿIlliyyīn} under the shadow of \emph{Him
Whom God shall make manifest} and the Letters of the Living. All other
letters beneath \emph{ʿIlliyyīn} belong to the gates of the fire.

6 The Letters of \emph{ʿIlliyyīn} will be gathered under His shadow,
while all other letters will be gathered under the shadow of negation.
The past is as the future, and there is no change in the decree of God,
just as the ``before'' of the past is as the ``after'' of the future.

7 The Qur'án comprises 114 surahs, with every six surahs corresponding
to the exalted station of each letter of \emph{Bismilláh al-Raḥmán
al-Raḥím} (``In the name of God, the Most Merciful, the Most
Compassionate''). The first six surahs are associated with the
\emph{Bá'}, and the last six surahs are associated with the \emph{Mím}.

8 All the Letters of \emph{ʿIlliyyīn} are connected to the
manifestations of the 19, while the gates of the fire are connected to
the First Gate. Similarly, all the gates of paradise return to the First
Gate. Ultimately, everything is encompassed in the phrase:

\emph{``There is no God but God.''}

9 All that lies beneath the Letters of \emph{ʿIlliyyīn} is gathered
under negation, while all that pertains to the Letters of
\emph{ʿIlliyyīn} is gathered under affirmation. All the Letters beneath
\emph{ʿIlliyyīn} in the Qur'án are subsumed under the shadow of the
First Gate of the fire and pass away.

10 In contrast, all the Letters of \emph{ʿIlliyyīn} in the Qur'án remain
under the shadow of the Word of Affirmation. \emph{``Thus does God
enrich whomsoever He wills and preserve whomsoever He desires. Verily,
He is powerful, mighty, and capable.''}

11 It is impermissible to interpret the Bayán except through what has
been interpreted by its Tree. All the names of His goodness are manifest
in the radiance of the hearts of His believers. Likewise, all the
letters beneath His \emph{ʿIlliyyīn} are reflected in the realities of
those who are not believers in Him.

Eternally, the Bayán is like the human soul, alive, with all its Letters
of Light and Fire apparent in the horizons and within the souls. Today,
anyone wishing to discern can do so, as this is the Day of the
Manifestation of God, the Mighty. However, after the Tree is lifted, no
one will have the capacity to discern the true reality except according
to the apparent judgments of those who remain within the bounds of God.

Whoever adheres to these bounds belongs to the Letters of
\emph{ʿIlliyyīn}, while whoever transgresses them is beneath
them---unless God, through His manifestation on the Day of Resurrection,
resolves this creation.

At that time, the judgments of the Point of the Bayán at its final stage
will reflect the judgments of its initial reality. Whoever believes in
it belongs to the Letters of \emph{ʿIlliyyīn}, while whoever does not
believe in it is beneath them. God will distinguish between them with
justice, for He is the best of judges.

The matter will reach the point where no mention of the Letters beneath
\emph{ʿIlliyyīn} remains except in the Book. That same Tree of Negation
will recognize its own negation in comparison to the Letters of
\emph{ʿIlliyyīn} and, unknowingly, curse itself until the sun of truth
rises again.

At that time, their lack of faith will become evident, just as all who
have witnessed this Resurrection have observed this matter with
certainty. Although they claim today to believe in God and the verses of
the Qur'án, the Tree of Reality, upon which the Qur'án descended,
resides in this mountain with a single soul.

In the appearance of the Sun of Truth, realities are unveiled, and veils
are removed. Those souls who never conceived anything less than the good
pleasure of God have given judgments contrary to divine
satisfaction---judgments that the pen hesitates to record---while the
true pleasure of God is only revealed through the good pleasure of His
Manifestation.

\emph{``Take heed, O people of insight, and guard yourselves against
disobeying the command of God.''}

\subsection*{Gate 3 (Two Proofs: The Verses and the
Manifestation)}\label{gate-3-two-proofs-the-verses-and-the-manifestation}
\addcontentsline{toc}{subsection}{Gate 3 (Two Proofs: The Verses and the
Manifestation)}

\textbf{The Third Gate of the One, the Second}: This pertains to the
truth that within the Bayán lies the decree for all things.

The essence of this gate is that God has established two proofs upon all
people: the verses of God and the soul upon whom these verses are
revealed. The first proof is an enduring and apparent one, lasting until
the Day of Resurrection. The second proof is manifest until the time of
His concealment, at which point it becomes a hidden proof over all
things, unknown to anyone.

For Him, there are witnesses who bear testimony from the moment of His
concealment. These witnesses are superior to the enduring proof, which
is the Bayán. They act as proof through His word, which He reveals until
the Day of His Manifestation.

If, at the time of His manifestation, they become veiled from the one
who has made them proofs, their condition will resemble the current
scholars who, despite the words of one of the Imáms, peace be upon
them---\emph{``Look to those who transmit our traditions.''} They are
regarded as leaders before the ruler and associate themselves with names
unworthy of their station. If they were truthful in their words, they
would not have veiled themselves from the one whose word establishes
divine authority and prophethood. Rather, when they observed that the
manifestation of truth conflicted with their position---based on the
claims they had made and the judgments they had rendered---they issued
decrees against God. Yet, in the Qur'an, God has not revealed anything
more grievous than the one who turns away from the signs of God after
hearing them.

There is no doubt that the signs of God are the radiant verses emanating
from this tree. This is because the Essence of the Everlasting, from
eternity to eternity, remains constant and unchanging. His signs
manifest through His primal will, in which nothing is seen except God
alone. Though today they do not pay attention to the signs of God, soon
they will extol these signs with the utmost reverence, offering
interpretations as though they were pure gold of a thousand measures,
and they will boast of them, claiming a connection to God.

For instance, the Qur'an, revealed over twenty-three years, found no one
during that time to record its original form. It was only the Commander
of the Faithful, upon him be peace, who inscribed it upon the shoulder
blade of a sheep and other available materials, as mentioned in the
narration of the Cloak. Today, we observe innumerable copies of the
Qur'an being written, ranging in value from one thousand to a hundred
thousand gold coins. Printed editions are now sold at such high prices,
accessible to most people. This illustrates the extent of human limits
before God, leaving no doubt that God has revealed the details of all
things with the utmost clarity through the enduring proof. Whoever
claims that there is anything whose ruling, as it is and as it stands,
is not mentioned in the Bayán has certainly not believed with unwavering
certainty. Everything falls into one of two categories: it is either
mentioned in the category of negation or in the category of affirmation.
That which is detested pertains to the former, and that which is beloved
by God pertains to the latter. Every name lower than the truth is
mentioned in the first category, and every name of the truth is
mentioned in the second.

This is the ink of the understanding of all things in the Bayán. Whoever
bears witness to this will testify that nothing has been neglected, and
God encompasses all things. There is no matter but that for the silent
Book, God has ordained a speaking Book. Neither exists without the
other, and whoever does not stray beyond the bounds of the silent Book,
it is as though they adhere to the speaking Book. The speaking Book is
He whom God will make manifest, and all things return to Him. If no one
transgresses the bounds of the Bayán, such a person is a servant who has
obeyed, and whoever is with Him is a witness over it before His
appearance. However, when He appears, faith in all who claim faith will
be severed, except for those who believe in Him. When faith is severed,
how can testimony remain for those who were witnesses? Testimony is a
branch of faith.

Therefore, fear God, O witnesses, and do not judge God, your Lord, with
the same judgment passed by those who were witnesses in relation to the
Qur'an concerning Me. Indeed, whoever judges Me has judged God, their
Lord. These have not even a mustard seed of goodness in the sight of
God, and such are the transgressors.

\subsection*{Gate 4 (Letters of the Most High and Those Below
Them)}\label{gate-4-letters-of-the-most-high-and-those-below-them}
\addcontentsline{toc}{subsection}{Gate 4 (Letters of the Most High and
Those Below Them)}

From the second unity in the Bayán, concerning the mention of the
letters of the Most High and those below them, the summary of this
chapter is as follows: No letter has God revealed without assigning to
it a spirit connected to it. This is why the believer is gladdened by
the mention of paradise and the pleasure of God, and saddened by the
mention of fire and that which is beneath God's satisfaction, such that
it is as though the former brings delight and the latter causes torment.

All words revealed by God in the Bayán fall into one of two categories:
either they are the words of the Most High or those below them. The
spirits of the words of the Most High reside in paradise, while the
spirits of those below them dwell in the fire. All letters below the
Most High refer to ``no god,'' and all the letters of the Most High
refer to ``but God.'' Just as the origin of all letters below the Most
High begins with this phrase, the origin of all the letters of the Most
High begins with that phrase which has eternally been in the exalted
paradise. The former fades into insignificance in the lowest depths,
while the latter rises in the highest paradise.

For example, if one observes the beginning of the tree of the Qur'an
today, they will surely witness how the five letters of
negation---first, second, third, fourth, and fifth---have faded into
insignificance in the lowest depths, while the five letters denoting
affirmation---Muhammad, `Ali, Fatimah, Hasan, and Husayn---have risen in
the highest paradise. The five letters of fire, when their letters are
multiplied, become nineteen, as God has revealed: ``Over it are
nineteen.'' Thus, these five letters are names, when multiplied, result
in the number one, just as all letters below the Most High pertain to
this word, while all the letters of the Most High pertain to the word of
affirmation. God created the realm of negation and decreed for it the
fire, and He created the realm of affirmation and decreed for it
paradise. This is because negation does not signify Him, while
affirmation signifies Him. This pertains to the creation of negation and
affirmation; their recording corresponds to their creation.

On the Day of Resurrection, all that branched out from the word of
negation returns to the word of negation, and all the letters below the
Most High, along with their associated spirits, are gathered under this
word. Similarly, all that branched out from the word of affirmation on
the Day of Resurrection returns to the word of affirmation, and all the
letters of the Most High, along with their associated spirits, are
gathered under it. Whoever dwells in negation resides in God's fire
until the Day of Him Whom God will make manifest, and whoever resides
under the shadow of affirmation abides in God's paradise until the Day
of Him Whom God will make manifest.

The exaltation of the servant lies in reflecting upon the branching of
these two words, observing how one diminishes endlessly while the other
ascends endlessly. God's proof applies equally to both, for the signs of
God are presented to both. The manifestations of negation, having
rejected them, were negated, while the manifestations of affirmation
accepted.

No fire is more intense than the manifestation of the word of negation,
and no paradise is greater than the manifestation of affirmation. In the
first case, the letters below the Most High revolve around it, and in
the latter, the letters of the Most High revolve around it, until all
return on the Day of Him Whom God will make manifest. If they turn
towards Him, they belong to the Most High, and if not, they belong to
that which is beneath it. Blessed is the one who clings to the cord of
God and places their trust in their Lord, for they shall not enter the
fire but shall surely enter paradise by the permission of their Lord.
Truly, this is the supreme bounty.

For this reason, when the servant recites the letters of the Most High,
they find tranquility, as the spirits of those letters connect with
them. This is the highest paradise of those who seek refuge, the
sanctified, those who glorify, magnify, and exalt. Whenever the servant
mentions the letters beneath the Most High, they desire that God's
justice descend upon them, for their spirits connect with those letters.
In such moments, one must seek refuge in God, exalted be His mention, to
be safeguarded from their spirits.

On the Day of Resurrection, all whom God promised paradise in the Qur'an
return to the tree of His love. This is the loftiest station of the
paradise of those who attain God's pleasure and are delighted by their
acknowledgment of His oneness. Conversely, those whom God promised the
fire are returned to the word of negation, where they are tormented
within that negation. No torment is greater than being veiled from God
and believing in anything but Him, and no paradise is greater than
belief in God and His signs.

Whenever a discerning observer reflects, they will see how the people of
paradise hasten to enter it, even though their sustenance was but the
leaves of trees. Meanwhile, the people of the fire enter the fire
willingly, deriving benefits from the word of negation, which they boast
about and through which they are tormented, though they are unaware. As
God has revealed: ``They consume fire in their bellies.'' In this way,
the letters below the Most High return to their own spirits, while the
letters of the Most High return to their own spirits.

No soul exists but that when they mention the letters of the Most High,
the spirits of the angels associated with those letters observe them and
send blessings upon them from God. Conversely, when mentioning the
letters below the Most High, if done out of love for them, the spirits
of the demons of fire observe them. If they do not seek refuge in God,
those influences will affect them, even if only as passing thoughts.
However, if they seek God's refuge and request His wrath upon those
influences, no harm can reach them, not even equivalent to a mustard
seed in religious detriment.

It is as though I see the letters of negation seeking refuge from
negation at the appearance of Him Whom God will make manifest, though
they themselves are the essence of that negation. At that time, they
will find no refuge from their own fire except in Him Whom God will make
manifest. When a servant says, ``I seek refuge in God,'' if they are not
within the religion of the Bayán, they will not be shielded from the
fire. Indeed, they do not utter this phrase except by entering the
religion, just as those who have entered the faith.

Seeking refuge in God is, in essence, seeking refuge in His proof.
Whoever believed in Muhammad was safeguarded from God's fire because
they sought refuge in Him. Even the letters of fire utter the phrase ``I
seek refuge in God,'' but it does not benefit them, for they do not seek
refuge in His proof. This is because God revealed in the Qur'an: ``And
whoever believes in God,'' pairing it with the phrase that follows. Yet,
they recite the phrase itself without understanding its implication.

In the early days of Islam, the interpretation of this phrase referred
to the second figure, even though the entirety of the Qur'an was
recited. At the time of the manifestation of the sign of divinity, it
was the Commander of the Faithful, upon him be peace. Had they sought
refuge in Him, they would have been saved from the succeeding word.
Thus, all, until the Day of Him Whom God will make manifest, seek refuge
in God and the Point of the Bayán. However, this brings no benefit to
them today, for seeking refuge in God on that day means seeking refuge
in Him. Seeking refuge in the Point of the Bayán is seeking refuge in
Him.

From the beginning of the manifestation of this tree, all have been
saying ``I seek refuge in God,'' yet they reside in the fire, except
those whom God has willed, who have recognized the manifestation of this
Name. Those who sought refuge in Him were safeguarded from the pure
fire. Otherwise, countless souls utter this phrase daily, yet it offers
no salvation for them. This is because God has paired seeking refuge in
Him with seeking refuge in His Messenger, and seeking refuge in His
Messenger with seeking refuge in His successors, and seeking refuge in
His successors with seeking refuge in the gates of His successors. The
first benefits none except through the last, nor does the outward avail
except through the inward, for seeking refuge in the Messenger is
identical to seeking refuge in God, and seeking refuge in the Imams is
identical to seeking refuge in the Messenger.

And seeking refuge in the gates is identical to seeking refuge in the
guardians. Whoever enters the Bayán today is safeguarded from the fire,
just as the letters of the Alif were not safeguarded from the fire until
they entered the letters of the Qur'an. The letters of the Bayán, until
the Day of Him Whom God will make manifest, reside in the Most High in
paradise, while those below the Most High remain in their respective
stations. On that day, whoever enters His Book will be saved from the
fire; otherwise, remaining in the Bayán will yield no benefit, just as
remaining in the Alif after the revelation of the Qur'an brought no
benefit, nor did remaining in the Qur'an after the revelation of the
Bayán.

Until God wills otherwise, the Most High ascends endlessly in its
loftiest heights, while those below are cast into the utmost depths of
nonexistence. Blessed is the one who nourishes their heart with the
letters of the Most High and, when mentioning what is below them, seeks
refuge in God, their Lord, for surely He will protect them. It is
inevitable that these will be mentioned, yet their mention will not harm
those who recite them, just as the believers in the Qur'an mentioned
those who had received the Book before them. Thus, God distinguishes the
signs so that you may attain certainty in His signs.

\subsection*{Gate 5 (Every Name Pertains to Him Whom God Will Make
Manifest)}\label{gate-5-every-name-pertains-to-him-whom-god-will-make-manifest}
\addcontentsline{toc}{subsection}{Gate 5 (Every Name Pertains to Him
Whom God Will Make Manifest)}

The fifth chapter of the second unity states that every good name
revealed by God in the Bayán primarily refers to Him Whom God will make
manifest in its truest essence. Similarly, every evil name revealed by
God in the Bayán refers, in its truest essence, to those who represent
the word of negation in that day.

The summary of this chapter is that every name of any matter revealed in
the Bayán pertains to Him Whom God will make manifest in its primary
truth and, secondarily, to the first who believes in Him, extending to
the final limit of existence. This is similar to the mention of the
earth. When the term ``earth'' is mentioned, it primarily refers to His
own essence, gradually descending until it reaches the physical earth
associated with Him, which is His dwelling place, regarded as the
loftiest chamber of paradise in the Book of God. Similarly, every lesser
name of goodness revealed pertains, in its primary essence, to the tree
that stands in opposition to negation. If the term ``earth'' is
mentioned in relation to those below the Most High, it refers first to
His own essence, gradually descending until it reaches the physical
earth that is His dwelling place, which constitutes the ultimate end of
the fire in the realm of fire---even if He is above the throne of
majesty.

Likewise, every mention of goodness in the Qur'an primarily refers to
the Messenger of God in its truest essence. Every mention of lesser
goodness pertains to the first one who stood in opposition to the
initial affirmation. If the term ``earth of the Most High'' is
mentioned, it refers to His own essence, descending until it reaches the
physical earth, which is the resting place of His body. All matters
return to the Qa'im of the Family of Muhammad, upon Him be peace. All
mentions of goodness in the Qur'an primarily pertain to Him, just as in
the Bayán, they have been explained as referring to Him Whom God will
make manifest.

Every mention of lesser goodness in the Qur'an, even if it involves the
mention of ``earth,'' refers to the earth of the essence of the first
who did not turn towards Him. What is true in the Qur'an is likewise
firmly established in the Bayán before God. Every name of goodness in
God's knowledge pertains, in its primary essence, to the Point of Will,
while those below it refer to the one who did not accept it. Just as
every mention of ``earth'' in existence today refers to the Point of the
Bayán.

The spirit descends from the earth of the spirit to the earth of the
self, from the earth of the self to the earth of the body, and from the
earth of the body to all things, approaching closer and closer until it
reaches that earth above the mountain, a third of which is marked with
``Sh'' in four parts. This is the essence of all earthly terrains at
that time, and the position changes according to the change in the
matter until it stabilizes in that which does not change.

Thus, under the shadow of the Most High, letter corresponds to letter,
and point corresponds to point. This is the highest earth of Ridván, and
that is the lowest earth of the fire. I seek refuge in God from that
which He does not love, and I ask God concerning all that He
encompasses, for He is bountiful and generous.

How often have the names of goodness in the Qur'an been manifest, rank
by rank, from the Messenger of God to each of His successors! Similarly,
this pattern applies to those below the Most High, until it reaches the
loftiest heights of the earth of paradise at the dwelling of the
martyrdom of the Master of Martyrs, and the lowest depths of fire at the
dwelling of the sovereignty of the self that opposes Him. This is the
decree with God, and it applies similarly in the appearance of every one
of the proofs of God.

Today, all names of goodness in their primary reality are encompassed in
the essence of the Point, even the mention of the earth as exemplified.
In their secondary reality, they are manifest in the letter ``S,''
continuing until the ultimate limit of existence. So fear God, O people,
all of you.

\subsection*{Gate 6 (The Bayan Is the Balance Until the Day of
Resurrection)}\label{gate-6-the-bayan-is-the-balance-until-the-day-of-resurrection}
\addcontentsline{toc}{subsection}{Gate 6 (The Bayan Is the Balance Until
the Day of Resurrection)}

The sixth chapter of the second unity discusses how the Bayán is the
balance of truth from God until the Day of Him Whom God will make
manifest. Whoever adheres to it is in light, and whoever turns away from
it is in fire.

The summary of this chapter is that the Bayán is the balance of truth
until the Day of Resurrection, which is the Day of Him Whom God will
make manifest. Whoever acts in accordance with what is revealed with
what is revealed in the Bayán is in paradise and under the shadow of
steadfastness, and the letters of the Most High will be gathered before
God. Whoever deviates, even by the measure of a mustard seed, will be in
the fire and gathered under the shadow of negation. This truth is also
evident in the Qur'an, where God has revealed on multiple occasions that
whoever judges by other than what God has revealed is a disbeliever.

The implications of this word and its consequences are clear: whoever
transgresses the decree of God is subject to His judgment. How much
greater is the transgression when directed at the very Manifestation of
God after He has revealed, ``I have not created jinn and humankind
except that they may worship Me.'' There is no doubt that obedience is
not accepted unless it is obedience to the proof of God. Indeed, actions
performed in opposition to the People of the House are mentioned before
God, yet today, many claim that they had no worship and that their deeds
bear no fruit. Similarly, those outside the Shi'a sect, though they act
according to the Qur'an, their deeds are nullified before God due to
their deviation from divine authority.

Today, few act in accordance with the balance of the Qur'an; in fact,
such individuals are rarely seen, except by the will of God. Even if
someone does adhere to it but does not enter the balance of the Bayán,
their piety will bear no fruit, just as the piety of the monks of the
Alif bore no fruit, despite their adherence to the balance during the
time of the Manifestation of the Messenger of God. Had they acted
according to the balance of the Qur'an, such judgments regarding the
tree of truth would not have been made.

The heavens nearly split apart, the earth almost tears asunder, and the
mountains nearly crumble from this. The hearts of these individuals are
harder than these mountains and remain unaffected. There is no paradise
greater or higher in the sight of God than being in His good pleasure.
Praise be to Him, for today this bounty is exclusive to the people of
the Bayán. Hereafter, whoever does not transgress its bounds will remain
in this bounty until the Day of Him Whom God will make manifest. But
should someone, God forbid, deviate, they have wronged none but
themselves, for God is self-sufficient above all the worlds.

At the beginning of His manifestation, all of the Bayán is obedience to
Him, and nothing else. Similarly, all religion during the Day of Alif,
at the appearance of the Messenger of God, consisted of following Him,
not merely adhering to the balance of their own understanding. For at
that time, judgment against those who clung only to the balance of their
past was issued. ``Whoever is guided, it is for their own soul, and
whoever veils themselves, it is against it. God is self-sufficient above
all the worlds.''

\subsection*{Gate 7 (The Appearance of the Tree of
Truth)}\label{gate-7-the-appearance-of-the-tree-of-truth}
\addcontentsline{toc}{subsection}{Gate 7 (The Appearance of the Tree of
Truth)}

The seventh chapter of the second unity discusses the meaning of the Day
of Resurrection. The summary of this chapter is that the Day of
Resurrection refers to the Day of the appearance of the Tree of Truth.
It is evident that none among the Shi'a has understood the Day of
Resurrection; rather, all have imagined a concept that has no reality
before God.

According to God and the understanding of the people of truth, the
meaning of the Day of Resurrection is the period from the appearance of
the Tree of Truth in any era until its setting. For example, the time
from the Day of the mission of Jesus until His ascension was the
Resurrection of Moses, during which the manifestation of God appeared in
that era in the form of that Truth. During that time, everyone who
believed in Moses was judged according to their own word, and whoever
disbelieved was recompensed according to their own word, for what God
testified in that era was what God testified in the Gospel.

From the time of the mission of the Messenger of God, may blessings and
peace be upon Him and His family, until His ascension, it was the
Resurrection of Jesus. The Tree of Truth appeared in the form of
Muhammad, and He rewarded those who believed in Jesus and punished,
according to His word, those who did not believe in Him. From the
appearance of the Tree of the Bayán until its setting, it is the
Resurrection of the Messenger of God, as God promised in the Qur'an.

The beginning of that day is calculated to be two hours and eleven
minutes after the night of the fifth of Jumada al-Awwal in the year
1260, (May 23, 1844) which corresponds to the year 1270 of the mission.
This marks the beginning of the Day of Resurrection for the Qur'an,
which continues until the setting of the Tree of Truth. For anything to
reach its Resurrection, it must first attain its station of perfection.
The perfection of the religion of Islam concluded at the beginning of
this appearance, and from that moment until its setting, the fruits of
the Tree of Islam will be revealed.

The Resurrection of the Bayán will occur with the appearance of Him Whom
God will make manifest, for today the Bayán is in its embryonic stage.
At the beginning of His appearance, the ultimate perfection of the Bayán
will be revealed, manifesting the fruits of the trees planted in it.
Just as the appearance of the Qa'im of the Family of Muhammad mirrors
the appearance of the Messenger of God, it does not occur except to
gather the fruits of Islam from the Qur'anic verses planted in the
hearts of the people.

The gathering of the fruits of Islam is nothing other than belief in Him
and acknowledgment of Him. Yet today, the opposite has occurred, for
during the zenith of Islam, He has appeared, and all proclaim Islam
relative to Him. However, they exile Him unjustly to the mountain of
Maku, even though in the Qur'an, God has promised all of creation the
Day of Resurrection, for it is the day when all will be presented before
God, which is to be presented before the Tree of Truth. All will attain
the meeting with God, which is the meeting with Him, for presentation to
the Most Sacred Essence is impossible, and direct encounter with Him is
unimaginable. Whatever pertains to the presentation or meeting is
related to the Primary Tree.

God has made clay His dwelling place, such that whoever on the Day of
Resurrection is presented before the Tree of Truth and affirms this
presentation will not be distant from encountering Him. Even a moment
from the Day of Resurrection surpasses in value all the years that
precede it, for the fruits of those years are manifested on the Day of
Resurrection. Just as the fruits of 1,270 years of Islam became evident
from the beginning of this manifestation until the end of this
manifestation---which is the initial setting of the Sun of Truth---they
will be fully revealed.

From the beginning of this manifestation to the appearance of Him Whom
God will make manifest, the fruits of this cycle will pertain to another
Resurrection, which will occur with His appearance. O people of the
Bayán, have mercy on yourselves and do not nullify the length of your
nights on the Day of Resurrection, as those veiled by the Qur'an have
done. They prided themselves on Islam for 1,270 years, yet on the day
when the fruits are gathered, which is the Day of Resurrection, judgment
was rendered against them as being outside Islam, and they were
invalidated by this very judgment until the next Resurrection.

How many souls from the beginning of their lives endured hardships and
struggled for the pleasure of God, only to be dreaming while asleep and
took pride in their dreams. Now, the Manifestation of God has appeared,
clearer than any previous manifestation, with evident signs by which the
religion of Islam is established. Presentation before God cannot occur
unless the fruits of one's faith are manifested in their rightful place,
nor can creation fulfill its purpose if it does not rise to its intended
station. Yet, they issue judgments against those who turn to God night
and day, declaring, ``You alone do we worship.'' Even if such
individuals are content with this, it does not satisfy them, as they
seek to bring sorrow to the friends of the Truth.

O people of the Bayán, do not act as the people of the Qur'an did,
invalidating the fruits of your nights. If you, who believe in the
Bayán, declare at the appearance of His signs, ``God is our Lord, and we
associate nothing with Him. This is what God has promised us of the
Manifestation of His Self, and we call upon nothing besides Him,'' and
if you obey Him in what He commands, you will have manifested the fruit
of the Bayán. Otherwise, you are unworthy of mention before God.

Have mercy on yourselves; if you do not assist the Manifestation of
Lordship, do not grieve Him. When He appears, creation will revert to a
state like the one in which I now appear. Yet, it has not even crossed
your hearts, apart from your faith, to hasten in responding to God and
affirming His signs. For it is He Whom God will make manifest, whose
words you must affirm. Do not veil yourselves from your Beloved in any
matter, for if a decree is issued from His word, it will remain until
the Day of Resurrection. In that Day, the people of paradise will enjoy
their bliss in paradise, and the people of the fire will be tormented in
the fire. Now that it is the Day of Resurrection, the place of judgment
is this mountain. All act as they think pleases Him, yet they accept
upon Him what they would not accept upon themselves.

If you were to make a covenant with your Lord to not accept for anyone
what you do not accept for yourselves, perhaps in the next Resurrection,
even if you do not attain the meeting with God, you would not have
caused sorrow to His signs. The benefit of all those who follow the
Bayán depends on your refraining from harming Him, though I know you
will not abstain. Just as I, in this Resurrection, refrained from
benefiting the followers of the Qur'an, you did not refrain from
bringing harm.

No fire is more intense for you in the sight of God than to turn to me
day and night in worship of God and then judge me in ways you would not
accept for yourselves. But God will judge between me and you with
justice, for He is the best of judges.

\subsection*{Gate 8 (Death Has Infinite
Meanings)}\label{gate-8-death-has-infinite-meanings}
\addcontentsline{toc}{subsection}{Gate 8 (Death Has Infinite Meanings)}

The eighth chapter of the second unity explains the reality of death and
affirms its truth. The summary of this chapter is that death has
infinite meanings in the sight of God, meanings that only He can
encompass. One of these meanings in common understanding is the physical
death perceived by all, which occurs at the moment the human soul is
taken. Whatever is considered ``death'' in the sight of God is truth,
and all who are accountable are obligated to acknowledge this.

It is true, not merely in the sense of the physical death understood by
the people, but rather as death before the manifestation of the Tree of
Truth, which pertains to what is beyond. This death is only realized in
five stages: through the words \textbf{``There is no God but He,''}
\textbf{``There is no God but I,''} \textbf{``There is no God but
God,''} \textbf{``There is no God but You,''} or \textbf{``There is no
God but the One in whom all are assured.''}

The essence of death is that, at the moment of the manifestation of the
Tree of Unity, represented by these five stages, all become ``dead'' by
negating negation and affirming affirmation. To fully expound on this
subtle mystery, which encompasses the seas of the heavens, the earth,
and all that lies between them, would exceed the capacity of any ink or
enumeration.

The core of the matter is this: whoever's will aligns entirely with the
will of Him Whom God will make manifest, whose intent is none other than
His intent, whose measure is His measure, whose decree is His decree,
whose permission is His permission, whose term is His term, and whose
Book is His Book---such a person truly comprehends death. For His will
is the essence of God's will, His intent the essence of God's intent,
His measure the essence of God's measure, His decree the essence of
God's decree, His permission the essence of God's permission, His term
the essence of God's term, and His Book the essence of God's Book.

Thus, in the Point of the Bayán, whoever ``died'' testified to the truth
of death. Otherwise, no benefit accrued to them from what they recited
in the Qur'an or supplications. How many have declared, ``Death is
true,'' yet their will deviated from His will, rendering them
invalidated and exposing their false claims before God! This continues
until the stage of the Book is reached, so that His Book, which was the
very essence of the Book of God, was revealed to those who considered
themselves the most learned of their time. Yet the pen is ashamed to
record what they did, even though they would proclaim day and night that
``death is true'' and would act according to the Book prior to His,
outwardly manifesting Islam and expending their knowledge. They claimed
for themselves the rights God had ordained in the Qur'an for Him, though
their own souls were not lawful for them, for they did not act out of
faith in God. This is the fruit of knowledge without action in the Book
of God.

Had they understood death, they would not have deviated from their own
acknowledgment, testifying that it is true, while veiling themselves
from the realities affirmed by that truth. This is the death that
benefits all on the Day of Resurrection, and beyond that, in the
intermediate world (barzakh), until God causes the Sun of Truth to rise.

What is meant by barzakh is the interval between two manifestations, not
the common understanding of the state after the physical death of
bodies. For this latter notion is below the responsibilities given to
humanity, as what is decreed upon them after death is known only to God.
What they are commanded in their lifetime, however, must be understood
and acknowledged.

Whoever journeys in the ocean of death beholds endless wonders. For
example, if someone had truly died during the time of the Messenger of
God, they would have perceived that all realities pertaining to those
who did not believe in Muhammad, from the realm of abstraction to the
realm of limitation, were pure negation and absolute fire. Conversely,
all realities pertaining to those who believed in Muhammad, from the
realm of abstraction to the furthest limits of limitation, belonged to
the Tree of Affirmation.

And the paradise of prophethood is established. The first was not
``dead,'' as he was annihilated in negation, while the second was
``dead,'' remaining steadfast in affirmation. Today, the fruits of the
``death'' of the believers are evident in how their mention is beloved
before God and creation, and the abundance of believers arises from
them. Conversely, the fruits of not ``dying'' among those beneath the
believers are that no mention of them remains. Even if their numbers
increased, they themselves would reject being associated with one
another and would disavow such connections.

For today, if someone refrains from speaking of the first tree in
opposition to the Truth, what was said of it would itself disavow them
and seek God's wrath on their behalf. In the year 1270, as the Tree of
Truth ascended, those opposing it descended further, becoming even more
hardened. Yet, as the manifestations have varied, they cannot be
discerned except by the proof that appears from God, for He alone knows
all things in their proper places. If He wills, He can distinguish a
particle of fire from a particle of paradise.

If a thought arises within a soul contrary to what is fitting for the
Point of the Bayán, the judgment of ``death'' is not rendered upon it at
the moment of its emergence. This matter is exceedingly subtle---indeed,
even more subtle than this---and only those with insight can comprehend
it. From the essence of the knowledge of divine unity, the attribution
of ``death'' is valid to the ultimate station of limitation.

For example, if someone in the station of ``Alif'' finds the ``Ba''
where it does not belong, they correct and elevate it. This pertains to
the functions of the Angel of Death, as such distinctions manifest from
Him. And if the ``Ba'' is raised to its proper station, it continually
calls upon its Lord, saying, ``Take me and revive me.'' When God wills
to answer its prayer, He inspires one of His chosen ones to remove from
it the spirit of ``Ba''-ness and bestow upon it the spirit of
``Alif''-ness. Then, it becomes capable of true comprehension. Prior to
this, the meaning of the word transforms, for ``after God'' is greater,
as it pertains to ``Alif.'' If the ``Ba'' is inscribed without its
essence appearing, what is intended by it will remain hidden. It is like
a vessel in both its macrocosmic and microcosmic dimensions, unperceived
by those who observe, even if it rests upon parchment.

Should one inscribe upon it something inferior, it must be erased, for
such a ``death'' grants life to that tablet. This is analogous to
removing from the human self anything that harms its faith. If someone
who does not believe in God possesses a tablet, one must declare, ``I am
dead,'' severing any connection to it and disregarding it entirely, for
it belongs to the realm of fire and remains within it. Conversely, if a
tablet is with one who believes in God, it must be preserved as one
safeguards their own being, for it belongs to the realm of light.

The matter, when it is most apparent, is also the most concealed. One
who understands death remains perpetually ``dead'' before God, willing
nothing but what God wills. This is death before the Point of the Bayán,
for nothing that God wills is revealed except through His decree. This
is the true essence of death for those who desire to die in God.

In the realm of creation, God has created nothing more honored than
death before Him. All aspire to align their will with the will of Him
Whom God will make manifest. Yet when He appears, they fail to fulfill
their love and claims, just as many who believed in the Qur'an elevated
themselves to a station where, if Muhammad were to return to life, they
would doubt His word. Indeed, He has returned in a form even more
manifest than before.

At the beginning of His appearance, it is evident that this is the
latter creation after the former. Those who declare that Muhammad is the
Messenger of God all gathered yet did not affirm Him. They were
unwilling to grant Him even what they desired for themselves in terms of
being associated with Islam, nor did they attribute to Him the rights
they claimed for themselves. This is something no Muslim would accept
from another Muslim. Such is the condition of creation before God.

If His prophethood had been established beforehand, they would affirm
it. Yet all remain veiled, countless in number, claiming adherence to
His religion but not believing in Him upon His return, except those whom
God wills. This persisted until what was to appear became manifest. For
those who do not recognize Him, no fire is more intense than the veils
they place between themselves and Him, the One who made Islam their
religion and the Qur'an their Book. Yet He has no greater honor than
that He attained the meeting with His Lord, conveyed His messages, and
turned entirely to Him with all that was within His power. This is the
true honor in which all should take pride.

If someone were to say, ``We did not recognize Him at the beginning of
His appearance,'' the response would be: It is universally known among
all who perceive reality that He was the first to respond in the
preexistent world when God asked, ``Am I not your Lord?'' He said,
``Yes, glorified are You! There is no God but You; You are the Lord of
all worlds.''

If they claim that they did not recognize the Manifestation of God, the
Qur'an, being the Book of God, was present among all and acknowledged as
such. When they heard or saw that the signs of God were manifest from a
soul, no doubt could remain in the hearts of those with perception that
this soul was the Manifestation of God. The signs before Him were His,
just as the signs after Him belong to Him.

And whoever responds first is the first creation, just as it was
previously said that the first to respond was Muhammad, and he was the
first creation. This remains acknowledged today. If they assert that the
first response occurred through him, then he is the primordial essence,
for above the throne of the heavens is indeed the place of the
Manifestation of God. God has eternally and will eternally remain
equidistant to all things; neither proximity nor distance applies to Him
uniquely among all creation. No thing is nearer to Him nor farther than
any other, whether the throne is above the heavens, as imagined by those
who speculate, or it is the station of the Tree that speaks by the
command of God.

Such beliefs are mere delusion and imagination. In the understanding of
the inhabitants of the realm of truth, ``place'' refers to the site of
manifestation. For example, all who visit the shrine of the Master of
Martyrs affirm what is recorded in tradition: ``Whoever visits Husayn,
upon him be peace, recognizing his station, it is as though they have
visited God above His throne.'' Among those with insight, it is evident
that this is the station of the Throne of God and the Throne of
Muhammad, the Messenger of God, may peace and blessings be upon Him and
His family.

It seems as though no one has ascended from the realm of limitation, for
all realities heard across the realms become embodied in this world.
Just as in this world the Commander of the Faithful, upon him be peace,
was the first to believe in Muhammad, this proves that he was a believer
across all realms. All realms are actualized under the shadow of this
world and manifest here before those with insight. Blessed is the one
who perceives all things as they truly are and does not imagine delusory
matters that have no reality before God or the people of insight.

The divine essence has eternally existed and will eternally remain,
whether manifest as the inner or the outer, or as the outer or the
inner. Whatever is mentioned regarding the manifestation of God refers
to the Tree of Truth, which signifies nothing but Him. This Tree is the
one that has sent forth all the messengers and revealed all the
scriptures, eternally existing as the throne of His manifestation and
concealment. It has always been among the people, manifesting in each
age according to His will.

For instance, during the revelation of the Qur'an, He manifested His
power through Muhammad, and during the revelation of the Bayán, He
manifested His power through the Point of the Bayán. At the appearance
of Him Whom God will make manifest, He will affirm His religion in
whatever manner He wills, by whatever means He wills, for whatever
purpose He wills. He is the one who has always been with all things, yet
nothing has been with Him. He is not within things, not above things,
and not alongside things.

When His ``sitting'' upon the throne is mentioned, it signifies the
manifestation of His power, not a physical throne or chair above the
earth, nor the celestial or terrestrial spheres that have existed
eternally. None has known Him, nor will anyone know Him, for everything
below Him is created by His command, and they continue to be created by
His command. He is exalted beyond any description or praise, sanctified
from any attribute or example. Nothing perceives Him, yet He perceives
all things.

Even the statement that ``nothing perceives Him'' ultimately relates to
the manifestation of His appearance, which is Him Whom God will make
manifest. He is the most exalted and the most supreme.

No one capable of pointing can gesture toward Him. Him Whom God will
make manifest is the first of His creation, and the mention of the
pronoun referring to Him returns to His heart, for both He and His heart
are His creation. God has eternally been Lord without being lorded over,
eternally been God without being worshiped, eternally been powerful
without any power exercised over Him, eternally been knowledgeable
without anything known to Him, and eternally been one without being
numerable.

The statement \textbf{``God has eternally been one without being
numerable''} pertains to the time when, at the manifestation of Him Whom
God will make manifest, one individual believes in Him. The hearts of
these believers testify to His oneness, and there are no others counted
among them. Similarly, all names and attributes are understood in this
way, without being confined by limits.

Do not focus on boundaries, for \textbf{``God has eternally been one''}
holds true whether or not you are a believer in this manifestation. If
you are not a believer now, consider the first manifestation: look to
the Messenger of God, where you can observe all names and attributes. If
you wish to say, \textbf{``He is a sovereign,''} you will see among His
community those who consider themselves mere servants of His. Yet the
sovereignty of His essence transcends association with this earthly
sovereignty.

If you wish to say, \textbf{``He is mighty,''} you will observe those
with insight honoring Him by expressing their pride in belonging to His
community. Yet the might of His essence is far above association with
this might. If you wish to say, \textbf{``He is knowledgeable,''} you
will find those endowed with knowledge taking pride in their connection
to Him. Yet the knowledge of His essence transcends any connection to
this knowledge of these learned ones. If you wish to say, \textbf{``He
is a ruler,''} you will observe many possessors of authority who take
pride in their rulership under His command. Yet the rulership of His
essence transcends any association with these manifestations of
rulership from Him.

Similarly, observe all names and attributes in their reality. When a
servant is knowledgeable, there is no knowledge except His. When a
servant is capable of an action, there is no power except His. In every
manifestation, all who are guided by that appearance are but aspects of
Him. For instance, if you consider the first manifestation, Adam the
first, to the infinite, nothing exists except through God. One cannot
truly recognize the manifestation of divinity except through the Tree of
His appearance, which is His primal will. Beyond this, recognition is
impossible, for such is the nature of possibility.

This is the meaning of the statement of the Master of Martyrs, upon Him
be peace: \textbf{``O my God, through the variation of traces and the
transformations of states, I have come to know that Your purpose for me
is that I recognize You in all things, so that I may not be ignorant of
You in anything.''} For this is the fruit of the existence of all
things: that all things recognize their being as established through the
primal will and see nothing in anything except the manifestation of God,
to the extent that the essence of each thing bears His appearance.
Beyond this, the association of His manifestation is equal in relation
to all things.

The signs of God manifest in one type of appearance, and from the same
source from which the signs of God originate, they descend into the
prophethood of a prophet and, below that, to whatever is fitting. The
relationship of His manifestation to these two aspects is equal, except
that one arises from the highest heights of affirmation, while the other
descends from the lowest depths of negation.

If you consider this meaning in a word during one manifestation, you
will perceive it in the essence of another. However, the intent is not
to observe the essence of God in all things, for this is impossible. He,
exalted is His mention, is beyond being within anything, with anything,
before anything, after anything, above anything, or below anything. What
gives reality to the existence of all things is His will, which is
itself self-sustaining. He has eternally been, and will eternally
remain, as all names exist under His shadow, while He is established in
the shadow of God.

The station of will is the station of the Point of the Bayán, which does
not manifest in anything except as an aspect of the appearances of His
command. The meaning is not that the essence of the will, which is the
essence of the Messenger of God, is seen in all things. Rather, it is
observed in all things that their reality is established through Him.

For example, if a person were to spend a thousand measures of gold for
the sake of the House of God, what would be seen in this act is nothing
but the command that the Messenger of God had given on behalf of God.
Similarly, if you ask how the essence of gold comes into existence, it
must necessarily relate to an aspect that ultimately returns to the Tree
of Truth, even if it manifests in one of His manifestations.

No thing can be described as having ``existence'' unless its reality is
established through His will. He is self-sustaining through God, the
Mighty and Glorious. He is the encompassing circle, eternally revolving
around Himself. He signifies and will always signify nothing but God,
exalted and glorified, to whom belong the most excellent names in the
kingdom of the heavens, the earth and all that lies between them. There
is no God but He, the Mighty, the Beloved. For every name, there is a
reality it refers to. For example, if it is said that God, exalted and
glorified, is the All-Mighty, then there must necessarily be two
manifestations that, in the presence of His primal will, are mentioned
and eternally established, signifying none but Him.

Blessed is the one who sees in every thing only the manifestation of
their Lord, who rests in nothing except in God, who perceives nothing
but Him, and who does not attribute to God what is attributed to His
creation. For God, glorified is He, is not within anything, nor of
anything, nor upon anything, nor toward anything, nor mentioned by
anything. All things beneath Him are created for Him. None but He can
comprehend His essence, and none but He can recognize His oneness in His
essence.

The will only knows itself, and all that creation perceives is what the
will manifests within them. God, exalted and glorified, in His essence,
cannot be known, perceived, celebrated, or sanctified. None can approach
Him except by recognizing their inability to comprehend Him and by
abiding in the shadow of His oneness and independence.

Everything exists through Him in its reality, essence, substance,
abstraction, beginning, end, outwardness, inwardness, purity, and
harmony. He is at the highest heights of His sovereign power and the
most exalted summit of His sacred kingship. He is beyond all description
and praise, sanctified above every attribute and eminence.

Eternally, God has been God---one, unique, self-sufficient, eternal,
everlasting, alive, sustaining, and independent. He has neither taken a
consort nor begotten a child. All beneath Him is His creation, brought
into being by His command. He has eternally been and will eternally
remain self-sufficient through Himself. How, then, could He not be
independent of all else and self-reliant in His essence? How, then,
could He not be independent of all others? Glorified and exalted is He
as befitting the loftiness of His sanctity and the exaltation of His
mention. Verily, He has always been the Most High, the Most Exalted.

\subsection*{Gate 9 (The Resurrection of Every
Soul)}\label{gate-9-the-resurrection-of-every-soul}
\addcontentsline{toc}{subsection}{Gate 9 (The Resurrection of Every
Soul)}

The ninth chapter of the second unity discusses the reality of the
grave. The summary of this chapter is that for every soul, a grave is
appointed within the bounds of its station, and all are brought to their
ultimate end at the appearance of Him Whom God will make manifest. His
resurrection is the resurrection of all; His gathering is the gathering
of all; His creation is the creation of all; and His emergence from His
grave is the emergence of all from their own graves.

For example, in the Point of the Bayán, when the Manifestation of
Divinity raised up the Messenger of God from His own essence, He raised
all those who were to be gathered under His shadow. Likewise, upon the
earth today, whatever judgment is issued in relation to the religion of
Islam will, by extension, apply to His religion.

All true souls guided by the Bayán are referred back to the primal
essence, which, on the Day of Resurrection, was the first to respond to
God's command and to acknowledge His oneness. Conversely, all souls
lacking truth are referred back to the Tree of Negation at the time of
His manifestation. Similarly, when the Messenger of God was raised, all
souls beneath the shadow of the Qur'an were resurrected under His
shadow. As the Tree of Affirmation was elevated, all true souls were
elevated under it.

This does not mean that one soul becomes connected to another, but
rather that all remain within the stations of their respective realms.
For instance, the letters of ``Hay'' remain within the stations of their
hearts, and at the first resurrection, they are resurrected according to
their designated positions. It is not that their souls surpass the
bounds of their stations. Likewise, the believers who exist under the
shadow of these letters remain within their respective realms. The grave
in which all are questioned pertains to their essential capacities. They
are questioned in the first mention until the limitless realities are
fully elucidated.

Similarly, those under the shadow of the Tree of Negation are questioned
regarding what is beneath the goodness encompassed by God's knowledge.
This is the meaning of the apparent tradition that, on the Day of
Resurrection, the Commander of the Faithful, upon him be peace, will
stand on the path before God and proclaim: \textbf{``Whatever goodness
exists is from me,''} while the second of the letters of negation will
confess: \textbf{``Whatever evil exists is from me.''}

The realms of goodness are infinitely abundant in the divine dominion,
as are the realms of the oppositional realms. For example, if for a
thousand years one soul causes sorrow to another, it is due to the
sorrow inflicted by the second letter of negation during the first
manifestation. All multiplied aspects of negation refer back to it,
while all affirmed aspects of goodness refer back to the first who
believed, who in turn refers to God.

If God does not raise Him up, He will not rise of His own essence. The
Tree of Negation, too, in its own bounds, ultimately refers back to God.
For if God does not raise it up, it remains in its initial state,
knowing nothing of itself or others.

This is the comprehensive grave encompassing the diverse aspects of all
souls. Each day, every soul is questioned about its faith in God and His
signs in the Bayán. If the soul responds, its grave is filled
withvlight, and the angels of mercy descend upon it. However, if the
soul does not respond, its grave becomes filled with fire, and the
manifestations of wrath descend upon it. This occurs within the
particular soul because it is an aspect of the comprehensive fiery
essence. Likewise, the one who responds to truth is an aspect of the
comprehensive luminous essence. For such a soul, a herald and bringer of
glad tidings are mentioned, while for others, the opposite applies.

This is the reason every soul benefits from acknowledging that the grave
is true. For if a soul belongs to the higher realms of the Most High, it
returns to Him and is then separated from Him in another sign---not
because its essence becomes His essence, but as a progression. Just as
those who believed in the Book of Alif returned to their faith in the
Book of Qaf, their creation spread, and they were nurtured within the
religion of Islam. Gradually, this continued until their time culminated
with the Day of the revelation of the Bayán. They returned to it, and
from it, their creation spread among another people until God willed
otherwise.

In this paradise, they grow and flourish until the Day of Him Whom God
will make manifest. Then, they will return to His Book, and from it,
they will spread among another creation until God wills otherwise, for
there is no limit to God's bounty. Similarly, if a soul under the shadow
of the letters of Alif does not believe, it returns to the first of
those who did not believe in the Messenger of God. From there, it
becomes separated and descends to the lower realms beneath the Most High
of the Bayán.

It remains in a station of ultimate negation until it returns to the
first letter of negation in the Bayán. Afterward, it becomes separated
again and proceeds toward its own annihilation until it ultimately
returns to the first who disbelieved in Him Whom God will make manifest.
Then it separates further, and in all these realms, its garment becomes
fire. Even if it be adorned with silk, its station becomes fire, even if
it is in the loftiest places above the earth. Its sustenance will be of
the same nature. Conversely, the garment of those beneath that station,
who are created for paradise, transforms their abode into the highest
chambers of Ridván. Their food becomes the choicest fruits of paradise,
even if their clothing is made only of cotton, their seat is but the
dust of the earth, and they consume nothing more than lettuce leaves.

There is no soul of a believer whose spirit is taken except that their
grave becomes a meadow from the gardens of the eternal paradise.
Whatever they desire, God creates for them there, and it is pure before
them. Similarly, there is no soul that does not believe in the Bayán
except that the pen is incapable of describing the divine wrath that
befalls them.

If one wishes to observe in this world the bounty bestowed upon the Tree
of Affirmation, they may perceive the countless realities emanating from
it, with every soul receiving from it according to what God has granted.
Conversely, if they wish to witness the wrath of God, they may consider
what has been revealed concerning the Tree of Negation, whose countless
realities are tormented in proportion to its essence, just as the
countless realities of the Most High are blessed in proportion to
theirs.

This is the explanation of the reality of the grave. Few are found who
truly understand the truth of it and recognize the grave as a reality.
Whoever has comprehended the Bayán of God regarding the truth has also
affirmed that \textbf{``the grave is true.''}

No spirit transcends the bounds of its station. For instance, a spirit
that pertains to creation has eternally been of creation, while a spirit
connected to truth has eternally been of truth. A spirit that serves as
a sign of God has eternally been a sign, without bounds or limitations.
No soul passes at the moment of death without God, the All-Knowing,
exalted in His sovereignty, commanding the angels who are purified,
sanctified, and glorifiers of His oneness and majesty to elevate that
soul to the highest station of paradise and the loftiest horizon of
Ridván. Such a soul will experience no sorrow after death if they were a
believer in what God revealed in the Bayán until the appearance of Him
Whom God will make manifest.

The first appearance will not be acceptable to those who pass without
faith in the Bayán, for the breezes of paradise will not reach them.
Similarly, if a soul, after the appearance of Him Whom God will make
manifest, hesitates even as much as to say ``yes'' or to gesture in
acknowledgment of His truth, the Bayán will yield no fruit for them. No
soul that departs without belief in the Bayán will taste even the
smallest degree of goodness or beauty after death. The pen trembles to
describe what has been prepared for such a soul.

Blessed is the one whose spirit is taken while believing in Him Whom God
will make manifest and His words, for such a person is a believer in the
Bayán and all that it contains. The grace of God toward the believers
has no bounds or limits. Eternally, the people of paradise belong to
paradise, except for those whom God wills, just as the people of the
Qur'an belonged to the Qur'an, except for those whom God willed at the
appearance of the Bayán by the will of the Point of the Bayán, which is
the will of God. The people of the fire remain in the fire eternally,
except for those whom God wills. This will becomes evident at the
appearance of Him Whom God will make manifest, as understood by those
endowed with knowledge.

For example, the people of the Gospel, prior to the mission of the
Messenger of God, were in paradise if they acted according to what God
had revealed. However, at the time of His mission, they were judged to
be in the fire unless they were guided during that time by God's will.
For such souls, it is said that they were saved from the fire and
entered paradise. Similarly, the believers in the Qur'an and what was
revealed within it are in paradise, except for those whom God wills
otherwise. This occurs if any of those believers fail to recognize the
Bayán, for they then enter the fire and leave paradise.

Eternally, the universal manifestation of the divine will appears in
every Resurrection and ascends. During the rising of the night, which
refers to the period of intermission (barzakh), the divine will becomes
concealed. God admits whom He wills into His paradise and prevents whom
He wills from entering it. None knows this except for those who
transgress the bounds of the Bayán, at which point their will aligns
with a will from the previous manifestation. Otherwise, God is always in
a state of action in every moment.

The universal manifestation of the divine will in the intermission may
open for the witnesses of the previous cycle a door of knowledge that
they are unable to bear. This has been the case from the beginning of
the appearance of the Messenger of God until His mission. Similarly,
prior to this appearance, from the ascension of Jesus, this matter
remained veiled. From the passing of the Messenger of God, peace and
blessings be upon Him and His family, until the initial revelation of
the Bayán, the divine will was concealed among the people and none could
recognize Him, for He has eternally remained in the station of the
primal Point and continues to do so. The Letters of Life dwell within
their stations, as do the other letters of the Most High and those
beneath them, each in their respective places. All draw sustenance from
Him, yet while He knows all, none can know Him. However, whoever turns
wholly toward Him is inevitably granted assistance in ways they do not
perceive, for He is the All-Powerful, the Most Glorious, the
All-Knowing.

\subsection*{Gate 10 (The Questioning of
Angels)}\label{gate-10-the-questioning-of-angels}
\addcontentsline{toc}{subsection}{Gate 10 (The Questioning of Angels)}

The tenth chapter of the second unity explains the questioning of the
angels in the grave. The summary of this chapter is that on the Day of
Resurrection, the believers in Him Whom God will make manifest will ask
people, ``Upon what is your religion established?'' They will respond,
``Upon the authority of the Bayán.'' If on that day they are believers
in the signs of God, they will answer the angels based on the authority
God revealed before and taught them. Otherwise, God's proof will be
completed against them.

If they do not believe, the word of wrath is decreed against them.
Afterward, the angels return to the presence of God, reporting the state
of these individuals. If they respond with faith, then the verses of
mercy are revealed upon them. If not, the words of wrath are
established, and whatever God decrees for them on that day will be
carried out.

For instance, whatever the Point of the Bayán decreed concerning anyone
will remain established until the Day of Resurrection. The
manifestations of negation will abide in the fire, while the
manifestations of light will remain in affirmation. The former will
suffer in the depths beneath the Letters of the Most High, while the
latter will eternally abide in God's mercy, granted within the Letters
of the Most High. This is a favor unmatched, extending until the Day of
Resurrection. Whatever is decreed for a soul in this grave applies also
to the physical grave. If the individual is a believer, their grave
becomes a garden from the gardens of paradise. If they are not a
believer, it becomes a pit from the pits of the fire.

Even during life, before those endowed with knowledge, the ruling of
death is established. If a believer sits upon a piece of earth, that
portion becomes a segment of Ridván on the Day of Resurrection by the
permission of God, the Exalted. Conversely, if they are not a believer,
that land becomes a portion of fire by the command of God, the Mighty.

The meaning of the angels' return to God and their presentation to Him
refers to the return of guides to Him Whom God will make manifest. No
path leads to the essence of the Eternal God---neither in the present,
nor in the return. Glorified is God above all that is mentioned in the
name of anything, past or future. To Him belong creation and command in
the dominion of the earth and heavens and all that lies between them.
There is no God but Him, the Most Great, the Most Exalted.

\subsection*{Gate 11 (The Resurrection of Every
Entity)}\label{gate-11-the-resurrection-of-every-entity}
\addcontentsline{toc}{subsection}{Gate 11 (The Resurrection of Every
Entity)}

The eleventh chapter of the second unity discusses the truth of
resurrection. The summary of this chapter is that God has created all
things in accordance with what He revealed in His Book. This includes
the souls of the Most High connected to their respective letters, and
those below them likewise connected to their letters. Every entity to
which the concept of existence applies will be resurrected on the Day of
Resurrection. The resurrection of every entity is through the mention of
Him Whom God will make manifest on that day, as the creation of that
entity also occurred through His mention at the beginning, even if it
belonged to a prior manifestation.

For example, this glass cup and saucer, now placed before God, will be
resurrected on the Day of Resurrection with their essential reality,
essence, and individual nature. This occurs when the Tree of Reality
speaks, proclaiming, \textbf{``This cup and saucer are, in essence,
manifestations of Him.''} Whatever is decreed for any cup or saucer
applies accordingly, just as before their existence, they were affirmed
by the word of the Point of the Bayán. This serves as an example,
illustrated at the level of the inanimate, to ensure understanding
across all realms.

For instance, the Letters of Life were placed into this cycle by His
command. In that Resurrection, He will resurrect these Letters in any
soul He wills, through His word. For nothing is created with essential
reality except through His word, as His word is the word of God. At the
moment of His utterance, the essential reality of a thing is formed. If
it belongs to the Most High, it is through the mention of truth; if it
is beneath the Most High, it is without such a mention.

Everything placed in the Qur'an by the Messenger of God stands by the
signs of God. Likewise, in this day, which is the day of the
resurrection of the Qur'an, everything is raised and established by the
command of God. All things upon the earth are in the presence of God,
and the resurrection of all depends upon the essence of all. Just as the
creation of all was established upon a single soul, so too is the
resurrection of all established upon a single soul. When that single
soul is raised, every entity is individually raised in its own station.

Even if the resurrection of a thing is through the mention of God---such
that none other than Him is aware of it---it eventually becomes evident
through the testimony of God, and all bear witness. It is as though this
foundation upon which the primal will rests is the very essence, even
though it has no beginning. The reality of each manifestation is
established through itself until it reaches its ultimate culmination.

For example, this foundation upon which the Messenger of God sat during
His time is the same principle described in the traditions that whatever
was with the Prophets is also with the Qa'im of the Family of Muhammad
in the stations He decrees. For everything they possess comes from Him,
and whatever He decrees regarding any thing is that very thing itself.
Even if during the early days of Islam Muhammad, the son of Abdullah,
manifested in a particular form and manner, today He appears in this
form and manner. Until someone becomes an observer of the Point of
Reality, they cannot comprehend the resurrection of all things in a
single soul.

Even so, any soul with understanding, through reflection, can grasp the
responsibility of belief in resurrection to the extent of their duty,
until the Day of the appearance of Him Whom God will make manifest. At
that time, whoever turns to Him will have their resurrection in the Most
High under the shadow of the Tree of Affirmation. Otherwise, their
resurrection will be in the lower realms under the shadow of the Tree of
Negation.

If something is not named explicitly in relation to resurrection, it
suffices that He proclaims, \textbf{``We have resurrected all things,''}
and all are raised in their respective stations under the shadow of the
single Point. If they belong to the Most High, they are under the Tree
of Affirmation; if they belong to the lower realms, they are under the
Tree of Negation. Nothing escapes the knowledge of God---neither in the
heavens, nor on the earth, nor between them. He resurrects all things by
His command, for He is all-powerful.

No soul is raised from the grave of clay as a dead body but instead
through the living souls of that time. If they are of the Most High,
they are among the believers; if they are of the lower realms, they are
from beneath that station. There is no escape from the resurrection of
all things on the Day of Resurrection before God, the Exalted. For God
initiates creation, then brings it forth again, and He measures the
creation of all things then resurrects it, for God is all-powerful over
all things.

\subsection*{Gate 12 (The Reality of the
Path)}\label{gate-12-the-reality-of-the-path}
\addcontentsline{toc}{subsection}{Gate 12 (The Reality of the Path)}

The twelfth chapter of the second unity explains the reality of the path
(\textbf{ṣirāṭ}) and affirms its truth. The summary of this chapter is
that the path refers to the manifestation of God and His command in
every age. Whoever is established on the true path is on the path of
truth, while others are on paths beneath it.

The example of the path is like the manifestation of the Point of the
Bayán. For those who believed in Him, the heavens and the earth became
more expansive than the heavens of acceptance and the earth of
receptivity, higher and broader. Those who traverse this path---the
verses of the Bayán---do so in varying degrees. Some, upon hearing the
message, immediately said, \textbf{``Yes, by our Lord, this is the truth
from God, without doubt, a revelation from the Lord of all worlds.''}
These passed the path closest to union with the essence itself.

Others believed by merely looking upon Him without needing to hear His
words, passing the path closer than the distance between the letter
\textbf{Kāf} and the \textbf{Nūn} in ``Kun'' (Be). Still others, having
heard the verses of God, paused and reflected to the extent of what is
deemed a ``thing.'' These lingered on the path, perplexed, for two
hundred and two thousand years---or rather, this duration is only a
figurative boundary for those who hesitated at the rising of the Sun of
Lordship. Otherwise, days, years, and months are irrelevant to those who
stop, for there is no ``beginning'' to the manifestation of God that can
be confined by limits.

All manifestations of the Divine return today to the Point of the Bayán.
This is why those who ponder deeply in the realm of the infinite
undertake their reflection, traversing the realms of the infinite faster
than the blink of an eye. Even if they proceed degree by degree, those
who reflect more deeply remain longer on the path, progressing further.

How many individuals remain on the path until the next Resurrection,
like the Letters of the Book of Alif who have remained on the path of
the Book of Qaf until now, even though its Resurrection has passed. If
someone were to consider the traditions regarding the path with a pure
heart, they would perceive everything clearly.

For those beneath the believers, the path is sharper than a single
strand of hair and keener than a sword, as they cannot escape the
authority of the verses that came before. They are unable to produce
anything like them or to acknowledge the truth beyond the confines of
their own selves. Thus, for them, the path becomes sharper than a sword
and finer than a strand of hair. Conversely, for the believers, it is
broader than paradise, as they declare, \textbf{``Our religion was
established by the Qur'an, which was the Book of God. All were incapable
of producing even a single verse like it.''}

Today, the same Tree that spoke the Qur'an now speaks the Bayán,
proclaiming, \textbf{``All is from God, without doubt, a revelation from
the Lord of all worlds.''} The incapacity of all to replicate the Qur'an
is equally apparent in the Bayán. For us, there is no hesitation in
traversing the distance between \textbf{Kāf} and \textbf{Nūn} in
affirming faith in God, His signs, and the reality of His Gate and His
words.

How many souls have passed the path and entered paradise? How many have
fallen into the fire while crossing? And how many remain halted, unable
to proceed? Those who traverse are saved, while those who deviate or
remain stagnant are described as under the shadow of the fire, alongside
the Tree of Negation.

How many souls worship God from the beginning to the end of their lives,
yet when the signs of God hear them, they do not acknowledge them. If
they reflect, they remain halted on the path, and they are gathered
under the shadow of the fire. If they deny the truth outright, they
enter the fire. Today, a reckoner is needed to count how many have
failed to cross the pure path, except those whom God wills. All others,
either halted or devoid of faith, are gathered under the shadows of the
fire, though they remain unaware of it.

God has decreed their judgment, which remains in effect until the Day of
Resurrection. Those still consider themselves in the heights of piety,
oblivious to the fact that God's judgment of fire has already been
pronounced against them. They will remain tormented by that judgment in
their stations until the Day of Resurrection. When God establishes the
path, all will comprehend it, yet each is veiled by something that holds
no value in the sight of God.

Today, the path of God is His signs, and all are capable of recognizing
their authority. Yet they veil themselves with things that bring them no
benefit. \textbf{Glorified are You, O God! Grasp the hands of those who
have believed in the Bayán on the Day of Resurrection by Your grace,
that they may cross the path more swiftly than anything else. Truly, You
are the one who watches over all things.}

\subsection*{Gate 13 (The Reality of the
Balance)}\label{gate-13-the-reality-of-the-balance}
\addcontentsline{toc}{subsection}{Gate 13 (The Reality of the Balance)}

The thirteenth chapter of the second unity discusses the reality of the
balance (\textbf{mīzān}) and affirms its truth. The summary of this
chapter is that from the Point of the primal will to infinity, in every
manifestation, the balance is the Point of Reality itself. Its commands
emanate from itself. The highest station of the balance in the unity of
God lies in the statement \textbf{``There is no God but He.''} Whoever
enters the balance of negation enters the balance of the fire, and
whoever enters the balance of affirmation enters the balance of
paradise. Entry into negation is only realized through allegiance to the
gates of the fire, while entry into paradise is only realized through
allegiance to the gates of paradise. Though all negation ultimately
leads to one tree, and all affirmation leads to another tree.

To observe the balance of a previous manifestation, one can look to the
balance of the appearance of the Messenger of God. From the time of His
mission until today, and to infinity, those in the scale of His justice
have gone to the fire, while those in the scale of His grace have gone
to paradise. Today, the balance is the Bayán. Whoever is not deviated
from it is in paradise and within the scale of grace. Otherwise, they
are in the scale of justice and the fire.

The origin of the creation of both fire and paradise stems from the
Point of Reality. God's judgment upon the one who turns toward Him
creates light, while His judgment upon the one who turns away creates
fire. God is the creator of both fire and light by His will, which is
the Point. He is the master of justice and grace, through the justice
and grace that manifest from this Tree.

For example, if the Point of the Qur'an had not decreed the authority of
the Commander of the Faithful, upon him be peace, the creation of
paradise would not have been realized. Similarly, if it had not been
revealed that whoever does not follow Him is not upon the truth, the
creation of fire would not have been realized.

Thus, all realities of goodness return through His word to the Tree of
Affirmation, while all realities beneath goodness return to the Tree of
Negation. The reality of the balance lies within the reality of the
primal Tree, extending infinitely through its expressions under its
shadow in the knowledge of God. For example, if someone today acts
according to the decree of the Messenger of God from the past, they are
enacting one of the realities of that balance. Conversely, anyone who
denies that is placed within the scale of justice.

Since the manifestation of the Bayán, all things have been resurrected
under the shadow of the primal Point. Whoever turns toward it remains
established in the scale of grace and affirmation. Conversely, anyone
who deviates even slightly is annihilated in the scale of justice, until
the Day of Him Whom God will make manifest. He is the Balance; His
command is the Balance; His conditions are the Balance; His character is
the Balance; His attributes are the Balance; His realities are the
Balance; His manifestation is the Balance; His words are the Balance;
and His indications are the Balance.

Whatever is attributed to the Balance is indeed the Balance if that
attribution is realized within the Book. Otherwise, to the extent the
attribution is severed, the judgment is also severed, until all
attribution is removed and negation remains in the fire. Truly, we seek
refuge in God, the One, the Radiant, from every mention of fire. There
is no command except from God, the One, the Creator.

\subsection*{Gate 14 (The Reality of
Reckoning)}\label{gate-14-the-reality-of-reckoning}
\addcontentsline{toc}{subsection}{Gate 14 (The Reality of Reckoning)}

The fourteenth chapter of the second unity discusses the reality of the
reckoning (\textbf{ḥisāb}). The summary of this chapter is that the
reckoning of all things lies in the hands of God, and none but God
possesses the power to reckon all. God, the All-Knowing, reckons all
things through the manifestation of the Tree of Reality in every
age---whether in its appearance or concealment. He reckons all things,
yet His judgment is not outwardly revealed except at the time of His
manifestation. He reckons on the Day of Resurrection with a single word.
Just as in this Resurrection, He has accounted for all creation with a
single word: \textbf{``Verily, I am God; there is no God but Me, the
Lord of all things.''} Whoever reflects upon the fire of negation is
reckoned with justice, while whoever affirms the light of affirmation is
reckoned with grace.

All beings upon the earth are accounted among them. The souls who are
subject to reckoning are referred to as the faithful souls of the
Qur'an, for others were already reckoned in the Resurrection of the
Messenger of God and were annihilated---not in their physical bodies,
but in their faith. The earth remains filled with their physical forms,
yet their spirits were extinguished.

For the faithful souls of the Qur'an, no proof was stronger than the
Book of God itself for the truth of their religion. For this reason, God
revealed the Tree of Reality with that same proof by which the religion
of these faithful souls had been affirmed. This word, which is the
reckoning of all, is to remain until the Day of Resurrection. It was
revealed in the language of Qur'anic verses, which is Arabic---the
clearest of all tongues.

The majority were reckoned with justice and consigned to the fire of
negation, becoming as nothing. Yet those souls who affirmed the oneness
of God through this word were reckoned with grace and rewarded with the
best reward: verses that will endure until the Day of Resurrection. All
goodness is contained within these verses, which will continue to reach
them until that day.

No verse has God revealed except that its related spirits dwell under
its shadow and are ultimately referred back to the soul for whom God
revealed these verses as a reward. This occurs during the long night
between two manifestations. If a soul were to say, \textbf{``We did not
attain certainty that this word is the word of God,''} it would be
answered, \textbf{``This very word was revealed in the Qur'an before,
and even prior to the Tree of Sinai.''}

For example, God revealed in the Qur'an a similar declaration, and from
the utterance of the Messenger of God, who recited this word, you
attained certainty that it was from God. By the same proof, certainty is
established here. For in that instance, it was said, \textbf{``Those
upon the earth are incapable of producing its like,''} and here, those
endowed with knowledge have observed the same matter.

What proof remains for denying that this word is the word of God? If any
soul claims they can produce its like, where is their evidence? From the
Tree of Reality, if a scribe can write, they may compose two thousand
verses in a single day---or at least as many as they are capable of
writing. Yet verses of this kind leave no doubt for those of insight
that they are from God.

These verses bear witness to themselves that they are the word of God.
No one can speak in the manner of these verses, for they are the words
of the will, which is the word of God. The eternal essence, which has
neither beginning nor end, remains as it has always been. Speech
pertains to the realm of creation and invention, for within the will,
nothing is seen except God. Thus, these words are attributed to God, as
none besides Him has the power to produce their like.

Whoever is connected to this Tree or will be, discerns that this word
aligns perfectly with their innate nature. To them, it is simpler and
closer than expressions of supplication, sermons, knowledge, or Persian
language compositions. This is because it is the word of their essential
reality, which points solely to God alone. For this reason, these are
called the verses of God and referred to as the word of God.

The eternal essence, which has always existed and will always remain, is
in a constant state. The act of speech, and what precedes or follows it,
are among the three conditions of the attribute of the primal will. Yet
God is far above being attributed such qualities or described in such
terms.

Every proof used to affirm the truth of the Qur'an is likewise
applicable to the Bayán. Thus, people should not disregard the reckoning
of God, whereby the pen records them as nothing from the beginning to
the end of their lives. How fearful a soul is regarding worldly
accounts, which are temporal and insignificant compared to those of
religion! Yet, where the measure pertains to faith and the unity of God,
and their deeds are rejected, leading to eternal annihilation, they fail
to reflect.

The people of insight would give up all that is upon the earth to hear
the Tree of Reality declare ``Yes'' concerning them on the Day of
Resurrection, rather than ``No.'' For the reckoning is made through this
word. All praise is due to God, for today every discerning soul
recognizes the majesty of the Day of Reckoning, even as veiled souls
remain unaware of these realities. This stems from their fear of worldly
obligations, reflecting the divine dislike for anyone carrying a debt
owed to another. People take great care to avoid becoming indebted in
worldly terms, yet they remain heedless of their souls and the purpose
for which they were created. For one thousand two hundred and seventy
years, they acted according to the Qur'an, only to have the pen inscribe
``nothingness'' over them on the Day of Reckoning and Judgment.

This demonstrates that people lack true spiritual awareness. If they
possessed faith's insight, they would give up all that is upon the earth
to have their reckoning conducted with grace rather than justice. For
justice places them in the fire until the Day of Resurrection, while
grace places them in paradise until the Day of Resurrection.

The seventh year of the appearance of Him Whom God will make manifest is
the year of reckoning, though it may also be determined as the seventh
month, the seventh week, or the seventh day. \textbf{He does as He wills
and decrees what He desires. None can question Him about what He does,
while all are questioned for what they do.}

\subsection*{Gate 15 (The Truth of the
Book)}\label{gate-15-the-truth-of-the-book}
\addcontentsline{toc}{subsection}{Gate 15 (The Truth of the Book)}

The fifteenth chapter of the second unity discusses the Book and affirms
its truth. The summary of this chapter is that the Book refers to
whatever is revealed from the Point of Reality. For the eternal essence,
which has always existed and will forever remain, undergoes no
alteration or change. Rather, the Book of the Point of the Bayán is a
book that points to God, for none but God is capable of producing such a
Book. It spans from a single letter to limitless expressions, for
whatever emanates from the Point of Reality becomes part of the Book.
Whatever is written by the hand of Him Whom God will make manifest is a
Book inscribed by the hand of God, for it is attributed to God and
remains so, as His Book is truth.

O spirits attached to the word of truth, in the words He
reveals---whatever their form---whether verses, which are the water of
the divine spring and the essence of Ridván; supplications, which are
unchanging milk; interpretations of verses, which are red wine; or
explanations of supplications, which are purified honey---all are part
of the Book of God. Even writings in Persian are akin to the verses, for
all flow from the ocean of reality. If someone contemplates Persian
writings with the eye of insight, they will perceive the eloquence of
the verses and attain certainty that none but God could produce such
words.

Yet how many acknowledge the Book as truth but remain veiled from the
Revealer of that truth? How many fail to recognize that the Point of
Reality, whose Book is the Book of God and even more exalted, is the
source of all these words? Indeed, one letter from His Book is more
exalted than all else.

The pride of the people causes the Revealer to hesitate in bestowing His
Book upon them, even though it could save them from the fire and admit
them to paradise. In some cases, out of His loftiness and mercy, He
grants it, yet those upon whom it is bestowed fail to recognize it. The
pen hesitates even to recount this, as they continually pray,
\textbf{``O God, grant me my book in my right hand.''} Yet when God
grants it, they refuse to accept it, rejecting the One who is a
Messenger among His Messengers. Yet they act contrary to what the Pen
hesitates to recount concerning them. The Book, however, is the Book of
their God, the Messenger is His Messenger, and it has been revealed out
of the exalted grace and generosity of God in a manner that allows them
to attain certainty that it is the Book of God, incomparable and
unmatched.

Just as today, the religion of all is established through the verses of
the Qur'an, the Tree of Reality has revealed its words upon all in the
same manner. Yet no one took notice, even though all acknowledged and
continue to acknowledge that the Book is true. With every appearance of
the Tree of Reality, believers in it and its Book are tested---whether
they affirm the truth of its preceding manifestation and Book or its
subsequent one. Through such tests, only the purest believers remain,
who are as rare as the red sulphur.

For instance, had the souls who believed in Jesus, the son of Mary, and
His Book recognized that the appearance of Muhammad was the same
appearance but in a loftier form, and that His Book was the Gospel in a
more exalted manifestation, none of the Christians would have deviated
from their religion. All would have believed in the Messenger of God and
confirmed the truth of His Book.

Similarly, if the believers in the Messenger of God and His Book had
certainty that the appearance of the Qa'im, upon Him be peace, and the
Bayán was the same as the appearance of the Messenger of God, and that
this Book is the Qur'an in a loftier form, no believer in the Qur'an
would have turned away from their religion. Instead, they would have
believed and confirmed the Bayán in a time quicker than the blink of an
eye.

However, their lack of certainty is unacceptable before God. Indeed,
what establishes certainty for them is evident. If they reflect upon the
proof by which Islam was established, they will observe that the same
proof is manifest here in an even loftier form. Their failure to affirm
and attain certainty signifies that their essential realities belong to
the Tree of Negation and return to it. Their faith and deeds performed
in Islam are provisional, not firmly rooted, and offer them no
benefit---not even the weight of a mustard seed.

If the deeds of the followers of the Book of Alif bring them no benefit
today, the deeds of these individuals will likewise yield no benefit.
Even if they act in full accordance with what God has revealed, without
altering a single letter, it will avail them nothing. This truth has
been realized by most jurists and religious authorities, who have
deferred to their own interpretations and deemed adherence to the
command of the will after its passing impermissible. Yet the truth of
the matter is evident before God and returns to the Tree of the Bayán.

The same source of judgment that revealed the law continues to operate
in all the interpretations of jurists and authorities. Whoever sees
nothing but the decree of God in any judgment is correct, even when
interpretations differ. Because all matters ultimately return to God,
they are acceptable---even though there is no fundamental disagreement
in what has been revealed in the Bayán from God. Each ruling is truthful
in its own context.

Discrepancies arise only when applied to specific cases, yet even in
such instances, they stem from variations in circumstances, times,
associations, and tools. These factors influence outcomes. However, God
remains ever engaged in wondrous activity, bringing forth new creations
and issuing fresh commands. Though everything emanates from the Point of
Reality until today has been and continues to be from God. Despite the
apparent infinite differences cited among the adherents of each
dispensation, it is evident that what originates from God contains no
discrepancies. With the insight of an expanded vision, the secret of
this truth becomes clear.

The laws are akin to the creation of beings; they are in perpetual
renewal and occurrence. In less than the blink of an eye, new creations
and commands are revealed. Yet the perspective of humanity is limited,
focusing only on the transitions from one manifestation to another. In
the realm of concealment, everything reverts to what was made manifest
in the previous appearance, though the inner reality supports all things
during its hidden phase. However, it extends aid specifically through
what has been revealed during its appearance, as certainty cannot be
attained by any soul except by the will of God, the All-Knowing and
All-Powerful.

\subsection*{Gate 16 (The Reality of
Paradise)}\label{gate-16-the-reality-of-paradise}
\addcontentsline{toc}{subsection}{Gate 16 (The Reality of Paradise)}

The sixteenth chapter of the second unity discusses the reality of
paradise and affirms its truth. The summary of this chapter is that
until now, no one except the manifestations whom God has chosen has
truly understood paradise or hell. All references to paradise in this
world pertain to this realm, which is the origin and culmination of all
worlds, where all are positioned in the station of the Throne of
Reality. This Throne is the first to be revealed through His revelation.

As mentioned in the supplication of ʿArafah: \textbf{``O You who has
established Your mercy over the Throne, such that the Throne becomes
hidden in Your essence, as all realms are hidden within Your Throne.''}
In the Qur'an, according to the people of the Bayán, the manifestation
of mercy in its primal reality is attributed to the Commander of the
Faithful, upon him be peace. This is because all that exists in the
worlds originates from Him and returns to Him.

For instance, from the time of Adam until today, with every
manifestation of truth, the first to believe in it opens the doors to
truth, while those beneath it remain subordinate. Ultimately, the
essence of all truth culminates in this manifestation, and everything
below it is realized in this appearance.

If one contemplates the essence, it becomes evident that the origin of
all worlds lies within the station of will, which brings existence into
being. This will, by its very nature, encompasses all worlds with a
comprehensive essence, for every rank-holder in this dispensation
acknowledges within themselves the exaltation of that rank in relation
to the will. The world of souls reflects the world of creation, as
evidenced by the manifestation of Muhammad, where He was revealed as the
first of creation. All believers in Him are convinced that He is,
indeed, the first of creation.

Once this reality is established, no soul can doubt that in the
knowledge of God, no paradise is greater than the manifestation of God
in the Point of Affirmation. With each appearance, the paradise of that
manifestation is tied to its era until all manifestations culminated in
the Point of the Qur'an. From the time of its revelation, no paradise in
the knowledge of God was greater than that of the resplendent soul of
its Manifestation. For within souls, no one greater than the Messenger
of God is recognized as an intermediary between themselves and God. This
truth of paradise is revealed in souls only through the appearance of
that primal soul. Afterward, in the knowledge of God, no paradise was
greater than the Commander of the Faithful, upon him be peace.
Similarly, within souls, no one was deemed greater than the Commander of
the Faithful after the Messenger of God. They focus solely on Him.
Similarly, degree by degree, the ranks of the gates of paradise multiply
until they culminate in the letters of unity. For instance, in the fifth
manifestation of paradise, no paradise, after those preceding it, was
greater than the one revealed at the end of His life, when He stood
alone on the plain of Ṭaff. The dwellers of that paradise were present
from the time of 'Alí ibn al-Ḥusayn, upon them both be peace.

In this way, observe all the paradises until they culminate in the
letter ``M,'' which then returns to the Point. From the moment this
Cause emerged, marked by the ninth hour and its most precise moment,
everything that could be enumerated began. The paradise of the Point of
the Qur'an in its latter stage became the paradise of the Bayán in its
initial phase.

In the knowledge of God, no paradise was greater than this until the Day
of Him Whom God will make manifest, and none will surpass it until then.
At His appearance, the initial paradise of the Point of the Bayán will
transform into its final paradise. After this, the paradise of the
letters of ``Ḥayy'' (the Living), representing the souls of the foremost
believers, will be the greatest of all paradises. This process
culminates in the paradise of ``S,'' the final letter of manifestation.

This does not mean that, during the appearance of each new paradise, the
prior paradise is no longer in its proper station. For example, at the
appearance of the paradise of ``S,'' the preceding paradise remains
enthroned in its position. The same applies to all the letters. For
every paradise that is the manifestation of His will, God has ordained
nineteen gates.

In this current manifestation, where this arrangement has been elevated,
it demonstrates that in each appearance this has always been the nature
of His will. As seen in the Qur'an, everything returns to the gates, the
gates lead to the People of the House, the People of the House to the
Messenger of God, and the Messenger of God to God. Thus, the fourth name
is sustained by the third from God, the third by the second, and the
second by the first, which is the essence of the name itself.

The first to enter this paradise was the Holy Spirit, who attained the
presence of God before all other souls and acknowledged His oneness. No
essence within the realm of possibility can be conceived as more exalted
or radiant in glorifying and sanctifying God than this spirit.
Similarly, for any soul in this world that reaches its ultimate
aspiration in the path of truth, the pinnacle of its joy lies in
offering gratitude to God for the blessings He has bestowed.

For instance, if a servant is elevated by God to the station of
gatehood, leadership, or prophethood, the ultimate fruit of their joy is
the act of thanking God, expressed in their words: \textbf{``Praise be
to God who has granted me this bounty.''} Likewise, in outward matters,
if a servant is seated by God upon the throne of sovereignty, their
ultimate delight is found when they turn to God and proclaim:
\textbf{``Praise be to Him who has honored me with this gift.''}

Thus, all joys ultimately return to the second station, the place of
praise. In the same way that essences return to that sanctified station,
so too do expressions return to their essence. This is why, for the
people of insight, the final fruit is revealed in the beginning. From
the praise that emerges following the manifestation of divine favors,
permission for such forms of gratitude is granted from the Source of
Glory and praise. This is why the first soul, upon entering the initial
paradise, attains every possible bounty within the realm of existence.
The ultimate fruit of all these bounties is expressed through the word
``praise,'' as that soul becomes the bearer of this spirit. Whatever
emanates from this soul emerges from the manifestations of its
paradisiacal reality, extending from the essence of understanding to the
outward form of the body. Even garments of silk are bestowed upon it,
and all pleasures in paradise become its enjoyment.

However, it delights only in its reflection, which is the station of the
Point. It is the revolving ``Káf,'' eternally turning around itself,
with no beginning or end, no limit, and no destination. Similarly,
anyone who enters the paradise of the Qur'an realizes that all blessings
they witness originate from the bounty of the Messenger of God. Even if
someone inherits a single ruby, it is only through the gift of that
primal reality, which has decreed it so. If it had decreed otherwise,
that specific soul would not have inherited it. Who could dare to
question this or challenge it?

Thus, all that exists in the presence of that Point, from the extent of
its being to its utmost ascension, derives from its generosity, which is
the generosity of God. Nothing possesses true existence except through
the grace of its essence. For example, if someone in this paradise wears
a garment of silk, it is solely through its generosity, even though they
may have been previously unable to do so.

This is why, if the eye of insight were opened, one would see that all
existence is but a handful from its boundless bounty. Everything who
entered the paradise of the Qur'an found themselves under the shade of
the first Gate, which is the Messenger of God. All other gates of
paradise are within the grasp of this primary Gate; indeed, it is
through His generosity that these gates have been opened. For instance,
anyone who ascends to the highest station in relation to the final
letter, achieving the utmost elevation and manifesting in their body
what is beyond conceivable in this world, does so only by the permission
of that one who has decreed: \textbf{``This is one of the gates of
paradise and My guidance.''}

Likewise, whatever is visibly manifested in the paradise of the tomb of
the Eighth Imam exists only by the decree of the Messenger of God, who
proclaimed him the tenth of His successors. Although such manifestations
may not be apparent at His tomb at present, all such revelations have
always been and remain within the grasp of His generosity and the power
of His authority. Under every gate of paradise, there exist countless
other gates. Every soul that has entered the guardianship of one of
these gates becomes, in itself, a paradise for that soul, within its own
rank. This truth, infinite in scope, has been evident and continues to
flow eternally.

Today, all conceivable pleasures lie within this paradise, and the
pleasures of previous paradises have been severed. For example, at the
appearance of the Messenger of God, all pleasures associated with the
paradise of the Gospel ceased. This is because true joy lies in the
recognition of God, in the knowledge of His pleasure, and in obedience
to the manifestations of His Cause.

Should a soul become detached from this, whatever paradise they once
enjoyed---no matter how exquisite their earthly pleasures---ultimately
returns to the fire of annihilation. Even if within the essence of the
Gospel letters there remains a sign of truth that once offered ultimate
delight, its relevance is confined to the manifestation of Jesus.

Today, that sign has been elevated, and those who hold on to the
illusion of its permanence persist in vain, thinking their patience has
sufficed. If all the adherents of the Gospel were certain that the
appearance of the Messenger of God was the very appearance of Jesus in a
more exalted form, not one soul among them would remain bound to the
Gospel or find delight in its paradise.

Similarly, after the Qur'an, any imagined delight today holds no truth.
If one acts between themselves and God for God's sake, imagining a
spiritual pleasure, it is misplaced because they are unaware that the
mirror of God has risen in another paradise. This is why all previous
delights have been severed, save for those who recognize God alone,
believe in Him, know His proof, and believe in it; who recognize His
book, believe in it, and embrace all that is revealed in His book.

Thus, the inhabitants of this paradise find their joy in it until the
first appearance of Him Whom God will make manifest. By the essence of
God, there is no greater paradise for them, at His appearance, than
faith in Him and obedience to Him. All other delights of the people of
the Bayán are thereby severed. Blessed is the one who enters His shadow
and takes delight in His remembrance, for they will endure with His
endurance into another cycle and beyond, with no end or limit. Even as
they remain in a manifestation, their faith and delight are severed at
the time of the next appearance.

I counsel all followers of the Bayán: If, at the appearance of Him Whom
God will make manifest, you are all granted success to attain the
greatest paradise and the supreme encounter, then blessed are you,
thrice blessed! If not, if you hear that a manifestation has appeared
with verses preceding the numerical value of the name of God,
\emph{al-Aghíth} (the Helper), then all of you must enter under His
shadow. But if it has not yet occurred and the cycle of
\emph{al-Mustagháth} (the One to be Invoked) has concluded, and you hear
that the Point has appeared yet remain uncertain, show mercy to your own
souls. Gather entirely under the shadow of that manifest Point, even if
you have not yet all entered its shade.

If you hear of a soul manifesting with signs, and the people of
knowledge in that era affirm Him---not the well-known scholars, but the
devout and discerning who see clearly, even though they may appear
clothed in poverty and humility---hasten to enter paradise, don robes of
silk, while those who cling to prior beliefs remain stagnant in earlier
judgments, failing to reflect. Enter completely under His shadow, for He
is the First and the Last, the Manifest and the Hidden.

If you have not heard of His appearance, then arise in supplication and
humility, so that God's grace to the \emph{Mustagháth} is not cut off
from you. However, if you hear that the \emph{Mustagháth} has indeed
appeared---He who is your Beloved and Mine, your Sovereign and
Mine---then let not even a breath's hesitation hold you back. Enter, all
of you, into God's shadow without questioning ``why'' or ``how.'' This
is the greatest of My commands to you, for it is through this that I
have created you.

O people of the Bayán, be ever thankful, for if a soul hesitates even
for the briefest breath after two thousand and one years, without doubt,
they are no longer of the Bayán and have entered the fire. Unless the
Manifestation of God has not yet appeared, all are then required to
supplicate and plead. It is not the case that you linger like the Jews
who awaited the return of Jesus, nor like the Christians who awaited
Muhammad, nor like the community of Muhammad awaiting the appearance of
the Qá'im of the lineage of Muhammad. Let not my heart be content if
anyone remains in the Bayán while the Point of the Bayán has come in its
finality, and you persist in what you are upon. By Him Whom God will
make manifest, there is no oath in the knowledge of God greater than
this: if He should appear and anyone remains in the Bayán, all the wrath
of the Bayán shall be upon them, doubled in intensity.

For today, the wrath of the Qur'án is multiplied against those who have
not entered the Bayán, just as the wrath of the Gospel afflicted those
who did not enter the Qur'án. Likewise, the Qur'ánic wrath doubled
against them. If such a soul exists, they have never truly entered the
Bayán, nor sought the shelter of the Tree of the Seven Letters.
Similarly, one who remains in the letters of the Qur'án without entering
the Bayán has not truly entered the Qur'án nor found refuge under the
shadow of the Muhammadan Tree. Otherwise, they would not have failed to
follow the more exalted appearance of Muhammad in the Hereafter.

The same applies to the Gospel. If a soul truly believed in Jesus, they
would have inevitably believed in Muhammad. Their failure to do so is
proof that they had no true faith in the Gospel or Jesus. Today, if
anyone reflects on the traditions regarding trials and tests, they will
see how much effort the true friends of God made to save people on the
Day of Manifestation.

Indeed, those with discerning insight have mentioned in their books that
the appearance of His Holiness, peace be upon Him, is the very
manifestation of the truth about which people have inquired, as
referenced in the tradition of Kumayl. Yet, God guides whom He wills and
admits them into paradise. This is the reality of paradise in the realm
of life and after death, known only to God. There, ``\textbf{no eye has
seen, no ear has heard, nor has it entered the heart of any human
being.}'' God has created therein everything that one may desire or ask
for. Even if the heavens were oceans of ink, all things pens, and every
soul a scribe, they could never comprehend even a single aspect of
paradise after death. The orbit of that paradise is tied to this very
paradise mentioned during the servant's life. If one enters the paradise
of God's Manifestation in this life, they will enter the paradise of the
hereafter after death. Otherwise, they will enter the fire upon their
passing. I seek refuge in God from such a fate.

In that paradise, I saw nothing but God; within it, only Him; before it,
only Him; after it, only Him; above it, only Him; beneath it, only Him.
To Him belong creation and command, before and after. There is no God
but Him, the Living, the Sustainer, the Self-Subsisting.

A soul may dwell in the highest heights of paradise in one Manifestation
and, after the next appearance, descend to the lowest depths of the
fire. I seek refuge in God and cling to the cord of the Point of the
Bayán, in its beginning and end. He is independent of anyone's faith in
Him or their entrance into His paradise. Indeed, if all refuse, they
enter the fire, while He Himself attains His paradise. Therefore, if one
soul guides another, it is better for them than owning all that exists
on earth. If they guide a soul to find refuge beneath the Tree of Unity,
God's mercy reaches both; otherwise, owning all the earth profits them
nothing. The earth is severed from one's possession at the moment of
death, yet the path of guidance remains founded on love and compassion,
not severity or dominance. This is God's eternal way, ``entering whom He
wills into His mercy,'' for He is the Most Gracious, the Most Generous.

There is no paradise greater for any soul than to recognize God's
Manifestation during His appearance, to hear His verses, and to believe
in the meeting with God, attaining the honor of union with Him. To
traverse the ocean of His good pleasure, which encompasses the realm of
divine delight, and to rejoice in the pinnacle of paradise---the station
of singleness---is the ultimate bliss. Praise be to God that today, many
remain veiled from this paradise, becoming captivated instead by that
which is severed from them at the time of their passing and casts them
into the fire.

Veiled from that for which they were created, they not only fail to
recognize their purpose but also oppose it. It would have been better if
they had simply remained indifferent, rather than hostile. And better
still if their opposition had not led them to manifest what is
unbefitting regarding the Tree of Truth. They profess belief, yet
abandon it on the mountain of seclusion. They claim reverence while
leaving it in solitude.

For the doers of deeds, there is no fire greater than that of their own
actions. Similarly, for the faithful, there is no paradise more exalted
than their own faith. ``There is no God but God,'' the God of paradise,
its Lord, Possessor, Sovereign, and Ruler of all that is within it.
Paradise subsists by His command, yet He is independent of it and all it
contains.

However, those within it rejoice in His presence during His
manifestation and later in His hidden state, even if they are unaware of
it. All within it ceaselessly glorify Him with praise at every moment
and beyond time itself and after time itself. They will proclaim the
oneness of God, their Lord, in every state, before every state, and
after every state. No soul will enter therein without declaring from the
depths of its being, ``Verily, I am God; there is no God but Me, the
Almighty, the Beloved. Verily, I am God; there is no God but Me, the
Watchful, the Sustainer. Verily, I am God; there is no God but Me, the
Lord, the Sovereign. Verily, I am God; there is no God but Me, the True,
the Inaccessible. Verily, I am God; there is no God but Me, the Lord of
all things, the Lord of the mighty throne.''

All things are ultimately drawn back to the form of the human temple,
whether masculine or feminine. If such a soul resides in paradise, then
everything under its dominion is also in paradise, even if it be a thing
unparalleled in its essence within its realm. For whatever emanates from
the divine will is reflected in the image of the soul, in its own
station. If stripped of all forms, nothing remains in its essence except
the human form in its true nature.

This is why all things seek from God to exist under the shadow of that
which is ascribed to the believer, not beneath something lesser.
Paradise is that which is associated with God, and what is ascribed to
the believer is ultimately linked to God. If a thing is associated with
a soul devoid of faith, everything related to it dwells in the fire,
regardless of how incomparable it may be in its realm.

For example, this chamber without known doors or limits, today the
highest pavilion of divine favor where the Tree of Truth is settled,
seems to resonate in its very atoms the refrain: ``Verily, I am God;
there is no God but Me, the Lord of all things.'' Even if other chambers
are adorned with gilded furnishings, should the Tree of Truth dwell in
one such chamber, at that time, the particles of mirrors would resonate,
just as the particles of the exalted mirrors of the supreme station of
leadership did, which in the days of the land of Ṣād reverberated and
continue to do so. The decree for anything is tied back to that soul,
and thus, any place where the people of paradise reside reflects the
utmost potential of their station. Those who gaze toward the truth will
witness it.

For instance, no resting place was lower than the site of the martyrdom
of the fifth word on that day. Yet, as it belonged to paradise, it
manifested at the highest conceivable level within its shadow. If an
observer had gazed upon it that day, they would have perceived its walls
worthy of being raised from rubies, not gold. The absence of such
manifestations was due to the frailty of the souls, not the incapacity
of the land, for it was receptive.

Similarly, the land upon which the opposing tree was established on that
day was the loftiest of its time. Yet, the discerning could see its
annihilation on that same day, as though it had already vanished. Today,
there is no trace of either the dwellers of that land or their abode.
Thus, God extinguishes negation and elevates affirmation by His command,
for He has power over all things.

Conversely, should the resting place of the Tree of Truth be exalted to
the highest heights of any land, it inevitably rises in loftiness,
eternally and continually ascending. Even in the shadow, if it be the
lowliest of lands, it descends to the utmost depths. Ultimately, all
things return to the human soul, which, in turn, returns to God in
paradise if it is faithful to Him Who will appear. Otherwise, it reverts
to falsehood in the fire if unfaithful.

\subsection*{Gate 17 (The Reality of the
Fire)}\label{gate-17-the-reality-of-the-fire}
\addcontentsline{toc}{subsection}{Gate 17 (The Reality of the Fire)}

This chapter highlights that the reality of the fire is undeniable, with
countless manifestations and levels. Its essence lies in the absence of
recognition of God, which becomes evident in each manifestation through
ignorance of the divine presence revealed in the truth of the
manifestation.

When the Tree of Reality is made manifest, those who fail to recognize
it are engulfed by the fire of doubt, denial, or neglect. Even if they
had previously proclaimed the oneness of God, recognized Muhammad as His
messenger, and acknowledged `Ali and the Imams as His proofs and the
doors to guidance, their lack of understanding and hesitation renders
all prior affirmations void.

For instance, the heart (fūʾād) that once affirmed ``There is no god but
God,'' recognized Muhammad, and upheld the truths of the Qur'an, the
Prophets, and the Imams, falls into the fire upon encountering the
Manifestation and failing to respond with belief and acceptance. Whether
due to heedlessness, denial, or hesitation, this failure results in a
negation of all truths once professed, as these truths were initially
derived from the Tree of Reality. Upon the new appearance of this Tree,
all spiritual recognition and affirmation must likewise be renewed, or
they are rendered null.

This passage elaborates on the spiritual dynamics of recognition and
rejection in the face of divine manifestations. It underscores that the
fruits of the previous faith were inherently derived from the Tree of
Reality. When this Tree reappears in a new and more exalted
manifestation, failure to recognize and embrace it nullifies all prior
affirmations and renders the individual as if they had never believed.

The essence of paradise for those who had recognized the earlier
manifestation transforms into fire when they fail to acknowledge the
latter, more sublime revelation. Even the foremost inhabitants of the
prior paradise may find themselves in the lowest depths of fire if they
resist the new manifestation. This is because the first act of
disconnection---turning away from God---places them in a state of
spiritual deprivation.

The first among the heedless, although given abundant grace and
opportunities to recognize the truth, including the sending of numerous
signs and envoys, chose rejection and became enshrouded in the ``garment
of the fire of denial.'' Conversely, the essence of paradise is embodied
by those who turn toward the manifestation and embrace its truth with
absolute sincerity.

Thus, the true paradise or fire of any age is not a fixed state but is
directly tied to the soul's response to the divine manifestation. Those
aligned with the manifestation enter paradise, while those who turn away
become engulfed in the fire of their own rejection, manifesting in their
detachment from the source of divine grace.

And what enters Paradise is that which relates to the first who
believed, while what pertains to the first who disbelieved returns to
the Fire. The aspects of this Fire are innumerable, though its principal
gates are mentioned as nineteen. Otherwise, no one but God knows their
full count. For every soul that enters through a gate of Fire becomes
itself a gate of Fire within its own station, and every soul that enters
through a gate of Paradise becomes itself a gate of Paradise within its
own station.

Even though all who are in the Fire trace their state back to its first
one, and all who are in the Light trace their state back to its first
one, both groups worship God, prostrate before Him, revere Him, glorify
Him, and affirm His oneness. However, the former group affirms His
oneness in the context of this world's appearance, while the latter
affirms it in the context of the Hereafter's appearance, which is also
the first appearance of this Dispensation and the final appearance of
the previous one.

Thus, the former becomes void, while the latter remains established.
That former group becomes the Fire, while this latter group becomes the
Light. That one becomes obliterated, while this one remains eternal.
That one is debased, while this one becomes exalted. That one is
impoverished, while this one becomes enriched, until nothing remains of
the former---neither name nor mention, even in itself.

Yet today, both proclaim, ``There is no God but God.'' However, what is
declared in the Bayán corresponds to what God loves, while what is
declared in the Qur'án corresponds to what God does not love. Similarly,
in previous times, whoever adhered to the Book of Alif (the Gospel)
would have declared the oneness of God. Yet, at the time of the
appearance of Qáf (the Qur'án), God, exalted and glorified, desired the
unity of His oneness to be declared through the oneness of Muhammad, the
Messenger of God, rather than through the oneness of Jesus.

Whoever followed the will of God had a spiritual essence established
within them by Him, except for those in whom God entrusted that spirit
temporarily; it would eventually depart. But whoever did not follow, the
spirit of separation (referred to as ``Shín'') would take hold within
them. The Spirit settled within them unless it was temporarily
entrusted, in which case it must inevitably depart. Yet both groups
worship God. This is why, following the refusal of the first ``Shín''
(denial) to prostrate, its essence and characteristics became manifest,
as expressed in the traditions. For instance, in the time of Muhammad,
this became the plea: ``Pardon me, O Messenger of God, from
acknowledging the authority of ʿAlī, the Commander of the Faithful,
peace be upon him.'' The divine response in that day, as conveyed
through Muhammad, aligned with God's decree: ``From where I will, not
from where you will. I desire to reveal My purpose as I will, not as you
will.''

Thus, this same utterance is now manifest in the Bayán. The first to
accept is the essence of Paradise, and the first to reject is the
essence of Hellfire. All attributes of good culminate in the first who
said ``Yes,'' while all attributes contrary to good culminate in the
first who said ``No.'' As stated in the Qur'án: \emph{``Do not be the
first to disbelieve in Him.''} Similarly, I exhort you all: \emph{``Be
the first to believe in Him Whom God shall make manifest on the Day of
Resurrection, that you may become the source of all goodness in the Book
of God, for this is the supreme bounty.''} And I warn you not to veil
yourselves from Him, lest you become the source of all that is contrary
to goodness, for this is the supreme chastisement.

If a discerning soul reflects, they will perceive that all goodness
manifest in the Bayán traces back to the first who said ``Yes,''
affirming the truth of the Primal Point and occupying the highest
station in the celestial realms. Conversely, all that opposes goodness
traces back to the first who said ``No,'' who is the essence of the Fire
and occupies the lowest depths of denial in the Bayán. This pattern will
remain until the advent of Him Whom God shall make manifest, whereupon
similar reasoning and outcomes will prevail: Paradise to the one who
accepts and Fire to the one who rejects.

The essence of Hellfire comes into existence only through the refusal to
prostrate before God, the Worshiped One, which, in turn, is established
by the lack of faith in the Point of the Bayán. This failure affirms its
fiery nature in opposition to divine truth. The manifestation of this
``Hellishness'' is not for others or itself but is ultimately directed
toward God.

For example, in their extreme caution within their own homes, such
individuals might, due to doubts, refrain from performing ablution,
abstain from fasting, or fail to offer prayers, even in their local
mosques. Yet, all these acts are invalid before God, being rooted in the
essence of denial and rejection. Such actions, perceived by them as for
God, are, in reality, the very signs and proofs of their veiling from
the Point of the Bayán.

Had they recognized Him, they would not have entertained even a fleeting
hesitation regarding Him. Yet they knew Him---for the signs revealed by
the First Point of the Qur'án, the verses of God, are manifest today in
the Bayán. Having seen and heard these verses, their refusal to bow and
acknowledge the Lord of Lords established within them the potentiality
of denial. They then donned the cloak of the first gate of Hellfire.

God forbid that one should meet such an end! For one who believes in
God, no harm can come upon them, and God protects His believing
servants. Every soul that stands in opposition to one of the gates of
the Paradise of the Qur'án becomes a universal gate of Hellfire. Other
fiery attributes gather under their shadow, all ultimately tracing back
to the primal heart of denial, which is the essence of pure rejection.
Conversely, all goodness and its attributes flow from the Primal Point
of the Qur'án, whose essence is the ultimate confirmation and source of
all affirmation. The most beloved thing in Hell is the veiling from the
truth, while the most detested thing in Paradise is veiling from
recognition of God.

Everything near such a soul becomes their own self, from which they
remain veiled. For instance, in the Qur'án, the love of everything was
manifest in the slayer of the Fifth Letter, who proclaimed ``God is the
Greatest'' while enduring all that he did. If he had known that this one
was the Manifestation of that very Greatness, he would have prostrated
before Him, and no other love would have entered his heart.

The most hateful thing to such a person is their very self, which is
veiled from recognizing its own reality. This aligns with what God has
revealed in all scriptures: that He does not wish for the command of
``Zayd'' (a generic name for a detached figure) to prevail. For example,
the first letter of fire in the Bayán---had they comprehended the words
of one who attributes themselves to the People of the House, the Imáms,
the successors of the Prophet, and the Prophet Himself as a
Manifestation sent from God---they would have been honored before their
own selves and revered before all.

How could one whose claim is rooted in affirming prophethood, which is
the Manifestation of pure Lordship and Divinity, entertain anything but
love and humility toward such a Being? Day and night, they would bow
before Him. Yet, they remain veiled and unaware of their Beloved, having
clad themselves in the garment of the most despised thing---the denial
of the first who turned away from Muhammad. Indeed, this is worse than
that, for each subsequent Manifestation becomes the cause for the denial
of the former, as the succeeding Revelation is exalted in its station.

For example, the creation of Christ was for the sake of the appearance
of Muhammad, just as the creation of the Qur'án was for the revelation
of the Bayán, and the creation of the Bayán is for the advent of ``He
Whom God shall make manifest.'' Even though the Pen hesitates to
inscribe the name of one who would conceive of anything less than
obedience to Him on the Day of His Manifestation with clear and mighty
signs from God, should it do so, it would be due to their rejection.
Such veiling arises from denial, and such denial arises from the failure
to accept.

Negation arises when one turns away, distance results from such
estrangement, and fire manifests when that distance becomes fire itself.
On the Day of His Manifestation, if anyone seeks refuge in ``He Whom God
shall make manifest,'' none of these outcomes will befall them. The
essence of seeking refuge in God on that Day is faith in Him, not
anything beyond this affirmation.

For even the first gate of fire would repeatedly utter words like
``There is none but God,'' countless and unquantifiable times, seeking
refuge from its own self, which was characterized by negation. Yet, this
availed it nothing. Had it been truthful, it would have sought refuge in
the Point of the Bayán through faith in Him, distancing itself from its
own essence, which had no belief in Him. Otherwise, what benefit could
accrue from prostrating night and day while adhering to the Qur'án but
rejecting His ordinances? Could thirty-four daily prostrations to God
compensate for rejecting the One who is the Manifestation of His
Command?

In such a state, one merely perpetuates what their nature requires---a
condition so shameful that the Pen hesitates to inscribe it. Though they
prostrate and seek closeness to God, they are, in truth, the farthest
from Him, the first among the denizens of fire. Similarly, in the
Bayán's Dispensation, just as the believers circle around the singular
divine Word in paradise, so too, in the realms of fire, it is the same.
Every multiplicity springs from the first gate, whether in light or
fire.

Whoever believes in the Bayán and what God has revealed therein remains
under the shade of paradise. Conversely, those who turn away dwell under
the shadow of fire. The term ``gate'' here does not signify something
like the gates of a city but rather serves as a representation. For
example, the gate of paradise in the realm of ``T'' refers to the
radiance within that world, all existing under the shadow of the primary
gate. In every instance, I seek refuge in God from all deniers of
affirmation---before, during, and after every moment. Place your trust
wholly in God.

The first gate of paradise is symbolically the Point, while the first
gate of fire represents its counterpart, extending until the
manifestation of ``He Whom God shall make manifest.'' Visualize the
waters of negation flowing through the veins of negative trees, causing
them to perish, while the waters of affirmation course through the veins
of steadfast trees, sustaining them until negation is completely
extinguished. When negation is annihilated, nothing will remain of it in
the Bayán except its mention in the sacred text. Affirmation, however,
will be established, its reality enduring as those who are connected to
it take pride in their association.

How many a fire does God transform into light through ``He Whom God
shall make manifest,'' and how many a light does He turn into fire? If
the Manifestation appears within the span of ``Ghiyath,'' and all enter
under His shadow, none will remain in fire. If He manifests during
``Mustaghath'' and all accept Him, no one will remain in fire either,
for all will be enveloped in light.

This sublime favor and supreme illumination have been sought from ``He
Whom God shall make manifest,'' a favor unparalleled and a light
unequaled. It ensures that the community does not remain in a state like
that of the followers of the Gospel, who waited for two succeeding
revelations from God but lingered in anticipation, still awaiting the
one promised, ``Whose name is Ahmad.'' Even if the Manifestation does
not appear within these two designations, He will inevitably manifest,
for His appearance is inescapable.

Although the reality of His manifestation might shine as clearly as the
sun at its zenith, some might view the decline of all others as akin to
stars fading into the brightness of day---not in the physical sense of
bodily forms or outward appearances, but in the station of faith and
truth. They might imagine circumstances akin to today and declare,
``That is God, your Lord. To Him belong creation and command. There is
none other but He, the Exalted, the Supreme.''

While the divine measures vary in every instance, no mention exists in
the Bayán except that of Him. Perhaps this ensures that, upon His
appearance, one will not witness sorrow among the faithful in their
hidden states.

In the verse mentioned, the believer is reminded of the absolute
reliance on God and His sufficiency as the ultimate sustainer and
protector. It is stated that all things depend on Him, and without faith
in Him, nothing bears fruit. Similarly, with faith in Him, no harm or
lack can truly affect one. The text emphasizes the singular importance
of God, declaring that nothing in the heavens, the earth, or between
them can suffice apart from Him, for He is the all-knowing,
all-powerful, and all-sufficient.

The mention transitions to life after death, stating that the human
heart cannot grasp its realities. The prayerful phrase, ``I seek refuge
in God,'' rejects any association with the torment of fire. The
narrative then turns to the symbolism of fire and light, asserting that
in the current manifestation of the Bayán, the distinction between these
elements becomes evident. When a servant enters a land belonging to
light, they enter paradise; if they enter a land associated with fire,
guided by love for the divine, they enter fire immediately unless their
intention is purely for God's sake.

It is described that believers in God refrain from entering locations
considered as domains of fire unless sanctioned by the Point of the
Bayán or ``He Whom God shall make manifest.'' If these figures see a
benefit for a believer's soul or if divine testimony necessitates it,
then entry may occur.

The passage concludes with a reference to historical and symbolic lands,
such as Kufa, where fiery opposition once resided. It invokes
acknowledgment of God's truthfulness in fulfilling His promise, urging
the faithful to observe and reflect on these manifestations of divine
will.

Thus does God obliterate denial until no trace of it remains, even in
the land itself. Then observe the station of light and proclaim, ``God
has fulfilled His promise.'' Likewise, God establishes affirmation by
His command, for He is all-knowing and all-powerful.

\subsection*{Gate 18 (The Hour)}\label{gate-18-the-hour}
\addcontentsline{toc}{subsection}{Gate 18 (The Hour)}

\textbf{Chapter Eighteen:} On the Explanation that ``The Hour is Coming,
Beyond Any Doubt''

The essence of this chapter is that in every manifestation of the divine
Will, the ``Hour'' is, in its primal reality, the Manifestation itself.
This ``Hour'' descends until it encompasses all rightful utterances
attributed to it in previous dispensations. If it is declared in a prior
manifestation that ``the Hour will come,'' this statement is true, for
it signifies the exaltation of the succeeding Manifestation. In the
current dispensation, the ``Hour'' refers to the Bayán itself, which has
come to humanity to judge them until the Day of Resurrection, leaving no
escape from its decree.

If the vast seas of the heavens were ink used to describe the ``Hour,''
even a drop of its reality could not be fully conveyed. In every true
manifestation, the ``Hour'' is established when the Divine Decree
confirms it as such. Contemplate the advent of God, for the ``Hour''
will come upon you suddenly, and before God, your Lord, you shall all be
presented.

\subsection*{Gate 19 (The Bayan is a Gift for Him Whome God Shall Make
Manifest)}\label{gate-19-the-bayan-is-a-gift-for-him-whome-god-shall-make-manifest}
\addcontentsline{toc}{subsection}{Gate 19 (The Bayan is a Gift for Him
Whome God Shall Make Manifest)}

\textbf{Chapter Nineteen:} That Whatever is in the Bayán is a Gift from
God for Him Whom God Shall Make Manifest

Glory be to You, O my God! How insignificant is my mention, and whatever
is attributed to me, when compared to Your boundless reality. Should I
seek to relate it to You, then accept it from me and all that is
ascribed to me, through Your grace, for You are the best of the
generous.

The essence of this chapter is that the traces and fruits of the
Manifestation of Truth, in every dispensation, are gifts from God
bestowed upon Him Who will appear in the subsequent Manifestation.

Whatever God revealed to Jesus was a gift from God for Muhammad, the
Messenger of God. Today, the meaning of Jesus' book lies within the
souls of those who believe in him, and even the smallest portion, like a
particle of clay, is counted as returning to its origin. Similarly,
whatever God revealed to Muhammad, now visible in those who believe in
the Qur'an, reflects the hearts of his believers as mirrors. This was a
gift from God to the Qá'im of the family of Muhammad.

Likewise, whatever is formed in the Bayán from its cherished dimensions
is a gift from the Point of the Bayán to Him Whom God shall make
manifest, who represents the final appearance of the Point of the Bayán.
It is an honor and pride for all to be accepted by Him, whether as a
soul or as an entity associated with His mention.

For example, today the least of the Qur'anic manifestations would not
accept the gift of the Gospel without aligning it under the shadow of
love for Muhammad. Similarly, the Bayán would not elevate even the
humblest aspect of a prior revelation unless it entered the love and
acceptance of its own shadow. In the same way, Him Whom God shall make
manifest will not accept anything attributed to the Bayán unless it is
also attributed to His own book.

This cycle continues indefinitely, as the first manifestation of God
serves as a gift in every subsequent appearance, drawing back to the
initial manifestation as a divine offering for the day of its
reappearance in the next world.

How distant is the soul that severs itself from its connection to Him
and removes this gift from itself! For instance, if believers in the
Qur'an today wish to deliver a gift to the Messenger of God, they must
all believe in the Bayán; otherwise, they sever this connection and the
divine grace is withheld from them.

The essence of exaltation lies in this: the Tree asks, ``Why have you
veiled yourselves from meeting your Beloved, who has always been and
remains the object of your hearts' desires?'' This is because, in your
pursuit of the world, you do not act unless you perceive in it the
pleasure of God. Yet today, when the Tree of Truth, the source of all
divine pleasure, has manifested itself through the fruit of its
existence---which is and has always been the culmination of your
souls---you have veiled yourselves. Whatever affects you arises from
your own selves, for God is indeed independent of you and of all that
you ascribe to yourselves. If you attribute yourselves to God, then it
is through this that you achieve piety and find cause for glory.
Otherwise, you will annihilate only your own selves, and you shall bear
witness to this and come to certainty.

\section*{Vahid 3 (The Manifestation of
God)}\label{vahid-3-the-manifestation-of-god}
\addcontentsline{toc}{section}{Vahid 3 (The Manifestation of God)}

\markright{Vahid 3 (The Manifestation of God)}

\subsection*{Gate 1 (Creation is For Who Points to
Him)}\label{gate-1-creation-is-for-who-points-to-him}
\addcontentsline{toc}{subsection}{Gate 1 (Creation is For Who Points to
Him)}

The Gate and the first tenth of the third Unity state that whatever is
mentioned as belonging to a thing is its dominion, and it is more
rightful for it than for any other. The summary of this principle is
that God, the Lord of all worlds, has created all things for the one who
points to Him. That one is the mirror of truth, who has eternally and
will forever manifest God. All things have been and are created through
him. He exists by his own essence, through God, while all things exist
through him. Nothing comes into being except by his will. Thus, he is
more worthy of all things than anything else, for all else belongs to
him by the granting of the sacred Essence, who has given ownership of
all things to him. He is, therefore, more rightful of all things than
all things themselves.

The fruit of this knowledge is that if the Point of Truth bestows all
things upon a single entity, it is and has always been more rightful
than any other, whether in practice or by decree alone. For example, if
the Messenger of God in the past had taken possession of all that is
upon the earth, he would have been more entitled to it than its owners,
for it is through God's granting that all things belong to Him.
Similarly, if Him Whom God Shall Make Manifest takes possession of the
realities of all things, He is more rightful.

From their realities to their own selves---yet He is far beyond and
greater than turning His gaze toward all things, for all things gaze
upon His generosity and grace. He is self-sufficient in His essence,
independent of all things, and dependent on God by His very being.

The fruit of this principle is that, at the time of manifestation, if He
issues a decree, all who recognize His truth must not question Him with
``why'' or ``how'' regarding His command. No one has the right to assert
ownership of any matter before Him if He commands regarding it, for He
is more entitled to it by His own essence. Although He does not command
except in accordance with the decree of the Bayán until such time as He
wills to renew His covenant with all things, if He commands one of those
who inherit according to the Bayán that one carat is the divine decree
regarding them in the Bayán, and if they act contrary to it, they have
disobeyed the command of their Lord.

For instance, if today the Messenger of God were to say that a command
revealed in the Qur'an should now be performed in a particular way,
there is no doubt that this command would still be the decree of the
Qur'an, even though it is now expressed differently, for all that was
revealed previously was from Him. The decree of the past and the present
is the same for those who know His truth. This is His entitlement.
However, the entitlement of the people is such that the pen of modesty
refrains from mentioning it.

All follow His command to perform the Friday prayer, yet if He directs
one of them not to pray at their designated place or declares another
soul more worthy of their position, they would not accept it. They would
have expressed faith in His first command but would not submit to the
second. For example, if one like the Muslims was not satisfied with Him
being, this situation would not have occurred. This is the entitlement
of the people, and that is His entitlement, which resides in hearts.

The verse of God's oneness, if replaced by another verse through His
command, becomes more rightful to Him than the former, as it originates
from His own essence. For instance, during the manifestation of the
Messenger of God, the verse of oneness that previously resided in the
hearts was elevated, and a new, radiant verse in the Qur'an was
revealed, illuminating all hearts. If it were not more rightful, how
could He have abrogated the prior verses? In the supreme exaltation of
servitude, when the servant declares His ownership in this manner, how
could the derivatives of such declarations even deserve mention? It
suffices that those who gaze upon the Sun of Truth know their place and
take pride in their association with Him, even if it is through the
attribution of ownership, just as all things derive their honor from the
fact that He is the sovereign of all things, and none other than Him.

If, in a later manifestation, others fail to take pride in Him, the
truth of their prior acknowledgment and pride in Him remains affirmed.
This reality is evident and manifest. If the letters of the alphabet
take no pride in the letters of rhyme and do not seek shelter under
their shadow, they are nonetheless honored through their association
with the tablets of the alphabet. Similarly, this has been true from the
earliest times until it reaches the first Adam, who has no beginning,
and beyond Him to what has no end.

Say: God begins all things and then restores them, and there is no
mention after God. Will you not then believe?

\subsection*{Gate 2 (The Word is Truth)}\label{gate-2-the-word-is-truth}
\addcontentsline{toc}{subsection}{Gate 2 (The Word is Truth)}

\textbf{The second gate of the third Unity}: By His word, a thing is
created, for His word is truth. The essence of this gate is that the
word of the Tree of Truth cannot be compared to the words of others
among created beings, for by His word the reality of a thing comes into
existence. For example, had the Tree not spoken in the Qur'anic
dispensation about the guardianship of the Commander of the Faithful
(peace be upon him), the creation of that guardianship would not have
come into being.

It would not have come into being, even though he has eternally been the
Guardian of God. However, the inception of manifestation depends upon
His word in that particular dispensation, not before it. In this manner,
all that is beneath the Truth, from the first spark of fire to its last,
is created by His word. If He had not declared, ``This is beneath the
Truth,'' it would neither have been realized in the horizons nor
manifested in the souls. This is why both light and fire circle around
His word.

Similarly, in the Dispensation of the Bayán, if the mention of the first
fire had not been made, how could its creation have been realized?
Likewise, in the case of light, if He had not mentioned it, how could
its guardianship have been established? Reflect on what befell the gates
of fire after the veiling of the Tree of Love in the Qur'anic
dispensation---why is there no mention of them? How is their reality
affirmed from before them? The repudiation of them has been apparent to
all, and there has never been any doubt for those endowed with hearts.

If He speaks with the letter of the Illiyyín, the Universal Soul is
created within its realm to proclaim the oneness of God with sincerity
and purity. If He speaks with the letter beneath the Illiyyín, the
Universal Soul is created within its realm in fire according to His
justice, such that the pen of modesty refrains from mentioning its true
worth.

In every manifestation of the Truth, no honor is greater for the people
of that manifestation than that the gaze of the Manifestation does not
regard what is beneath their worth. If it does, then by necessity He
will mention His decree, and through His decree, their creation is
realized, becoming a fire for the people of that manifestation. If
people understood how beneficial it is that the gaze of the
Manifestation does not regard anything beneath the Truth, then all would
strive with the utmost strength to ensure this.

That which is beneath the Truth should not be mentioned in His presence,
for if a judgment is made concerning it, it becomes a fire in that
manifestation---a fire in which people perish. For instance, at the
beginning of this manifestation, had all strived to ensure that nothing
unworthy of the Tree would be mentioned, it would not have become a fire
in which the veiled ones are condemned to dwell eternally. Creation
comes into being through His mention. Otherwise, why does no one mention
the prior letter in comparison to this second one, even though the
injustice of both, in relation to creation, is equal? Rather, because
this one, in opposition to the Truth, performed an unworthy act, its
existence is realized in this way until the Day of Resurrection, when
all will renounce it and it will be consumed in the fire of its own
annihilation.

No honor is greater than this: in the manifestation of every Truth, the
people of that manifestation take hold of the Words of God, for the
creation of their realities depends upon Him. For example, if a verse is
revealed without specifying a particular subject or command---such as
the verse: \emph{``To God belongs the kingdom of the heavens and the
earth and all that is between them, and God is powerful over all
things''}---a manifestation in the Bayán arises that points to this
verse, and even to an infinite extent, manifestations under the shadow
of that manifestation continue to emerge, all pointing to this verse.
Similarly, verses of this nature are few in the Qur'an, yet the
manifestations of divine authority from the time of the appearance until
today are beyond count.

Thus, under the shadow of each verse, infinite forms are realized. Even
if the matter pertains to a specific command, all adhere to it until the
Day of Resurrection, such as the obligation of the Khums or other
prescribed ordinances. This is why His word creates things, and it is
unlike the word of anyone else, for in it nothing is seen but God, and
there is no reality other than God.

The Creator of all things, yet not the sustainer of all things, not the
protector of all things, not the giver of life to all things, nor the
originator of all things, nor the initiator of all things---except that
to Him belongs creation and command, from before and after. Such is the
Lord of all worlds.

If a soul recognizes the manifestation of \emph{Him Whom God Shall Make
Manifest}, there is no greater honor for it and all others than ensuring
that no mention of fiery attributes is made in His presence, implying
that anything other than God has created the fire. Each soul, to the
extent of its own reality, receives from that source of grace until the
next manifestation, so that between the two manifestations, it may take
pride in His bounties, even if it is through a single verse. Indeed, the
spirits attached to it will inevitably become manifest concerning it.
And who is more truthful in speech than God, if only you would believe?

\subsection*{Gate 3 (The Bayan Revolves Around He Whom God Shall Make
Manifest)}\label{gate-3-the-bayan-revolves-around-he-whom-god-shall-make-manifest}
\addcontentsline{toc}{subsection}{Gate 3 (The Bayan Revolves Around He
Whom God Shall Make Manifest)}

\textbf{The third gate of the third Unity}: The Bayán and all within it
revolve around the utterance of \emph{Him Whom God Shall Make Manifest,}
just as the letter ``A'' and all within it revolved around the utterance
of Muhammad, the Messenger of God, and that which God revealed to Him in
His first manifestation. Likewise, all within it revolved around His
word during His latter manifestation.

The essence of this principle is that the focus of the Bayán is none
other than \emph{Him Whom God Shall Make Manifest,} for none other than
Him has ever, or will ever, fulfill its purpose. Indeed, His revealer is
none other than Himself. The Bayán and its believers yearn for Him more
than any lover longs for their beloved. Similarly, the Qur'an and the
spirits attached to it longed for the manifestation of its Revealer, and
they looked, and continue to look, toward none other than Him.

Today, the Qur'an sends blessings upon those letters that have ascended
to and been incorporated into the Bayán. At the same time, it calls for
vengeance upon those spirits that have not entered into the Bayán nor
recognized its Revealer. And He did not grant His word to it. Likewise,
the Bayán sends blessings upon the believing souls among itself, who are
the letters of Illiyyín, for they believe in *Him Whom God Shall Make
Manifest* and elevate Him toward His Book. It seeks vengeance from the
Almighty God upon the letters beneath the Illiyyín, who, at the time of
His manifestation, do not prostrate to God in acknowledgment of Him and
remain veiled from the presence of God.

If someone observes with the eye of the heart, they would hear today
from the letters of the Qur'an: \emph{``Help, help, O our God and the
God of all things! Rescue us and deliver us from the fire of our
association with that to which we were once ascribed. Attribute us
instead to You, admit us into the Bayán, for indeed, we have been among
those who supplicate for Your grace.''} This is the very cry of the
letters of the ``A'' from before and will be spoken word for word by the
Bayán in the future.

The Bayán bestows mercy upon those souls who have not squandered their
due rights, nor purchased for themselves wrath, and who have prostrated
themselves before its Revealer. At the time of the manifestation of
\emph{Him Whom God Shall Make Manifest,} the Bayán looks toward its
believers and asks, \emph{``Is there any spirit among you who will come
today to acknowledge Him Whom God Shall Make Manifest, or remain
faithful to the covenant of their Lord within Me?''} The Bayán rejoices
in the acceptance of its believers, as they turn to its Revealer, and
grieves if any sorrow from the believers reaches it concerning its
Revealer.

Today, there is nothing more sorrowful than the Qur'an, for all recite
it, yet they possess none of its mercy---only its vengeance. This
mirrors those who, at the time of the revelation of the Qur'an, recited
the Book of the ``A,'' yet held nothing of its blessings. The people of
the Bayán have not acted like the people of the Qur'an, who became
veiled from their Beloved by various things. Instead, the elevation of
the Bayán is its ascent, and its delight is its afterlife. Its souls cry
out, \emph{``O Bayán!''} They contemplate the command of God and
prostrate before Him to whom they were commanded to prostrate, for the
Bayán is not enriched by them unless they believe in \emph{Him Whom God
Shall Make Manifest,} who is its Revealer and the Revealer of all Books.
It intercedes for its believers before Him, and its intercession is
accepted in His presence. There is no servant who calls upon God through
the Bayán whose prayer is not answered---until the appearance of
\emph{Him Whom God Shall Make Manifest.} At that time, if someone calls
upon God for something incompatible with the Bayán, their prayer will
not be accepted.

O Lord, from Your grace and bounty, we beseech You concerning the Bayán
and all within it, for what You love and not for what You do not love:
that You may have mercy upon it and upon those who believe in it at the
time of Your manifestation, and that You may elevate it and those who
believe in it on that Day with what You reveal from Yourself, for You
are the most merciful of the merciful.

\subsection*{Gate 4 (The Next Manifestation is Greater Than the
Prior)}\label{gate-4-the-next-manifestation-is-greater-than-the-prior}
\addcontentsline{toc}{subsection}{Gate 4 (The Next Manifestation is
Greater Than the Prior)}

\textbf{The fourth gate of the third Unity}: That which God has revealed
to Him in terms of verses and words is greater and loftier than what was
revealed before.

The essence of this principle is that in every manifestation, the divine
will elevates the manifestation itself, and its words are loftier than
those of the previous manifestation. Indeed, the first is a sign of the
second, and the second is a sign of the third, as is evident in the
sight of God and those endowed with hearts. In reality, the first exists
only for the sake of the second, the second only for the sake of the
third, and the third only for the sake of the fourth, extending
infinitely toward what has no end. This is the measure of its
progression.

The favor of the Messenger of God, may peace be upon Him, over Jesus is
akin to the favor of the Qur'an over the previous Book. Similarly, this
holds true for the next manifestation and the one after that. For the
command of God has no limit, nor is there any cessation in what God
manifests. Blessed is the soul that, at the time of every manifestation,
dons the garment of the tradition of `Askarī, peace be upon Him. The
Holy Spirit, in the gardens of Sagur, tastes from our pristine orchards.

The meaning of wondrous words is the Tree of Truth in every
manifestation. If there is doubt among those endowed with hearts about
the greatness of the Qur'an compared to the Book of the ``A'' in their
manifestations, that doubt will persist in subsequent manifestations.
There is no later manifestation except that it is the very manifestation
of the first in a more exalted form. Likewise, His Book is none other
than the first Book, in a more elevated manner. This is why all are
veiled---they cannot comprehend. Otherwise, the command of God is more
manifest than anything, for to God is the ultimate return and end, both
in the beginning and in the end.

The manifestation of God's will in each manifestation is a sovereign
manifestation over all things. Lesser manifestations are mentioned under
its shadow. For instance, whatever is exalted from the Imams or the
Shi'a in the manifestation of God is sheltered under the shadow of the
manifestation of the Messenger of God. This holds true for what came
before and for what follows thereafter. No manifestation of His exists
except that it prevails over all and dominates all particles of what is
mentioned in its name. In its manifestation, it is a shadow over all,
and all must follow it. This is the command of God, from before and
after, and we are all devoted to Him.

\subsection*{Gate 5 (The Command Is Entirely the
Manifestation's)}\label{gate-5-the-command-is-entirely-the-manifestations}
\addcontentsline{toc}{subsection}{Gate 5 (The Command Is Entirely the
Manifestation's)}

\textbf{The fifth gate of the third Unity}: Elevated stations above the
earth are raised if He permits, and if He does not permit, they remain
fixed---for the matter is in His hands. The essence of this principle is
that once the manifestation of the Tree of Truth has been realized, the
command is entirely His.

In all that He commands and forbids, His prohibition is decisive in what
He forbids. There is no escape: souls in the exalted stations beloved of
God must be raised from their graves, and at the time of their
resurrection, the stations return to His command. If He permits their
elevation, they are elevated; otherwise, they remain fixed. To Him
belongs creation and command. He does as He wills and decrees as He
desires. He is not questioned about what He does, but all are questioned
concerning everything He does. What He does is what God does, and what
He decrees is what God decrees. He is not questioned about His actions,
but everything else is questioned, for He is the mirror that reflects
none but God, the One Lord of all things---the Lord of all that is seen
and unseen, the Lord of the worlds.

\subsection*{Gate 6 (God Has Created Everything and Everything is Below
Him)}\label{gate-6-god-has-created-everything-and-everything-is-below-him}
\addcontentsline{toc}{subsection}{Gate 6 (God Has Created Everything and
Everything is Below Him)}

\textbf{The sixth gate of the third Unity}: Whatever is mentioned as
bearing the name of a thing, God has created it in the realm of
origination.

The essence of this principle is that God has revealed in the Bayán a
word that encompasses all knowledge, which is this: \emph{``Verily, I am
God; there is no God but Me. All that is below Me is My creation. O My
creation, fear Me!''} Every thing to which the concept of ``thingness''
applies is, beneath God, His creation in the realms of origination,
invention, bringing forth, and manifestation. However, within these
levels, the manifestations of truth exist, who are all signs leading to
God. They are the ocean of eternal names and attributes, always pointing
to God. Yet, they are mentioned only in terms of names and attributes,
not in terms of the essence of divinity or His essential being.

The reality of their existence is that they are, beneath God, His
creation, and all are His worshipers. Every thing to which the concept
of ``thingness'' applies is brought into being by God through His will,
and His will itself is brought into being by His own self. Thus, today
all things are ascribed to the Bayán, for the spirit of their
``thingness'' is derived from it.

Within these truths and realities, all have been created by the very
essence of the Seven Letters, which is the manifestation of the primal
will. It becomes evident in the manifestation of the mystery of God and
remains hidden in every inner reality of God's command. Indeed, all
belong to God, and all return to Him. \emph{God originates all things,
then brings them back, and we are all devoted to Him.}

\subsection*{Gate 7 (Meeting God is Meeting the
Manifestation)}\label{gate-7-meeting-god-is-meeting-the-manifestation}
\addcontentsline{toc}{subsection}{Gate 7 (Meeting God is Meeting the
Manifestation)}

\textbf{The seventh gate of the third Unity}: That which God has
revealed about the mention of meeting Him or encountering the Lord
refers to \emph{Him Whom God Shall Make Manifest,} for God cannot be
seen in His essence.

The essence of this principle is that the eternal essence, in itself,
cannot be comprehended, described, characterized, unified, or seen.
Although all perceive, describe, praise, and glorify Him, and He is
seen, what is mentioned in the heavenly scriptures as the encounter with
Him refers to meeting His manifestation. The intended meaning is the
Point of Truth, who is the primal will.

What is mentioned in the Qur'an about meeting God or encountering the
Lord refers, in its essence, to meeting the Messenger of God. It
gradually descends from the primal reality until it appears in the form
of a thing that points solely to God under the shadow of that primal
reality. For instance, what is revealed concerning the Imams of
Guidance, such as \emph{``Whoso recognizes you has recognized God''} and
similar expressions, stems from this knowledge. Through understanding
this principle, the meaning is unlocked.

Similarly, regarding the believer, it is said that their joy is the joy
of the Messenger of God, and the joy of the Messenger is the joy of God.
Likewise, their sorrow is the sorrow of the Messenger, and their sorrow
is the sorrow of God. By this, the believer intended is the primal
reality---the gates of guidance---and from there it descends until it
applies to every believing soul. Even if a believer commits a sin, the
connection to this reality remains intact.

In him, nothing is seen except God, for he is attributed to Him.
However, if it resides in the possession of one beneath the believer,
nothing is seen in it except fire, for it is attributed to them. The
same applies to the dust upon which they are settled and to everything
attributed to them. All things have been created solely for the meeting
with God, which is the meeting with the primal will in its true essence.

Mentions of what lies beneath Him are by nature indirect and do not
possess independent existence. His likeness in all conditions is like
the sun, while those beneath Him are like mirrors, reflecting the sun's
radiance. If the mention of meeting is applied to Him, it is through the
reflection of the divine unity that originates from Him and appears
within Him. Otherwise, the application of this term is only rightful for
Him. Whoever attains the presence of \emph{Him Whom God Shall Make
Manifest} has attained the presence of God and has succeeded in the
presence of the Lord, provided they are believers in Him. Otherwise,
even if they witness Him in a particular manner during His ascent, they
will not have truly attained the meeting with God.

For such individuals, what benefit is derived for them? Indeed, it would
have been better for them to have remained in eternal nothingness than
to exist without faith and without the meeting of the primal will. The
meeting of the divine will with the primal will is akin to the
reflection of the sun in mirrors, continuing thus to the end of
existence. How, then, could one compare the direct meeting with the sun
itself to the reflection in mirrors? Although the reflection is nothing
but the sun and speaks of nothing but the sun, such is the condition of
contingent existence when faced with the eternal essence, and the state
of createdness when encountering the preexistent truth.

Whoever associates the meeting with \emph{Him Whom God Shall Make
Manifest} with another meeting, or attributes to Him an equal, a
counterpart, a semblance, a peer, or any likeness in this meeting, or
describes Him with what applies to others, has failed to recognize Him
and is unworthy of mention.

Whoever ascends in their existence does not surpass their own capacity
to recognize Him. If knowledge of Him is impossible, how could knowledge
of the Eternal Essence be conceivable? \emph{Glorified is God above what
the speakers say, with great glorification, and exalted is God above
what those who remember mention, with great exaltation.}

\subsection*{Gate 8 (Everything Is Contained Within the
Bayan)}\label{gate-8-everything-is-contained-within-the-bayan}
\addcontentsline{toc}{subsection}{Gate 8 (Everything Is Contained Within
the Bayan)}

\textbf{The eighth gate of the third Unity}: That everything in the
greater world is contained within the Bayán.

The essence of this principle is that whatever is named as a ``thing''
belongs to the Bayán. Its name and the spirit attached to it relate to
the name, not to the thing itself, which pertains to its essence. All
that exists in the Bayán is encompassed by this verse: \emph{``If We
were to reveal this Bayán to those in the dominion of the heavens and
the earth and all that is between them, then all, by their Lord, the
Merciful, would believe. God, there is no God but Him---the Living, the
Watchful, the Self-Subsisting. God, there is no God but Him---the
Overpowering, the Manifest, the Unique, the Inaccessible, the Exalted,
the Holy. His are the most excellent names. All in the heavens and the
earth and all that is between them glorify Him. Glorified and exalted is
He above what they describe. Say: Verily, God, He is the Sovereign, the
Supreme Ruler, the All-Powerful, the All-Knowing. His are the highest
attributes, and all in the heavens and the earth and all that is between
them prostrate to Him, for He is the Mighty, the Beloved.''}

These are nineteen names that point to God, under which all names and
likenesses are mentioned. Opposite them are the nineteen letters of
negation, which correspond to the nineteen gates of fire, as opposed to
the nineteen gates of paradise.

Whoever does not believe in the Bayán but recites these four verses,
which pertain to creation, sustenance, death, and life, with hearts
reliant upon these names, will not be considered a believer. Such a
person belongs to the gates of fire, and one must seek refuge in God
from them.

And anyone who is purified and dissociated from falsehood, attributing
all mentions of good things in the Bayán to its names and likenesses,
while considering all mentions of that which is below as falling within
the silence of the letters of negation, effectively recites the entire
Bayán and confirms all that God has revealed previously. These four
verses return to this verse: \emph{``God bears witness that there is no
God but Him. To Him belongs the kingdom and the dominion, then might and
power, then authority and divinity, then strength and the jewel, then
sovereignty and humanity. He gives life and causes death, then causes
death and gives life. Verily, He is the Living who does not die, the
Sovereign who does not fade, the Just who does not wrong, the Ruler who
does not shift, and the Unique who cannot escape anything from His
grasp, whether in the heavens, the earth, or what is between them.
Verily, He has power over all things.''}

This verse, in turn, relates to another verse: \emph{``God bears witness
that there is no God but Him. To Him belongs creation and command. He
gives life and causes death, then causes death and gives life. Verily,
He is the Living who does not die. In His grasp is the dominion of all
things. He creates whatever He wills by His command. Verily, He has
power over all things.''}

This verse connects to the phrase \emph{``In the Name of God, the Most
Mighty, the Most Holy.''} All the letters of the Basmala return to the
point of the letter \emph{B,} for all began with the point, and the
entirety of the Bayán is an elaboration of the point. Its appearance is
reflected in the mirrors, and its likeness is that of the sun, while all
the letters are like mirrors reflecting its radiance. Within no letter
is there a first except Him, no last except Him, no manifest except Him,
and no hidden except Him.

Likewise, whoever enters under the shadow of belief in the Qur'an, in
their reality, nothing is seen except the reflection of the Cause of the
Messenger of God, for by His grace, they are sheltered under His shadow.
The entirety of the Bayán is the manifestation of the point, and the
point holds the station.

The Will of God's manifestation is the essence of the appearance of God,
and all things return to \emph{Him Whom God Shall Make Manifest,} for He
is the one to whom all the Bayán and all within it return with utmost
humility and profound reverence. He is the one seen in the mirrors of
the Bayán. For example, if the Bayán manifests justice, He is the Just;
if it manifests grace, He is the Gracious; if it manifests sovereignty,
He is the Sovereign; if it manifests knowledge, He is the All-Knowing;
and if it manifests power, He is the All-Powerful. For in the mirrors,
nothing is seen except the sun, even if they exist in their reflective
capacity.

All the letters of the Illiyyín of the Bayán return to *Him Whom God
Shall Make Manifest,* who is the first gate of paradise and the Most
Great Name revealed in divinity. All beneath the letters of the Illiyyín
return to the first gate of fire, which exists under the shadow of this
Name and draws its sustenance from annihilation. If one looks with a
discerning eye, all goodness is observed in the grasp of \emph{Him Whom
God Shall Make Manifest,} and all shadows opposing Him are seen before
Him.

For example, today He is manifest in the Point of the Bayán, just as
previously He was manifest in the Point of the Furqán. No greater honor
exists for the Bayán and its believing souls than that nothing is seen
in them during the appearance of \emph{Him Whom God Shall Make Manifest}
except Him. In the prior manifestation, there was no doubt that He was
manifest in them, and nothing was seen in them except Him. Similarly,
today no one doubts the Point of the Furqán, and yet the Point of the
Bayán is the very manifestation of the Point of the Furqán, but in a
more exalted form.

Indeed, in all the mirrors of the Qur'anic revelation, it is He who is
manifest. However, His concealment is due to the loftiness of the
manifestations and the intensity of the light, which veils Him from
view. This is why today the manifestations of the Qur'anic revelation
draw near to God through Him and seek nothing other than His pleasure.
How, then, could they issue decrees of their own? Thus, those within the
Bayán cannot be saved except through the recognition of \emph{Him Whom
God Shall Make Manifest} in their own realities, not through the
recognition of one who has already been manifest. For He is ever
manifest, dwelling in the cradle of honor, independence, exaltation,
power, and sovereignty. All the attributes of His actions are reflected
in the Bayán. His most excellent names are the signs of His paradise,
while the names beneath them, veiled in fire, are the apparent
manifestations of His justice.

At such a time, it becomes evident that the Sovereign of the Bayán is
the manifestation of His name, and through mention of Him, all deeds are
carried out. Likewise, He is the manifestation of His majesty, wealth,
power, and invincibility, and of all praiseworthy attributes. This is
evident in the Point of the Bayán as well, now manifest on the mountain.
He observes that nothing exists but His own manifestation, and all
perform their actions through Him, whether they are Qur'anic
manifestations or those beneath them. However, since they have veiled
themselves from their Beloved, they are eternally consumed in the fire
of veiling and derive no joy from knowing their Beloved.

If a thing becomes pleasing within the realm of existence, it is He who
has been pleased. If a thing becomes despised, it is He who has been
despised, for nothing is seen within that thing but the sun of His will,
which grants it its existence. Without this, it would be pure
nonexistence. Indeed, even the reality of the word ``nonexistence'' does
not come into being except through His mention. Otherwise, it would not
even be recognized or mentioned. This is the meaning of the saying of
the Messenger of God: \emph{``O God, Show me the realities of things as
they truly are.''} This means not that in the essence of every thing you
observe the Point of the Bayán, but that its manifestation appears at
various levels: in the rank of earth as earth, in the rank of water as
water, in the rank of air as air, and in the rank of fire as fire. This
does not diminish anything from Him or cause any increase to Him. For if
infinite mirrors are placed before the sun---whether they be of ruby,
diamond, crystal, or glass, or of any material conceivable---their
reflective nature reflects the sun without anything being added to or
subtracted from the sun itself.

For example, consider the manifestation of the Messenger of God after
His declaration. His prophetic mission lasted twenty-three years. If He
had not declared His prophethood by God's permission, no soul would have
been guided. Yet nothing would have been diminished from Him, nor would
anything have been added to His reality. Even without the declaration of
His mission, after the appointed time of His earthly existence, He would
have ascended to His sanctified horizon, eternally radiant as He always
has been and always will be.

Thus, all are sustained by the bounty of the Point of Truth without even
a drop being diminished from the ocean of His grace or any increase
being added to it. Likewise, if the radiant traces of Him were to be
written infinitely in the most exalted forms beyond all conceivable
imagination, and if infinite souls were guided by Him, nothing would
diminish from Him or add to Him. He remains as He always has been. In
this way, God creates the radiance of the sun as itself in its station
of mention. Verily, He has power over all things.

\subsection*{Gate 9 (The Bayan Exists Within a Verse of 19 Names - For
Any
Soul)}\label{gate-9-the-bayan-exists-within-a-verse-of-19-names---for-any-soul}
\addcontentsline{toc}{subsection}{Gate 9 (The Bayan Exists Within a
Verse of 19 Names - For Any Soul)}

\textbf{The ninth gate of the third Unity}: That all things in the Bayán
exist within the verse of the Bayán.

The essence of this principle is the Bayán exists within a verse in
which nineteen names are mentioned, and under their shadow, nineteen
letters of negation are also mentioned. All things that pertain to the
human reality, if they signify affirmation, are included among His names
and most excellent attributes, even if they consist of a single particle
of dust. However, if they signify negation, they are mentioned among
those described in \emph{``that which they remain silent about.''}

Blessed is the one who believes in Him who pours forth these verses from
the ocean of His power, brings into being these likenesses through the
manifestation of the sun of His grandeur, and fulfills all that is
described with the name of negation, unyielding to the majesty of His
holy exaltation. This is the Point of the Bayán in that manifestation,
the primal will in every manifestation, and the very essence of
\emph{Him Whom God Shall Make Manifest} at the time of His appearance by
the permission of His Lord.

Watch for this, O people of insight, for we too have been watching. He
rises only for Himself and sets only for Himself, like the sun that
rises in the heavens of hearts. Those who face it with the nature of
mirrors reflect it without altering the essence of that radiance in its
exalted mention and its sanctified glory. To God belongs the command,
from before and after, and on that day, the believers shall be pleased.

\subsection*{Gate 10 (The Bayan Exists Within a Verse of 19 Names - For
Prophets)}\label{gate-10-the-bayan-exists-within-a-verse-of-19-names---for-prophets}
\addcontentsline{toc}{subsection}{Gate 10 (The Bayan Exists Within a
Verse of 19 Names - For Prophets)}

\textbf{The tenth gate of the third Unity}: That what is in this verse
is contained within the first verse, from \emph{``God bears witness''}
to \emph{``God, Almighty and Powerful.''}

The essence of this principle is that its details have been mentioned in
the gate preceding this one, but in a manner that any soul might
comprehend. This pertains to the mention of souls who, in the station of
prophethood, grasp these truths---not in the station of hearts, which is
the station of the manifestation of the names of God. For these
manifestations have no limit, boundaries, manifestations, hidden
aspects, elevations, nearness, risings, or settings. Everything bound in
its station, the soul, when clothed in the garment of limitation,
assumes boundaries; otherwise, in the station of hearts, nothing is seen
but God and His names. \emph{To Him belong creation and command, from
before and after; indeed, we are all His worshippers.}

If a soul today reflects upon the first manifestation of the Messenger
of God, which represented the determined station of the will in that
dispensation, they will realize that whatever ascribed existence to
itself---whether of truth or falsehood---drew its reality from the ocean
of His manifestation. Thus, all exist through Him. The intended meaning
of the first verse is His very essence in the Furqán, and the intended
meaning of the phrase \emph{``all things emanate from the B of
Bismillah''} is also Him.

Similarly, consider the Bayán: all that the believer in God or anything
beneath Him attains is realized through the manifestation of the Point
of the Bayán. The meaning of the verse \emph{``all is within Him''}
refers to Him, for He is the sign of the creation of this verse, and He
is the B of Bismillah in the realm of creation. This B signifies Him,
just as words and letters are realized through the point and multiplied
infinitely. Similarly, the spirits of all realities are created and
multiplied through Him.

When the mention of \emph{those endowed with hearts} arises, it refers
to those who are signs pointing to the phrase \emph{``There is no God
but God.''} When the mention of \emph{those endowed with souls} arises,
it refers to those who are signs pointing to the Messenger of God, peace
be upon Him. When the mention of \emph{those endowed with spirits}
arises, it refers to those who are signs pointing to the Imams of
Guidance, peace be upon them. When the mention of \emph{those endowed
with bodies} arises, it refers to those who are signs pointing to the
gates, peace be upon them.

The names and attributes are manifestations of the multiplicity of that
primal unity. Reflect upon the verbal letters of the Bayán: all
multiplicity originates from the first unity, even if it extends
infinitely. And in the multiplicity of the universal manifestations,
there arises a strength in the manifestation surpassing that of the
primal unity.

Yet, all things are realized through Him, and all return to Him, just as
they originate from Him. It is not that all the letters of the Bayán
become that single letter, but rather that each letter, within its own
limit, represents a will derived from Him. Similarly, each soul, within
its own station, points to Him. Reflect upon the return in the same way
you reflect upon the origin: if today a soul arises in the east of
origin, its origin is nothing but what it assumes of the garment of the
Bayán upon its heart, spirit, soul, and body. Similarly, if a soul
returns in the west of return and witnesses the appearance of \emph{Him
Whom God Shall Make Manifest,} it returns to Him in the garment of
recognition that it assumes, for the mirrors of origin are from Him and
return to Him.

Thus, all things exist within their own limits, originating from the
point, without returning to the essence of the point, even though they
arise from it. Indeed, consider all things as mirrors and the point as
the sun in the sky. If a white mirror faces it, it manifests the verse
of hearts. If yellow, the verse of spirits; if green, the verse of
souls; if red, the verse of bodies. If it is of colors beyond these, it
reflects according to what is inherent within it. Even if, God forbid, a
soul devoid of faith stands before the mirror, it reflects its own lower
colors and desires.

This is why the statement \emph{``I am your Lord, the Most High''}
arises in opposition to the statement \emph{``From God, the Almighty,
the Glorious.''} Thus, in every manifestation, pure truth becomes
indistinguishable from falsehood for those without insight, except for
those endowed with vision, who perceive the realities of things as they
truly are. These individuals are always observing the sun of truth, and
the manifestations and boundaries of the mirrors do not veil them from
the Manifest One.

They are the truly righteous in the Book of God, from before and after,
and they are the rightly guided. For instance, if someone observed
during this manifestation, they would have witnessed this type of
reflection in the second letter of negation. That letter, while
remembering the sun of the unseen, appeared in the station of God's
testimony, and that unseen sun acted upon it as it did. The pen is too
modest to describe it.

\emph{Reflect upon this, O possessors of understanding, all of you
together. And be certain, O you mirrored suns, all of you together.}

\subsection*{Gate 11 (Verbal Letters and Their Spirits Are Created
Through the
Point)}\label{gate-11-verbal-letters-and-their-spirits-are-created-through-the-point}
\addcontentsline{toc}{subsection}{Gate 11 (Verbal Letters and Their
Spirits Are Created Through the Point)}

\textbf{The eleventh gate of the second Unity}: That which is in the
verse is contained within the phrase \emph{``In the Name of God, the
Most Mighty, the Most Holy.''}

The essence of this principle is that all verbal letters are created
through the point, and their spirits are realized through the point of
truth. In the Furqán, that point is Muhammad, the Messenger of God, may
peace and blessings be upon Him and His family. In the Bayán, it is the
essence of the Seven Letters. In the manifestation of \emph{Him Whom God
Shall Make Manifest,} it is the divine reality, the heavenly essence,
the essential light of the divine, and the absolute purity of essence,
which is the sun of truth. Its radiance constitutes its signs, and
everything beneath it is merely shadows in mirrors.

This has been explained in prior gates and will appear in its
appropriate places. God is the protector of the righteous.

\subsection*{Gate 12 (The Sun and the
Mirrors)}\label{gate-12-the-sun-and-the-mirrors}
\addcontentsline{toc}{subsection}{Gate 12 (The Sun and the Mirrors)}

\textbf{The twelfth gate of the third Unity}: The likeness of the point
is like that of the sun, while the likeness of the other letters is like
that of mirrors facing it. Everything in the Basmala is contained within
the point.

Whoever says, \emph{``God, God is my Lord, and I ascribe no partner to
my Lord,''} has remembered God in accordance with what is decreed in the
point.

The essence of this principle is that the mention of the point signifies
the reality of the primal will. If it is mentioned in the station of
\emph{``In the Name of God, the Most Mighty, the Most Holy, from God,
the Most Mighty, the Most Holy,''} it is then the mention of the essence
of that reality.

The will becomes evident through the eye of perception, for if the
letter \emph{B} is separated, the first manifestation of it becomes the
letter \emph{'Ayn.} Thus, in every aspect, He is manifest upon and
within all things. This is the mystery behind the statement of Amīr
al-Mu'minīn, peace be upon Him: \emph{``The point is beneath the letter
B,''} referring to a correspondence that speaks of letters and numbers,
not of essence or intrinsic reality.

In the manifestation of the Point of Truth, which in the Qur'an appeared
as the name of the Messenger of God, its likeness is taken as the sun,
and those guided by it are outward suns reflected in mirrors. The fruit
of this principle is that knowledge is taken rather than action. The
station of action is this: today, as the same Point is manifested in the
Bayán, every soul that believes in Him must recognize nothing within
themselves as their honor except as a shadow seen in a mirror before the
sun in the sky.

For example, if one says \emph{``God is greater than my soul,''} or if a
believer in the Qur'an utters this phrase, before the true \emph{Allahu
Akbar} spoken by the Point of the Furqán in its conclusion, it resembles
the sun in the sky relative to its shadow in the mirrors. This truth,
however, was not fully apparent in this cycle, except as knowledge for
some souls, not as action. It is like the supreme elevation of what
ennobles a servant, a branch extending from its root, but not fully
realized.

In the manifestation of \emph{Him Whom God Shall Make Manifest,} the
believers in the Bayán must act upon this principle, which is the
essence of both knowledge and action. They must perceive no existence
for themselves except through Him. Just as the reality of their hearts,
spirits, souls, and intrinsic bodies has been realized through the
manifestation of the Point of the Bayán, He has placed these verses
within the selves of creation so that they may turn toward Him. Thus,
all knowledge and faith lies in this: if a person observes and draws
conclusions from the fruits of their own existence and, at the
appearance of every divine will, regards themselves as utterly dead---as
all souls were in the previous manifestation---then they achieve true
recognition. For instance, those who today believe in the Messenger of
God do not consider themselves significant in His presence. Instead,
they find honor and pride in their faith in Him, firmly established in
the pleasure of God. However, they remain veiled from realizing that the
manifestation of the Point of the Bayán is the very manifestation of the
ultimate appearance of the Messenger, but in a more exalted form.

If someone possesses both this knowledge and the actions aligned with
it, they will never remain veiled in any manifestation. In every
manifestation, they will attain the presence of their Beloved and
partake of the fruits of that manifestation. \emph{Blessed is the one
whom God has taught this knowledge and enabled to act upon it.}

Had this knowledge been widespread among people, the station of Amīr
al-Mu'minīn, peace be upon Him, would have been seen as equal in favor
to that of the Messenger of God, even though the sun of his virtue
reveals nothing except that he is the sun of truth. As someone might
say, this statement points not to the ocean of the essence of eternity
but rather to the ocean of the divine will. It signifies the idea:
\emph{``The ocean is as it was in pre-eternity; the events are its waves
and forms.''} These are exactly like the reflections in mirrors that
point to the sun, for in the realm of existence, which is the station of
mirrors, nothing else is possible.

The first mirror to reflect the sun of truth throughout all worlds has
been Amīr al-Mu'minīn, peace be upon Him. In every manifestation, He has
appeared under a different name until it reaches this manifestation.
Indeed, He is the mirror of all manifestations. \emph{Blessed is the one
who has sought shelter under His shade,} for such individuals are the
companions of destiny, the signs of the All-Powerful, the companions of
the Day of Origin. They ascend to a horizon that none among the worlds
can surpass, save for the one God, when He decrees a thing, creates
whatever He wills and even higher than what He wills. Verily, He has
power over all things.

The fruit of this knowledge is that in the manifestation of \emph{Him
Whom God Shall Make Manifest}, if all on earth bear witness to Him and
He bears witness without their bearing witness, His testimony is like
the sun, while their testimony is like the shadow of the sun, which does
not align unless placed in direct correspondence. By the sacred essence
of God, one line of His words is better than the words of all on earth.
Indeed, I seek forgiveness for even mentioning this preference, for how
could the reflections of suns in mirrors compare to the effects of the
sun in the sky? The former lies in the realm of non-existence, while the
latter is in the station of the reality of existence through God, the
Almighty and Glorious.

If someone receives the reward of a single \emph{Lá iláha illa'lláh}
from Him, it is better than the reward of all things having acknowledged
God's oneness. Yet, I repent for mentioning this comparison, for it is
beyond preference---it is as I have stated before. Whatever He testifies
to is the testimony of God regarding that thing, and whatever He speaks
concerning it is the speech of God about it.

If, during His manifestation, a king exists and declares his
sovereignty, he is like a mirror claiming in the presence of the sun
that light resides within him. Similarly, if a scholar boasts of his
knowledge before Him, it is the same. If a wealthy person proclaims his
wealth, it is likewise. If a powerful one displays his power, it is the
same. If someone claims honor, it is the same. Even those of the same
kind as him would laugh at such claims, knowing that the sun of truth.

For example, countless kings have existed on the earth, both during the
manifestation of the Bayán and in manifestations preceding the Bayán.
Yet all kings, along with their manifestations, ultimately terminate in
the point of the divine will, regardless of which manifestation it may
occur in. The same is true for those of knowledge, wealth, power, and
honor. Reflect upon this, gain certainty, and act accordingly concerning
all names and attributes---and indeed regarding every thing to which the
concept of ``thingness'' is applied.

Do not display any claim of existence before Him, for such claims arise
from nonexistence. Observe the condition of those who sought to affirm
the Messenger of God by affirming the monks of the Book of the ``A.''
Similarly, consider the Bayán: there were those who sought to affirm the
Point of the Bayán through the affirmation of individuals who, for 1,270
years, had been veiled in various degrees by the first manifestation of
His in the Qur'an.

This happens even in the time of acceptance---so how could the pen turn
to mention the condition of those in a state of rejection? These
individuals attempt to bring faith in the one who resides in the fire by
affirming the reality of paradise and hell. Through their affirmations,
they seek to validate the Word of God, and through their testimony, they
try to confirm the truth of the Realized Truth, whose testimony is the
very testimony of God regarding all things.

The one to whom all believe through faith in Him is the true believer,
while disbelief in Him leads to judgment beneath the station of faith.
If someone acts on a ruling other than that which pertains to the name
``Believer,'' which is one of His names, what limit can be placed upon
them? It is a waste to even mention the letter \emph{``H''} in
describing them, for purity can be decreed in their case, but by their
own ruling, it cannot be applied to Him.

\emph{``By the One who split the seed and created the soul, who stands
alone in might, sanctified in greatness, and unique in oneness''}

By His majesty and might, no fire is greater than the veiling of people
from their Beloved, to whom they turn night and day. And no paradise is
greater than beholding the sun of truth and seeing everything else as
mere shadows in mirrors, acting only in accordance with His will. If a
soul that has issued judgments in the name of God were to uncover the
reality of its actions, it would fully grasp the torment of being veiled
within itself. Such torment would be inevitable for it, even in the
darkest of nights. And God guides whomever He wills to the path of true
certainty.

God does not love to mention those who do not believe in Him and His
signs, but God loves the righteous.

\subsection*{Gate 13 (Impermissible to Ask About Him Whom God Shall Make
Manifest)}\label{gate-13-impermissible-to-ask-about-him-whom-god-shall-make-manifest}
\addcontentsline{toc}{subsection}{Gate 13 (Impermissible to Ask About
Him Whom God Shall Make Manifest)}

\textbf{The thirteenth gate of the third Unity}: It is impermissible to
ask about \emph{Him Whom God Shall Make Manifest} except in accordance
with the Book, and to act in accordance with it is better for those who
have spoken of faith with their tongues. This may apply to some in
relation to others. God is a witness over all things.

The essence of this principle is that questioning about \emph{Him Whom
God Shall Make Manifest} is not permissible except concerning matters
befitting His station. His station is the pure manifestation of divine
appearance, even the very essence of appearance itself. The essences of
all manifestations are revealed under His shadow. If there is any virtue
in existence, it is a reflection of His grace. If there is any thing, it
is by His reality.

Whoever seeks to observe His knowledge should look to the scholars of
the Bayán, who have knowledge of His Book from the prior manifestation.
Whoever desires to witness His power should look to the possessors of
power in the Bayán, who have been empowered by the manifestation of His
prior might. Similarly, the possessors of honor reflect His prior glory,
the possessors of greatness reflect His prior grandeur, and the
possessors of strength reflect His prior strength.

From beginning to end, the Bayán is the repository of all His attributes
and the treasury of fire. The light of His essence and the spirits
associated with it on the earth are signs pointing to His words from His
prior manifestation. All were created for the manifestation that
follows. For example, His name \emph{``the Generous''} in the Bayán was
created for His name \emph{``the Generous''} in His current
manifestation, just as the name \emph{``the Bestower''} in the Qur'an
was created for the name \emph{``the Bestower''} in the Bayán. This
pattern applies to all names and attributes of the Truth, both in the
realm of Truth and beneath it.

Whoever comes to know Him becomes as if lifeless before His will, and no
proof is greater for them or for their own soul than His very presence.
This is true for all, as all verses and words arise from the ocean of
His grace, are illuminated by the vastness of His generosity, and are
enlightened by the majesty of His glory.

If one seeks to question Him, it is impermissible except within the
context of the Book, so that the line of His response may be understood
as it is---becoming a verse from the Beloved to the questioner. For
instance, in the Bayán, whoever embodies this praiseworthy attribute of
refraining from inappropriate questioning is beloved in both His hidden
and manifest stations. However, one must not inquire about matters
unbefitting His station. For example, if someone asks a jeweler about
the value of straw, such a person is deeply veiled and rejected.

The same applies to the highest heights of creation in relation to Him,
except for what He describes of Himself on the Day of His Manifestation.
Imagine a scenario: someone questions Him in His Book about matters
revealed in the Bayán concerning certain boundaries that they have
established for themselves. In response, He reveals not from His own
self but from God:

\emph{``Indeed, I am God, there is no God but Me. I have created all
things, sent the messengers before, and revealed the scriptures upon
them so that you may worship none but God, your Lord and the Lord of all
things. Indeed, that is the true certainty. It is the same to Me whether
you believe in Me, for you only prepare the way for your own selves. If
you do not believe in Me or in what God has revealed to Me, you only
veil yourselves. I have been independent of you from before, and I will
remain independent of you hereafter. So, help your own souls, O creation
of God, and believe in My verses. For whoever does not believe in Me or
in what God has revealed to Me is, in the sight of God, like one who did
not believe in the essence of the Seven Letters and the Bayán, though
they believed before in Muhammad, the Messenger of God, and the Furqán.}

How will you act on that day when you would refuse to associate
yourselves with those souls? Likewise, if you truly understand, you will
enter into the religion of God and refuse to remain in the Bayán after
what God has revealed to Me of His verses. Verily, He has power over all
things. I am the Point of the Bayán, and before this, God manifested Me
as He had manifested Me previously. Blessed are those who have
recognized Me and believed in the Bayán before. Then hasten in the Cause
of God and fear Him alone. See nothing in Me but God, your Lord and the
Lord of all things---the Lord of the heavens and the earth, the Lord of
all that is seen and unseen, the Lord of the worlds.

What you possess is like the reflection of the sun in a mirror, and thus
is your understanding from your books compared to what We have revealed
to you previously in the Bayán. So, O servants of God, be mindful of
Him. Let your inquiries to your Beloved be brief, concerning only the
loftiness of His oneness, the exaltation of His sanctity, the elevation
of His glorification, and the transcendence of His greatness---in
knowledge, in word, in deed, both outwardly and inwardly. For He loves
hearts that point solely to God and are centered on His love, and
spirits, souls, and bodies that direct solely toward Him.

The letters of His ``Living'' essence are the same as the Living Letters
of the Bayán, which are identical to the Living Letters of the Furqán,
and likewise to those of the Book of A to Z, continuing until the Book
of Adam. From the appearance of Adam to the first manifestation and the
Point of the Bayán, only 12,210 years of this world's time have passed.
Beyond this, there is no doubt that God has created countless Adams,
whose number is known to none but Him. In every world, the manifestation
of the divine will has been none other than the Point of the Bayán, the
essence of the Seven Letters. The Living Letters are exclusively the
Living Letters of the Bayán, His names are only the names of the Bayán,
and His likenesses are only the likenesses of the Bayán.

He is universally recognized as a prophet with a book attributed to God,
yet all remain veiled from His true knowledge and unaware of His book,
except for the believers in the Bayán. In the same way, observe the
manifestation of \emph{Him Whom God Shall Make Manifest,} for He is none
other than the primal will in all worlds. His book is the book of that
primal will in all worlds. He has been and will always be a sign
pointing to God. His book has been and will always be the voice of God.
His names have been and will always be manifestations of the name of
God, exalted and glorified. His likenesses have been and will always
dwell under the shadow of God, exalted and glorified.

\emph{To Him belong creation and command, from before and after. There
is no God but Him. Indeed, we are all devoted to Him.}

The very Point of the Bayán is the same as the Adam of the primal
creation, the first of its kind, and the same as the seal that is in His
hand, which has been preserved by God from that time until now. The sign
engraved upon it is the very same sign that was engraved upon it before.
This explanation is offered due to the weakness of human understanding;
otherwise, Adam in the station of the Point is the same as this Adam.

Consider a youth who has reached the age of twelve. He would not say,
\emph{``I am from the drop that descended from a certain heaven and was
established in a certain earth.''} If he did, it would show a
regression, and those endowed with knowledge would not deem him fully
mature in intellect. Similarly, the Point of the Bayán does not say
today, \emph{``I am the manifestation of the will from Adam until
today,''} for such a statement would reflect the same kind of
regression. This is why the Messenger of God did not say, \emph{``I am
Jesus,''} for that would imply that Jesus had ascended from his own
station to reach that level.

In the same way, \emph{Him Whom God Shall Make Manifest}---when He is
mentioned in the station of being beloved at fourteen years---would not
say, \emph{``I was that twelve-year-old,''} for such a statement would
account for the weakness of human understanding. All things progress
toward loftiness, not regression, even though the fourteen-year-old at
one time was the twelve-year-old, just as he had previously been in the
form of Adam's seed, gradually advancing until he became twelve. From
that age, he progresses further until he reaches fourteen.

If today one of the believers in the Qur'an considers it acceptable to
say, \emph{``I am among the believers in the Gospel,''} then the Point
of Truth also considers it acceptable to speak in this way. Similarly,
this principle applies in the Bayán, and the Bayán, in relation to
\emph{Him Whom God Shall Make Manifest,} follows the same progression
infinitely. The manifestation of God appears in each dispensation as He
wills, but each subsequent manifestation is the maturity of the
preceding one. Therefore, at the point of maturity, it would not befit
Him to refer to His prior station as His own, for the station of
maturity encompasses all that belongs to its previous station, while
also surpassing it.

For example, the letter \emph{Ghayn} exists, but not as \emph{Ṣad,} even
though \emph{Ṣad} contains what \emph{Ghayn} possesses and more.
However, \emph{Ṣad} does not encompass the thousand \emph{Ghayns}
contained in \emph{Ẓā.} This pattern applies in the numerical and
creative orders as well.

At the beginning of each manifestation, those who enter that
manifestation through God, the Mighty and Glorious, and recognize the
previous manifestation alongside what it bestowed, are those who
progress. Otherwise, they remain veiled from the new bounties of God and
are consumed within what was. For instance, the will that manifested
from Adam to the Point of the Bayán retains the thrones and the
followers associated with its prior manifestations, but the former do
not possess what this new manifestation holds. Even though the throne is
of His essence and the thrones signify the signs pointing to Him, they
do not encompass what belongs to this appearance.

Every soul who was sincere in faith---or indeed every thing that existed
in the \emph{'Illiyyīn} of the previous cycle---has inevitably entered
the paradise of the new manifestation. Today, all have entered the Bayán
and continue to do so. However, those who remain veiled are not judged
to be within paradise; instead, the decree of fire applies to them. This
is because paradise in every age is the perfection of that age. Today,
perfection lies in the Bayán, not beneath it. In the manifestation of
\emph{Him Whom God Shall Make Manifest,} perfection will be found in His
Book, and so it will ascend infinitely.

Each subsequent manifestation incorporates the previous one along with
its own perfections. In no manifestation is there a will other than that
which strives to bring all into that manifestation, saving them from the
fire and admitting them into paradise. The degree of this depends on the
extent to which the means for it have been provided and carried out.
Otherwise, matters remain until it becomes clear how the brave ones of
the Bayán and its forerunners act---whether they bring everyone into
paradise with their effort or, according to what resources are
available, advance the Cause of God as much as possible.

\emph{By the sacred essence of God, the Mighty and Glorious,} if the
possessors of authority and knowledge in the time of \emph{Him Whom God
Shall Make Manifest} were united in faith in Him, He would not be
content to leave even one soul of the Bayán upon the earth. How, then,
could one who is not of the Bayán presume to exert effort or claim to
assist the Absolute Truth until no thing remains except that it has
entered paradise? This is the greatest bounty and supreme success in
every manifestation. When all on earth come under the shadow of that
manifestation and owe their allegiance to its Cause, only then is the
essence of the divine will satisfied. Otherwise, it remains eternally
supplicating for the grace of God until that state is realized. And it
is inevitable that it will be realized, for God is and always has been
powerful over all things. In every cycle, according to what benefits the
entirety of creation, God will bring about the necessary means.
Ultimately, all who dwell upon the earth will be granted a portion of
the supreme paradise. \emph{Indeed, He is the All-Knowing, the
All-Powerful.}

\subsection*{Gate 14 (Preserve the Bayan With Utmost
Care)}\label{gate-14-preserve-the-bayan-with-utmost-care}
\addcontentsline{toc}{subsection}{Gate 14 (Preserve the Bayan With
Utmost Care)}

\textbf{The fourteenth gate of the third Unity}: On preserving the Bayán
with the utmost care by every soul.

The essence of this principle is that what remains among the people from
the Tree of Truth are His words and the spirits associated with them.
The extent to which people preserve these words, honor them, elevate
them, and safeguard them reflects in their own spirits. It is not
permissible for the tablets to remain hidden; they must be collected,
bound, and preserved in the most honorable manner possible by every
soul, ensuring they do not become like the Qur'an, with scattered
fragments found in corners of mosques and inappropriately treated.

No soul invests even a mustard seed's worth of effort in the Bayán
without God guaranteeing to grant them two thousand times its reward. If
such recompense does not reach them in this world, they will certainly
attain it in the next, and no hand will sever the connection between
them and God. \emph{Blessed is the one who preserves the words of God
with the utmost care and embellishes them as best as they can,} for the
honor and adornment of their spirits lie in this.

It is not that preserving the Bayán requires gold worth a thousand
measures, but that a soul in whom the spirit of the Bayán resides,
through their faith. Even a single measure of the Bayán becomes a
necessity for one who seeks its worth. All exist in their respective
stations, apparent to those who are perceptive. No servant who possesses
the entirety of the Bayán will remain unblessed; instead, their good
deeds will be multiplied, and they will be nourished with uncountable
provisions. The angels, who are the stewards of all things, will shower
blessings upon them, seeking mercy and forgiveness for them.

The more effort expended in elevating its craftsmanship, lightening its
weight, magnifying its calligraphy, and embellishing its tablets, the
more beloved it becomes to God compared to lesser acts. It is unworthy
to inscribe in its margins the trivial notes of students that diminish
its majesty. The essence of this principle is that each person, in their
station, should preserve the Bayán in a manner reflecting its
unparalleled nature, even though there exist infinite levels above and
below them.

Permission has not been granted to write it with anything other than
excellent script, for its recitation, observation, and contemplation
bring divine pleasure. These actions must ultimately lead to the
application of what God has revealed within it. For there is no word
that does not draw its spirit from its recitation. Whoever gazes upon
the Bayán with reverence and sends blessings upon it with the prayer,\\
\emph{``O God, bless the Bayán and those who have believed in it in
every condition, with might and majesty. O God, chastise those who have
not believed in it with Your power and justice,''} fulfills the rights
of its words.

\emph{This is the grace of God upon His servants, for He is bountiful
and self-sufficient.} All the majesty of the Bayán reflects \emph{Him
Whom God Shall Make Manifest.} Mercy is for those who believe in Him,
and punishment is for those who do not believe in Him.

\subsection*{Gate 15 (Belief in Him Whom God Shall Make Manifest is
Belief in
God)}\label{gate-15-belief-in-him-whom-god-shall-make-manifest-is-belief-in-god}
\addcontentsline{toc}{subsection}{Gate 15 (Belief in Him Whom God Shall
Make Manifest is Belief in God)}

\textbf{The fifteenth gate of the third Unity}: Whoever believes in
\emph{Him Whom God Shall Make Manifest} is as though they have believed
in God. Whoever does not believe in \emph{Him Whom God Shall Make
Manifest}, even if they profess belief in God and in what they believed
before, it is as if they have never believed. They will be cast into the
fire.

The essence of this principle is that the manifestation of God in every
cycle---embodying the primal will---is and has always been the radiance
of God (\emph{Bahá'u'lláh}). All things, in relation to His radiance,
are nothing. Every soul that believes in His subsequent manifestation
has, in effect, believed in all His manifestations, both past and
future, within that cycle. The future manifestation is encompassed
within the current one, as the current is a reflection of the former.

For instance, the reality of the soul at fourteen years, before it
reaches nineteen, contains its nineteen-year maturity within the
fourteen-year self. Thus, whoever believes in \emph{Him Whom God Shall
Make Manifest} and in what He commands, believes in God from the eternal
beginning---without beginning---and attains the pleasure of God in every
manifestation. But whoever does not believe in Him, even if they claim
faith in all worlds and enjoy God's favor, their deeds become as
scattered dust, as if they never believed in God even for a moment.

This truth is evident in the Point of the Bayán to those endowed with
insight, and in the Furqán, it is even clearer to all. Whoever believes
in Muhammad, peace be upon Him and His family, has certainly believed in
God and His commands in all worlds. Whoever does not believe in Him has
not believed in God or His commands in any world.

Thus, those without faith are judged beneath the station of the
believers in the Qur'an of their time. \emph{Be certain of this, O
people of knowledge, all of you together.} It is possible for a soul to
be a believer in one manifestation but cast into the fire in the next.
Conversely, a soul in the fire in a previous manifestation may become a
believer in the subsequent one.

In the subsequent manifestation, one may find themselves in paradise,
for the manifestation of God has neither a beginning nor an end that can
be defined. If someone is not a believer across a thousand thousand
manifestations but becomes a believer in the one following, all their
previous worlds are transformed into faith. Conversely, if, God forbid,
the reverse occurs, all their faith is reduced to nothing. This is
because, in each manifestation, whatever constitutes God's pleasure
resides solely with that manifestation---whether it is before or after.

For example, during the appearance of the Messenger of God, whatever was
the pleasure of God, from the beginning of creation up to the advent of
the Qa'im, peace be upon Him, was contained within the pleasure of that
day's manifestation. Similarly, the pleasure of \emph{Him Whom God Shall
Make Manifest} today is the Bayán, up until His own manifestation. At
that point, His pleasure is renewed in His appearance, eternally as it
has been and will be. This is the station of divine will before God and
will remain so.

No manifestation occurs without a covenant being taken from the
believers in that appearance to affirm faith in the next manifestation.
If they fulfill this covenant, no one remains in the fire. For instance,
had the adherents of the Book of the ``A'' fulfilled their covenant with
Jesus, no one would have remained in the fire during the manifestation
of the Messenger of God. Likewise, in the Furqán, had all accepted the
divine radiance of that manifestation, no one would have been judged
beneath the station of faith or left in the fire.

Similarly, if all the believers in the Bayán were to believe in
\emph{Him Whom God Shall Make Manifest,} no one would remain in the
fire, and no one would be judged below the station of faith. However,
vigilance toward the next manifestation is necessary, ensuring that no
delay occurs between its appearance and universal belief among the
faithful of the Bayán. The interval of time---even if only to the
station of the Mustagháth---is unworthy if precaution does not protect
them from delay.

This caution lies within the fire, for it always has and always will.
Yet, there is hope in the grace of God, who is compassionate and
merciful, that at the time of the manifestation, through His exalted
commands and sacred tablets, He will awaken all His servants from their
slumber and not allow them to remain in the fire under the decree of the
Bayán until the station of \emph{Mustagháth} is reached. Who but God
knows the timing of the manifestation? Whenever it occurs, all must
affirm the Point of Truth and give thanks to God, even though there is
hope in His grace that before \emph{Mustagháth}, the Word of God will be
exalted and the divine proof will manifest.

\emph{The proof lies in His verses, and the evidence of His being is
Himself. He is recognized through Himself and not through anything other
than Him. Exalted is God above what they describe.}

\subsection*{Gate 16 (Impermissible to Act Except Through the Traces of
the
Point)}\label{gate-16-impermissible-to-act-except-through-the-traces-of-the-point}
\addcontentsline{toc}{subsection}{Gate 16 (Impermissible to Act Except
Through the Traces of the Point)}

\textbf{The sixteenth gate of the third Unity}: It is impermissible to
act except through the traces of the Point.

The essence of this principle is that no action is permissible except
through the traces of the Point of the Bayán. In this manifestation, for
the Living Letters, their traces appear through the sun of truth, for
the verses are specific to the Point, the supplications are specific to
the Messenger of God, the commentaries are specific to the Imams of
Guidance, and the scholarly forms are specific to the Gates. Yet, all
these arise from the same ocean to ensure that all these traces are
observed in their primal reality in the most exalted form.

There is no honor greater for them than the precedence of faith, which
is the most glorious of all things before God and the people of
knowledge. All grace dwells under its shadow. From the time of sunset
until the rising of \emph{Him Whom God Shall Make Manifest}, the
obligatory traces are elevated, and the Living Letters and all who
believe in God and the Bayán remain under their shadow. Except for one
who soars in the knowledge of a decree of God, whether concerning
principles or branches, if one manifests any effect through their
actions without exceeding the bounds of the Bayán, they will remain
under its shadow. Otherwise, they are unworthy of mention before God and
the people of knowledge. During this cycle, if one reflects, they will
perceive the clarity and ease in the statements, arguments, and delights
of what has been revealed from the Point of the Sun of Truth. For His
traces are akin to the radiance of the sun compared to the light of
stars. Is it possible for anyone to equate the two? \emph{Exalted is God
above such a comparison, infinitely glorified and greatly exalted.}

However, one should strive to understand the knowledge of letters, the
correspondences of numbers, the names of God, and the connections
between similar words and effects. In their rightful place, where
permission has been granted, the order of the Bayán should be presented
in whatever form makes it most pleasing. Even if it manifests in a
thousand variations, all ultimately return to the essence of the Bayán.
For no letter is added to it, nor is any letter removed from it, except
that the arrangement, preservation, and correspondence between parts may
differ slightly in one version compared to another.

It appears evident that no new arrangement is given except that it
becomes sweeter and more harmonious than the previous one. \emph{Blessed
is the one who reflects upon the arrangement of the Bayán and gives
thanks to their Lord,} for He reveals signs and purposes from God within
the Bayán, until God raises what He wills and brings forth what He
desires. Verily, He is mighty and powerful.

The best arrangements are those that adhere to outward boundaries. For
example, if ten prayers, each a hundred verses, are listed together,
they should not intermix or confuse the arrangement. The five streams
should not be blended. Verses should remain in their loftiness,
supplications in their elevation, commentaries in their seat of majesty,
and the sublime words in their celestial heights.

The loftiness of its sanctity and the elevation of its Persian words
reflect an unparalleled majesty, as no subtle observer could overlook
the essential connections therein. \emph{This is a grace of God upon the
scholars of the Bayán,} who rise to what they are commanded until God
relieves them through the manifestation of that which they turn to day
and night, seeking their Lord.

If none appears to clarify or organize the Bayán with its structure, it
remains like the sun among the stars, as its clarity outshines all else.
This occurs when all are veiled, much like the Point of the Bayán
writing three commentaries on the Qur'an: two following the structure of
its verses to completion and one on the Surah of the Cow, structured in
scientific discourse. Even a single letter of His commentaries outweighs
the collective works of all commentators from the Qur'an's revelation to
its ascension.

\emph{Who compares the radiance of the sun to the light of the stars?
Such a comparison holds no validity before God. Will you not reflect?}

\subsection*{Gate 17 (Impermissible To Transcribe the Bayan Except With
Calligraphy)}\label{gate-17-impermissible-to-transcribe-the-bayan-except-with-calligraphy}
\addcontentsline{toc}{subsection}{Gate 17 (Impermissible To Transcribe
the Bayan Except With Calligraphy)}

\textbf{The seventeenth gate of the third Unity}: It is impermissible to
transcribe the traces of the Point except in the finest calligraphy. If
anyone possesses even one letter written without excellence, their work
is rendered void, and they are not among the believers.

The essence of this principle is that all traces of the Point are
referred to as the Bayán. However, this name, in its primal truth,
applies exclusively to the verses. It then applies in a secondary sense
to supplications, in a tertiary sense to commentaries, in a fourth sense
to scientific forms, and in a fifth sense to Persian words. However, the
true designation of this name belongs solely to the verses and none
other, as befitting their rank. Permission has been given that the
designation of ``Servant of the Bayán'' may be passed among souls, for
the name \emph{Bayán} is derived from the essence of God's name. The
first to name Himself with it was God, exalted and glorified, as He
revealed: \emph{``Indeed, I am God, there is no God but Me, the One, the
Bayán.''} All the mysteries of the Bayán are manifest in this name, for
the numerical value of \emph{Bayán} aligns with \emph{One}, forming the
comprehensive number of God's name.

Thus, it becomes a mirror reflecting the Point of the Bayán, which is
itself a mirror for God, and for \emph{Him Whom God Shall Make
Manifest}, who is the mirror of God. Whatever in the Bayán belongs to
God has been and will remain for God. In its comprehensive expression,
it reflects Him; in the mention of each letter of the One, it reaches
the numerical completeness of that exalted word. For they are, and
always will be, for God. However, all ultimately return to the First
Gate.

As is evident in the Point of the Bayán, what lies within the mirrors
cannot equate to the sun in the sky. \emph{``There is no God but God,
the Lord of the heavens, the Lord of the earth, the Lord of all things,
the Lord of what is seen and unseen, the Lord of the worlds. Truly, your
God is one God; there is no God but He, the Most Merciful, the Exalted,
the Inaccessible.''}

It is not permissible for anyone to write a single letter of the Bayán
except in the finest script. Excellence is defined by what is most
suitable for each individual in their capacity---not beyond it, nor
below it. This ensures that the spirit associated with that letter in
the Bayán is elevated to the utmost potential within its realm. Among
the believers in the Bayán, nothing is seen unless it has reached
perfection within its station.

Consider how today the \emph{Letters of Alif} are distinct in their
excellence among nations. In the same way, those in the Bayán must excel
such that if one from the Bayán were in the East on the earth, by virtue
of their refinement and the excellence of what they possess in their
station, they would become beloved. This is the greatest path for
attracting all religions to the true faith of God, the Merciful. Yet,
this depends on one's capability; it should not bring hardship upon
oneself in any matter. God does not desire to see a believer in sorrow;
rather, all duties are prescribed according to one's ability.

It seems as though, even now in this mountain, the adorned Bayán is
cherished by the people of the Bayán, who find joy in its recitation.
Through this, they prepare themselves to present their copies of the
Bayán before \emph{Him Whom God Shall Make Manifest}, which is the
ultimate act of presenting oneself to God. Such presentation has always
been and will always be a sign of devotion. They conceal any display of
their own existence before Him, for anything less would be contrary to
humility and grace. \emph{Fear God, O people of the Bayán, all of you
together.}

\subsection*{Gate 18 (Interpretations Must Be in The Finest
Script)}\label{gate-18-interpretations-must-be-in-the-finest-script}
\addcontentsline{toc}{subsection}{Gate 18 (Interpretations Must Be in
The Finest Script)}

\textbf{The eighteenth gate of the third Unity}: Whoever desires to
interpret anything from the traces of the Point or to compose a work for
the pleasure of God must not allow their copy to be presented to anyone
unless they have transcribed it themselves in the finest script, or at
least in a script beneath their capacity. Only then is it permissible;
otherwise, it is not.

The essence of this principle is that God loves in the Bayán that
whatever arises from any matter---be it derived from the station of the
Point or reaching the utmost of its perfection---attains the highest
refinement in both form and essence. Thus, no soul experiences even a
mustard seed's worth of aversion or bears what God does not love. Among
these decrees is the ruling that anyone who writes a commentary on the
words of the Bayán or composes a book in any science must transcribe the
original version themselves with the finest script they can muster. If
they cannot, they must give it to someone else capable of doing so, then
preserve it with themselves before offering a copy to another. The
traces of each soul should remain in the possession of that soul before
being shared. If this standard is not met, their actions may be rendered
void, except in cases where it is not feasible, or where a better copy
is produced for the individual. In such cases, permission is granted to
provide the work to another before transcribing it oneself, so that the
improved version may be completed and returned. This is permissible in
situations where capability is lacking; otherwise, completing it oneself
remains the preferred course in the sight of God.

This stands in contrast to the practice of the scholars of this age, who
preserve incomplete manuscripts with numerous contradictions on each
page, merely because they are considered original versions. Such
carelessness resembles the ``Book of Sorrows'' now found in this
mountain, attributed to its compiler. If a book contains even a single
misplaced mark in an unsecure place, it is unworthy of preservation and
unsuitable for the boundless ocean of God's grace. Let your works, O
people of the Bayán, reflect the utmost refinement according to what is
within your capacity.

Say: \emph{``God is more gracious than all that is gracious. None can
escape the sovereignty of His gracious dominion---not in the heavens,
nor on the earth, nor in what lies between them. Verily, He is supremely
gracious and infinitely merciful.''}

\subsection*{Gate 19 (Give for the Glorification of the
Bayan)}\label{gate-19-give-for-the-glorification-of-the-bayan}
\addcontentsline{toc}{subsection}{Gate 19 (Give for the Glorification of
the Bayan)}

\textbf{The nineteenth gate of the third Unity}: Permission is granted
for those who wish to spend their possessions on the traces of the
Point, in any manner they desire, as long as it elevates them through
the love of God.

The essence of this principle is that God, out of His boundless grace,
has permitted each person to spend as much as they are able for the
glorification of the Bayán. If it were possible to dedicate all that is
on earth to the exaltation of the Bayán, such permission would have been
granted to Him, such permission would have been granted. The fruit of
this principle is that, in the time of \emph{Him Whom God Shall Make
Manifest}, one should reflect: if His traces are given such rulings, how
might His very self be treated? Let not what occurs today repeat itself,
where countless copies of the Qur'an from the era of the ``A'' remain
widely disseminated within Islam, yet His abode lies in a mountain,
where the chamber of His rest consists of mere clay bricks.

However, wherever His dwelling is, it is upon the Throne of God, whether
it is upon the seat of majesty or over the soil of the earth. This
statement serves as a reminder for the believers in the Bayán, that they
should not act toward Him as those veiled from the Qur'an acted.
Nonetheless, before Him, \emph{light and darkness are equal; both
glorify His praise in the evening and the morning.}

\section*{Vahid 4}\label{vahid-4}
\addcontentsline{toc}{section}{Vahid 4}

\markright{Vahid 4}

\subsection*{Gate 1 (The Point Has Two
Stations)}\label{gate-1-the-point-has-two-stations}
\addcontentsline{toc}{subsection}{Gate 1 (The Point Has Two Stations)}

\textbf{The first gate of the fourth Unity}: The Point has two stations:
one in which it speaks on behalf of God and one in which it speaks on
behalf of what is below God. The latter is the station of servitude
through which God is worshipped day and night, and His praise is sung
morning and evening.

The essence of this principle is that God has created two stations for
the Sun of Truth. One is the hidden station of His essence, which
manifests divinity. Whatever verses are revealed, they are from Him. He
cannot be described by any description or defined by any attribute, for
He is exalted above all mention and praise and sanctified beyond any
essence or substance. None but Him can know Him, and He is not found
except by Himself. \emph{To Him belong creation and command. There is no
God but Him, the One, the Majestic, the Most Exalted.}

This is the station of the mighty verse: \emph{``Say, God is Truth.
Whatever is other than God is creation, and all worship Him.''} Beneath
this verse lies His creation, and within it, nothing is seen except God.
Whatever is below, He is the Creator, and this is a verse in which no
sign of being a sign can be discerned; rather, it is the very
Manifestation of God, the Essence of the Mystery of God, the Loftiness
of the Loftiness of God, the Sublimity of the Sublimity of God, the
Eternity of the Eternal, the Essence of the Everlasting, and the Pure,
Absolute, Eternal Countenance.

A sign exists for the sake of recognition; otherwise, the sign itself is
not perceived. For if the sign is regarded, it becomes creation, and He
is not mentioned in Himself except by that through which the Essence of
the Eternal is made manifest. For Him, there are neither places nor
boundaries. His nearness is identical to His distance, and His distance
is identical to His nearness. His Firstness is identical to His
Lastness, and His Lastness is identical to His Firstness. His
Manifestness is identical to His Hiddenness, and His Hiddenness is
identical to His Manifestness. His Loftiness is identical to His
Proximity, and His Proximity is identical to His Loftiness. His coolness
is identical to His warmth, and His warmth is identical to His coolness.
His Essence is identical to His Being, and His Being is identical to His
Essence.

God has ever been a God, and there was none worshiped there; He has ever
been a Lord, and there was none to be lorded over there; He has ever
been Beloved, and there was none to love Him there; He has ever been
Worshipped, and there was none to worship Him there; He has ever been
Sought, and there was none to seek Him there. Glorified and Exalted is
He above all that is mentioned of Him in terms of name, attribute, or
quality. God has ever been a God, and there is no god besides Him. God
has ever been a Lord, and there is no lord besides Him. God has ever
been a Sovereign, and there is no sovereign beneath Him. God has ever
been a King, and there is no king besides Him. God has ever been a
Ruler, and there is no ruler besides Him. God has ever been All-Knowing,
and there is none knowing besides Him. God has ever been All-Powerful,
and there is none powerful besides Him. God has ever been Being, and
there is no being beneath Him. God has ever been Creator, and there is
no creator besides Him. All names are in His grasp, and all attributes
are within His hold. Whatever is in the heavens, on the earth, and
between them glorifies Him. There is no god but He, the Mighty, the
Beloved.

What sign is this, beyond which all else is mentioned as creation?
Whoever has recited or recites the prayer \emph{``Glory be to Him who is
Ever-Living and will never die''} until its end will find the ocean of
this verse manifest within their heart. This is because such prayers
appear at the beginning of every Dispensation as unique and singular,
and its Manifestations are rarer than red sulfur. At the end of each
Dispensation, they become more unattainable than the unattainable and
loftier than the loftiest, like the phrase, \emph{``Glory be to Him who
is an everlasting Judge.''} At the culmination of each Dispensation, it
attains an elevation among the hearts of those who glorify, rendering
judgment for all, such that none even conceive of issuing judgment upon
them. Yet, at the beginning of every Dispensation, none contemplate
their judgment.

Similarly, the phrase \emph{``Glory be to Him who is the Truthful One''}
at the end of a Dispensation reaches such loftiness that, from the
height of dignity and the majesty of grandeur, no one considers
themselves worthy of affirming His truth. Rather, He affirms whomever He
wills, and many take pride in having been affirmed by Him. Yet, at the
start of a Dispensation, He is so rare and exalted that no one affirms
Him, even though all truth finds its validation in His affirmation.

This is the case with all names and attributes, for in them nothing is
seen but Him. He manifests through those names and attributes, while in
His presence, everything else sees itself as utterly nonexistent.
Consider today the Dispensation of the Qur'an: Who among men is worthy
of assuming the name of the Knower, while the heart knows nothing? How
much does one regard themselves as effaced before the Messenger of God?
Reflect similarly on the Manifestations of the Names.

The attributes and names are but an immense ocean; when the servant
contemplates the essence of the verse, they are drowned within it. Many
seekers of this ocean have perished by turning their gaze to anything
other than the apparent within it. Were it otherwise, the manifestations
of the Ever-Living would perceive their own mortality. Similarly, this
holds true for all examples and attributes, for within these mirrors
nothing is seen but the One who is Ever-Living, everlasting and
unceasing. It is by His life that these are alive.

Likewise, He is the Sovereign, as there is no sovereignty beside Him;
the All-Knowing, as there is no knower apart from Him; the All-Powerful,
as there is no power save His; and the Truthful One, as no one is
truthful besides Him. Indeed, the divine names have neither beginning
nor end, neither enumeration nor limit. In the essence of all things,
there have been and will be signs from God that bear witness to His
unity. Such a sign proceeds from His will and reveals only Him. The will
itself, being His very essence, acts as a mirror of God, reflecting
nothing but God, Exalted and Glorious.

This is the verse in which nothing is perceived but God in every entity.
When one turns their focus to God, creation is unseen; and when one
mentions creation, they see creation as originating from God, for they
recognize that this creation is the Creator's creation. All else is
creation in every condition.

Consider today the Dispensation of the Bayán: whoever affirms God's
unity does so through the teachings of the Point of the Bayán, from whom
the word of unity has risen, shining forth from that exalted Source.
Yet, at the moment of its radiance, it indicates nothing but God. In the
station where His manifestation occurs, it points only to Him.

Indeed, the aspect of creation is a testimony to servitude before God,
just as the aspect of the essence of every thing testifies to its own
servitude for God. Everything is created from the shadow of His example.
Thus, it has been narrated that there exists a ``verse of truth'' and a
``verse of creation,'' by which creation worships its Lord. It is
through this verse of creation that prostration before Him occurs.

In this manner, all things worship their Beloved through the verse of
creation, yet none reach or can reach Him except through the verse that,
in the sight of God, belongs to Him and points to Him---not to itself.
If a verse points to itself, it belongs to creation. Yet none among the
possessors of hearts has looked or will look at anything except unto God
alone, for within all names, the hearts see nothing but Him. Were it
otherwise, the worship of the servants would be severed, while every
thing has been created solely for the purpose of worshipping Him, as
revealed in the Qur'an: \emph{``I have not created the jinn and
humankind except that they should worship Me.''}

Similarly, just as nothing but God is seen within the names, nothing but
God is seen within the mirrors of these names, which are the hearts of
the monotheists. If a soul, while considering a name, turns to something
other than God, in that moment they are veiled and drowned in the ocean
of veiling.

For example, when one gazes upon a mirror, they perceive only their
reflection within it and do not intend the mirror itself. In the mirror,
the mirror is seen only as it reflects, not as itself. The reflection
within the mirror is established by the one who manifests, not by the
mirror itself. If the mirror were the essence, it would need to be
visible even before the manifestation. Similarly, reflect upon the
mirrors of the living letters: if the mirror itself were the focus, then
it would necessarily precede the act of reflection.

If the name ``Ever-Living'' (\emph{Ḥayy}) were to be perceived in the
mirrors before the manifestation, it would have to be seen prior to the
act of manifestation itself. However, from the moment of manifestation,
it is the Manifested One that is seen, as though nothing exists within
this Ever-Living save Him---just as within a mirror, there is nothing
but your reflection, not your essence. This reflection exists within the
reality of the reflection itself, not within the essence of the mirror.
If it were within the mirror itself, the living letters (\emph{ḥurūf
ḥayy}) would have to be visible before the reflection of the Manifested
One appeared in them. And if they are mentioned as living, their life is
attributed to the manifestation and not inherent to the mirror itself.

This is a matter hidden and contained within the ocean, where all
meticulous observers have drowned, mistaking the reflection in the
mirror for the essence. However, the reality of the throne's reflection
is that the reflection itself is the essence of the example, and its
manifestation comes through the one who manifests and the mirror itself.

Reflect upon all names and attributes; within them, nothing is apparent
but God alone. Do not fix your gaze upon their essence, form, or
attribution, lest you become veiled from the intended purpose. This path
is so subtle that in the Qur'an, the doors of the names have not been
opened, and the Manifestations of Oneness have not disclosed their
mysteries before their explicit expression. If mention has been made, it
is akin to the words of the Master of Martyrs (\emph{Sayyid al-Shuhadā},
peace be upon him), who said:

\emph{``O my God! You have commanded me to return to the traces, so
return me to them cloaked in the garments of light and guided by the
insights of discernment, so that I may return to You from them just as I
entered into You through them, safeguarded from gazing upon them and
elevated in aspiration beyond reliance upon them. Verily, You have power
over all things.''}

Even today, traversing this ocean remains exalted above all, for one
cannot, while knowing the reality of the mirror, look away from the
mirror. Thus, this decree remains elevated above all except for those
who see nothing in the names but God, who do not mix the ocean of
creation with the ocean of truth, nor the ocean of truth with the ocean
of creation. They see nothing in the ocean of truth but God, and if they
perceive the aspect of truth that is the essence of the verse, they
enter the ocean.

Creation does not remain at the pinnacle of truth; rather, it calls upon
the name of Truth, as nothing is seen therein except God. In the human
form and, indeed, in all things, God has placed this reality: that
through the first sign, they may recognize Him, affirm His unity, and
see no independence except Him, perceiving all else as His creation.
Through the sign of their own selves, they worship God, prostrating
before Him, and severing themselves toward Him from any love for what is
other than Him.

If someone thus turns to God with such focus, they are perpetually
connected to that which is the grace of possibility within the realm of
existence. Indeed, the servant perceives in themselves nothing but their
servitude. These two stations are mentioned only in the Point of Truth,
for none other possesses the power to speak on behalf of God. Rather,
all things that bear the name ``thing'' must journey in the ocean of
servitude, for none are capable otherwise. Only through the will of God
has permission been granted, and no servant can attain the ultimate
beyond.

After the setting of this Sun, no one but \emph{Him whom God will make
manifest} will possess the power for such a manifestation of divine
appearance. He alone is the pure manifestation through whom all signs
that indicate God appear in the hearts of all. If a soul ascends and
enters the ocean of their own heart, they will then witness the majesty
of God, perceiving that all else is and has ever been His creation. Yet
that same soul remains forever a worshipper of God and severed toward
Him.

Just as the ocean of the Divine Will is the pure realm of manifestation,
all hearts are created from the shadow of its signs. Each journeys to
the extent possible in the station of servitude. Even the most
infinitesimal particle proclaims its servitude, declaring in its station
of servitude the words: \emph{``Indeed, I am God; there is no god but I.
I have ever been and shall forever be.''} From the ocean of the
manifestation of His essence, which is the mirror of God, all existence
has risen and continues to rise. Within this, His essence is not
mentioned but rather the essence of God. His essence has always been and
continues to be the creation of God.

This is the path (\emph{ṣirāṭ}) than which nothing finer has existed or
can exist in the knowledge of God. When one gazes upon God through this
sign, all that God has described for Himself is ascribed to Him. But
when one gazes upon this sign, all that is described of creation is
ascribed to it. God has ever been a God, and all else has been and
continues to be His creation. There has never been a third between God
and His creation; rather, any third is His creation. There is no god but
He, and all are His worshippers.

\subsection*{Gate 2 (All That Returns to the Point Returns to
God)}\label{gate-2-all-that-returns-to-the-point-returns-to-god}
\addcontentsline{toc}{subsection}{Gate 2 (All That Returns to the Point
Returns to God)}

\textbf{The Second Gate of the Fourth Unity} concerns the principle that
all that returns to the Point returns to God, and what does not return
to the Point does not return to God. Likewise, all that returns to God
returns to the Point, and what does not return to God does not return to
the Point.

The essence of this gate is that the Eternal Essence has ever been and
will ever be beyond comprehension, description, attribution, or vision.
What descends from Him does so through the word of His will, and what
returns to Him returns likewise. In every dispensation, the sign of His
will has particular stations and manifestations that the possessors of
knowledge perceive and recognize.

For instance, in the present Dispensation, all that is exalted within
the Bayán pertains to God and belongs to Him until the appearance of
\emph{Him whom God will make manifest.} At that time, if anyone acts for
the sake of another, such actions will not return to God, even if
performed with the utmost sincerity in unity. For example, the
monotheists of the \emph{Kitáb al-Alif} {[}Book of the Alif{]} bore no
fruit after the appearance of the Messenger of God (\emph{Muḥammad}),
except for those who entered the Furqán {[}Qur'án{]}.

Similarly, whatever is exalted in the Qur'an through praiseworthy
attributes, the essence of concealed knowledge, and the sovereignty of
divine praise becomes void if it does not enter the Bayán and brings no
fruit. Likewise, for the people of the Bayán, at the appearance of
\emph{Him whom God will make manifest}, none of their deeds will return
to God unless they believe in Him---even if they utter the phrase
\emph{``There is no god but God.''} Before that appearance, however,
whatever exists in the Bayán pertains to God and returns to Him---not to
the Eternal Essence directly, but through \emph{Him whom God will make
manifest}.

This concept of returning to Him is akin to the relation of the Ka`bah,
which is referred to as ``His House'' (\emph{Bayt Allāh}). This
attribution is due to its exalted and lofty station, but beyond this
symbolic relationship, no other connection is possible in the realm of
existence, as the Eternal Essence cannot be coupled with anything.

At the beginning of creation, the Will attaches itself to a thing, and
in its ultimate return, the thing returns to Him. If, even nine-tenths
of a tenth of a moment before the command of \emph{Him whom God will
make manifest} to exalt the Bayán, someone were to pick up even a grain
from the ground with the intention of dedicating it to God, it would be
accepted and their existence would bear fruit for the divine cause of
the Seven Letters (\emph{Ḥurūf-i-Sab`}). Such an act would receive the
reward of all things from Him.

However, if this action coincides with the appearance of His command, it
must then be performed with His permission. If it is in accordance with
His pleasure, it becomes the paradise of that act, in what He commands
and enjoins through what is done or will be done. This is the mystery of
all existence, the unseen essence of every fire and light. Whoever
clings to this bond of truth will never dwell under the shadow of fire
in any matter. They will not dwell in the shadows of fire but will
instead be sheltered in the shade of paradise. Otherwise, in every
religion, when you observe its adherents, you find them claiming, ``We
act for God.'' For instance, the sorrow that occurred in the Qur'an for
the mirrors of God---those who said, ``We act for God,'' while in truth,
they acted against God. Similarly, in the Bayán, whatever befell its
believers at the hands of the followers of the Furqán (\emph{the
Qur'an}) was committed while they claimed to act for God, though in
reality, they acted against God.

By the truth of \emph{Him whom God will make manifest}, nothing in the
knowledge of God is greater than this: no soul directs anything against
Him or His believers without claiming they act for God. They say, ``We
do this for God,'' but they lie and falsely claim their actions are for
God, while in reality, they act against Him.

O people of the Bayán, have mercy on yourselves! Judge with your reason,
accept the truth, and do not remain veiled by concealed matters, for the
proof of God at the time of His appearance is manifest over all things.
Consider what you now see: those who believe in the Qur'an. At the time
of its revelation, there was not a single believer in it other than one
person for seven years, though the proof was as it has always been. This
was due to the lack of understanding among the people of that time.
Likewise, in the Bayán, until the Day of Resurrection, whoever enters
will encounter the same proof as at the beginning. The servants who
remained veiled at the outset did so due to a lack of understanding.
Otherwise, the proof of God at the time of His appearance is complete
and manifest over every particle.

When the Bayán was revealed, if all the believers in the Qur'an had
wished to believe in it---indeed, if every soul had sought to
believe---then the proof of God would have been complete and perfect
against them. God testifies against all, just as He did to the first
believer, saying through the same proof that the first believer
accepted, the same proof applies universally. Why, then, have they
remained veiled? It is with this very proof that God chastises them
until they enter His religion. Let this serve as a warning to all who
possess insight, for you are all held accountable.

\subsection*{Gate 3 (God's Manifestation of His
Will)}\label{gate-3-gods-manifestation-of-his-will}
\addcontentsline{toc}{subsection}{Gate 3 (God's Manifestation of His
Will)}

\textbf{The Third Gate of the Fourth Unity:} On the truth of \emph{badá}
(God's manifestation of His will).

The essence of this gate is that no act of worship surpasses the
acknowledgment of \emph{badá}, for \emph{badá} is an affirmation of
God's absolute power to do as He wills. If a soul worships Him with a
devotion surpassing anything conceivable within creation, yet
acknowledges \emph{badá}, this act of worship becomes greater than all
other devotions. This is because it acknowledges His ability, if He so
wills, to cast them into the fire, while remaining just and beyond
question in His actions. None can say ``why'' or ``how'' concerning His
decrees, for He is just in His judgment.

Conversely, if someone disobeys Him in every conceivable way yet fails
to perceive \emph{badá}, this disobedience is considered greater in His
sight than all their transgressions. If He wills to admit them into
paradise, who has the right to question His judgment, saying ``why'' or
``how''? For He remains praiseworthy in His divine justice and exalted
in His decree.

The \emph{badá} of God is distinct from the \emph{badá} of His creation.
The \emph{badá} of creation arises from weakness and incapacity, while
the \emph{badá} of God arises from His power. In every Dispensation, His
\emph{badá} manifests through His will, demonstrating His absolute
authority. Even before \emph{badá} is revealed, He has decreed faith and
paradise for the believers, and He upholds His authority over all.
Through \emph{badá}, He reveals His power, and His sovereignty remains
unchallenged.

For Him, whether it is the Qur'an, the Bayán, or the judgments of the
Furqán or the Bayán, they are all but expressions of His supreme power
and unmatched sovereignty. This serves to remind all that everything is
in the grasp of His power and under His control, now and forever. For
them, there is nothing except by His permission. This is but the
\emph{badá} (manifestation) of His will, for the Eternal Essence has
ever been and shall forever remain in one constant state. The
manifestation and concealment are attributes of His will, and the
beginning and the end are the structure of His intention.

If one considers Him, His beginning is identical to His end, and His
manifestation is identical to His concealment. O observer of the Bayán,
do not become veiled by the use of these terms. The name ``beginning''
(\emph{awwal}) may refer at times to the hidden essence, at other times
to the first mention, at times to the first creation, and at times to
the first fire.

The ``first'' mentioned in the realm of His will refers to the primary
creation within the realm of the Eternal Essence. Similarly, the
``first'' mentioned in creation is akin to the shadow of the sun as
reflected in a mirror, in relation to the primal will. Likewise, the
mention of ``first'' in the context of fire pertains to the eternity of
its nature, which in creation is perishable, not everlasting. Do not
become veiled by the manifestations of His names, for each thing abides
in its own station and does not exceed its ordained limit.

After understanding that the \emph{badá} of the Eternal Essence is the
\emph{badá} of His will, and that His decree is His decree, look upon
each manifestation and its primal \emph{badá}. This \emph{badá} is like
an infinite ocean, ever surging. As all who have existed within this
manifestation have observed, the nature of this appearance in the ocean
of will is apparent. Similarly, in the previous Dispensation of the
Furqán (\emph{the Qur'an}), it was known by the name of glorification
(\emph{tasbīḥ}). In the traditions, it is stated that once the station
of decree (\emph{qaḍā}) is reached, there is no \emph{badá} for it.

This means that if faith reaches the fourth station, which is the pillar
of magnification (\emph{takbīr}), there is no \emph{badá} in what has
been decreed, as a bounty from God. However, if it pertains to justice
(\emph{'adl}), \emph{badá} remains forever applicable. It has ever been
and will forever be true for all things, never separating from any
matter in any station unless God so wills. Indeed, He has power over all
things. The meaning of the station of decree (\emph{qaḍā}) in the pillar
of magnification (\emph{takbīr}) is the ascent of the will, which rises
from the pillar of glorification (\emph{tasbīḥ}) and culminates at its
descent in the pillar of dust (\emph{turāb}), reaching its ultimate end
in itself.

For example, what was decreed during the life of the Messenger of God
and the \emph{badá} manifested therein remains firm until His ascent.
After that, what He made lawful remains lawful until the Day of
Resurrection when another manifestation appears at its conclusion, and
what He forbade remains forbidden until the appearance of the next
manifestation at its conclusion. His martyrs reside in the \emph{badá}
of His ocean of decrees, except in cases where certainty of His pleasure
has been established. If something has been mentioned explicitly for the
Messenger of God, it has not been established by mere analogy. Rather,
it never exceeded even a mustard seed beyond the bounds of the Book of
God.

However, after the setting of the Sun of Truth, the \emph{badá} of His
hidden reality has been and continues to be with those who believe in
Him. Yet, since no one recognizes Him, they do not become illumined by
it. Even in the days of concealment, He does not command anything other
than what He decreed in the days of manifestation. This is the meaning
of the divine \emph{badá}, Exalted and Glorious is He: that the servant
can never find rest on the throne of hope, even if they ascend to the
highest conceivable station within the realm of possibility, for the
consideration of \emph{badá} surpasses that station.

Similarly, if one descends to the lowest conceivable station, they
cannot find solace on the throne of fear. The contemplation of the
\emph{badá} of God is greater than remaining in that state. Yet the
manifestation of this \emph{badá} proceeds from the source of His will,
not from anything else, and it becomes apparent in the Bayán. If
paradise is decreed for something, it remains so until the Day of
Resurrection, firmly established for it. If in regard to something,
paradise is not decreed, then until the Day of Resurrection, it remains
in fire, subject to divine justice, and perishes. However, at the
appearance of the Resurrection, if the very tree of fire accepts the
manifestation, it becomes a tree of paradise. Conversely, if the tree of
paradise becomes veiled, it turns into a tree of fire.

This reality has been witnessed in every Dispensation. Those present in
this Dispensation observed it: in every manifestation, individuals
exerted their utmost effort to leave a mention of good for themselves in
the Book of God until the next manifestation. This effort is greater
than expending countless wealth in exalted places to establish one's
mention among others. However, those very exalted places might turn into
places of humiliation, leaving no trace of their mention, if they
deviate from the truth. Otherwise, their reward is with God, for God
rewards the righteous.

\subsection*{Gate 4 (All Mentions of the Servant are Bonds of Servitude
to
Me)}\label{gate-4-all-mentions-of-the-servant-are-bonds-of-servitude-to-me}
\addcontentsline{toc}{subsection}{Gate 4 (All Mentions of the Servant
are Bonds of Servitude to Me)}

\textbf{The Fourth Gate of the Fourth Unity:} On the principle that all
mentions of the servant are bonds of servitude to Me.

The essence of this gate is that the spirits of all things return to the
human form, and the paradise of all things is within the paradise of
humanity. An example of this is if an unmatched diamond is in the
possession of a believer, its value is magnified through its association
with the believer. Similarly, all things derive their worth and essence
in relation to this form. This human form, in turn, is associated with
the divine names and attributes.

If the heart points to God, it is mentioned under the shadow of
\emph{His Most Beautiful Names} (\emph{asmā' al-ḥusnā}) before God. If
it is associated with an attribute, it is mentioned among His
attributes. This corresponds to the saying, \emph{``There is nothing but
God, His names, and His attributes.''} This form does not bear the
judgment of name and attribute except insofar as it enters under the
shadow of each manifestation at the time of that manifestation.
Otherwise, the station of faith is lifted from it.

How, then, can it bear the judgment of name or attribute, for God's gaze
is upon the reality of things, not upon outward appearances. Consider
today: though every praiseworthy attribute resides within the Letters of
the Alif, they are instead mentioned by attributes of fire. If today in
the Bayán, a soul is a believer in God and His signs, even if seated on
the dust, such a person is seated upon the throne of honor. Conversely,
anyone else, even if sitting in the highest seat of prestige, is in the
utmost abasement before God and the possessors of knowledge.

At the time of the appearance of \emph{Him whom God will make manifest},
every believing man who becomes a believer in Him is worthy for the sun
of His name to rise in their heart and for that name to be enshrined in
their essence. Every soul that is a believer in the Bayán is and has
always been a servant bound to His servitude, just as the
\emph{possessors of forms} in the Qur'an were and remain bound servants
to the Messenger of God.

But woe to the servant bound in servitude who, before their Master,
commits acts contrary to the requirements of servitude! Reflect upon
this and be saddened. See how many have placed the yoke of servitude
around their necks and taken pride in it, only to act unjustly towards
the noblest manifestation of their Master in the end. How could such a
matter be permitted, that they reside in the mountain while all believe
in Him? Yet they were not content with the very essence of faith in Him
under the name of Believer. If they had been, such events would not have
occurred.

Thus, all are drowned in the ocean of veils. The suns shine in mirrors
through His radiance but remain veiled from Him. The fish of the sea are
alive through Him, yet they question Him today. All believers in the
Qur'an await the appearance of the Qá'im of the family of
Muhammad---peace and blessings be upon Him and His family. They
supplicate and plead for His appearance, take pride in seeing Him in
their dreams, and then send Him with their own hands to prison,
confining Him to the mountain.

This is the meaning of the tradition: \emph{``In Him is a likeness to
the one who is described, sold, and bought.''} O people of the Bayán,
take refuge in your Beloved and place the yoke of His servitude around
your necks. Do not permit for Him what you would not permit for
yourselves as His bound servants. Surely, you have heard of the
appearance of the Messenger of God before, and of His companions and
those who awaited Him. After His appearance, what befell Him over seven
years---and in another narration, nine years---is well known. Yet the
lessons of those events have brought no benefit to you, for if they had,
such actions would not have occurred in this manifestation.

Have shame before your Beloved and refrain from treating \emph{Him whom
God will make manifest} in this way. Resolve within yourselves to
refrain from doing to any soul what you would not want done to
yourselves, whether among those of power or those of humility. If you
act thus, by God's will, you will not cause Him sorrow. For the suns in
the mirrors do not recognize the Sun except through its reflection
within themselves. Thus, they see Him as one like themselves, while in
reality, all are upheld by Him. It is from His bounty that the heavens
and all between them were created.

Consider the statement of one among the Letters of the Bayán, who said:
\emph{``From His bounty is the world and all it contains; from His
knowledge is the knowledge of the Tablet and the Pen.''} Yet even the
phrase ``from'' (\emph{min}) is insufficient, for everything to which
the concept of ``thing'' applies is but a single drop from the vast
ocean of His bounty and a mere particle from the immense sea of His
favor.

In His time of manifestation, you should understand His worth. Do not,
during His days of appearance, weep for Him in places of honor while He
resides in a mountain, enduring sorrows. By His grace and bounty, you
are free from all sorrow, yet He bears the burden of grief. Reflect on
this O possessors of insight, reflect, and then consider further.

\subsection*{Gate 5 (All Primal Points are His
Signs)}\label{gate-5-all-primal-points-are-his-signs}
\addcontentsline{toc}{subsection}{Gate 5 (All Primal Points are His
Signs)}

\textbf{The Fifth Gate of the Fourth Unity:} On the fact that all primal
points are His signs.

The essence of this gate is similar to that of the previous one,
elaborating on the exaltation of all that is attributed to Him. Yet, it
does not explicitly mention the handmaidens, for He is more deserving of
dominion over His creation than any other. When He wills a matter, none
can question Him with ``why'' or ``how,'' whether it pertains to the
highest or the lowest of creation or any station. He decrees as He
wills, is not questioned about what He decrees, while all are questioned
about what they do.

The fruit of this gate is that if the Master does not grant permission
to His servants, they cannot rightfully establish connections.
Similarly, in the time of His manifestation, if He forbids anyone from
partaking in one of the signs of His dominion, it becomes forbidden for
them. For example, He has forbidden it to all who do not believe in Him,
and such disbelief is graver than any other transgression. Thus, no one
should hasten to deny Him on the Day of His appearance. If someone does
not believe, let them remain silent, for many have heard mention of Him
yet refrained from affirming Him. They approach one of His signs while
it is forbidden for them due to their lack of faith, and it will not
become permissible for them except through belief.

Take heed, O people of the Bayán, for the manifestation of God will
appear as God wills. Do not let that which is beneath the truth come to
represent the truth, for His signs are proofs over all. If you reflect
deeply, you will immediately affirm this. Consider: what difference
exists between the miraculous nature of the entirety of the Qur'an and
the single chapter of Surah al-Tawhid (\emph{The Unity})? Similarly,
there is no difference between faith in the last manifestation, whose
heart pointed to God, and faith in the first. Always look to the
appearance of the Point of the Bayán, so that perhaps on that Day, you
will not veil yourselves from your Beloved. If you had reflected on the
appearance of the Point of the Qur'an, you would today comprehend the
Cause of God. This has only been decreed so that you may exert your
utmost effort---not that you strive in all aspects of religion while
veiling yourselves from the one who is the essence of that religion.
Consider the Qur'anic dispensation: you observe all striving with utmost
precision and effort, yet they issued decrees against the very one who
is the Manifestation of the Faith, imagining that their decrees and
precautions established the religion through their interpretations.

When the jurists of the Qur'an elevated their caution to this extent,
how will it be in the Bayán? Yet in the Bayán, the decree of taking life
has been prohibited more strictly than anything else. No act has been as
forbidden as this: if anyone even contemplates taking the life of
another, they exit the religion of God and are subject to chastisement
as long as God wills. That which was lawful for them in the Bayán
becomes forbidden, even the very breath they take. God has not permitted
anyone in the Bayán to issue a decree of death against another under any
circumstance, in any station, or in any matter---even if the person
appears in opposition to their faith in matters of knowledge or belief.

Such judgment belongs to no one but God. O perverted people, what right
do you have to issue a decree for taking life? Do you not realize that
it is through the blessing of that soul that you bear the name of Islam?
Reflect on what befell the People of the House in the past under the
name of Islam. Those deeds were committed in the name of Islam, yet in
reality, they acted against the essence of Islam, for there is no Islam
without faith in Him.

From the time of the setting of the Sun of Truth, it has not been and
will not be permissible for anyone in the religion of the Bayán to
decree death. If anyone does so, they act outside the Bayán. Such an act
does not belong to the teachings of the Bayán. There is no sin greater
than this for such a person, for God has revealed in the Qur'an:
\emph{``Whoever kills a soul, it is as if they have killed all of
humanity.''} This decree applies to the souls of the believers. How much
greater is the sin if that soul has been likened to the Ka'bah,
attributed to God, and called the ``Self of God''? Despite such emphasis
in the Book of God, the fire prepared for those who ascribe themselves
to the Qur'an has yielded no effect from the beginning of Islam until
today. For them, no chastisement is greater than disobeying their
Beloved while professing worship and prostration to Him.

God has decreed upon His sacred essence that He will not forgive anyone
who even contemplates the killing of one of the believers in the Bayán.
How much more severe will it be if someone issues such a judgment or,
God forbid, commits such an act? Every ruler who recalls the sufferings
of the first and last Manifestations and refrains from harming any soul,
acknowledging the trials that befell the \emph{Letters of the First}
{[}followers of the first Manifestation{]}, fulfills a right upon God.
God will bless whatever has been bestowed upon such a ruler and multiply
what has been decreed for them.

If the days of such a ruler coincide with the appearance of \emph{Him
whom God will make manifest}, it will be their duty to assist the
Manifestation of the Faith and seek His guidance in alleviating
burdensome acts of worship. They should request sovereignty from Him
until the Day of Resurrection, for He is the Giver of all things and the
rightful authority over all matters. If He bestows sovereignty upon
them, no one has the right to take it away by divine decree, for He is
more worthy of all things than anyone else. This is evident in all
decrees that proceed from God's command. No one can breach what is
established by Him, nor should anyone become veiled from the Beloved of
all by anything, for all that one possesses has come from Him, even
before His appearance. Support Him, and if He commands striving in His
path, then be exalted in the honor of being martyred for His cause, that
your name may be inscribed in the register of the near ones. Otherwise,
you will depart from this world without leaving any mention behind.

It is not enough to engage in acts of mourning for the Truth during
blessed days, as was the custom among all believers in the Qur'an, from
the highest to the lowest, who commemorated the sorrows of the Messenger
(\emph{peace be upon Him}) and His successors, and recounted the
sufferings inflicted upon His followers. Yet, despite their mourning and
weeping for the successors of His religion, many of these same
individuals became oppressors and tyrants toward the very ones they
grieved for. The learned among them issued verdicts against those
successors, while others remained veiled by their own ignorance.

Reflect on the events of the past so that you may not commit acts that
would lead to your eternal abode in the fire. In the Bayán, much
emphasis has been placed on ensuring that no one causes grief to
another, so that the Sovereign of creation may not be grieved by the
shadows in the mirrors. If they turn toward Him, they have significance;
otherwise, they are absolute nothingness. From the beginning of their
lives to the end, even with utmost caution and effort in worship and
action for God's sake, they remain lifeless in the presence of the
Manifestation of God.

What value do their deeds hold? Rather, their worth is no greater than
the mention of \emph{nothingness} itself, which is naught before God. No
one has the authority to decree except \emph{Him whom God will make
manifest}. He is the Blessed Tree that appears with clear signs for the
pleasure of God, to renew His religion, and to judge justly among the
people of the Bayán. He will also render fair judgment upon those who
have not entered the Faith, should that Day arrive. Otherwise, how could
the sovereigns of the Bayán enjoy comfort for themselves while a soul on
the earth remains that does not believe in God and His religion. For
those not among the sovereigns of the Faith, it is not permissible in
the religion of the Bayán to tolerate such a condition. However, even
for them, it is not permissible to harm those associated with the Bayán
under any circumstance or in any station---except at the appearance of
\emph{Him whom God will make manifest}. Even then, the treatment of
non-believers is subject to His command and the guidance He provides.

The believers in Him and the Bayán have been granted permission to open
lands, bringing all under the shadow of faith in God and His religion
without causing even the slightest sorrow to those associated with the
Bayán. Even in the conquest of territories, this principle has been
upheld: the people of those lands were not subjected to harm but,
wherever possible, were invited to the religion of God through other
means. For those unperceiving of divine insight, the goods of this world
often hold value; by acquiring them, there is hope that they might enter
the religion of God.

This ruling applies to those outside the Bayán, not to its
followers---except at the appearance of \emph{Him whom God will make
manifest}. His command is the divine command, arising from the sacred
realm of His holiness. Through His command, all are elevated,
manifesting their potential in existence according to what He ordains
and determines. For the servant possesses no power to ordain or manage,
except by God's permission. Indeed, He is the best of planners and
decreers.

At the time of elevation and divine decree, nothing escapes His
knowledge---neither in the heavens, nor on the earth, nor between them.
Nothing is beyond His power, whether in the dominion of command,
creation, or anything beneath them. He creates as He wills by His
command, for He has power over all things.

\subsection*{Gate 6 (He is Not Questioned About What He
Does)}\label{gate-6-he-is-not-questioned-about-what-he-does}
\addcontentsline{toc}{subsection}{Gate 6 (He is Not Questioned About
What He Does)}

\textbf{The Sixth Gate of the Fourth Unity:} On the principle that He is
not questioned about what He does, while all else is questioned about
everything.

The essence of this gate is that the Eternal Essence neither has nor has
ever had, by Its very nature, a manifestation or a concealment. Beyond
that, there has been and will be no manifestation or concealment. No one
has access to His realm, which is characterized by the statement
\emph{``He is not questioned''} (\emph{lā yus'al}). He has no action
inherently tied to His essence that could be described as \emph{``He
acts''} (\emph{yaf`al}), nor is there any judgment established for Him.
Rather, He created the primal will (\emph{mashiyyat}) in the same manner
that He created all things---through itself. He created it by itself and
attributed all to it, and it to Himself, because of its loftiness and
sublimity.

Just as He called the Ka'bah His house, while in truth all places are
equal before Him, He assigned the Ka'bah a special status due to its
exalted and sublime association with Him. Similarly, He made the primal
will the manifestation of \emph{``He is not questioned''} because
nothing greater or loftier in His knowledge could exist that would merit
such a bestowal. It is akin to the celestial sphere that neither
oppresses nor points (\emph{lā yajīr wa lā yashīr}) and similar
examples.

He made the will the manifestation of \emph{``He does what He wills and
ordains what He desires''} because His essence is the very will itself.
He has desired and will desire only what God wills, and nothing of God's
will manifests except through the will. Likewise, nothing that God
ordains becomes apparent except through what the will ordains. None
other is worthy of such a bestowal, for through this will, all creation
arises.

If God were to establish two separate wills, no single manifestation
could appear in creation, and between the two, confusion would arise.
The order of the dominion of the heavens, the earth, and all that lies
between them would collapse. From the One who has neither beginning nor
end, there has ever been and will ever be one will that manifests anew
in each Dispensation. This will becomes the manifestation of \emph{``He
is not questioned about what He does''} and \emph{``He does what He
wills.''}

In the realm of the letters of the living (\emph{ḥurūf ḥayy}), nothing
is seen but His will, just as in the realm of the multiple letters,
nothing is observed but His will. If anything enters a heart other than
what His will has determined, in that moment they are not within the
paradise of that manifestation, and they are deemed disobedient. The
exaltation of the mirrors lies in the fact that nothing is seen within
them except the sun. If the essence of the mirror itself is observed,
how could it be worthy of mention before the Sun of Truth? Conversely,
if the sun is observed, its movements are the movements of the sun, for
there exists nothing for it other than the essence, attributes, actions,
and orientation of the sun itself.

In the Dispensation of the Furqán (\emph{the Qur'an}), Muḥammad, the
Messenger of God (\emph{peace be upon Him}), was the primal will
(\emph{mashiyyat}), and the Letters of the Living (\emph{Ḥurūf Ḥayy})
were the essence of His will, revolving around His command. They desired
only what He desired, for they were perpetually oriented around His
decree. Similarly, the multiplied letters of this unity, from the first
to this day, have sought nearness through aligning their will with that
of Muḥammad. Without this alignment, they would not be worthy of
mention.

Consider everything that occurred in Islam: actions done without love
for Him were carried out under the pretense of His will. Were it not for
this pretense, they would not have possessed the strength of even a
gnat. This was also true in the dispensation preceding Muḥammad, and it
remains so in the Dispensation of the Bayán. Here, the Sun of Truth has
risen in this cycle under the name of the Gate (\emph{Báb}), and all
matters occur according to His will until the appearance of \emph{Him
whom God will make manifest}.

Whatever transpires in this Dispensation aligns with His will, and if it
goes against His pleasure, it is still falsely attributed to His will.
Were it not for this pretense, no thing would possess even the slightest
power. This is the meaning of \emph{``There is no power or strength
except through God, the Most High, the Most Great.''} Without this
reality, no manifestation of truth would find comfort in any
dispensation, for it is through this connection that all things find
their strength.

At the end of each manifestation, creation becomes so fully aligned with
His will that they desire nothing but what He desires. How, then, can
they recognize their Beloved or seek anything other than love for Him
and that which He loves? If you wish to observe this truth clearly,
consider the moment of severance: the one who severs themselves is, by
their very act, cut off from their own essence. How, then, can they pass
judgment upon another?

Whether in legislative matters or in the order of creation, the same
principle applies. For example, if a Muslim severs themselves from
Islam, they can no longer judge themselves as being Muslim---how, then,
can they judge another? At that moment, they must inevitably find
themselves in some other position within religion, even if it means
joining the community of Adam. In such a case, any judgment they issue
reverts to the judgment of Adam, as that was the manifestation of the
will of God at that time.

All actions return to God, for He has denied the existence of any power
or strength except through Himself. This remains true whether one
considers the legislative realm or the order of creation. Should one
seek to understand this while not being within the Bayán, they will find
no power or strength remaining, for the essence of all has been gathered
into the Bayán. Beyond it, there is no power or strength left to be
claimed.

In the community of Jesus, no power or strength remained except through
God. Were it otherwise, they would have been judged by the truth, yet
such judgments were not forthcoming. The same holds for Islam after the
appearance of the Bayán and for the Bayán after the appearance of
\emph{Him whom God will make manifest}. From the day of His appearance,
all power and strength belong to God alone, manifesting through that
reality which is the embodiment of His will. For the people of the
Bayán, no power or strength remains in truth unless they dwell under His
shadow and so it continues unto infinity. Consider this: if power and
strength remained in falsehood, it would hold the same relation to truth
as it did before. Today, those in error act based on the illusion of
such a relationship, yet in reality, there is no power or strength
except through God, manifesting in every appearance through that which
proceeds from God. If you look with the eye of the heart, you will see
that all power and strength reside with \emph{Him whom God will make
manifest,} through God, the Exalted and Glorious, and within His bounds.

This manifestation, from Adam to today, has always been supported by the
aid of this appearance. Do not focus on places or boundaries, for the
will is the sign of God and has always pointed to God. If the first Adam
had not been supported by this aid, no believer would have found shelter
under the shadow of faith today. Similarly, every prophet in every
manifestation has been a believer in that manifestation. Without the aid
of this appearance from \emph{Him whom God will make manifest,} how
could they return to Him and turn completely toward Him at the time of
each appearance?

Even if a thousand thousand manifestations were to occur after this
one---indeed, extending infinitely---the aid for all these
manifestations would still originate from the Manifestation of His will.
This is the very same primal Adam, without beginning, as seen within the
realm of existence. This is the secret of truth: whoever journeys within
this ocean perceives the meaning of \emph{``There is no power and no
strength except through God.''} They see both compulsion (\emph{jabr})
and absolute freedom (\emph{tafwīḍ}) as nullified and walk upon a path
wider than the expanse between the heavens and the earth.

In every age, they recognize the Manifestation of \emph{``He does what
He wills and decrees what He desires''} at the time of His appearance,
and they weep during His concealment until the day of His next
appearance. For at that time, all understanding is severed from
everything else, and the Manifestation of \emph{``He is not questioned
about what He does, while all are questioned by His command''} is
recognized. They bow in submission at the moment of His appearance and
do not veil themselves from Him. The separation between the ``Kaf'' and
the ``Nun'' (\emph{Be, and it is}) is never apart from Him, for they see
that whatever is manifested in every appearance arises by His word. They
focus on the essence of the matter, not on the various outward aspects
of each manifestation. In every appearance, such a path leads to the
paradise of union, which surpasses anything conceivable in existence.

All previous manifestations are witnessed in their entirety in the
current manifestation. For instance, if someone looked upon the
Messenger of God (\emph{Muḥammad}), they would, in doing so, have
observed all the prophets in His appearance. Similarly, all future
manifestations are contained within what is possible in the current
manifestation. Such a soul is worthy of recognizing \emph{Him whom God
will make manifest} and addressing Him while turning toward God, saying:

\emph{``O my God, You are the First, and there was nothing before
You.''}

For all previous manifestations culminated in the Bayán, and all within
the Bayán culminates in \emph{Him whom God will make manifest.} Before
Him, nothing remains hidden, for were it so, judgment concerning faith
and belief would not rest solely on Him.

\emph{``You are the Last, and nothing is beyond You.''}

This is because all future manifestations emerge through His appearance,
not apart from Him.

\emph{``You are the Manifest, and nothing is above You.''}

For everything that appears in His time is revealed through Him, not
independently or through those who do not believe in Him. Were there
another manifestation independent of Him, it would necessitate assigning
superiority to it. However, judgment remains upon the absence of belief
in Him, not upon anything superior to Him.

\emph{``You are the Hidden, and nothing is beneath You.''}

If there were another reality hidden apart from Him, it would have to be
the essence of existence itself. Yet, every essence is annulled before
each manifestation unless it enters that manifestation, which is the
Manifestation of God.

\emph{``Indeed, You have power over all things.''}

For no one else is the manifestation of the essence itself, and nothing
can arise outside the bounds of His will and decree. For if it were
otherwise, a decree from the Possessor of the Command would remain, and
yet, before every manifestation, what existed in the previous
appearances becomes as nothing unless it enters into the new
manifestation. This is why the one who knows God, at the time of the
rising Sun of Truth, reaches the ultimate limit of their existence. At
the time of its setting until the next rising, they remain in tears,
pained by the concealment of the Sun, experiencing a sorrow beyond all
other sorrows. For all sorrows exist within the shadow of this sorrow.

Consider the weaning of a child from milk: since milk sustains the
physical nature of the child, its absence causes great pain. Now reflect
on what sustains the essence of the heart---if it becomes veiled, how
much greater is the anguish! These worldly loves, which sometimes arise
between people, are mere reflections of the love that manifests within
the essence. How much greater is the effect of separation in the
manifestations of the King of existence, while these lesser loves pale
in comparison to the love of the soul. Likewise, the love of the soul is
insignificant compared to the love of the heart.

By the sanctified essence of God, all in the world of nature are in
motion. Should they ascend and enter the realm of essential existence,
they cannot distance themselves even for a fraction of a moment from the
manifestations of glorification (\emph{takbīr}). Rather, they will
always remain settled near the grave of their final abode of
annihilation. Similarly, if they enter the realm of love for the self,
they become cloistered in the manifestations of divine separation
(\emph{taḥlīl}). Should they ascend from there, they become cloistered
in the love of the manifestations of glorification (\emph{tasbīḥ}).

If they ascend further and contemplate these manifestations outwardly,
they will prostrate themselves before God. For all love, if it exists
after faith in Him, becomes a divine love; otherwise, it remains a
worldly attachment. In the fire, consider the essence of this essence,
the purity of this purity, the mixture of this mixture, and the coolness
of this coolness. All delights are created under the shadow of His
dominion and directed toward Him in relation. For all pleasure resides
in His good pleasure, as the servant drinks the cup of tribulation
knowing it is His satisfaction, just as one drinks the cup of pure wine
in paradise.

What, then, of one who delights in His love? All other loves become as
nothing before Him. This is why the traveler in this paradise, at the
time of any manifestation, is absorbed in the radiance of that
Countenance. At the time of its setting, they burn with separation until
its rising. Their sorrow in separation from Him becomes the most
exquisite delight, surpassing all other pleasures, both for themselves
and for those endowed with knowledge.

How can the pen describe the removal of veils while all dwell within the
ocean of veils? None have yet ascended from the transient corporeal
world, which turns to dust after death, to the realm of essential
existence. How, then, can they reach the vanishing body or the stations
beyond, as God has ordained?

Thus, those who find delight in the essential pleasure are fewer than
the rarest red sulfur. Indeed, such souls are almost nonexistent, for
most are absorbed in things whose pleasures are confined to the span of
their lives, if even that. For many, such pleasures are not even
decreed. No soul is always in the state of experiencing what it finds
pleasurable.

The fruit of all these explanations is this: recognize the Point of
Truth at the time of its manifestation. Under the shadow of this
recognition, you will witness all good. Do not become veiled, for you
will then be deprived of all good. At the very least, traverse the realm
of essential existence so that you may, even for a brief morning, be
with Him in a manner befitting, for otherwise, in the transient
corporeal world, all have been and are with Him night and day. This is
because His concealment resides in His own throne within creation, and
that same place is the highest paradise (\emph{Firdaws al-A`lá}) and the
Garden of Contentment (\emph{Jannat al-Ridwán}), just as it was before
His manifestation, during it, and will remain after His setting. In
every exalted throne that God has decreed for Him, it remains so.

All the people of the Bayán are like fish in the waters of His command,
living by His decree and seeing nothing but Him, perceiving their very
existence as dependent on Him. Yet they remain veiled from Him when each
decree is issued. They focus instead on its fruit, striving not to
remain veiled from Him in any manifestation, for this is the essence of
knowledge and action.

If a servant becomes aware, blessed is the one whom God remembers, for
when He remembers, all things remember Him, and God is the best of those
who remember. He is the manifestation of the divine will in this
Dispensation, the Point of the Bayán. At the time of \emph{Him whom God
will make manifest}, His essence, which is exactly the same essence,
appears in its most exalted form. If anyone remembers Him, they remember
God. Without this, nothing else is worthy of remembrance.

Do not stray into your own desires, for you will find nothing there. The
command of God emerges from His treasures, and He is the essence of the
treasures of God's command, from before and after. Will you not then be
grateful?

\subsection*{Gate 7 (All Below God is His
Creation)}\label{gate-7-all-below-god-is-his-creation}
\addcontentsline{toc}{subsection}{Gate 7 (All Below God is His
Creation)}

\textbf{The Seventh Gate of the Fourth Unity:} On the principle that the
beginning is from God, and the return is to Him.

The essence of this gate is that all below God is His creation. The
beginning of all things is from God by His command, and the return of
all things is to God by His command. Consider the prayer: its beginning
was through the command of God given to the Messenger of God, and its
return is to God through the command of the Point of the Bayán.
Similarly, in \emph{Him whom God will make manifest}, all the ordinances
of the Bayán are fulfilled. He is none other than the Point of the
Bayán, for the return cannot be attributed to other than God, nor can
the beginning arise from other than Him. In every instance where a
beginning is conceivable, it pertains to God; and where a return is
conceivable, it pertains to God. However, the manifestation of this
beginning and return is through the manifestation of that time. For
instance, all prayers in Islam revolved around the command
\emph{``Establish the prayer''} (\emph{aqim al-salāt}), which issued
from the tongue of the divine will. Similarly, the return of all these
believers, if they alter the command, is only by the decree of God.

For the people of understanding, the beginning itself is the return, and
the return itself is the beginning. Yet this truth must not lead one to
neglect deriving its fruits. Though this phrase---\emph{the beginning is
the return, and the return is the beginning}---is found innumerable
times in the scriptures of the past, few have derived its fruit. The
fruit lies in this: if one looks at the Point of the Bayán today, they
see nothing but the essence of the beginning, which is the Point of the
Furqán and its ordinances. Similarly, in \emph{Him whom God will make
manifest,} one sees nothing but the Point of the Bayán.

This understanding must manifest in action; otherwise, what benefit is
there in merely uttering phrases while remaining veiled from their
fruit? For those endowed with insight, this explanation and reminder
suffice if they reflect. God begins all things and then returns them,
and to God do all things return---specifically to the command of
\emph{Him whom God will make manifest.} Blessed is the one who attains
His presence, for He is the Mighty, the Beloved.

\subsection*{Gate 8 (By His Signs, He Creates the Essence of All
Things)}\label{gate-8-by-his-signs-he-creates-the-essence-of-all-things}
\addcontentsline{toc}{subsection}{Gate 8 (By His Signs, He Creates the
Essence of All Things)}

\textbf{The eighth Gate of the Fourth Unity:} On the principle that by
His signs, He creates the essence of all things, provides sustenance,
causes death, and grants life.

The essence of this gate is that all things return to the decree of the
human self, which is created from four signs:

\begin{itemize}
\item
  \textbf{The sign of creation}: the heart (\emph{fu'ād}).
\item
  \textbf{The sign of sustenance}: the spirit (\emph{rūḥ}).
\item
  \textbf{The sign of death}: the soul (\emph{nafs}).
\item
  \textbf{The sign of life}: the body (\emph{jism}).
\end{itemize}

All these elements are mentioned within the nineteen souls who are the
gates of paradise. For every creation and the sustenance, death, and
life that occur through the appearance of these manifestations are
established by them. Similarly, in opposition, in the nineteen gates of
fire, whatever pertains to these aspects---creation, sustenance, death,
or life---also occurs through God's decree, for those entities existed
through God before, even though they later emerged outside the fold of
divine will.

Thus, judgment is passed on those attributes that are not associated
with the essence of the Seven Letters (\emph{Ḥurūf-i-Sab`}). However,
those that are tied to the remembrance of the Messenger of God remain
connected. For example, if one of the people of the Bayán gives a leaf
of paper to a believer, this act is not separate from the deeds of the
Letters of Unity (\emph{Ḥurūf-i-Wāḥid}), as this is merely the
multiplied form of their collective will. Such an act is only performed
in obedience to them, and in this act, nothing is seen except the act of
God.

Thus, if in the East or the West, one of the people of the Bayán
manifests an aspect of creation, sustenance, death, or life to another,
it is God who manifests it, and at that moment, the individual becomes
the manifestation of that act. In such a manifestation, nothing is
perceived but the Primal Unity (\emph{al-Wāḥid al-Awwal}), and in that
unity, nothing is seen except God. This is why none but God is the
creator of any thing, none but God is the sustainer of any thing, none
but God is the cause of death for any thing, and none but God is the
giver of life to any thing.

On the other hand, if one of the letters of fire manifests these
aspects---creation, sustenance, death, or life---to another, it is only
through God that it occurs, but it does not pertain to God, nor does it
return to Him, nor can it be attributed as coming from Him. This is
because, in that moment, something other than God is seen, and such
perception does not arise except from viewing something apart from the
Point of the Bayán. Just as the meeting with God (\emph{liqā'-ullāh})
cannot be realized except through the vision of that Point.

Everything that is established within the Bayán concerning these four
aspects---creation, sustenance, death, and life---is mediated through
the verses and words revealed in the Bayán. Reflect on the past: before
the verse of pilgrimage (\emph{ḥajj}) was revealed, there were no
specific ordinances tied to it. Only after its revelation did these
matters come into existence that revolve around creation, sustenance,
death, and life in the context of pilgrimage (\emph{ḥajj}), how could
they have been realized otherwise? Similarly, consider all ordinances
and observe that none are manifest except through God. None other than
Him has been or will be the Creator, Sustainer, Cause of Death, or Giver
of Life.

At the time of each manifestation, do not veil yourself from the Point,
for if you remain veiled even for a moment, you will not be mentioned
within the sphere of God's acts. For instance, if someone within the
Qur'anic Dispensation manifests sustenance, even though it is through
God that sustenance is given, their act does not return to God if they
fail to recognize Him and remain veiled from His manifestation. Such an
individual cannot attain the worth of being mentioned under the shadow
of the acts of His manifestations.

Consider this: if today a believer in the Bayán gives a single cup of
water to another, it is sweeter in the sight of the one who knows God
than if someone who does not believe in the Bayán were to present the
entire earth at the highest level of love. This is because the cup of
water is given under the shadow of the acts of His manifestations, while
the other act, though through God, is not judged as worthy today. God
does not attribute such an act to His manifestations because the
manifestations the individual believes in have been fulfilled and have
reached their stations, yet they fail to recognize this.

How, then, can such an act be judged under the shadow of the acts of the
manifestations of Truth? \emph{Exalted is God beyond that, in utmost
glorification!}

If someone comprehends even one aspect of the attributes of \emph{Him
whom God will make manifest,} it is better than comprehending all the
attributes of the Bayán after His appearance, though not before. This is
because, in such an act, the act of God is manifest, while previously,
though it existed, it was not worthy of being attributed to God unless
the very essence of that act became aligned with the attributes of the
Bayán. At that point, it is worthy of being realized within the
manifestation of His act. For the Letters of the Living might appear and
yet people continue to cling to the Letters of the Living, prostrate
before their stations as if nothing has changed. This is evident even
today. Though the Letters remain the same, their decree is lifted after
the manifestation, and the acts of God become manifest through the new
outward manifestations. Therefore, judgment is rendered through them.
\emph{O possessors of insight, wait expectantly, for we too have been
waiting.}

\subsection*{Gate 9 (A Believer of the Bayan Deserves Their Name to Be
Recorded)}\label{gate-9-a-believer-of-the-bayan-deserves-their-name-to-be-recorded}
\addcontentsline{toc}{subsection}{Gate 9 (A Believer of the Bayan
Deserves Their Name to Be Recorded)}

\textbf{The Ninth Gate of the Fourth Unity:} On the record of those who
rise with the Bayán and its decrees.

All those who rise with the Bayán, by virtue of their dominion, should
have their names recorded along with what God has decreed for them. This
record will last until the Day when the Point of the Bayán is made
manifest again, to recompense every soul for what they have earned.
Truly, God is swift in reckoning and is indeed self-sufficient and
exalted.

The summary of this gate is from the time of the revelation of the Point
of the Bayán to the appearance of \emph{Him whom God will make
manifest,} any person of dominion who is exalted by the Bayán and
manifests what they are capable of in the path of love for the Point of
Truth, elevating His word, deserves to have their name recorded in a
book, inscribed upon the essence and purity of the Bayán, until the
return of the Point. At that time, they will receive the finest reward.

What reward could be greater than this: that one whose mention is the
mention of God should mention every soul and express satisfaction with
them? This alone is an honor sufficient for anyone, as it pertains to
the primal oneness of multiplicity, the first manifestation of divine
power.

From the time of the Messenger of God (\emph{Muḥammad}) until today, if
anyone's deeds were recorded precisely, it would be evident that no one
has acted in exact accordance with the religion, word by word. This has
never been heard. However, in the Bayán, whoever arises will do so with
the manifestation of divine power and eternal sovereignty, rendering
unnecessary the conquest of all lands, to take even a mustard seed's
worth from the soul of a believer. Truly, God enriches whomever He wills
by His grace. Indeed, He is bountiful and generous.

\subsection*{Gate 10 (Prohibition of Studying Texts Outside the
Bayan)}\label{gate-10-prohibition-of-studying-texts-outside-the-bayan}
\addcontentsline{toc}{subsection}{Gate 10 (Prohibition of Studying Texts
Outside the Bayan)}

\textbf{The Tenth Gate of the Fourth Unity:} On the prohibition of
studying texts outside the Bayán, except as permitted by God in matters
related to theology.

The summary of this gate is the Point of the Bayán has been manifested
by God as the embodiment of infinite appearances in this Dispensation.
From the highest heights, it calls people to God with the proclamation,
\emph{``Indeed, I am God; there is no God but Me.''} Likewise, it
proclaims, \emph{``I am more insignificant than every particle,''}
issuing from His tongue. In every matter and by infinite means,
expressions from Him have been revealed---whether in the form of verses,
supplications, interpretations, philosophical sciences, or Persian
responses---such that no one remains in need of anything else.

Permission has not been granted to learn from sources other than the
Bayán's revealed works. However, if someone creates knowledge in a field
and it bears the name of God---indicating their faith---it is
permissible to learn from them, provided their words are connected to
the utterances of the Point. Otherwise, how can it point to the Tree of
Truth if its words lack the remembrance of the Reminder? In such cases,
there is no mention of God within their remembrance.

It is forbidden to engage in the composition of matters that
\emph{``neither nourish nor satisfy,''} such as principles of logic,
jurisprudential rules, legalistic reasoning, metaphysical philosophy,
non-practical linguistic studies, and similar subjects. Likewise,
detailed studies in grammar and morphology are discouraged unless they
serve a practical purpose. For instance, understanding only what is
necessary to distinguish the subject from the object and their related
elements is sufficient for learners. Beyond that, God does not forgive
those who occupy themselves unnecessarily with such pursuits, even if
most people in this Dispensation are content with Persian expressions.

If someone wishes to understand the Bayán, they should only acquire the
amount of knowledge they need, no more beyond the essence of the Bayán,
and not apart from it. This is the straight path for learners and
educators until the day when the Tree of Truth is made manifest. On that
day, the Book of Truth will speak, and to return to the silent Book is
to veil oneself from the speaking Book. This silent Book is sanctified
and exalted beyond the scientific and practical pursuits of humanity.

In this Dispensation of the Point, all who have observed His reality
have seen that He is free from the knowledge of grammar, morphology,
logic, jurisprudence, principles, and anything derived from these. For
all such sciences exist to comprehend the command of God in His word.
Yet, for one whose purpose aligns with God's purpose and whose word is
God's word, what need has He for such pursuits? God has endowed Him with
a power and eloquence such that, if a swift scribe were to write
continuously for two days and nights without pause, the amount revealed
from that source of utterance would equal an entire Qur'an. If all the
thinkers on earth gathered, they would not comprehend even a single
verse from it, let alone produce its equal or speak with its knowledge.

This is a divine gift bestowed upon whomever God wills, as He wills,
through what He wills, for whatever purpose He wills. Indeed, He is the
All-Powerful, the All-Knowing. Let none become veiled from their Beloved
during His manifestation by such pursuits. God revealed in the Qur'an:

\emph{``It is God who created seven heavens and of the earth their like.
The command descends among them so that you may know that God has power
over all things and that God encompasses all things in knowledge.''}

Yet at the time of fruition, people sought only to express their
incapacity before the manifestation of power and to display their lack
of knowledge before the manifestation of knowledge. All the while, they
were created for this very acknowledgment.

If someone attains certainty in the verses revealed by \emph{Him whom
God will make manifest}, who is the manifestation of God's power and
knowledge, they will have achieved the fruit of the Bayán. Otherwise,
there are countless individuals who recite the Qur'an yet remain veiled
from its fruit---acknowledging the power and knowledge of God as
manifested in the one who is the \emph{Qá'im of the Family of Muhammad}.

Consider how, in this verse, nothing is seen but the power and knowledge
of God. Similarly, in the manifestation of \emph{Him whom God will make
manifest}, nothing is seen but the power and knowledge of God.
Everything has been created for this purpose, yet today, people remain
veiled from Him, failing to recognize the manifestation of His power and
knowledge. His knowledge exists within Himself by Himself, and in His
creation through His creation. His power is over all things, but only
within Himself through His divinity, and within His creation through His
lordship.

This is the essence of all essences, yet people remain veiled from it
due to their reliance on speculative sciences. If such sciences are not
coupled with faith in Him, they become as nothing. The believer's faith
grants them true dignity, even if they lack such knowledge. The essence
of all knowledge is established for the believer, as they affirm it by
their faith.

Thus, it is as if all others are as lifeless beings, failing to perceive
the fruit of knowledge and wisdom. The gravest of people at the time of
\emph{Him whom God will make manifest} are those who take pride in their
learning, yet fail to comprehend that all their knowledge was meant to
understand His words prior to His appearance. At the time of His
manifestation, what need remains for these sciences, when His words are
evident, and His purpose is dominant?

This has been observed in every manifestation, as evidenced in this one,
where all have witnessed this truth. Indeed, God establishes truth, and
He is the best of judges.

\subsection*{Gate 11 (Those Who Exceed the Limits of the
Bayan}\label{gate-11-those-who-exceed-the-limits-of-the-bayan}
\addcontentsline{toc}{subsection}{Gate 11 (Those Who Exceed the Limits
of the Bayan}

\textbf{The Eleventh Gate of the Fourth Unity}

On those who exceed the limits of the Bayán and cannot be judged as
believers, regardless of their knowledge or ignorance, whether a king, a
servant, or a slave.

The summary of the gate is at the time of the revelation of the Bayán,
God gazed upon all His creation and ordained boundaries for every soul,
regardless of their station. These boundaries ensure that no soul in the
paradise of the Bayán suffers distress or deprivation. Instead, God
decreed for everything to be elevated to the pinnacle of its potential
through the actions of those with the power to do so, ensuring no one is
excluded from their rightful paradise.

For instance, consider a tablet upon which beautiful verses have been
written. Its paradise lies in being adorned with gold, skillfully
crafted designs, and embellishments of the highest attainable standards.
When its owner brings it to the highest degree of excellence within
their capacity, they fulfill their obligation. However, if a higher
degree of beauty is possible in their knowledge but they fail to provide
it, they deprive it of its paradise. God will hold them accountable for
withholding grace despite their ability.

Any soul that transgresses the boundaries of the Bayán, whether the
highest or lowest of creation, is not judged as a believer.
Nevertheless, every soul must be mindful not to pass judgment on
\emph{Him whom God will make manifest}. Before His appearance, none can
fully know the essence of their own soul, which is their highest
paradise. At the time of His manifestation, He alone will decree upon
every soul.

At the moment of His appearance, no one has the right to question His
decrees with ``why'' or ``how,'' for all the boundaries of the Bayán
originate from Him. The stages of His manifestation will unfold in the
same way as the revelation of the Bayán, letter by letter, and verse by
verse. He will manifest to the dwellers of His sovereignty just as the
Point of the Bayán manifested, beginning with His illumination upon two
souls gradually and continually, be vigilant at the time of His
appearance, for when it occurs to a soul, it should not be overwhelmed
by the magnitude of what it witnesses. They will see that all the people
of the Bayán rise by His name, and in an instant, they may hear that He
is the one who has always been with them, night and day.

Reflect on the same scenario with the Point of the Qur'an, and before
Him, with Adam, and afterward, continuing to infinity. Seek refuge in
God so that the first stirrings within their soul do not occur without
proper recognition of the Truth. Strive to attain the honor of being the
first to believe, even if the unity of His cause appears overwhelming.
God's proof remains conclusive, even at that moment of unity.

Consider this: today, the greatest proof of the Messenger of God is the
Qur'an. Countless people believe in it now, yet this very proof existed
in its entirety in the early days of Islam, during which, for seven
years, not a single soul acknowledged that supreme Light. Regardless,
each person must remain vigilant not to transgress the bounds of the
Bayán.

This principle also applies to the Qur'an. Consider the verse:

\emph{``And whoever does not judge by what God has revealed, it is they
who are disbelievers.''}

This verse, along with others found in the Qur'an, was revealed as a
warning, yet observe how many have transgressed the bounds of God's
decrees. Be careful not to misapply this verse to the decree of
\emph{Him whom God will make manifest,} for what God reveals in His time
is the criterion. Just as the revelation of the Qur'an was the Word of
God in the time of the Messenger of God, not the Gospel, so too is the
Bayán the Word of God today, not the Qur'an.

Thus, people may act with the intention of serving God, yet they act
against Him doing what they do because the Eternal Essence is sanctified
above being subject to any command or judgment. Instead, these matters
pertain to the manifestation of His primal will in the reality of the
First Truth. As stated explicitly in the Qur'an:

\emph{``Indeed, those who pledge allegiance to you are actually pledging
allegiance to God.''}

This principle applies to all matters: judgment upon Him is judgment
upon God. Furthermore, this understanding has been expounded by the
people of the House (\emph{Ahl al-Bayt}) and applied to themselves as a
secondary reality. For example, the \emph{Ziyárat Jámia Kabíra} is
filled with such expressions:

\emph{``Whoever knows you has known God, and whoever is ignorant of you
is ignorant of God. Whoever loves you has loved God, and whoever hates
you has hated God.''}

God knows how extensively this supplication is recited by the Shí`a
sect. Yet, despite their devotion, the Imams returned to the transient
world, and no one recognized them. Day and night, people recite this
supplication, unaware of its true meaning. This does not mean that the
proof of God has not been made clear.

Consider the time of the Messenger of God: Jesus foretold His coming,
saying, \emph{``One will come after me, whose name will be Ahmad.''}
Despite this, many Christians failed to recognize Him, and to this day,
they remain awaiting the fulfillment of that promise. Vast numbers
across Europe cling to this belief, awaiting its realization. This
failure stems from their lack of reflection on the proof of the
Messenger of God, which is the Qur'an. Had they reflected, they would
have observed the fulfillment of Jesus's promise at the time of His
appearance.

Their continued veiling has rendered them eternal inhabitants of the
fire, while in their minds, they remain followers of the true faith,
still waiting.

O people of the Bayán, have mercy on yourselves! Do not fall into this
state at the time of the appearance of \emph{Him whom God will make
manifest} and the position of the Manifestation, as the arbiter of the
Ahl al-Bayt's decree, has been applied to a third reality regarding
their Shia followers. This is evident in the veneration of Islamic
jurists (\emph{mujtahideen}), whose authority derives from the sayings
of the Imams (\emph{peace be upon them}). One such saying, in an
extended tradition, underscores the principle that:

\emph{``He who rejects them rejects us, and he who rejects us rejects
the Messenger of God, and he who rejects the Messenger of God rejects
God.''}

The principle has even been extended further, applying to believers. For
example, in the \emph{Tafsir of Surah al-Baqara,} a tradition states:

\emph{``Whoever gladdens a believer has gladdened me, and whoever
gladdens me has gladdened God.''}

Reflect, then, on how far humanity has descended through these four
degrees. Yet, in the realm of reality where none is seen but God, the
verse \emph{``Do not place yourselves before God''} applies directly to
Him. How can people bear to transgress this principle?

While some have interpreted this verse as referring to refraining from
placing oneself before the Messenger, this is contrary to its intent.
The verse explicitly mentions the Messenger, but its ultimate purpose is
the Sun of Truth, which is the highest aspect of the Messenger. In Him,
nothing is seen but God.

The multiplicity of these matters in this gate is intended to ensure
that none become veiled from the purpose. Indeed, God is the best of
judges.

\subsection*{Gate 12 (All Prior Holy Sites are
Abrogated)}\label{gate-12-all-prior-holy-sites-are-abrogated}
\addcontentsline{toc}{subsection}{Gate 12 (All Prior Holy Sites are
Abrogated)}

\textbf{The Twelfth Gate of the Fourth Unity:} On the cessation of all
holy sites that existed before the new Dispensation.

The summary of this gate is this: In every manifestation that appears
from God, the holy sites associated with the previous Dispensation are
abrogated. For example, today, no one in the religion of the Messenger
of God knows the locations or even the names of the successors of Jesus,
let alone the sites of their graves.

Similarly, after each manifestation, the sanctity of burial sites that
were revered above the earth is lifted. This applies to all past sacred
places after the appearance of the new Dispensation, from the
Manifestation of Prophethood to the Manifestation recognized by the
Shia, even to the graves attributed to previous prophets in places like
Kufa or other locations---these must also be abrogated. Similarly, sites
like the Holy House (\emph{Bayt al-Maqdis}) have their significance
superseded with each new manifestation. Consider how the followers of
Moses revere Jerusalem, the followers of Jesus honor sites associated
with him, and the followers of the Messenger of God (\emph{Muḥammad})
journey to the Kaaba.

The sanctity of physical locations lies not in the clay itself but in
the essence of divine command (\emph{amrullāh}). Wherever it appears, it
bestows sanctity. Observe today how people visit the graves of
descendants of the Imams, believing them connected to divine truth,
while remaining veiled from the Manifestation who establishes
prophethood, guardianship, and the entirety of religion through the
verses revealed to Him. Otherwise, He would not reside in solitude on
Mount Maku.

Do not be astonished by this phenomenon. Reflect on the appearance of
the Messenger of God in the past. The followers of Jesus adhered to
their religious practices, and monks worshipped in their monasteries,
while the Sun of Truth remained hidden for seven years on a mountain in
Mecca, veiled by individuals who failed to recognize His station. From
the moment of His mission, the truth was severed from every rightful
claimant among the followers of Jesus, except for those who came under
His shadow.

After the appearance of the Point of the Bayán, consider how every year
countless people travel to Mecca, performing circumambulation, while the
one whose word establishes the sanctity of the Kaaba resides alone on a
mountain. That very person is none other than the Messenger of God, for
the command of God is like the sun. Even if it rises infinitely, it is
still the same sun, upon which all depend. This is why, in every
subsequent manifestation, the appearance of God is the most exalted.

Regarding the Manifestation, in every station where He appears, it is
evident that all previous manifestations were created for the Messenger
of God (\emph{Muḥammad}), and all manifestations including His own were
created for the \emph{Qá'im} of the Family of Muhammad. Similarly, all
manifestations, including this one and the appearance of \emph{Him whom
God will make manifest,} were created for the subsequent manifestation
after Him. All of these appearances, as well as those that follow, were
created for the manifestation after the next, and so on to infinity. The
Sun of Truth rises and sets eternally, without beginning or end.

Blessed is the soul who, in every manifestation, understands the will of
God specific to that appearance. Such a person does not look to previous
matters and become veiled from the present manifestation. Today,
everything established in Islam was made possible through the mission of
the Messenger of God, just as all that was abrogated from the religion
of Jesus occurred due to the same mission. Similarly, in the
manifestation of \emph{Him whom God will make manifest,} it will become
clear that everything in the Bayán was brought about through the
manifestation of the Seven Letters (\emph{Ḥurūf-i-Sab`}).

On the day of His appearance, if people focus on the origin of the
matter, they will not remain veiled from Him by the elevated
manifestations in the Bayán. For every decree, countless souls lie
dormant, clinging to those past ordinances with pride and action, as is
observed today. All these manifestations of the Bayán, however, are
under the shadow of a single word from Him. If He declares, \emph{``We
have raised it,''} it will be elevated. Yet He does not command its
elevation until something greater is revealed.

Similarly, observe this manifestation: God will not command its
abrogation until something greater than the previous manifestation
appears. Indeed, God has power over all things.

\subsection*{Gate 13 (Elevated Sites May Only Be Utilized for Their
Intended
Purposes)}\label{gate-13-elevated-sites-may-only-be-utilized-for-their-intended-purposes}
\addcontentsline{toc}{subsection}{Gate 13 (Elevated Sites May Only Be
Utilized for Their Intended Purposes)}

\textbf{The Thirteenth Gate of the Fourth Unity:} On the cessation of
former sites and the decree concerning those who possess something of
value of elevated sites above the earth, and that they may only be
utilized for their intended purposes.

Let it not remain hidden that there is no day or night without the
presence of manifestations of truth and falsehood appearing in all
realms upon this earth, both outwardly and inwardly. All human souls,
from the time of Adam until today, have been embodied in human forms,
enjoying their stations. Likewise, disbelieving souls are tormented in
infernal forms. Yet in every manifestation, a specific decree is
revealed, and all are commanded to follow it. After the setting of each
manifestation, its ruling is abrogated until the appearance of the next,
where it remains latent.

In this Dispensation of the Bayán, God has not desired to see elevated
places except for those associated with Muhammad, the Family of
Muhammad, and the Gates of Guidance. These are expressed in this
manifestation as the ``Letters of Unity'' (\emph{Ḥurūf-i-Wāḥid}),
representing nineteen exalted stations. God has also desired that these
exalted stations include those of the prophets, the truthful ones, the
martyrs, and the believers, whose hearts are the loci of divine names
and attributes. These are illuminated under the shadow of these nineteen
exalted stations.

This ensures that the matter is not made overly burdensome for the
people. Were someone to attempt to enumerate all the exalted stations,
they would fail, for all multiplicity is encompassed within the
singularity of the First Unity, which is accounted for before God and
those endowed with knowledge.

Blessed is the soul that believes in the elevation of these exalted
stations, for they are the loci where the angels of the heavens, the
earth, and what lies between them descend. It has been decreed in God's
eternal knowledge that they are elevated and will continue to be so.
None can resist God's will, for He is dominant over all possibilities
and manifest over all existences.

Blessed is the soul that becomes an instrument for the manifestation of
His will, for such a soul is the finest guardian of the boundaries God
has promised in the Bayán that for every measure of gold given, He will
multiply it manifold, record a thousandfold reward for the giver, and
bestow upon them what pleases their soul. Truly, God has power over all
things. Regarding the elevated sites above the earth, no one is
permitted to utilize them except for their specific purpose within those
exalted stations. These sites belong to themselves unless the Sun of
Truth appears and grants permission otherwise, for He is the decree of
God concerning all things.

Until this day, no ruling has been revealed other than this. God, who is
all-knowing, is aware of what will come to pass. Before the appearance
of \emph{Him whom God will make manifest,} there is no obligation upon
individuals beyond what has been decreed. Indeed, God is all-knowing of
all things.

\subsection*{Gate 14 (Exalted Sites are a Place of
Refuge)}\label{gate-14-exalted-sites-are-a-place-of-refuge}
\addcontentsline{toc}{subsection}{Gate 14 (Exalted Sites are a Place of
Refuge)}

\textbf{The Fourteenth Gate of the Fourth Unity:} On those who seek
refuge in these exalted sites and are pardoned as God has decreed.

The summary of this gate is anyone who seeks refuge in the elevated
stations of the Letters of Unity (\emph{Ḥurūf-i-Wāḥid}) has the right to
be granted protection by the people. These lands are the manifestations
of divine shelter across the earth, representing the attribute
\emph{``He grants refuge but none can grant refuge against Him.''} This
decree ensures that on the Day of the appearance of \emph{Him whom God
will make manifest,} the Day of Resurrection for this Dispensation,
those who take shelter in the Letters of Unity may be spared from
destruction---not physical destruction, but destruction of faith.

All deeds are performed with the intention of drawing nearer to God.
Yet, if people fail to take refuge under the shadow of the true
manifestation on that day, they will be spiritually annihilated. If a
person of insight exists upon the earth, they would give everything they
possess to avoid being excluded and instead be included among those
protected, for this is the ultimate fruit of human creation: to ensure
humanity is not wholly obliterated.

This does not mean seeking refuge in physical locations but rather
aligning with the spiritual reality of these elevated stations when
visiting or mentioning them, as was seen during the Dispensation of the
Qur'an. People would visit the graves of these Letters and remember them
day and night. However, when the Letters themselves appeared with the
proof upon which their religion was based, delivering the message of the
Point of the Bayán, they became veiled by worldly affairs. Thus, what
transpired did indeed come to pass.

Today in Islam, the people act upon the sayings of their predecessors,
issuing rulings upon themselves based on those teachings. Were it merely
a matter of this, no harm would come upon them. Yet, they expend
countless wealth in commemorating the martyrdom of the Master of Martyrs
(\emph{Sayyid al-Shuhadá,} peace be upon him) while disregarding what
they did to his predecessors. They have committed deeds under the guise
of Islam, actions so grave that even the word ``falsehood'' (\emph{ifk})
is too noble to describe them.

Under the name of Islam, they have carried out such acts, including
those against the Manifestation of Islam, who is the Messenger of God
(\emph{Muḥammad}). Take heed, O possessors of insight, and observe with
clarity in the religion of God. They attribute the name of God to their
deeds, yet they act against those manifestations upon whom judgment is
equivalent to judgment upon God. Unknowingly, they stray from the
religion while imagining themselves to be walking the highest path of
caution and diligence.

O people of the Bayán, resolve among yourselves not to impose upon any
soul what you would not accept for yourselves. Perhaps, on the Day of
Manifestation, you will not act toward the Letters of Unity as others
have acted. Reflect on all the rewards promised for reciting Qur'anic
chapters, as narrated from your Imams, and the rewards associated with
the prayers you recite. All these, on the Day of Resurrection, return to
the shadow of Muhammad and the Family of Muhammad, under whom you will
be gathered. This is the greatest favor that the proofs of God have
bestowed upon their loved ones, promising them such rewards. Praise be
to God, who has established the Day of Resurrection, allowing them to
come forth, witness the proofs of divine oneness, and endure what they
endured. Yet, they continue to recite Qur'anic verses and prayers in
pursuit of these rewards.

Do not be astonished by this. Consider the followers of Jesus, who are
still awaiting the fulfillment of the promise, \emph{``One will come
after me, whose name is Ahmad.''} Day and night, they supplicate for his
arrival, yet he came, and 1,270 years have passed since his advent. His
Resurrection occurred, yet they remain in expectation.

Have mercy upon yourselves and do not render your deeds as scattered
dust (\emph{habā'an manthūrā}). Elevate your souls such that, if they
are not for any particular purpose, they are at least not against
themselves. This is the path to salvation in this world and the
next---if you act upon it.

No manifestation occurs without its representative being the epitome of
sincerity and offering guidance in their time. This is so that the Day
of Fruition---the subsequent manifestation---does not render the
previous followers annihilated. To this day, the fruit of past followers
remains unfulfilled. Strive, then, in the Dispensation of the Bayán to
make yourselves an honored community among the nations on the Day of
Resurrection.

In the Day of \emph{Him whom God will make manifest,} all who have been
created since the time of Adam will be present upon the earth, along
with souls from realms before Adam and possibilities from realms after.
Among all nations, distinguish yourselves with faith in Him. If one of
you is veiled, you will be the most condemned among all peoples in His
sight. Yet, if you manifest the fruit of your existence, you will be the
most honored among all.

Do not disgrace yourselves with what emerges from Him on the Day of
Resurrection to come. For as you have heard, when judgment is pronounced
on the Day of Resurrection, all will become aware. This is intended,
that the essence of all creation on that day consists of believers in
that manifestation. Until the next manifestation, the decree pronounced
will be recited, witnessed, and acted upon by all. For instance, the
mention of \emph{Abū Lahab} in the Qur'an has, from the moment of the
verse's revelation to today, been read by countless individuals, serving
as a testament to his actions. This is his disgrace.

Reflect on how his veiling from the truth during the few days of his
earthly life has caused him to be dishonored before every soul who
recites the Qur'an, even to himself, as he exists in the station of his
own fire. This is the intended meaning of the judgment of the Day of
Resurrection being heard by all.

\emph{Watch for it, O possessors of insight, and then guard yourselves.}

\subsection*{Gate 15 (No One May Prevent Refuge at an Exalted
Site)}\label{gate-15-no-one-may-prevent-refuge-at-an-exalted-site}
\addcontentsline{toc}{subsection}{Gate 15 (No One May Prevent Refuge at
an Exalted Site)}

\textbf{The Fifteenth Gate of the Fourth Unity:} That no one may prevent
another from reciting or seeking refuge in these exalted sites.

The summary of this gate is if anyone seeks refuge in the elevated
stations, no one has the right to prevent them. Such a prohibition would
remove the oppressor's dominion over the seeker. If the seeker is
honorable, they are granted refuge even within their station. For
example, if someone in the East of the earth seeks refuge from another,
they are to be given refuge by these manifestations of divine shelter,
out of reverence for God, the Almighty and Exalted, who is the best of
protectors and helpers.

Mention of the Sacred House (\emph{Bayt al-Ḥarām}): God has never had,
nor will He ever have, a fixed location. In every manifestation, the
land associated with His will becomes His House, the site of
circumambulation for the angels of heaven and the inhabitants of the
earth. All revolve around the divine decree, which is made manifest
within the earthly clay.

If the sanctity were in the clay itself, it would have remained
eternally unchanging. However, it is evident to the pure-hearted that
such clay is a reflection of the divine decree, just as the decree
itself is reflected within creation, and just as the divine decree
(\emph{amr}) is like the sun, so too is the Sacred House (\emph{Bayt}).
Even if the locations of the Sacred House change infinitely, it remains
one House. This is why the transition from one site to another parallels
the appearance of the divine will in subsequent manifestations.

The same clay associated with God in the Day of Adam is the very clay
associated with Him today. Similarly, the decree that sanctified the
House in those days is the same decree that sanctifies it now. It is the
station of \emph{Him whom God will make manifest}, the true locus of
divine revelation. That House is the manifestation of the Sacred House,
and its dimensions align with the name of God in its length, breadth,
and height. Its structure stands as a reflection of divine intention.

If the believers in God had the capacity, they would be commanded to
build it from pure diamond, filling it with water to its height and
transforming its soil into the philosopher's stone. The scent emanating
from it would be that of the finest fragrance. However, since such
capabilities are not observed, the House is made appropriate to whatever
state fulfills the concept of elevation.

Its outward and inward aspects, if crafted with clarity like a mirror,
would be purer and closer to transparency than any other earthly design.
Today, this understanding is present among the skilled craftsmen who
excel in their crafts. On the earth, there exists a mosque with a
structure modeled after the Kaaba at its center. This design was not
established except as a sign in anticipation of God's decree for
elevating the Sacred House in that land.

It serves as a token of divine grace for that region. Blessed is the one
who remembers God there, for I have remembered God there for those who
raise it. In this way, God rewards the righteous, and He recalls those
who remember Him, even through such acts. Truly, He is the best of those
who remember.

The significance of the Sacred House lies in its symbolic representation
of the House of divine unity, glorification, and praise. The elevation
of the House have built the House, looking upon its manifestations, so
that at the time of the appearance of \emph{Him whom God will make
manifest,} they would not remain veiled from the True Reality of the
House. It was this very structure that, 1,270 years ago, was decreed for
pilgrimage (\emph{ḥajj}). Each year, seventy thousand souls
circumambulate around it, yet they remain unaware of the true purpose of
the House and fail to attain its fruit. Despite the passage of time
since the appearance of the true reality of the House, not a single soul
has fully understood its purpose or grasped its fruit.

The real \emph{Bayt al-Ḥarām} (\emph{House of God}) is the hearts of
those who believe in Him, which are the hearts of those who believe in
\emph{Him whom God will make manifest.} Today, those who consider
themselves believers in the Bayán act upon what was revealed in the
Qur'an, but they do not truly believe in Him. If the people had truly
circumambulated the House of Reality, the command of \emph{ḥajj} would
not have been tied to the physical House.

Because they failed to do this, their burden has been placed upon their
own shoulders, compelling them to circumambulate the clay associated
with Him. This act was decreed so that they might recognize their limits
and, on the Day of His Appearance, not remain veiled from Him. This is
the fruit of \emph{ḥajj,} which was instituted in service of His
command, so that by this means they might ascend toward His very self on
the Day of His Manifestation.

In the Dispensation of the Qur'an, this fruit was not realized. Seventy
thousand souls circumambulate around the House, yet the one who is the
true reality of that House resides upon Mount Maku, with only one soul
in His presence. How could the fruit have been attained? It would have
been fitting that, at the time of His appearance, all the believers in
the Qur'an---those who circumambulate the clay associated with His
command---would, in His presence and to no end, circumambulate His very
self.

However, the same individuals who once journeyed on foot to the House
now render such judgments against Him. And those who spent their wealth
in service to His House now fail to offer even a single piece of land to
reside upon. Instead, they prevent Him from even inhabiting the earth
itself. This is the condition of the people, always moving without
awareness, and on the Day of Attaining the Result, they become like the
dead, oblivious to God's intention behind His commands.

Awaken, O people of the Bayán, and prepare yourselves for the appearance
of the true reality of the House (\emph{Bayt}). He takes pride in those
who circumambulate His House in their innermost hearts, witnessing them
and showering them with forgiveness. Even if, during the pilgrimage,
some show kindness to one another, such acts are beloved. I myself
witnessed, during my journey to Mecca, individuals making significant
expenditures while withholding something as small as a cup of water from
their companions who shared the same quarters. This occurred aboard a
ship, where scarcity rendered water precious. From Bushihr to Muscat, a
journey lasting ten days, I had to subsist on brackish water due to the
impossibility of carrying fresh supplies.

Be vigilant over yourselves and ensure that you bring no sorrow upon any
soul. For the hearts of the believers are nearer to God than the House
of clay. Seek nearness to God within His House, where prayers are
accepted. Whoever spends even the smallest measure in the path of God
will be rewarded two thousandfold in this world at the Plain of `Arafát,
the site of God's recognition.

Whoever has the ability to journey without causing sorrow and chooses
not to go, at the time of death will have their soul taken by one who
believes only in the book of the previous Dispensation. However, nothing
in the context of pilgrimage is greater than acquiring noble character,
ensuring that one neither causes grief nor experiences grief through
interactions with others. On the path to Mecca, there are actions more
detestable than any other before God, nullifying their deeds, as is the
case with disputes among pilgrims. Such disputes are forbidden under all
circumstances. The conduct of believers has always been marked by
patience, forbearance, modesty, and tranquility, and it will remain so.
The House itself rejects such individuals who engage in disputes while
circumambulating it.

Guard yourselves, for the religion of God is broader than all things.
For someone whose journey to the House requires crossing the sea, they
are absolved from such travel if they cannot endure the hardships. If,
however, they demonstrate steadfastness and contribute even minimally to
the wellbeing of a believing soul from among their close relations,
their pilgrimage will be accepted by God and forgiven in His presence.

This command has not been given except to ensure that sorrow does not
arise in the path to God. In the depths of the sea, fear alone is
conceivable, and it is not possible to perform the pilgrimage relying
solely on human means. If traders are also prohibited from traveling by
sea in places where alternatives exist, this is closer to the spirit of
this religion, fostering their tranquility. Otherwise, the world's order
would be disrupted, as those who rely on the sea have no other means.
With their actions, they draw nearer to God, and God rewards the
righteous, whether they are on land or sea.

God doubles the reward for His servants who travel by sea, given the
hardships they endure. If they act in accordance with God's religion and
move together in the spirit of love, their efforts will be accepted. God
rewards the righteous.

\subsection*{Gate 16 (Sovereign Leaders Should Ensure the Flow of
Information)}\label{gate-16-sovereign-leaders-should-ensure-the-flow-of-information}
\addcontentsline{toc}{subsection}{Gate 16 (Sovereign Leaders Should
Ensure the Flow of Information)}

It is incumbent upon any sovereign whose kingdom contains the sacred
precincts (\emph{ḥaram Allāh})---indeed, upon every ruler of any
domain---to establish workers throughout their lands. These workers must
ensure the flow of information and communication across their
territories. In European lands (\emph{Farang}), this system is arranged
with great excellence, efficiently managing communications across vast
distances. The crescent moon would be observed over several nights and
days, but this matter should now be made universal so that all may
benefit from this means of gaining news. \emph{Him whom God will make
manifest} will inevitably appear, and if, in the lands where He arises,
systems of delivering messages and exchanging letters are established
universally, the servants of God will sooner attain the honor of His
guidance.

If a soul hears of the manifestation even the smallest fraction of a
moment earlier and believes, it is better for them than owning all that
is upon the earth and spending it in the path of God. Thus, this command
has been given to organize systems of communication so that, on the day
of the appearance of that Supreme Luminary, the means for guiding His
servants are readily accessible.

Until this organization becomes universal, it will not extend its
benefits to all the servants of that court, except when all people have
access to means of communication. Even today, couriers (\emph{chāpār})
exist for the ruling class, but what benefit is derived if the weak and
disadvantaged have no access to such systems?

It is incumbent upon every ruler to ensure that communication spans
their entire land, accessible to all, so that if even the humblest
person in the most remote region desires to find the Sun of Guidance,
they may do so through a well-ordered system. God indeed loves those who
maintain order.

\subsection*{Gate 17 (Any Authority Who Wishes to Elevate the House May
Take Possession of the Area Surrounding the
House)}\label{gate-17-any-authority-who-wishes-to-elevate-the-house-may-take-possession-of-the-area-surrounding-the-house}
\addcontentsline{toc}{subsection}{Gate 17 (Any Authority Who Wishes to
Elevate the House May Take Possession of the Area Surrounding the
House)}

\textbf{Seventeenth Gate of the Fourth Unity:} The area surrounding the
House may not be sold, and any authority who wishes to elevate the House
has the right to take possession of its vicinity, even without the
consent of its current owner, for God has a greater right over His
dominion than any servant who possesses it for a number of years.

Any person with authority who desires to elevate the Sacred House
(\emph{Bayt}) along with the Mosque of the Sacred Precinct (\emph{Masjid
al-Ḥarām}) has the right to acquire whatever they require from the
surrounding area. No individual may object to this, for ownership
belongs solely and independently to God, who is the true Master of all
things. He alone has the ultimate right over His dominion, especially
over the House that represents His very essence. This is the decree of
God, even if it is displeasing to someone, for what is expressed is the
will of God. It is incumbent upon all to be content with what God has
commanded, for He has created them so that they may recognize His
ownership over all things. God loves those who are pious.

\subsection*{Gate 18 (Pilgrimage Without
Hardship)}\label{gate-18-pilgrimage-without-hardship}
\addcontentsline{toc}{subsection}{Gate 18 (Pilgrimage Without Hardship)}

\textbf{Eighteenth Gate of the Fourth Unity:} No one may ascend to the
Sacred House except through wealth sufficient to ensure they face no
hardship along the way. Upon arrival, four \emph{mithqāl} of gold must
be given to those who serve the first, second, third, and fourth pillars
of the House, to be divided among themselves. It is forbidden for them
to request anything beyond what is offered voluntarily, except from
those who visit them. Servants, those traveling on the path, the poor,
and those unable to make the journey are exempt.

The summary of this gate is the obligation to perform the \emph{ḥajj}
has been prescribed only so that those ascending toward it may rejoice
in God's good pleasure. The duty is lifted from those who are unable to
afford it so that they are not burdened on the path.

This obligation applies only once in a lifetime for each individual, so
as not to impose undue hardship. Additionally, the purchase of
pilgrimage on behalf of the deceased has been prohibited, ensuring that
individuals may, during the appearance of the Truth, personally attain
the meeting with their Lord or, in times of concealment, achieve His
proximity through their place of repose or by acts undertaken during the
preceding manifestation.

If someone is required to perform the pilgrimage but does not, and death
overtakes them, yet they had intended to go, it is upon God to grant
them the best of rewards and admit them into Paradise with the greatest
blessings. The obligation of pilgrimage has been lifted from women so
that no undue hardship befalls them on the path. Permission has been
granted for the residents of the Sacred Precinct (\emph{ḥaram Allāh}) to
perform the pilgrimage annually, as it is less difficult for them
compared to others. What soul is there that dwells in the land of God's
House and does not circumambulate around it?

It has been ordained that visitors to the House should gift four
\emph{mithqāl} of gold, measured according to the Bayán standard, where
each \emph{mithqāl} equals nineteen grains. This gift is to be given to
the nineteen souls who reside near the House, elevated upon their
appointed thrones. These individuals are tasked with being steadfast at
the Pillar of Praise, symbolizing the manifestation of the Point of the
Bayán.

These appointed ones have been instructed to show utmost respect and
honor to those who visit the House, refraining from requesting offerings
from them, thereby ensuring that the visitors fulfill their
responsibilities willingly and without coercion. This approach is nearer
to dignity and exaltation.

The nineteen individuals are to divide the divine gifts equally among
themselves every year, expressing gratitude to their Beloved for His
bounty. They are to use these gifts to adorn the sacred surroundings
with divinely ordained materials, as described in the Arabic text:
elevated, colorful thrones -- white for the first pillar, yellow for the
second, green for the third, and red for the fourth. These arrangements
symbolize sublime and incomprehensible truths known only to the people
of insight.

All of this is to prepare for the day of the appearance of \emph{Him
whom God will make manifest}. On that day, no proof will remain for the
pilgrims except to turn toward Him, no preservation of the House will be
required other than the preservation of His command, no service of the
appointed places will be needed other than service to Him, and no aspect
of religion will have a function other than directing hearts toward His
will except His command. This is the true purpose, if one comprehends.
Exemption from the obligation of the four \emph{mithqāl} of gold has
been granted to those who cannot afford it, to servants, the young, and
those who face hardships on the path. This exemption is out of God's
grace and mercy, ensuring no undue burden falls upon those fulfilling
their duties.

All these ordinances are decreed by the One whose will resides between
\emph{kāf} and \emph{nūn} (``Be, and it is''). Perhaps a soul may reap
the fruits of these ordinances on the day of the manifestation of His
command. In every age, sacred souls have been custodians of the House of
Truth, bearing witness to the innermost mysteries and embodying the
outward manifestations of divine will.

The structure of the House itself, marked by eight days and the measures
of its height, symbolizes the honor and dignity bestowed upon it, a
burden carried by all before and after. There exists no being, in any
rank, that is not subject to the ordinances of God and humble before His
primal truth.

From the first day of Adam to the present, all have remained in
obedience to His will, whether consciously or not. Even those veiled
from the manifestation of truth, from the highest ranks to the lowest,
live their lives under His decrees, unknowing participants in His plan.

For instance, though the followers of Christ did not prostrate before
the Messenger of God, the commands of Christ upon their lives were, in
essence, their prostration to Muhammad. The manifestation of Christ in
His time was the manifestation of truth, just as it has been in every
prior dispensation and will be in every subsequent one.

In all things, no existence has independent being apart from its essence
being grounded in the reality of God's command. They prostrated,
humbled, and submissive to the Point of Truth, in remembrance, devotion,
and worship of God through Him. Yet they remained unaware, for if they
had recognized Him, they would have wholly turned away from their own
selves and devoted themselves entirely to Him.

Consider those who recognized the Messenger of God---how they believed
in Him---and those who did not, remaining in the fire of veils. Reflect
likewise on the past manifestations and those to come. This is the
majesty of God: all take pride in their servitude to Him if they truly
accept it, but He only accepts from the sincere.

Twelve hundred and seventy years have passed since the advent of
Muhammad, and countless have circumambulated the House each year. Yet,
in the final year, the one who established the House went on pilgrimage,
witnessing multitudes from every group. Yet none recognized Him, though
He knew them all, moving within the embrace of His command.

Among them, only one person, enduring the test of eight unique days, was
recognized by Him in full sincerity and detachment for the sake of His
good pleasure. God glorified this soul in the heavenly realms for their
pure devotion---not because they received a special favor, but because
they alone did not veil themselves from the favor extended to all.

That year, the unveiling of the Book interpreting the Surah of Joseph
reached all, yet when people examined it, they found no companions in
faith to confirm it, so they all hesitated. They failed to consider that
the very Qur'an, which now has countless believers, had only one
apparent believer in its first seven years---Amir al-Mu'minin (the
Commander of the Faithful), peace be upon him. However, this soul,
seeing the proof of the Proof, remained steadfast and assured. Others
considered none but their own understanding. On the Day of Resurrection,
God will question every soul according to their comprehension, not by
their imitation of another.

Many a soul, upon hearing the verses, becomes humbled and acknowledges
the truth but fails to follow it. Thus, all are individually
accountable, not reliant upon another. During the advent of \emph{Man
Yuzhiruhu'llah} (He Whom God shall make manifest), the most learned of
scholars will stand equal in judgment to the simplest of beings. Often,
the humblest will recognize and believe, while the most learned will
remain veiled.

In every dispensation, some follow others into the fire of veils. If
every soul acted in accordance with their own understanding, the innate
purity of people would remain untainted. They would not regard the
majesty of worldly knowledge but rather the knowledge that magnifies
true glory.

Consider that one soul in the past manifestation who recognized the
proof without looking for the affirmation of others. In God's sight,
this individual stood in truth. But those who depended on others for
validation were veiled, thereby losing the essence of the pilgrimage to
the House of God, which is to know the sanctity of the One associated
with the House.

That single soul performed a true pilgrimage with believers who, though
few, affirmed the Truth. The majority, however, who did not affirm Him,
lost their station despite their outward actions. They did not lack the
message; they had heard but dismissed it, clinging to their assumptions
of worship and pilgrimage in the name of God while remaining veiled from
true faith.

Oh, people of the Bayán, have mercy upon yourselves. Do not nullify your
deeds by failing to recognize the proof when it appears. Strive with
utmost care and diligence, for this is the day of revelation, and only
through your own vigilance can you attain certainty in the Truth, so
that His proof may be clear and evident to you and all, as God is the
best of judges.

\subsection*{Gate 19 (Women May Enter the Sacred
House)}\label{gate-19-women-may-enter-the-sacred-house}
\addcontentsline{toc}{subsection}{Gate 19 (Women May Enter the Sacred
House)}

\textbf{Nineteenth gate of the Fourth Unity:} Regarding the Entry of
Women at Night into the Sacred Mosque

It has been ordained that women of that land and its vicinity may enter
the Sacred Mosque during the night, circumambulate the four sanctified
pavilions---upon each of which nineteen names are inscribed---and
partake in the glorification, exaltation, praise, unification, and
magnification of God. After this, they are to return to their abodes.
Each individual is granted a one-time allowance of four mithqáls of gold
during their lifetime for every successful pilgrimage to the House.

What brings these women closer to divine acceptance is the display of
love and care toward their families and the nurturing of their
offspring. If one, with all they possess, demonstrates affection and
compassion to their children prior to the age of responsibility, this
act is considered far superior to any other form of devotion to God.

God has commanded those of faith to treat their children and close
relatives with the utmost kindness and refinement, as befitting the
customs of their age. This ensures that no trace of sorrow clouds their
hearts, thus honoring the religion of Him Whom God shall make manifest.
This kindness extends to siblings, kin, and all others, for all creation
revolves around a singular purpose: to be created, sustained, and
ultimately revived by Him, the Eternal and Everlasting, who appears in
every dispensation under a new name while remaining concealed in His
highest majesty in periods of occultation.

If this reality were to be otherwise where could one utter ``There is no
God but God,'' for it is a sign that points to that Word, just as this
Word, composed of letters, signifies the oneness of God, exalted and
glorified be He. That divine essence also testifies to the sanctified,
singular essence of God. There is no deity but God. Truly, we are all
circumbulating around Him.

\section*{Vahid 5}\label{vahid-5}
\addcontentsline{toc}{section}{Vahid 5}

\markright{Vahid 5}

\subsection*{Gate 1 (The Mosque of
Honor)}\label{gate-1-the-mosque-of-honor}
\addcontentsline{toc}{subsection}{Gate 1 (The Mosque of Honor)}

The First Gate of the Fifth Unity, concerning the mosque:

The essence of this gate is that the first land where the Manifestation
of ``He Whom God Shall Make Manifest'' appeared and is manifested was
and remains the ``Mosque of Honor.'' Similarly, in the Point of the
Bayán, the secret of truth has been and remains evident. Whatever extent
the Cause of God can be exalted, it has been and remains worthy, such
that the day will come when the abode of ``God's clay of unity'' will be
established in a portion of that supreme land, becoming a place of
worship for the believers, just as is now manifest in Mecca. Its initial
boundary was not this expansive; rather, it has grown fourfold in
extent.

The intention of this matter is that lands associated with the
appearance of His body are elevated in this manner, becoming the locus
of consecration for pilgrims performing circumambulation around His
House. How, then, would be the lands of the essential realities, which
bear witness to His glorification; and the lands of the souls, which
testify to His oneness; and the lands of the spirits, which proclaim His
praise; and the lands of the hearts, which extol His sanctity? In the
first, the fire of love arises; in the second, the air of devotion
ascends; in the third, the water of oneness flows; and in the fourth,
the dust of existence is exalted. God initiates all things and then
returns them. Do you not see?

\subsection*{Gate 2 (Mosques Named After the Letters of the
Living)}\label{gate-2-mosques-named-after-the-letters-of-the-living}
\addcontentsline{toc}{subsection}{Gate 2 (Mosques Named After the
Letters of the Living)}

The Second Gate of the Fifth Unity, regarding the mention of the eight
mosques preceding the ten:

The essence of this gate is that it behooves those of power in the Bayán
that eighteen mosques, distinct and new, should be built in the name of
the Letters of the Living of ``He Whom God Shall Make Manifest,''
signifying the First Letters of the Living. Within them, the believers
should perform acts of glorification, sanctification, oneness, and
magnification of the divine. They should elevate them as much as
possible, and the lamps therein, free from extravagance, should be
raised. It is as though one sees lamps equal to the number of the name
``Mustagháth'' suspended in those mosques, shining like the stars of the
heavens.

In those places, believers in God and His signs gather to pray. Yet
beware of the day when these same letters turn towards worldly life, to
the extent that they do not even refrain from barring their own places
from themselves. This occurred in the manifestation of the Point of the
Bayán: all under the guise of previous traditions assumed leadership in
mosques, barring the original believers from the places where the
mention of God was exalted.

Those who witnessed this manifestation and recorded its events have
noted how, even in the grand mosques established for them, they were not
content with what had been designated for the Letters of the Living.
They descended further in their behavior until that which was destined
came to pass. It is not that no proof exists for them; indeed, the very
proof by which these mosques are erected is the same proof that had been
established before. If ``Whoever constructs the mosques of God'' had not
been revealed, how could all these mosques have been built?

But these people fail to reflect upon the essence of the proof. For this
reason, they are veiled from the truth. They observe and see a multitude
of people following the evidences, yet they fail to consider under the
shade of which tree all these movements occur and by what proof they are
content with this.

All endure hardship in the path, yet, as it becomes a source of pride,
they remain unaware. Meanwhile, all of these deeds of the masses stem
from the command ``And pilgrimage to the House is a duty upon mankind
for God.'' This same word exists in the Day of the Manifestation of ``He
Whom God Shall Make Manifest,'' as it does today and did during the
descent of the Qur'an. However, when they see no semblance of themselves
in the Cause, they remain veiled from the Source of the Command. Later,
when they fashion a semblance for themselves, it becomes a source of
pride, and they act accordingly.

If the same self that acts today had existed in that Day, it would not
have heeded the command of God, just as today you see the same proof
present, yet they remain similarly veiled. The veiled wander in streams
branched from the ocean of the proof of the Book of God, yet they are
veiled from the ocean itself. Thus, the decree of shadows falls upon the
masses, while the decree of light shines upon the believers with
insight. The sun, which is the sign of truth itself, is exalted above
association with remembrance, for none sees from it but God alone. Such
is the Lord of all worlds.

\subsection*{Gate 3 (Establishment of a
Calendar)}\label{gate-3-establishment-of-a-calendar}
\addcontentsline{toc}{subsection}{Gate 3 (Establishment of a Calendar)}

The Third Gate of the Fifth Unity, concerning the knowledge of years and
months:

The essence of this gate is that the Lord of all has established all
years by His command, appointing from the Manifestation of the Bayán
that the number of each year corresponds to the number of ``all
things.'' He has divided it into nineteen months, and each month into
nineteen days, so that all may traverse, from the vernal equinox to its
culmination in Pisces, through the nineteen stages of the Letters of
Unity.

The first month He named Bahá (Splendor), and the last He named 'Alá
(Loftiness). He has structured this arrangement upon this number and
decreed that each day serves as the springtime of a particular decree.
Thus, those who find delight in this paradise may rejoice with the
utmost elevation that can be attained in existence.

In the first three months, which are the months of glorification, the
fire of the hearts of beings is kindled. In the following four months,
which are the months of praise, the spirits of possible realities are
created, wherein they are sustained. In the next six months, which are
the months of oneness, God causes the beings to die---not a physical
death, but a death from negation and a life in affirmation. In the
subsequent six months, which are the months of magnification, God, the
Exalted and Glorious, grants life to those who have died in love for
anything other than Him and remained steadfast in their love for Him.

The first three months represent the fire of God. The next four months
represent the eternal air. The following six months represent the water
of oneness, which flows over the souls of all things from the eternal
air that is sustained by the fire of God. The final six months pertain
to the dust, wherein all that has appeared from the other three elements
becomes established in this element, and its fruits are gathered. All
creation is multiplied from this unity within unity.

The first month is the month of the Point, around which the months of
the Living revolve. Its likeness among the months is like the sun, while
the other months resemble mirrors that reflect the radiance of that
month. Nothing is seen in them except that month. God has named it the
month of Bahá, meaning the splendor of all months is contained within
it. He has reserved it for ``He Whom God Shall Make Manifest.'' Each day
of this month has been attributed to one of the Letters of Unity.

The first day, which is Naw-Rúz, is the day of ``There is no God but
God,'' analogous to the Point in the Bayán, from which all are created
and to which they return. Its manifestation in the Point of the Bayán
has been placed in the essence of the Seven Letters.

In this manifestation, the Throne has been appointed for ``He Whom God
Shall Make Manifest,'' for He is the dawning-place upon this Throne. He
is the Revealer of the verses in this manner, and none is seen in Him
but God, the Exalted and Glorious. He is the One who cannot be known by
the first through the first, nor by the last through the last. He is the
Manifest One who cannot be recognized by the apparent, and the Hidden
One who cannot be perceived by the concealed. He is the One through whom
the essence of all things exists, and His essence is through God
Himself.

Whoever believes in His Day and declares the oneness of God 361 times in
that year shall be protected from whatever descends from the heaven of
fate. Verily, God is the Guardian over all things. The fruit of this
knowledge is that in this manifestation, the months---being an aspect of
creation, all of which are signified by the Letters of Unity---indicate
how much more profound the realities of God's creation must be. No thing
perceives its essence except through the manifestations of the divine
Cause. This is not mere knowledge; rather, it is the attainment of its
fruit. In the Day of Manifestation, these very manifestations, even if
they possess all the lands, see in themselves only the manifestations
and humble themselves before them, even if they appear in simple
garments like this manifestation. This benefits the God-fearing.

\subsection*{Gate 4 (Servants May Name Themselves With the Names of
God)}\label{gate-4-servants-may-name-themselves-with-the-names-of-god}
\addcontentsline{toc}{subsection}{Gate 4 (Servants May Name Themselves
With the Names of God)}

The Fourth Gate of the Fifth Unity, concerning the ruling on naming with
the names of God or with the names of Muhammad, `Ali, Fatimah, or both,
or Hasan and Husayn, peace be upon them:

The essence of this gate is that God has permitted His servants to name
themselves with names that signify Him, such as `Azíz (Mighty), Jabbár
(Compeller), and similar names. The best names are those associated with
God, such as Bahá'u'lláh (Glory of God), Jalál'u'lláh (Majesty of God),
or Jamál'u'lláh (Beauty of God), O Light of God, O Grace of God, O
Generosity of God, and similar names of this kind are exalted. Names
such as Abdu'lláh (Servant of God) and Dhikru'lláh (Remembrance of God)
also constitute a station through which one may ascend infinitely. If
one wishes to name within the ocean of Prophethood, Guardianship, or
Love, they should not exceed the Five Names, and the combination of the
names Muhammad and Alí is the greatest of all names.

If all generations and eras ascend gradually, step by step, they will
reach a manifestation in which all things are named with the names of
God. No thing will be named without it resembling one of the divine
names of the True One, Glorious and Exalted. For example, \emph{halím}
(mild), which is a type of food, resembles the name of God,
\emph{Haleem} (the Forbearing). In this Manifestation of the Bayán, such
associations will be abundant, so much so that permission has been
granted to all that in the Day of the Manifestation of the Sun of Truth,
if humanity reaches the limit of perfection, nothing will be named
except that it corresponds to one of the names of God, Glorious and
Exalted.

If this is not achieved in that manifestation, it will occur gradually
in subsequent manifestations until the heavens and the earth and all
that is between them are filled with the names of God. What difference
does it make if the clay is a sign of God or if humanity is? Both are
created, except that one was created for the other, for God has placed
the spirit of oneness in all things within the human spirit.

For example, if a believer sits upon a piece of land, the spirit of that
land becomes still and delighted. But if one unworthy sits upon it, it
becomes agitated to an extent that none but God can encompass. It
continuously beseeches God to raise the unworthy from it. Similarly, the
metaphor of clay has been extended to include all things.

Reflect and understand: blessed are the people of the time when no name
is attributed to anything except with a name of God. That time may well
be called the beginning of the realms of paradise. No thing reaches its
paradise except by manifesting the utmost perfection within its own
limit. For example, the crystal of paradise originates from stone, which
is its material. Similarly, this crystal has degrees within paradise,
until it reaches the point where, in the presence of water and oil, the
poets sing of it as ruby. At that point, it has reached the pinnacle of
paradise, for when it was merely stone, it had no value. But today, if a
carat of it attains the full quality of ruby, which lies within its
potential, how valuable does it become!

In the same way, consider all things. Their ultimate exaltation lies in
their faith in God during each manifestation and in that which is
revealed before them---not in knowledge alone, for every nation has its
scholars in various fields; and not in wealth, for it is evident that
every nation has its wealthy individuals in their ranks; nor in other
such distinctions. Rather, true knowledge is knowledge of God, which is
none other than knowledge of His Manifestation in every appearance. True
wealth is poverty in relation to Him and independence from all else.
This is not manifest except in relation to the Manifestation of the
Cause.

It is not that one should fail to give thanks for previous
manifestations---this is impossible. For instance, a person at the age
of nineteen must give thanks for the day of conception, for without that
initial stage, they would not have reached their current position.
Similarly, had the religion of Adam not existed, today's religion would
not have reached its current state. And so it continues.

Reflect endlessly upon the Cause of God, and give thanks to Him for
every manifestation He has revealed in each appearance. For it is
through gratitude to Him that one attains nearness to Him, and God
guides whomever He wills to the path of true certainty. The fruit of
this matter is that by remembering these names, one does not step
outside the realities of the Named. Perhaps a soul, through its
attraction to the Named One in this Manifestation, may become worthy of
a name that signifies ``He Whom God Shall Make Manifest'' and not
something other than Him.

Do not veil yourself with the mere name, for in this cycle, the killer
of the Prince of Martyrs bore the very name of the honored one himself.
In the Manifestation of ``He Whom God Shall Make Manifest,'' there is no
doubt that all are called by these beloved names. Yet, if a soul remains
steadfast in its indication of that Reality, then its name becomes one
whose essence is created from the ocean of God's bounty, deserving to be
called a ``Most Beauteous Name.'' Otherwise, it is but a false shadow
buried beneath the dust, annihilated in negation.

You witnessed, on the Day of Resurrection, how many individuals bore the
name of the Messenger of God---a name unsurpassed in its
excellence---yet remained veiled from their Beloved. Verily, God is
witness over all things. Know that the likeness of ``He Whom God Shall
Make Manifest'' is that of a touchstone, distinguishing pure gold from
all else. For example, if someone is named Bahá'u'lláh, and they believe
in His splendor, which is the first to believe in Him, then this name is
affirmed for them and becomes established. Otherwise, they are
annihilated in negation as though they were never mentioned.

\subsection*{Gate 5 (Sovereign Rulers May Take the Lands of
Unbelievers)}\label{gate-5-sovereign-rulers-may-take-the-lands-of-unbelievers}
\addcontentsline{toc}{subsection}{Gate 5 (Sovereign Rulers May Take the
Lands of Unbelievers)}

The Fifth Gate of the Fifth Unity, concerning the law of seizing the
possessions of those who do not believe in the Bayán and the ruling on
returning it should they enter.

In every Dispensation, all that exists on earth should come under the
shadow of the subsequent Manifestation. For example, during the
Dispensation of the Messenger of God, it was fitting that everything on
earth should believe in Him and come under His protection. Whatever did
not occur was due to the weakness of the believers, not due to any lack
of worthiness in that religion. On the Day of the Manifestation of the
Messenger of God, peace be upon Him, no one was permitted to take
another's life or possessions unless that person entered into faith in
Him. Only then would it become permissible to partake of what God, in
His bounty, had granted them.

Similarly, in the Manifestation of ``He Whom God Shall Make Manifest,''
no soul has rights over another unless that soul believes in Him.
Everything belongs to Him and is to be brought under His shadow, except
for those who enter His faith. This is the grace of God toward His
creation. For instance, if lands were conquered during the early days of
Islam, those who were compelled to enter Islam through force and power
ultimately attained the fruits of faith. However, those who did not
submit remained deprived of mercy and continue, even now, to suffer in
fire.

Likewise, in this Dispensation, nothing associated with those who do not
believe in the truth is permissible unless they enter faith. When they
do so, what God has granted to them through His bounty becomes lawful.
This ruling applies to the sovereign rulers who possess authority in the
religion, not to everyone, and not in lands where its implementation
would cause harm or sorrow to any soul. God has not permitted its
declaration in such cases. For example, merchants conducting trade in
Western lands must maintain the utmost precision in their accounts and
dealings.

Let them conduct themselves in such a way that no humiliation befalls
them unless God grants them victory through a power that establishes
them over all that is on earth. When that time comes, all will dwell in
the mercy of God, even if they themselves do not desire it. Yet, through
the power of God, they will be brought within, saved from the fire, and
transformed into light. Verily, God is powerful over all things.

It is not for those in power to merely wait for something to descend
from heaven that would bring all on earth into the religion. Rather, as
in the religion of Islam, when all entered through the command of the
Messenger of God, the same must be manifested in every Dispensation.
Whatever comes from God is permission, but what is required from
humanity is obedience to Him. If those in power during the Dispensation
of the Messenger of God had acted in accordance with the Qur'anic
commands, today everyone on earth would be a believer in the Qur'an.
Since this has not occurred, the shortcoming lies with the servants, not
the Source, for everything that was to be manifested in the Qur'an has
already been revealed. Verily, God aids whomever He wills by His
command, and God is mighty and powerful.

The essence of this ruling is that, at the time of the Manifestation of
``He Whom God Shall Make Manifest,'' all should have been nurtured
through the teachings of the Bayán so that none of the believers in the
Bayán would turn away from faith in Him. If any do so, their judgment is
the same as those who do not believe in God. By the sanctified essence
of God, if all in the Manifestation of ``He Whom God Shall Make
Manifest'' were to unite in supporting Him, no soul would remain on the
face of the earth except that they would enter paradise. Indeed, there
would be no thing left. Let each be vigilant over themselves, for the
true support of the religion lies in this, not in deeds that\ldots{}

At the time of His Manifestation, whatever has been revealed in the
Bayán must be fulfilled. However, before His Manifestation, anyone who
deviates even slightly from His command has truly turned away. Seek
refuge in God from that which distances you from the source of the
Cause, and hold fast to His cord. Whoever clings to obedience to Him in
all worlds is saved and will be saved. This is by the grace of God,
which He bestows upon whomsoever He wills. Verily, God is the Possessor
of immense grace.

\subsection*{Gate 6 (The Distribution of Wealth In the
Religion)}\label{gate-6-the-distribution-of-wealth-in-the-religion}
\addcontentsline{toc}{subsection}{Gate 6 (The Distribution of Wealth In
the Religion)}

The Sixth Gate of the Fifth Unity, concerning the ruling on wealth
acquired in this religion and its distribution if it has no equivalent
value:

Wealth that has no equal belongs solely to the Point of the Bayán and
remains so as long as the Sun of Truth continues to shine. If the Sun
has set, its authority is to be preserved by trusted believers in the
Bayán until the Day of the Manifestation of Truth, at which point they
must return it to ``He Whom God Shall Make Manifest.'' No one is
permitted to use it for personal purposes.

It is like the wealth of God's Proof, which certain learned ones have
taken and used without His permission. If even the smallest portion---a
carat---was used without His authorization, the recompense for such
action would be the fire of punishment.

When God, out of His grace, grants victory to the believers by enabling
the conquest of lands that have not yet embraced the faith, whatever
wealth is without equal belongs to the Point. This remains so as long as
the Sun of Truth is present and returns to Him. Upon its setting, it is
entrusted to the trustworthy among the believers in the Bayán, who will
preserve it until the Day of ``He Whom God Shall Make Manifest,'' at
which time it will be restored to Him.

Beyond this, what remains must be distributed among those who opened
these lands by permission, each according to their need. Anything
surplus should be given to the poor and spent on communities.
Furthermore, even providing for an unborn child in the womb is
preferable to spending it elsewhere. This is a gift from God, who is the
Ever-Generous and Mighty.

All things belong to God, and who is closer to Him than the Point of His
Will? Whatever belongs to the Proof of God cannot rightfully belong to
another without His permission, not even the smallest amount. If someone
gives it to another without authorization, it carries no validity, even
if the one who gave it was considered knowledgeable in their time. They
would have failed in their duty toward the Proof of God by transferring
it to others without His permission. Both the giver and the receiver are
in the fire, for its rightful owner is alive and more deserving of what
God has granted him through His bounty in the Qur'an. He is
self-sufficient beyond need.

How could one become independent through wealth itself? Whoever seeks to
save themselves from the fire does so by offering their share, yet the
Proof of God remains ever-abundant and self-sufficient. All are created
from the ocean of His bounty; how could they take what flows from His
existence?

Today, which is the Day of Resurrection, even the most learned are
asked: ``By whose permission did you build a grand mosque with the
wealth of the Proof?'' This single question weighs heavier on them than
any other torment for those with true knowledge, provided their spirit
of faith remains intact. If not, they may hear a thousand times the
verse \emph{`All things are perishable except His Face'} and still act
as if they only heard a meaningless word.

To those who truly know God, giving away all that exists on earth is
better than being questioned on the Day of Resurrection about an act
carried out without the approval of their Beloved. As for wealth without
equal, it is taken from all in accordance with their faith and then
allocated to the people of the Bayán, from the highest rank to the
lowest. After this, God bestows His victory upon His chosen ones,
granting every soul what befits their station.

And what is fitting for them comes from the bounty of their Beloved. If
there is any excess, it should be spent on the places commanded, or
distributed among all the people of the Bayán, even if it includes a
six-month-old child still in the womb, whether they are in the East or
the West of the earth. Spending on the places takes precedence unless it
has already been fulfilled; otherwise, this duty comes first. This is
the decree of God in this matter.

The result is that on the Day of the Manifestation of ``He Whom God
Shall Make Manifest,'' all that possesses existence on earth belongs to
Him. The custodians of the Bayán must recognize the right of God, for
all that they have received since the beginning of the Dispensation of
the Bayán has been from His bounty, even prior to His Manifestation.
Whether it pertains to this world or any other, they must act with a
degree of reverence and ensure no sorrow comes to Him through the
actions of His servants, withholding none of the rights decreed for Him
by the Point of Truth.

Whatever is without equal cannot belong to anyone except Him, for He is
the sign of \emph{``There is nothing like unto Him.''} Anything that
reaches this station in its essence is worthy of Him alone, not another.
From the highest peak of loftiness to the lowest point of nearness,
everything belongs to Him, that no sorrow may touch His blessed heart at
the time of His Manifestation due to His creation, who were formed
through His bounty. For His sorrow cannot be compared to the sorrow of
all things combined, as all things derive their essence from Him.
Likewise, His satisfaction surpasses the satisfaction of all things, for
the satisfaction of all things was created through Him.

By Him, in the knowledge of God, there has been and will be nothing
greater than Him. A single gesture from Him is of greater significance
to God than all the deeds of creation, even if they have reached the
pinnacle of their potential exaltation.

For all existence derives its reality from Him, and so too do all deeds.
Likewise, proximity to Him is arranged in degrees: the closest are the
Letters of the Living, then the Names and Similitudes, followed by the
Prophets, the Truthful, the Martyrs, and the Near Ones. Each is ranked
according to what has been decreed for them, for all have stations in
the sight of their Lord, and all are devoted to Him.

If the verse \emph{``There is nothing like unto Him''} cannot be
preserved by the trusted ones of the Bayán and becomes altered, it is
incumbent upon them to safeguard it through their own means, even by
trading on behalf of its owner and deducting their rightful share---one
hundred out of every thousand. This practice is to be established among
all so that everyone benefits from this method equally. It is
unimaginable that anyone would transgress in a matter where the Proof of
God has acted in such a manner, for His commands are the source of all
grace. There is no power or strength except through God, the Exalted,
the Mighty.

\subsection*{Gate 7 (Believers May Buy and Sell All Good Things to
Purify
Them)}\label{gate-7-believers-may-buy-and-sell-all-good-things-to-purify-them}
\addcontentsline{toc}{subsection}{Gate 7 (Believers May Buy and Sell All
Good Things to Purify Them)}

The Seventh Gate of the Fifth Unity, concerning God's permission for
those who follow the Bayán to engage in buying and selling of all
things, with certain conditions for those who do not adhere to the
faith:

The essence of this gate is that God has granted a bounty to the people
of the Bayán. Even if they were given all that exists on earth, it would
not lead them astray from Him. For this reason, God has decreed a
severance of association with non-believers in the Bayán while affirming
association with its believers. This purifies the acts of buying,
selling, and other exchanges of gifts.

For example, if an item originally belonged to a Christian and is
transferred to a believer, it is immediately purified upon its transfer.
However, if the item is rejected and remains in the possession of its
original owner, the original ruling applies. The moment it is
transferred, even if the reason for separation is a gift from a
non-believer to a believer, the item is purified.

If an item is sent by a non-believer to a believer, from the moment it
is declared to be for a specific believer, it is purified---even if many
years pass before it reaches the intended believer. God has permitted
the believers of the Bayán to seek and acquire any good thing in any
land, so that perhaps, on the Day of the Manifestation of Truth,
something might come into the presence of the created world that becomes
beloved to Him. For all that appears as beauty in creation is but a drop
from the ocean of His bounty and grace.

Indeed, God, the Glorious and Exalted, cannot be described as ``kind''
in the way that His creation is described, for He is beyond and above
such comparisons. His grace is not akin to the grace of His creation. In
this way, all the names and similitudes you witness reflect His essence
but do not fully encompass it.

\subsection*{Gate 8 (Recite the Verses of the Bayan At Least 19
Times)}\label{gate-8-recite-the-verses-of-the-bayan-at-least-19-times}
\addcontentsline{toc}{subsection}{Gate 8 (Recite the Verses of the Bayan
At Least 19 Times)}

The Eighth Gate of the Fifth Unity, concerning the right of every soul
to recite the verses of the Bayán and the prohibition against reducing
the count below the number of Unity (19). For those unable to do so, the
utterance of \emph{``God, God is my Lord, and I associate none with my
Lord''} 19 times suffices:

The essence of this gate is that the creation of the Bayán may be
likened to believing souls, where each soul is stationed at a particular
limit and manifests in a specific way. Similarly, consider the verses of
the Bayán---observe how they stand, and delve into this ocean to uncover
the pearls that God has created within it. Recite them with a spirit of
joy and contentment, as much as your heart delights, but no less than
the count of Unity (19).

The reason for this is that the manifestations of Unity must not be
exceeded. Through the blessedness of reciting these verses, you may be
guided by the guidance of their revelation on the Day of Resurrection.
The entirety of the Bayán can be seen as capital given to someone to
trade; its true owner is ``He Whom God Shall Make Manifest.''

The wealth of the Bayán has been entrusted to the believing souls by the
Point of the Bayán so that they may trade with it in preparation for the
Day of His Manifestation. If He wishes to reclaim this capital, no one
may question why or for what reason. Every individual is immersed in
countless rulings, as you see today---when someone teaches the purity
laws of the Qur'an, they turn it into a means of trading both religion
and worldly gain. However, on the Day of the Manifestation of its Owner,
when He seeks an account from the people, they will feign blindness.

On the Day of the Manifestation of ``He Whom God Shall Make Manifest,''
envision a paradise being established where gardens are taken into
account. Imagine one garden as belonging to a sovereign, another to a
scholar, another to a merchant, and so on among the various classes of
people. On that Day, you will witness and comprehend your own existence
and what is derived from it. Do not withhold from Him what belongs to
Him. Reflect on how today, in the Bayán, some claim the title of Judge
of the Bayán, some Sheikh-ul-Islam, others Mujtahid, or Imam of the
Friday congregation. They are proud of these titles, yet they remain
veiled from the One from whom these names originated.

Similarly, on the Day of the Manifestation of ``He Whom God Shall Make
Manifest,'' the Bayán will be recited endlessly, yet its true source
will remain veiled from many. Do you not perceive that the time of the
Bayán's revelation is like the time of its Manifestation? Just as you
heard the Qur'an when it was revealed but did not take its fruits, today
you see its worth, and all are proud of the honor associated with its
name. Yet remember, the Qur'an was revealed over twenty-three years, and
even then, a complete and proper copy was not prepared. Had it been
completed, the Commander of the Faithful, `Alí, peace be upon Him, would
not have carried it in His blessed cloak.

By the Most Sacred Essence of God, Glorious and Exalted, on the Day of
the Manifestation of ``He Whom God Shall Make Manifest,'' if someone
hears a single verse from Him and recites it, it is better than reciting
the Bayán a thousand times. Reflect for a moment and observe how, even
today, what is considered the culmination of Islam parallels what will
occur on the Day of His Manifestation. Imagine that the Source of all
proofs will be in His hand, and yet He will remain veiled by false and
misguided preoccupations. He is exalted above such things, for all
circumstances branch from the Book of God, and He Himself is the Proof.
Indeed, all are incapable of producing the like of Him.

There are countless scholars of logic, grammar, morphology,
jurisprudence, and principles, yet if they do not believe in the Book of
God, they are considered beneath the station of faith. Thus, the essence
lies in the Proof Himself, not in the secondary matters that branch from
Him.

Know that every word revealed in the Bayán has intended that you obey
``He Whom God Shall Make Manifest,'' for He is the One who revealed the
Bayán prior to His own Manifestation. If you are unable to recite its
verses, then say, nineteen times from dawn to dawn: \emph{``God, God is
my Lord, and I associate none with God, my Lord.''} If you say this with
insight, there is no doubt that you will be guided by the truth on the
Day of Resurrection. The reward of reciting the entire Bayán will be
granted to you, for God bestows His grace upon whomsoever He wills.
Verily, He is the Possessor of immense grace.

\subsection*{Gate 9 (Mention a Name of God Over Anything You Intend to
Use)}\label{gate-9-mention-a-name-of-god-over-anything-you-intend-to-use}
\addcontentsline{toc}{subsection}{Gate 9 (Mention a Name of God Over
Anything You Intend to Use)}

The Ninth Gate of the Fifth Unity, concerning the mention of one of the
Names of God over everything one intends to use:

The essence of this gate is that whenever a person seeks to use
something, they should mention one of the Names of God over it.

Whether in secret, openly, or with inner reflection, the essence of this
gate is as follows: No thing possesses true existence except through
God, Glorious and Exalted, and for each thing a day has been ordained
when it will reach the fullness of its potential, such that all that can
possibly manifest within it becomes actualized. At that point, it
becomes worthy of having the Name of God mentioned over it, in
accordance with its rank, not beyond its station.

Since God has commanded the people of the Bayán to bring every thing to
the utmost perfection of its potential, He has permitted them to invoke
His Names over all things. This ensures that no soul sees anything
within any object except the radiance of His Will. In the manifestation
of that Will, nothing is perceived but God. For example, consider a
stone: at its highest level of perfection, one might see only the
attribute \emph{``Sanctified''} in its ``S,'' \emph{``Light''} in its
``N,'' and \emph{``Generous''} in its ``K.'' Whether this invocation is
verbal, arises in the heart, or comes as a simple reflection upon the
object without words, the mention of God's Name sanctifies it.

If one cannot explicitly invoke a specific Name of God, they may instead
mention \emph{``The Most Guarded, the Most Sacred Name,''} and God,
Glorious and Exalted, will accept what is intended in this matter. The
aim is that nothing is observed within creation except ``He Whom God
Shall Make Manifest,'' who is the source of all divine Names and
Attributes.

On the Day of the Manifestation of the Sun of Truth, no being will
assert its own independent existence, nor see in itself anything but
Him. It will realize that it stands and exists only through Him. Yet, it
cannot truly perceive Him in itself; rather, it resembles a mirror
placed before the sun---it sees the sun reflected within itself, yet
what appears is but its image. Thus, every entity that bears the
designation of a ``thing'' in the presence of ``He Whom God Shall Make
Manifest'' will follow this principle.

Consider the realm of abstraction, from its loftiest heights to the
furthest limit of definition, and know that whatever exists within it
reflects the radiance of the Sun of His Manifestation, even before His
appearance. For there has never been, nor will there ever be, another
besides Him truly manifest. He desires that on the day of His
appearance, nothing is seen in His knowledge except that all things have
ascended to the fullness of their perfection in the Bayán.

There is not a single particle of clay in the depths of an ocean that He
has not made His own, through the possession of a believer among the
people beloved to Him. All things have become mirrors prepared for the
dawning of the Sun. The moment the light of His radiance---the verses of
His revelation---shines upon them, all will point to Him. This is the
fruit of this decree for those who reflect. Verily, God illumines the
hearts of His righteous servants.

\subsection*{Gate 10 (The Haykal for Men and Circles for Women Are
Bounties)}\label{gate-10-the-haykal-for-men-and-circles-for-women-are-bounties}
\addcontentsline{toc}{subsection}{Gate 10 (The Haykal for Men and
Circles for Women Are Bounties)}

The Tenth Gate of the Fifth Unity, concerning God's decree of forms for
men and circles for women, wherein they may write whatever they wish
from the Bayán:

The essence of this gate is that God has ordained two great favors for
the people of the Bayán and bestowed them upon them as bounties. The
first is the form (\emph{haykal}) for men, whose outer manifestation is
the letter \emph{H} and whose inner reality is \emph{W} (\emph{huwa},
``He''). Permission has been granted for them to inscribe within it
whatever has arisen from the Sun of the Point. Each individual may draw
from the ocean of His bounty and write within the form, for whatever is
written there manifests its effects upon the soul, letter for letter and
point for point.

For the manifestations of the \emph{B}, God has permitted a circular
form, symbolizing the garment of the Sun of Truth. He has divided it
into five units, each representing unity. These signify the letters of
the phrase \emph{``To God belongs whatever is in the heavens and the
earth and what is between them,''} and that \emph{``God encompasses all
things.''} Through these, the truths of their hearts are expressed.

The garment of the Sun of Truth becomes manifest so that, on the Day of
the appearance of that Supreme Luminary, it may signify nothing but Him.
Permission has been granted for them to inscribe within it whatever they
desire from the words that have arisen from the Sun of Existence.
Whatever they write within it will manifest its effect upon their souls,
letter for letter and point for point.

Although the beginning of this circle is in the manner mentioned by the
Commander of the Faithful, peace be upon Him, where apparent names are
inscribed within singular points: in the Point, \emph{Individuality}; in
the \emph{Ḥay}, \emph{Life}; in the \emph{Alif},
\emph{Self-Subsistence}; in the \emph{Bá}, \emph{Sovereignty}; and in
the \emph{Jím}, \emph{Sanctity}. Similarly, in this Dispensation, the
phrase \emph{Bismillah al-Amna` al-Aqdas} has been revealed,
encompassing within it the four stages of creation: \emph{origination
(khalq)}, \emph{provision (rizq)}, \emph{death (mawt)}, and \emph{life
(ḥayát)}, spanning from the rank of the Point to the rank of the
\emph{Jím}.

If one acts according to the teachings that have spread forth like an
ocean of bounty, they will find their purpose fulfilled, provided that
the Will of God flows within them. The essence of these two decrees is
that the Bayán itself is the creation of the greater world. The moment
it becomes manifest in a form or a circle, without exceeding the limit
of the \emph{Há}, it may, within five years of the Manifestation of ``He
Whom God Shall Make Manifest,'' attain the honor of faith in that Sun of
Truth. All that has been written within these forms and circles praises
Him and exalts Him beyond what His detractors claim.

If someone remains near Him, partaking of the gifts of His bounty, yet
does not manifest its fruits in the five lines or six houses of a single
unity, how can they claim any merit? Such a one nullifies the purpose of
their existence. This ensures that the people of the Bayán do not stray
beyond the bounds of these five.

For within the structure of the Five (\emph{Haykal-i-Khams}), the
\emph{Wáw} is encompassed, and within the circle, the \emph{Wáw}
preserves the \emph{Há}. This is the purpose of its revelation, so that
people may recognize these two great gifts and supreme bounties at the
time of the Manifestation of that Eternal Sun and the Radiant
Countenance. The foremost of the forms are within the \emph{Há}, and the
foremost of the circles are within the \emph{Wáw}. Indeed, all should be
severed from attachment, even as the greatest cycle progresses.

The advancement from five years may reduce to five months, then to five
weeks, five days, five hours, five minutes, or even five-tenths of a
ninth. Whatever can be expressed in proximity to Him is due to the fact
that no separateness exists between the Will of God and that which
arises from it. There has always been a distinction of attributes
between them, like the fire and its burning: the fire may be realized,
but its burning may remain unmanifest. Or a lamp may shine, but its
light might not yet illumine the places around it.

In this way, consider all beloved realities, from the essence of
existence to the ultimate limits of boundaries. Do not observe any
station within all these circles and forms except as a single form. For
example, what you see in the Qur'an reveals nothing but the form of the
Messenger of God. If it were not infused with the light of His essence,
no one would find faith in Him, His manifestations, or that which God
revealed before Him.

Similarly, in every Manifestation, observe with the eye of certainty
this singular essence that pervades all things. If, on the Day of the
Manifestation of ``He Whom God Shall Make Manifest,'' another form
besides His own is seen, that form itself has veiled itself from Him.
Yet He is more manifest to it than it is to itself. Everything He does
and every intention He holds aims solely at Him, even though, at times
if veiled from Him, everything ultimately returns to God. This is the
decree: \emph{``To God belongs the entirety of the matter, if you but
know.''}

\subsection*{Gate 11 (The Prayers for the Newborn and the
Deceased)}\label{gate-11-the-prayers-for-the-newborn-and-the-deceased}
\addcontentsline{toc}{subsection}{Gate 11 (The Prayers for the Newborn
and the Deceased)}

The Eleventh Gate of the Fifth Unity, concerning the prayer for the
newborn and the deceased, where five takbírs are recited for the newborn
and six for the deceased:

In the prayer for the newborn, after the first takbír, recite nineteen
times: \emph{``Indeed, we are all believers in God.''} After the second
takbír, recite: \emph{``Indeed, we are all certain in God.''} After the
third takbír: \emph{``Indeed, we are all loving toward God.''} After the
fourth takbír: \emph{``Indeed, we are all dying in God.''} After the
fifth takbír: \emph{``Indeed, we are all content with God.''}

In the prayer for the deceased, six takbírs are recited. After the first
takbír, recite nineteen times: \emph{``Indeed, we are all worshippers of
God.''} After the second: \emph{``Indeed, we are all prostrate to
God.''} After the third: \emph{``Indeed, we are all devout toward
God.''} After the fourth: \emph{``Indeed, we are all remembering God.''}
After the fifth: \emph{``Indeed, we are all thankful to God.''} After
the sixth: \emph{``Indeed, we are all patient for God.''}

The essence of the gate is what God has revealed as His decree is a
bounty from Him for ``He Whom God Shall Make Manifest,'' and all else
follows in the shadow of the divine ordinances. Among these decrees, He
has ordained that for every newborn---whether male or female---a prayer
should be raised with five takbírs, that the Name of God may be
mentioned for them. Perhaps if they live to witness the Day of
Resurrection, they may become believers in ``He Whom God Shall Make
Manifest.''

Similarly, for the deceased ascending to their station, He has decreed a
prayer with six takbírs in a single sequence. This signifies that their
beginning is with the \emph{Há} and their return is to the \emph{Wáw,}
illustrating the journey from origin to return. Perhaps, on the Day of
Resurrection, they may join the first who believed in ``He Whom God
Shall Make Manifest,'' without being veiled by the material concerns of
the world. For this Cause is difficult for all but the true monotheists,
and how often a soul is unknown, yet all consider themselves to be
known. Consider, as an example, the Dispensation of the Qur'an: after
the ascension of the Tree of Truth, the faith of those who remained
steadfast in the identity of their being did not endure, despite their
apparent adherence to the actions prescribed by the Qur'an. Nonetheless,
in the early days of Islam, they were judged as being beneath true
faith.

Similarly, in the Dispensation of the Bayán, reflect on how those who
enter the gate with insight are saved through this act, not through
other deeds. This is because the soul in question is the origin of all
things through God, and the return of all things is to Him through God.
Many souls, from the beginning of the manifestation of the Tree of the
Bayán, have relied on their own understanding, thereby veiling
themselves from Him, despite all being accountable before God for what
they are charged to do. Before God, there is no distinction between the
exalted and the lowly.

Today, you see all declaring their belief in the Commander of the
Faithful, peace be upon Him. This is because they have not witnessed
contradictions and have been nurtured in the shadow of His love.
However, if these same souls had lived during the early days of Islam,
you would have seen among them the three groups mentioned in the
traditions. Similarly, in this Dispensation, God has shown favor to the
believers through manifestations of love equivalent to the number of the
name \emph{Rahmán}. This has been granted by the very essence of the
Point, and if it culminates in Him, the people will be tested. The truly
sincere believer, rarer than red sulfur, will emerge.

The decree for this prayer is intended so that, on the Day of
Resurrection, souls may not be veiled. After all belief is declared, the
Day of Fruition will reveal just how challenging it is for souls to
remain steadfast.

On the Day of His Manifestation, even one who possesses no worldly
wealth or adornment and enters into faith in God will be tested in ways
similar to those who professed belief in the Commander of the Faithful,
peace be upon Him. They themselves were not tested until they saw the
apparent luster of gold and the grandeur of His station among all
people. If His Day had arrived and they were among the three groups,
only then could their sincerity be proven. Yet now, they fail to mention
the days of Salman, Abú Dharr, and Miqdád in a manner befitting their
worth.

This is the root of discord in every Dispensation. Blessed is the soul
that contemplates what makes the Proof a Proof, for in that instant,
closer than the blink of an eye, even those on earth would enter
paradise through its gates. They would witness the vastness of the path,
broader than the heavens of potentialities and the earth of acceptances.

In all circumstances, remain vigilant over yourself, so that you do not
falter in the trials of God and instead cling firmly to the cord of the
Book, which is the guide for all who follow it. Verily, God initiates
all things and returns them, and to Him do all return.

\subsection*{Gate 12 (Burial of the
Dead)}\label{gate-12-burial-of-the-dead}
\addcontentsline{toc}{subsection}{Gate 12 (Burial of the Dead)}

The Twelfth Gate of the Fifth Unity, concerning the burial of the dead
in marble tombs and placing agate rings on their hands:

The essence of this gate is as follows: since the outer body is the
throne of the inner body, whatever affects the inner body will also
govern the outer body. However, it is the inner essence that feels
delight or pain through this outer body, not the body itself. For this
reason, God has decreed that the body, being the throne of the inner
essence, must be preserved to the utmost. Nothing that may cause
aversion or harm should come upon it, for the essential body observes
its throne.

If one observes anything contrary to this---where the body is not
treated with dignity---it is as though they have been deceived. Should
the body be subjected to anything less, it will inevitably experience
what it should not. For this reason, the command to honor and respect
the body has been emphasized to its fullest extent. Permission has been
given for the body to be placed in crystal or polished stone, so that it
may remain protected from anything that could bring discomfort to its
essential self while resting on its throne. This is the fruit of this
decree. Thus does God bestow His favor upon whomever He wills among His
servants, for He is the Guardian, the Self-Subsisting.

Permission has also been given for the use of an agate ring, engraved
with a sacred verse, to ensure that no sorrow comes upon the essential
self of the body and that it remains veiled from the fire, established
under the shadow of light. Whoever wears such a ring engraved with the
Name of God on their hand---if they are believers in the Bayán and
adherents to its laws---has the right to be granted life through God.
God, in His infinite grace, will bestow upon them such honor and bounty
that they will be content. Who speaks more truthfully than God? If only
you would believe in Him and His signs.

The fruit of this decree is as follows: the return of all things leads
to the manifestations of divine glorification. The element of dust
reaches its highest station, beginning as stone and ascending to its
utmost purity, becoming crystal. At this stage, the attribute of
\emph{Ṣamad} (Self-Sufficient) is manifested in it. Thus, at the return
of all things, essential selves may be established within such purified
bodies, signifying the truth of His glorification. God guides whomever
He wills by His command, for He is Mighty and Exalted.

\subsection*{Gate 13 (Permission to Write a Will and
Testament)}\label{gate-13-permission-to-write-a-will-and-testament}
\addcontentsline{toc}{subsection}{Gate 13 (Permission to Write a Will
and Testament)}

The Thirteenth Gate of the Fifth Unity, concerning the explanation of
the Book of the Will.

The essence of the gate is there is no greater bounty from God for His
servants than His permission for them to worship Him and His teaching of
glorification, praise, oneness, and magnification. He has granted
permission that, at the time of death, a testament be written,
containing a declaration of His oneness and justice, affirming that
creation and command belong to Him, acknowledging the Divine Point and
the Letters of the Living, expressing love for the manifestations of His
Names and Similitudes, and disavowing all that is displeasing to His
Beloved.

In this testament, the individual may also express their hopes and
desires and include a directive that the testament be conveyed to ``He
Whom God Shall Make Manifest.'' If His Will decrees to answer it, such a
response will be as if God Himself has revealed it concerning that soul.
The preservation of this testament becomes the responsibility of their
heirs, who are to pass it down, hand to hand, until it reaches ``He Whom
God Shall Make Manifest,'' written in the finest script and with utmost
beauty.

The interval between one Manifestation and the next is well defined, and
preserving what exists between them is easier than any other task.
However, utmost care must be taken to ensure that, God forbid, the
Manifestation of Truth does not appear while the individual remains
veiled, or that the testament intended for Him is not delivered.

It is evident that, during each Manifestation, people become attached to
something that veils them. They fail to realize that the object of their
attachment derives its legitimacy from the previous Manifestation, yet
in the subsequent Manifestation, it no longer holds the same relevance
due to the exaltation of the new appearance.

For instance, in the Manifestation of the Messenger of God, Muhammad,
peace be upon Him, the appearance of Jesus was acknowledged as a prelude
to His own, yet He did not declare Himself universally known through
that prior Manifestation. There is no doubt that the people of the
Bayán, though they act with utmost propriety according to the principles
of their religion, the moment of reaping the fruits of their deeds is
what remains in question. For instance, everyone, at the time of their
death, declares \emph{``There is no god but God''} and departs,
revolving around the sovereignty of the preceding command. They utter
this word at the moment of death, yet the Manifestation of this word,
which signifies God, is veiled from them by other attachments. Thus, all
their deeds are rendered as \emph{``scattered dust.''}

In the same way, you will all write your testaments and bear witness to
the oneness of God, declaring, \emph{``I associate nothing with God.''}
Yet, on the Day of the Manifestation, your own souls will become the
embodiment of the very polytheism you claimed to negate, without you
realizing it. This is why, in a single moment, your religion is
nullified, and you remain unaware.

Have mercy upon yourselves, and ensure that your actions are performed
with insight. Perhaps on the Day of Resurrection, you may attain
salvation. Know that the origin of this Book is from God, through the
utterances of the Point, and its return is to God, through the return to
``He Whom God Shall Make Manifest.'' You write it, yet you do not know
to whom you write it. He will introduce Himself to you through the proof
by which the religion of all is established. However, the veils of your
own selves will prevent you from recognizing Him.

For instance, today, all the people of the Qur'an adhere to the command
of the preceding Manifestation, yet they remain veiled from the true One
who established that command. Despite observing His proof, which
surpasses that of the previous Manifestation in clarity and majesty,
they fail to perceive it.

Similarly, those who encounter these words may not comprehend them.
Instead, their understanding remains limited by their imaginations. They
imagine that, if the Manifestation of Truth were to appear, they would
be the closest of all creation to Him. Yet, these very souls, in every
Dispensation, are the ones who, toward the end of that Dispensation,
reach the highest ranks within its religion, only to experience what
must unfold. God is fully aware of all that you do.

\subsection*{Gate 14 (The Purification of Materials Sanctified by Divine
Decree)}\label{gate-14-the-purification-of-materials-sanctified-by-divine-decree}
\addcontentsline{toc}{subsection}{Gate 14 (The Purification of Materials
Sanctified by Divine Decree)}

The Fourteenth Gate of the Fifth Unity, concerning the purification of
fire, air, water, earth, the Book of God, the Point and its traces, and
other materials sanctified by divine decree:

God, out of the bounty of His grace, has commanded the purifying power
of specific elements and manifestations. If a soul were to give all that
is on earth as recompense, they could not claim mastery over even one of
these purifiers. All are made purifying by the decree of God,
originating from the Tree of Command. The true purifier is the divine
command itself, not the object in essence. Rather, the object becomes
the throne upon which that command is manifested.

Consider the one whose word establishes purifiers. Reflect on what is
said about the elemental realities that signify God. If they judged
according to their own beliefs, no argument could be made against them.
Yet, out of excessive caution, they use water to such an extent that it
becomes a burden. Still, they do not accept the decree of purity from
the one whose word makes water purifying, refusing to recognize its
power.

Their behavior resembles that of a soul who bears responsibility for the
slaying of the Prince of Martyrs, peace be upon Him. Despite all
evidence of the divine command, they reject what is clear and cling to
practices that obscure the essence of purity and faith.

Yet they question the blood of a mosquito in place of what truly
matters. This is the condition of such people, though even calling them
animals is unbefitting, for animals do not harm themselves. These
individuals, however, are clearly causing harm by their actions. God has
not commanded these purifiers except as signs of the purity of ``He Whom
God Shall Make Manifest'' and as evidence of His exalted station. But
who truly considers the fruits and the purpose of God's decree?

Had the people of the Qur'an acted with insight, matters would not have
reached this state. May God preserve the people of the Bayán from
becoming veiled from the purpose, so they do not bring calamity upon
that command. Otherwise, the structures of religion, in every
Dispensation, remain true to their context. For example, when the
Messenger of God appeared, whoever entered the religion of Islam gained
its associated privileges. However, consider the reality that all these
structures rest beneath the shadow of faith in Him.

This is why some grasp these privileges while remaining veiled from
their source. If someone possesses knowledge of the source and acts in
alignment with it, the religious and worldly aspects of that
Dispensation belong to them. Otherwise, they are rendered null and void,
as though they had never existed in the Book of God.

\textbf{Purifiers and Their Hierarchies:}\\
The manifestations of purity begin with the following:

\begin{enumerate}
\def\labelenumi{\arabic{enumi}.}
\item
  \textbf{Faith in the Bayán:} This transforms the individual's body
  into a state of purity.
\item
  \textbf{The Book of God:} Simply encountering a verse from it purifies
  even something devoid of existence.
\item
  \textbf{The Name of God:} By reciting the name \emph{Alláh} sixty-six
  times (\emph{Allah-u-Aṭhar}), any object becomes purified.
\item
  \textbf{Severing association from non-believers in the Bayán and
  establishing connection with its followers:} This act purifies.
\item
  \textbf{The Tree of Truth:} The ultimate purifier and the source of
  all purity.
\end{enumerate}

This hierarchy establishes the divine purpose, yet many fail to grasp
the essence of these commands, remaining fixated on superficialities
while neglecting the true origin of purity and faith.

On the Day of His Manifestation, all of His traces will be purified.

\begin{enumerate}
\def\labelenumi{\arabic{enumi}.}
\setcounter{enumi}{5}
\item
  \textbf{The four elements}
\item
  \textbf{The Sun}
\item
  \textbf{Any object whose essence undergoes transformation.}
\end{enumerate}

These apply when the object in question lacks an intrinsic impurity. For
instance, blood from the mouth caused by using a toothpick or miswak is
permitted and forgiven. However, God loves those who purify themselves
in every condition, and there is nothing more beloved in the Bayán than
purity, refinement, and cleanliness.

As for the hoof of an animal that steps into a muddy area and then
enters a room, it is exempt from burden. God, in the Bayán, does not
wish to witness anything less than joy and comfort in His servants'
lives. He desires that all maintain both spiritual and physical purity
at all times, ensuring that their own souls remain at peace with
themselves and others. There is no harm, for instance, in the presence
of animal fur or hair near one who prays, such as the materials brought
from the West, including ivory, bones, or similar items.

All these laws are designed to ensure that people experience the
expansiveness of God's mercy, so that, on the Day of the Manifestation
of Truth, they may be grateful to Him for the ordinances revealed
previously. God does not desire individuals to become excessively
cautious about trivial matters, such as a single strand of hair, causing
them to repeat their prayers unnecessarily. However, this does not mean
that, at the time of issuing religious decrees, one should be neglectful
of the principles of the faith.

Reflect on the past Dispensation of the Qur'an: those who oppressed the
House of the Prophet meticulously observed the minutiae of the law.
Similarly, in the Dispensation of the Bayán, anyone who reflects deeply
will recognize that great efforts were made to refine and elaborate upon
minor legal issues, often producing vast commentaries on a single minor
matter.

However, when it comes to affirming the Truth, which is the foundation
upon which religion is established, if this affirmation is neglected,
such a soul is no better than an animal, albeit a harmless one.
Conversely, if they write in opposition to the Truth, while engrossed in
trivialities, they veil themselves from the Truth and become deluded by
these displays of sanctity. Before God, these deeds amount to nothing
but scattered dust unless accompanied by faith in the Truth.

In the past, within the field of legal principles, some composed works
comprising hundreds of thousands of verses, ensuring that every aspect
of religious caution was accounted for. Yet, when it came to affirming
God and His signs, they hesitated. Had their hesitation been sincere,
they would have invalidated their own existence. Instead, they persist,
profiting from the very concept of divine unity---the foundation of
religion---which arises from the ocean of God's bounty, while their
actions betray a disregard for its sanctity.

O people of the Bayán, do not fall into the same errors as the people of
the Qur'an. At the very least, wherever you ascend in your actions,
emulate the harmlessness of animals. If you cannot bring benefit, ensure
that you cause no harm. Perhaps on the Day of the Manifestation of
Truth, you may avoid actions that destroy your faith without your
realizing it. This is the divine counsel to all: if you act upon it, you
will never face harm. The Sun of Truth, in its mercy and forgiveness,
encompasses all creation.

As long as a soul does not bear witness against itself by expressing
animosity, but instead shows love for the Truth, God may, out of His
bounty and grace, guide them to faith in Him and grant them the honor of
remembrance. Even if they are unaware of this grace, on the Day of
Reckoning, those who caused no harm will be granted mercy by the Point
of Truth.

He rewarded them with His signs, even though one was in the sea and
another on the land. God knows when His recompense will reach them, and
God rewards those who act righteously.

\subsection*{Gate 15 (Water and
Purification)}\label{gate-15-water-and-purification}
\addcontentsline{toc}{subsection}{Gate 15 (Water and Purification)}

The Fifteenth Gate of the Fifth Unity, concerning the water through
which you are created and its appearance in the Book:

Water is a single point worthy of being associated with purity, as it
symbolizes the radiance of the Sun of His bounty. These suns, reflected
in mirrors, represent the manifestations of His grace. By invoking His
name, God has made all things shaded under its purity. While permission
for purification has been given, God desires that this element be
refined to the utmost perfection. It should be preserved from excessive
use in circumstances of hardship.

No one should conceive of impurity when considering this supreme light
and its signs. All purifiers originate from the ocean of His bounty, and
the initial essence of purity stems from this primordial substance. It
is imperative that the people of the Bayán remain vigilant, ensuring
that no judgment of impurity is imposed on any soul among them. Anyone
who does so departs from faith. This protects the manifestations of
Truth between two Manifestations from sorrow caused by such actions.

Reflect on the extent to which, in the Qur'an, God's decree was violated
when judgments other than purity were imposed upon souls signifying God.
In reality, the purity of the believers' souls was a blessing arising
from their very existence.

During the days when I resided in the sacred shrines, there was an
occasion when the late Siyyid A visited a household. Upon his departure,
the owner of the house ordered that the door handle touched by the Tree
of Purity's hand be washed. Yet, in their own religion, where two dry
objects meet, no ruling of impurity is prescribed. How then could
impurity have been decreed in such a case?

This illustrates how that they exceed the rulings of their own religion
under the pretense of caution, while the foundation itself is eroded.
How, then, can the branches remain? \emph{``O servants of God, fear
Me.''}

\subsection*{Gate 16 (Love For One
Another)}\label{gate-16-love-for-one-another}
\addcontentsline{toc}{subsection}{Gate 16 (Love For One Another)}

The Sixteenth Gate of the Fifth Unity, concerning what belongs to God
and has no equivalent value, as long as the sun continues to shine. It
must be presented before God, and when the sun sets, permission is given
by God for any soul to possess it until the sun rises again from its
place of origin. At that point, it is no longer lawful to possess it,
and it must be returned to the equivalent of the number one (unity), and
nothing beyond that should be retained. If not, no obligation is imposed
upon them in this regard.

If something without an equivalent value exits someone's possession due
to their craft or trade, it is pardoned as a favor from God, provided
their livelihood depends upon it. If not, they are required to return
it, for God has granted them provision from another place and they are
not in need of this. However, if they are compelled by necessity, it is
forgiven as a grace from God, for He is the Most Generous and Bountiful.

The essence of the gate is that the greatest thing God desires to see
among the people of the Bayán is their love for one another. Whatever
arises among them---whether in the realms of knowledge, principles,
branches, the visible or invisible, the beginning or the end, elevation
or descent---they should not reject one another. Whoever enters the
Bayán is a believer, regardless of the station at which they stand.

If a soul within the Bayán rejects another soul from among its
followers, they become obligated to return ninety-five mithqáls of gold
to ``He Whom God Shall Make Manifest,'' not to anyone else. No one is
entitled to demand this from them except Him. Rather, it remains a
responsibility between them and God until that Sun of Truth either
forgives or claims it.

The purpose of this limit is to ensure that no one dares within the
Bayán to reject another soul or speak ill of them in matters other than
faith. Otherwise, the ranks of each person manifest according to their
station, and whoever occupies any position for the sake of God is
praiseworthy in their actions---whether they engage with the simplest of
subsidiary matters or stand firm in the loftiest principles.

Perhaps on the Day of the Manifestation of Truth, no one will be
satisfied with anything less than expressions of love for that Supreme
Luminary. Through this love, the people of the Bayán may avoid perishing
on the Day of Resurrection and attain salvation through their love for
Him, which is the essence of all faith. Since He is the sign of
\emph{``There is nothing like unto Him,''} God has commanded that
anything in His creation that reaches this station belongs to Him.

As long as the Sun of Truth continues to shine, no one is permitted to
claim ownership of what is befitting of Him except by His permission,
after the dues of the Manifestations of Unity have been fulfilled. From
the time of sunset, permission is given to the believers to take
possession of such things from His bounty and to express gratitude to
their Beloved until the next Manifestation. However, upon the Day of His
Manifestation, no one is permitted to delay even a fraction of a second
in returning to the true Owner whatever in creation has reached its
ultimate perfection, in accordance with the number one (\emph{waḥid}).

Anything beyond this is permitted only by His generosity before His
Manifestation. But for anyone who does not recognize God's decree on the
Day of the Manifestation and prevents what is without equivalent value
from being returned, it becomes obligatory for them to pay ninety-five
\emph{mithqáls} of gold. This ensures that no one can obstruct another
from contributing to the Cause of God.

Permission has also been granted to a craftsman whose livelihood depends
on producing something over the course of a year, provided that all may
dwell in the expanse of His mercy. Whenever someone takes possession of
something, they must return it to its true Owner upon His Manifestation.
Whoever desires to obey God will comply with His command. What greater
honor exists than for a servant to take pride in their obedience to
their Master? This is none other than a bounty from Him, for He has
permitted His creation to partake in this station. Otherwise, He is the
Self-Sufficient, independent of all else, through His very essence. All
are created from the ocean of His bounty and are held in the palm of His
grace and justice.

\emph{``To God belongs whatever is in the heavens and the earth and what
is between them, and God is Self-Sufficient, Almighty.''}

It is not that a soul prevents others for its own sake; rather, no soul
acts in any matter except for the sake of God. However, it is when a
soul becomes veiled at the time of any Manifestation that it fails to
recognize what pleases Him. This is when their deeds, which they think
are for God, become invalid. Even so, when something is rendered
invalid, it does not leave God's dominion, for all things belong to God.

This is why, when anyone approaches the Truth, they do so for Him and by
His will. Yet, when the owner veils themselves from it, they face
consequences. It is not that the proof of God was incomplete in their
case; for if the proof were insufficient, they would not be subject to
His decree.

During the Manifestation of the Messenger of God, the proof of God was
complete for the people of the Gospel, and the fulfillment of their
promise was made manifest through the same proof. Yet, because they
remained veiled, their actions were rendered null, even though they
could not conceive that the promised one of Jesus had come and they had
failed to believe. The matter, however, was evident. This is an
indication that at every Manifestation of Truth, the adherents of the
preceding Manifestation remain veiled by their assumptions of caution
and their interpretations of their religion. Yet God bestows His grace
upon whomever He wills through His guidance, for this guidance is more
precious to God than anything else. Without it, even if all else exists,
it is as though nothing truly exists. Conversely, with it, even if
nothing else exists, no goodness is diminished.

For those who attain it, everything promised to them in paradise will be
fulfilled. Consider the creation of paradise as akin to the creation of
all things: whatever is possible within its realm, God will bring into
existence, for He is All-Powerful and encompasses all things.
\emph{``There is no God but God; indeed, we are all devoted to Him.''}

\subsection*{Gate 17 (Recite the Sacred Names 95 Times Each
Day)}\label{gate-17-recite-the-sacred-names-95-times-each-day}
\addcontentsline{toc}{subsection}{Gate 17 (Recite the Sacred Names 95
Times Each Day)}

The Seventeenth Gate of the Fifth Unity, concerning the obligation of
remembrance in each month, that God should be mentioned ninety-five
times each day:

In this gate, it is decreed that from sunrise to sunset, every soul is
permitted to recite ninety-five times phrases such as \emph{``God is
Most Glorious'' (Alláh-u-Abhá),} \emph{``God is Most Great''
(Alláh-u-A`ẓam),} \emph{``God is Most Manifest'' (Alláh-u-Aẓhar),}
\emph{``God is Most Radiant'' (Alláh-u-Anwar),} \emph{``God is Most
Exalted'' (Alláh-u-Akbar),} or similar exalted expressions.

The purpose of this practice is that, through the blessing of reciting
these sacred Names, one may, on the Day of Resurrection, attain the
honor of guidance by the Supreme Luminary and the Radiant Countenance.
In doing so, one may be guided by the Letters of the Living on that day.

However, this act of remembrance should not become a veil preventing one
from recognizing the proofs of that Supreme Manifestation. These Names,
like the realities they signify, serve as guides to God. Just as the
Letters indicate that God is far greater than what can be described.
Similarly, these realities point to the truth that God is far beyond any
attribute that can be ascribed to Him. The Letters of Unity
(\emph{Ḥurúf-i-Wáḥid}) never exceed their symbolic station, for no one
has ever had or will have a pathway to the Essence of the Eternal.
Whatever exists within the realm of possibility is His creation, and the
Letters of Unity are the signs of His Names and the mysteries of His
creation. Within them, nothing is seen but God alone.

In their outward station, they are the creation of God. Yet in the
station where nothing is seen in them but God, they are the Letters of
Truth. \emph{``O servants of God, be mindful of Him!''} Do not
exaggerate concerning them, for even if you were to do so, you would
still be unable to comprehend them. Whatever station is attributed to
them must be rooted in their servitude, which has always been and will
always remain.

Their exaltation lies in their reflection of their inner hearts, which
indicate nothing but God alone. This is why they have been made the
signs of God's glorification, praise, oneness, and magnification.
However, when you observe them, do not become veiled by their station,
for when you do so, you become heedless. Instead, as with the words
\emph{``God is the Greatest'' (Alláh-u-Akbar)} where nothing but God is
seen, so too in these realities, nothing but God should be observed.

\emph{``Say: All are God's creation, and all worship Him.''}

\subsection*{Gate 18 (Transactions are Valid Through Mutual
Consent)}\label{gate-18-transactions-are-valid-through-mutual-consent}
\addcontentsline{toc}{subsection}{Gate 18 (Transactions are Valid
Through Mutual Consent)}

The Eighteenth Gate of the Fifth Unity, concerning transactions of
buying and selling when mutual consent has been established:

When mutual consent between the buyer and seller is realized, the
transaction becomes valid by any means, even through action alone. The
transaction is permissible for both minors and adults, and it is lawful
to negotiate as long as the terms are agreeable to both the buyer and
the seller. It is based on mutual consent, with terms agreed upon for
the transaction.

The essence of this gate isGod has permitted buying and selling when
mutual consent is achieved between the two parties. He has granted
permission for both minors and adults, as well as free persons and
slaves, to engage in transactions, provided that mutual consent is
ascertained. In the Bayán, such transactions are valid even if consent
is expressed through action alone.

God has also granted merchants permission to negotiate and adjust terms,
a practice common among them today. They may establish increases or
decreases with due time in their dealings, ensuring that all remain
within the expanse of God's grace and mercy, thereby fostering
gratitude. On the Day of Manifestation, this principle may apply to the
exchange of the light of the Sun of Truth for all else, through His
consent and the surrender of their own selves. By this, all who act
righteously may be saved on that day.

\subsection*{Gate 19 (The Right of God -
Huquq'u'llah)}\label{gate-19-the-right-of-god---huququllah}
\addcontentsline{toc}{subsection}{Gate 19 (The Right of God -
Huquq'u'llah)}

The Nineteenth Gate of the Fifth Unity, concerning the measure of a
\emph{mithqál}, defined as nineteen grains, and its equivalent in gold
and silver:

When a year has passed and the amount exceeds five hundred and forty
\emph{mithqáls}, and a portion equivalent to nineteen divisions has not
been reduced by the believers, then five hundred dinars for each
\emph{mithqál} of gold (nineteen grains) and fifty dinars for each
\emph{mithqál} of silver (nineteen grains) should be given to the
Sovereign of the Bayán. This is so that on the Day of the Manifestation
of ``He Whom God Shall Make Manifest,'' He may be self-sufficient from
His own bounty and that no sorrow may reach Him---even if the nearest
soul to Him rejects Him.

This is a right of God, assigned to ``He Whom God Shall Make Manifest,''
and within it lies the mystery of \emph{Qáf}, apparent to those endowed
with understanding. For those awaiting His appearance, He has permitted
such provisions to be made so that He may not experience sorrow nor feel
compelled to transgress the bounds of God, causing grief to any soul.

In the Bayán, no act of worship is more beloved to God than benefiting
another soul, even if it is merely by bringing joy to their heart.
Conversely, no action is more distant from His pleasure than causing
harm to another soul, even if it is simply by introducing sorrow into
their heart.

It is not that people perform these deeds in His name and then, on the
Day of His Manifestation, use His name as an excuse for hesitating to
affirm Him. Consider how, today, all that exists on earth is attributed
to that initial Divine Will, which manifests under the name of the
Prophet of the Dispensation in every age. Yet, despite this attribution,
they act as they do.

It is evident in this Dispensation that all prior Manifestations are
like shadows compared to Him. Indeed, the harm inflicted by those
associated with the previous Manifestation becomes apparent among those
who claim allegiance to the current Manifestation. This was not the case
with those associated with the Manifestation preceding the last.
Although they may not have affirmed Him, they did not cause harm to His
followers. However, harm becomes apparent when claimed adherents of a
previous Manifestation reject and act against the followers of the
current one.

If all observe His justice, there is no doubt that the believers in
truth will fulfill their divine obligations. Even if, in this
Dispensation, He walks upon the path of God's pleasure, His sufficiency
is independent of worldly conquests. He bestows grace upon those who
seek shelter in His dominion, pardoning their obligations to God's
rights. For God has always had at His disposal all that exists on earth,
such that if all were to arise in support of His Cause, everyone would
find themselves enriched.

Permission has been granted to define the measure of a \emph{mithqál} of
gold as nineteen grains, and likewise for silver. The value of both is
established according to what is apparent today. If anyone wishes to
expend these, they should divide them into nineteen portions, ensuring
that the increase among the believers in the Bayán is not diminished by
anything other than gold and silver. This will enable all to express
gratitude within the expansive grace of God.

In this ruling lies a hidden treasure of wisdom for those endowed with
understanding. If one reflects with insight, they will perceive all the
mysteries of divine unity with the eye of certainty. No limits have been
imposed on what is taken, as long as the bounds of the matter remain
within their proper station, and it aligns with the truth of \emph{Qáf}
under its deserving shade.

Although, for merchants today, some discrepancy in the value of gold and
silver may appear, such discrepancies will be resolved after the process
is properly enacted. If someone owes even a single \emph{qirán}
(twenty-eight grains), it is their responsibility to pay it, ensuring
that its value remains intact. This is the measure for gold as twenty
grains, not its equivalent, unless it is through buying and selling. All
these rulings are designed to ensure that no soul becomes sorrowful on
the path of God on the Day of His Manifestation. This is so that no
occurrence at the time of His Manifestation causes the people to suffer
in their spirits, particularly those who, in the Bayán, are judged
according to the law of faith. \emph{``God guides whomever He wills to
the path of certain truth.''}

There is no paradise greater for the monotheists than that which results
from acting upon God's commands, nor is there any fire more intense than
exceeding the bounds of God or one soul wronging another, even by the
weight of a mustard seed. For those who act in accordance with faith in
God and His signs, \emph{``God will decide with justice on the Day of
Resurrection, and indeed, we are all supplicants for His grace.''}

\section*{Vahid 6}\label{vahid-6}
\addcontentsline{toc}{section}{Vahid 6}

\markright{Vahid 6}

\subsection*{Gate 1 (Organization of the
Bayan)}\label{gate-1-organization-of-the-bayan}
\addcontentsline{toc}{subsection}{Gate 1 (Organization of the Bayan)}

The First Gate of the Sixth Unity, concerning the organization of the
Bayán, which must not exceed nineteen volumes:

The Bayán should be organized into no more than nineteen volumes. In the
first three, the verses are to be written; in the next four, the
supplications; in the following six, the interpretations; and in the
final six, the scientific illustrations. The gates within each of these
divisions should range from the unity (\emph{waḥid}) to the name
\emph{Mustagháth}. Additionally, every individual should have a booklet,
even if it contains no less than one thousand verses, so they may recite
from it as they wish.

The alphabet consists of thirty letters, and the diacritical marks are
ten in total.

\textbf{Summary of the Gate:}\\
Nothing is more beloved to God than moderation. For example, if someone
owns land, and they wish to display moderation in their actions
concerning it, they should do so in such a way that the owner of an
astrolabe, observing with utmost precision, sees no excess between the
beginning and end of their work. If such balance is beloved to God in
the physical realm, how much more so must it be in matters where
moderation can manifest?

God has permitted and ordained in the system of the Bayán that it should
not exceed nineteen volumes, and that in all matters, utmost moderation,
order, and arrangement should be recorded. If someone examines it in
sorrow, they will find not a single letter in excess from the beginning.
This is not because such is commanded, but rather because it is in the
highest degree of moderation that it is mentioned. Moreover, what is
even more just has already passed in the knowledge of God, in that the
numerical values of the letters should be observed, which is in no way
fully possible for humanity.

All that was revealed in the first atom and the second atom was not
commanded, and the title of Bayán, in its primary reality, is applied
solely to the verses. For it is the supreme proof and the greatest
evidence that indicates nothing but God alone. In its secondary reality,
it is applied to supplications; in the third, to commentaries; in the
fourth, to scientific words; and in the fifth, to Persian words. Yet all
are mentioned under the shadow of the verses, though the secret of
eloquence that is manifest in the first is also concealed in the last.
However, since all cannot comprehend it, it is not explicitly mentioned.

The purpose of this arrangement is that, on the Day of the Manifestation
of the Lord of the Bayán, all beings may conform to the letters of the
Unity and may be traversing the ocean of oneness. Thus, on the Day of
the Manifestation of Truth, they may become illumined by the radiance of
the sun of reality and draw near to their hearts the presence of that
most exalted Luminary, free from religious and worldly concerns that are
apparent.

In each of the five degrees, if all the levels are mentioned, it is most
beloved in the sight of God when done with the utmost moderation. Even
though in the first third, unpolluted water flows forth. He demolished
four after bricks whose taste had not changed, and in the fifth, after
the wine of unity, and in the sixth after that, what is mentioned in the
corner with glorification flows forth. Yet in each, all outward and
inward realities were and are present. Indeed, in each letter, the
observer perceives, and in each point, as is the nature of the four
elements, the form is evidence of its essential nature. However, it is
not the case that within this oneness anything other than unity is
observed. If one is added, it becomes twenty, not two.

The mystery is that within this oneness, nothing is seen but the
likeness of unity, beyond enumeration. For this reason, they may
approach, with their hearts, the presence of the Primal One at the
beginning of the Day of Resurrection. Reflect upon the Bayán: although
it is ordained to be compiled, all has, from the outset of
manifestation, been one water. Within all these letters of unity, it is
that same animate reality. It is not the case that there exists a
creator other than God, or a provider besides Him, or one who causes
death or gives life apart from Him. Rather, this Primal Unity is His
creation, signifying within the realm of possibility a oneness that can
appear in existence.

Otherwise, that oneness of the Essence is and will remain unknown,
unassociated with mention. It is not within enumeration nor known by
measurement. The result of this order is that, on the Day of
Resurrection, perhaps all believers in the Bayán may be guided by the
letters of unity. Yet they should not be deluded by love for the Point
and the living letters, for that day is the Day of Testing. If someone
loves that Point and those letters and is guided by them, it is evidence
that this Point and these letters are loved by them. Otherwise, observe
how many claim to love and obey the Point of the Qur'án and its living
letters, yet not even one in a thousand has truly entered the allegiance
of this Point and these letters. In that Manifestation, as in this one,
imagine the same reality, and be vigilant that you do not become veiled
by anything on that Day. All of the Bayán and its spirits are, in the
presence of Him Whom God shall make manifest, like a garden in His
grasp. Just as today, the entire Qur'án and its spirits, if flowing from
the Point of the Bayán, are indeed under the mercy of God, so they have
been and will continue to be. But if they proceed contrary to that mercy
and its decrees become manifest, observe whether they are lighter than a
garden or not. This is the true decree in the sight of God.

Even though under the shadow of each letter, countless souls take
shelter, and each one is honored and ennobled by a command from the
Qur'án, all are upheld by that single thread. That thread rests in the
hand of the One who revealed it, not in the hand of anyone else. Imagine
if the Messenger of God had revealed a verse stating that all believers
in the Gospel were encompassed by His mercy---could anyone then decree
anything less than mercy for them?

Now consider those who cannot even be enumerated, let alone all of
them---how could anyone reach the entirety? Even when He did not state
this explicitly, but instead decreed something less, observe how many
remained confined within their limits. This is the essence of all
knowledge: with a single ``Yes,'' all come to life, and with a single
``No'' from Him, all perish. This is the truth---not through imagined
qualities such as wrath, dominance, or any other conceivable attribute,
but by the decree of God. This is the decree of reality, beyond which
nothing is conceivable. It proceeds from the letter exactly as it is,
and encompasses all things.

It has commanded the utmost effort and diligence so that, on the Day of
the Manifestation of Truth, none may argue against Him using His own
words. For all the Bayán is the speech of the previous Manifestation
before Him, and He is more knowledgeable of what He has revealed than
all creation, as the spirits of all are in His grasp. In the presence of
all, there is nothing but a shadow, provided they are firmly established
in the truth; otherwise, they are unworthy of mention.

For instance, if someone today is like the Plato of their time in every
field of knowledge, but does not acknowledge the Truth, will their
knowledge bear any fruit for them? Exalted is God above such things!
Indeed, for them, there is no knowledge. True knowledge is knowledge of
God, His Messenger, the Manifestations of His Cause, and the
Manifestations of His decree. Anything below this is not considered
knowledge by those endowed with hearts. Similarly, in the time of Jesus,
the mere acknowledgment of His prophethood did not benefit those who
failed to affirm Him. Consider this as well in relation to Him Whom God
shall make manifest. If all the people of the Bayán were to attain the
essence of knowledge to His level, it would bear no fruit unless they
affirm Him.

Reflect deeply, O people of knowledge, and guard yourselves concerning
Him. In the arrangement of verses, the structure progresses from surah
to surah, from a single verse up to the invocation of ``Mustagháth.'' It
is fitting that each soul should possess a scroll of at least one
thousand verses, comprising the teachings of the Bayán, with each
finding joy in what delights them. Each verse consists of thirty
letters, and with diacritical marks, it is counted as forty letters.

The entirety of the Bayán is likened to a jewel entrusted by one to
another as a trust. On the Day of the Manifestation of Him Whom God
shall make manifest, if He wishes, He may take it from all what has been
given to them, they would not endure even for a moment with patience. It
is not because of being veiled by the branches of the matter, or one by
its principles, or another by the aspects of its decree, or yet another
by the aspects of its grandeur. Rather, all are from Him and return to
Him. They recognize Him through His verses but fail to exercise caution
in understanding Him. To the extent of this neglect, they will remain
veiled in the fire.

If you reflect between yourselves and God, that same verse by which you
turn to Him is a likeness of Him, as it is within your hearts. Do not be
veiled from Him by it, and do not fail to recognize Him for whom all
were created to know. Be vigilant, for if He makes Himself known by His
name, He also makes Himself known by other aspects through His verses.
Thus, no one will have a proof greater than their capacity.

Sometimes, one may sit within their house, gazing into the Bayán,
imagining that His Messenger or His book will come to them. But, being
blind in their religion, they remain without certainty and in an instant
enter the fire. The Bayán, which they had followed, was merely from
themselves, while the book He revealed with His verses is far greater in
His sight. For it is a divine gift for the hereafter, which is more
exalted before God than the manifestations of the past that He revealed
from the treasury of His grace and mystery.

It has not been as it is today, where each soul possesses a Qur'án to
which they adhere in the religion of Islam, yet they are veiled from its
sender and the one unto whom it was sent. How often the value of the
Qur'án may be worth thousands of measures of gold, yet they are veiled
by some aspect of the sender or the receiver. If they knew, they would
wish they were not created to remain deprived of the fruits of their
existence. The utmost precision has been exercised, for the path is more
delicate than anything, yet in its knowledge, it is broader than all
things. ``God begins all things and then returns them. Say: `Do you,
then, believe with certainty in what God has revealed in the Bayán?'\,''

The knowledge of the Bayán is dependent on understanding its revelation
from its beginning to its end. What was revealed at the start aligns
with Qur'anic laws, and later, the measures of the Bayán became
manifest. It is not the case that the beginning is veiled from the end.
Until one observes the end, they have not truly considered the
beginning. Every letter, in its station, is described by its fire or
light within its domain, and this has been and will be so. Yet, the more
novel it becomes, the more apparent is the divine intent.

The entirety of the Bayán is the utterance of the Point of Reality, just
as all religion consists of recognizing Him and understanding His
commands. ``God guides whom He wills to the path of truth with
certainty.''

\subsection*{Gate 2 (Purification of the Body and
Essence)}\label{gate-2-purification-of-the-body-and-essence}
\addcontentsline{toc}{subsection}{Gate 2 (Purification of the Body and
Essence)}

The Second Gate of the Sixth Unity, concerning the ruling of God
regarding wells, the law of ``kurr'' has been abrogated, and all have
been commanded to use pure water. Verily, water is pure, purifying, and
inherently sanctified in itself, by itself, and for others through
itself, provided it is not altered by any of the three forms of change.
The ruling of a part of it is the ruling of the whole, and vice versa.

The summary of this gate is that God, in His grace and bounty, has
decreed the ruling of a small quantity of water to be like that of a
large quantity, so that all may partake of His expansive favor and
mercy. The earlier ruling concerning wells, which was based on what
might fall into them, has been abrogated. However, He has desired that
in every land where His servants dwell, there should be a reservoir of
water to facilitate purification and cleansing. And there is no house in
which a reservoir is not constructed, angels do not pass through it.
However, if the command comes from God, they will pass through it at
dawn. All the people of the Bayán have observed this practice with
precision, ensuring that no soul is deprived. The manifestation of the
law of purification in this water has not occurred except by the decree
of the water of the ocean of unity. Just as a single drop of it
signifies God, so too does the whole signify God. Water, in itself, is
pure, purifying, and inherently sanctified, and this has always been so.

In purification, the dominance of water over the object is better than
the object entering the water, for compulsion becomes evident in the
latter and is removed in the former. Consider first that all
purification is decreed by the word of God, and it is through water that
purity is attained. If, on the Day of the Manifestation of Truth, He
decrees the purity of an object, let it not seem strange to you, for it
is His word that sanctifies, not the object itself. Though the initial
appearance does not occur except by His command, it is as if one sees
that all the people of the Bayán follow the earlier law of purification
through water with utmost diligence, yet on the Day of His
Manifestation, they become veiled by those same practices.

Just as it is apparent today, O people of the Bayán, do not act as the
people of the Qur'án have done. They exert the utmost effort and
diligence in purifying their physical bodies with physical water, yet
they become veiled from purifying their essential selves with the water
of unity. Cleanse yourselves, for any heart in which there is love for
anything other than God is not pure in the sight of God. Similarly, any
spirit, soul, or body in which love for anything other than the letters
of unity and those who love them resides is not pure in the sight of
God.

This is because the ruling on the purity of the body depends on the
purity of the essence, which is determined by faith, The body is
purified not by anything other than this. For if it were otherwise, no
nation among the Christians would surpass them today in outward bodily
refinement. Yet their ruling is manifest. In the same way, at the time
of the appearance of Him Whom God shall make manifest, all must purify
themselves through faith in Him. Even if, on a single day, they immerse
themselves a thousand times in the sea and emerge, the ruling of bodily
purification will not apply, let alone essential purification.

Do not let such acts veil you from the sanctity of Truth, as they are
veiled today by the outward purification of the body. They exercise such
meticulous caution that it is unparalleled, composing fifty thousand
verses about it. Yet, when reflection on this purification invalidates
their essential reality, they pay no heed to the ruling of purity.
Instead, they issue judgments on matters so improper that the pen
hesitates to record them.

Exercise utmost precision so that you are not veiled from either the
outward or the inward. Rather, perfect both to their fullest extent.
Perhaps, on the Day of the Manifestation of Truth, you may become
illumined by the radiance of the Sun of Reality. If that Sun of Reality,
in a single moment, decrees purity for something---whatever it may
be---its ruling is truth for all. It is incumbent upon all to obey and
act accordingly.

What people practice today is derived from the command of the previous
Manifestation, and in His sight, the origin of the command remains the
same in all cases. The ruling on a part of water applies to the whole of
it. When poured over something that no longer retains its inherent form,
it purifies it a second time. However, if something enters water and the
water is such that it does not cause repulsion in the soul, a single
occurrence suffices for purification, even if the water is minimal.
Otherwise, purification with such water is not beloved. If something
impure changes the nature of water, its use becomes undesirable, though
the inherent purity of the water itself is not nullified unless it
ceases to meet the criteria of water. If mixed with a pure substance,
even if its nature changes, it remains pure. However, if mixed with
soil, it does not take on the status of ``added water,'' and
purification with added water is not permitted. This ensures that the
wealthy do not assert superiority over the poor. Otherwise, in the sight
of God, the matter is closer than the blink of an eye.

Yet nothing is more beloved to God than preserving water in its utmost
purity. If knowledge reveals that a cup of water has passed over impure
ground, it will not feel refreshing to the clean heart of a believer.
The ruling of purity after change has been made to ensure inclusivity,
but otherwise, what soul would desire proximity to such water for
purification?

God does not favor excessive scrutiny in purification that leads to harm
for the soul in the end. Rather, the soul of the believer is far nobler
than to be affected by anything impure. The ordinances of purification
are wholly intended to train souls in refinement and purity to such an
extent that no soul finds its own self undesirable, let alone others.

Perhaps, on the Day of the Manifestation of God, a matter may arise, and
aversion to it may appear---something less than His pleasure, less than
God's pleasure. For His pleasure is God's pleasure. Thus, it is fitting
that a believing servant should guard their senses, especially their
sense of smell, so as not to encounter any scent contrary to love for
Him, not to inhale any scent contrary to His love. Likewise, during
one's journey, do not perceive scents contrary to those of the
believers, lest you turn away from the pleasure of your Beloved and
remain unaware.

\textbf{Say:} ``It is water that purifies you by the permission of your
Lord. O servants of God, be grateful.''

\subsection*{Gate 3 (House of Truth in Every
Village)}\label{gate-3-house-of-truth-in-every-village}
\addcontentsline{toc}{subsection}{Gate 3 (House of Truth in Every
Village)}

The Third Gate of the Sixth Unity, concerning God's decree that His
servants should construct a House of Truth in every village.

The summary of this gate is that in this Manifestation, as what was
destined has occurred, and all have been graced by the blessing of the
decree of the Messenger of God, they have been content with the origin
of this decree as it has been revealed. For this reason, it was ordained
in the Bayán that no land should be inhabited without a place of
refinement being constructed there, such that if, on any day, a believer
is afflicted, they may perform acts of refinement to the utmost degree
within the Faith of the Bayán.

This is because no command in other matters has been made with as much
emphasis on refinement as in this one, nor has it been forbidden for
anyone to exhibit anything with visible imperfection, so that all might
emulate its perfection. For instance, if someone constructs a building
but does not complete it to the fullest extent possible, not a moment
passes over it without the angels beseeching God for retribution upon
that person. Even the particles of the building itself request this, for
every entity desires, within its station, to attain the utmost
perfection of its potential.

When someone possesses the ability to manifest this perfection but does
not do so, they will be questioned about it. Perhaps, on some day, the
Sun of Reality may pass by during the final Manifestation and, upon that
land, witness the obedience of His believers and be pleased. In this
religion, it has been commanded to establish doors that pertain to any
position should be constructed in such a way that a tall person can
enter without lowering their head. On every day and in every
circumstance that is elevated to such a position, suitable seats should
be prepared for them to be made manifest. Perhaps, on the Day of the
Manifestation of Truth, nothing will be observed in His dominion that
causes sorrow. For all the fruits of the previous Manifestation and all
that has transpired over the years must become evident on the Day of
Resurrection.

In the Manifestation of Him Whom God shall make manifest, God knows at
what age He will reveal Him. Yet, from the beginning of the
Manifestation to the number ``One,'' it has been decreed that in every
year, faith in one letter must be manifested from all creation. After
the divine permission, no one can express the fruits of the previous
Manifestation except through the subsequent Manifestation.

As has been observed in this Manifestation, no one has allowed the
fruits of the twelve hundred and seventy years to exceed the limit of
the Báb. And if one fails to manifest the fruit of their existence, they
remain veiled by their own self. However, in the Manifestation of Truth,
utmost haste has been commanded, for the decree of God is swifter than
anything.

At the moment of hearing the Manifestation, all should reveal the fruits
of the Bayán to the extent possible. If they fail to do so even for a
moment, everything they have done may be obliterated. It is possible
that nothing will be accepted from them unless the fruit of the
subsequent Manifestation is revealed. God does what He wills and decrees
what He desires.

\subsection*{Gate 4 (Do Not Dwell Beyond the Measure of 5 - the Letter
H)}\label{gate-4-do-not-dwell-beyond-the-measure-of-5---the-letter-h}
\addcontentsline{toc}{subsection}{Gate 4 (Do Not Dwell Beyond the
Measure of 5 - the Letter H)}

The Fourth Gate of the Sixth Unity: God has not permitted any to dwell
beyond the measure of five except for the Letters of the Bayán, even if
time stretches long upon them.

The summary of this gate is that, in the same way that God takes
everything unto Himself. Similarly, in the Manifestation of the Letter
``H,'' the essences of the earth were drawn in and confined within the
boundaries of the ``H.'' This is because the radiance of this Word
appeared closer and more manifest within these five boundaries than in
other places. Indeed, this influence extended from this limit to other
limits, just as the souls in these lands became guides for other souls.

In like manner, in the lands of the Cause of God, this is evident to
those endowed with knowledge. Within the five ranks of the unity of
hearts, the hearts of those who belong to them ascend toward God. In the
land of ``F,'' the manifestations of unity proclaim ``There is no God
but He'' and are illumined within the mirrors of hearts. In the
manifestations of ``Ayn,'' unity is expressed as ``There is no God but
I'' within the sanctified hearts. In the manifestations of ``Alif,''
unity is declared as ``There is no God but God'' within the hearts of
those who affirm the unity of God.

In the manifestations of ``Kha,'' unity is expressed as ``There is no
God but You'' within the hearts of those who magnify Him. In the
manifestations of ``Mim,'' unity is declared as ``There is no God but
the One who created all things by His command'' within the hearts of
those who glorify Him, as they are illumined in the mirrors of creation.
If all these ranks appear in each land, and indeed within every atom,
they do so with the purpose of revealing the manifestation and conveying
assistance from the Source. This is so that those belonging to it may
ascend to the universal essence and enter the five paradises of names,
expressed within the boundaries of the ``H'' in the human form.

If one observes all lands, they will notice the lines of the ``H''
within these five. However, this decree remains fixed until the
Manifestation of Him Whom God shall make manifest. At that time, this
location will become the dawning place of the Source's radiance. For
during the Manifestation of the Furqan, Mecca became the point of
radiance for the lands, and during the Manifestation of the Bayán, the
land of ``F'' served that purpose. God alone knows from which horizon of
the earth that Sun of Reality will rise.

For this reason, it has been decreed that not even an atom of what is
beloved by God should remain on this expanse. If the power of
observation existed, it would have been commanded that the area be
elevated above the waters to the lofty boundaries and adorned with
diamonds, so that no knowledge, other than that of the beloved, would
encompass it. However, since this is beyond the capacity of this
creation---except as God wills---one soul's merit on these five expanses
is superior to twelve thousand years of worship, even if firmly
established in faith in the Truth. Otherwise, they remain under the
shadow of that which has been taken away.

If it were possible for all the walls to be made of red ruby, God's
command would have been carried out to distinguish the form of the ``H''
in these lands from all others. The expanse chosen as the dawning place
of the Manifestation of God would be elevated above all others. However,
this elevation will not occur unless it accepts the prostration for the
expanse of Him Whom God shall make manifest. Otherwise, it would not
have been created. This is its distinction over all lands, just as souls
are distinguished---had they not chosen to accept faith in Him, they
would not have been created.

In every matter, one should supplicate for His bounty, so that He may
illuminate all possible realities with what is worthy of His ocean of
generosity, for all are sustained by Him. Reflect today upon the Qur'án:
how many prayers are offered daily? All were gathered under the shadow
of ``Establish prayer'' and derived assistance from it. Even if all that
is on earth bore this command, it would still have been sustained.

Indeed, if the Day of Resurrection of this Manifestation had not come,
the assistance for those who pray would have continued endlessly,
without limit. There would have been no cessation of His aid. Such is
the exaltation of God's Cause and the loftiness of His decree, by which
all are guided. That Word, in its station, is like the sun is
established at the point of culmination. God has created nothing more
exalted than the essence of knowledge that is coupled with action. There
is no true knowledge except knowledge of the Source of the Cause and
comprehension of all matters branching from it until the Day all return
to Him. For the distinction between a believer and others is solely
rooted in knowledge.

Consider the believers in the Qur'án: they became believers because of
the knowledge that it is the Book of God. Those below them, comprising
all that is on earth, remained without this knowledge and therefore
became non-believers. In the same way, view reality on the Day of
Ashura: one person gives their life for Him in truth, while another
commits acts against Him that are unjust. This distinction lies purely
in knowledge. Outwardly, both appear human, but their essence differs.

Compare the Truth to the sun and the believer to a mirror. The moment
the mirror faces the sun, it reflects it. Now compare the non-believer
to a stone: no matter how much the sun shines upon it, it cannot reflect
its light. This is why one gives their life for the Truth, and the other
acts unjustly against it. However, if God wills, He can transform even a
stone into a mirror, for He is all-powerful. Yet, the stone itself has
been satisfied with its state; had it desired to be crystal, God would
have created it as such.

Consider that on that day, the very cause that led believers to faith
was the same for others. But those veiled by their own will became
veiled by that same cause. This is evident today: those who turn toward
the Truth are drawn to the Bayán, while those veiled remain veiled by
the same.

Similarly, on the Day of the Manifestation of Him Whom God shall make
manifest, reflect that all existence, whether present in creation or yet
to be formed, owes its being to Him. Even if before the Manifestation,
He may remain unknown within His own house. Even before His physical
manifestation, while in the cradle crying for milk, He has always been
the source of assistance for all. What has been is from the
Manifestations before Him, and what will come is from the Manifestations
after Him. And this single reality is like the sun: if it rises without
limit, it remains singular in its essence within the realm of
origination. And if it sets, it remains singular in itself within the
exaltation of innovation. All are sustained by His radiance, from the
realm of hearts, where the essence of unity shines forth, to the
furthest extent of inanimate objects, which represent the final limit of
the outpouring of grace.

\textbf{``God shall settle whomever He wills of His servants in the
chambers of Paradise. Truly, He is one of immense bounty.''}

\subsection*{Gate 5 (Greetings To Magnify and Exalt
God)}\label{gate-5-greetings-to-magnify-and-exalt-god}
\addcontentsline{toc}{subsection}{Gate 5 (Greetings To Magnify and Exalt
God)}

The Fifth Gate of the Sixth Unity, concerning the ordinance of greeting
and responding in the spirit of magnification.

Men should greet with ``Alláh-u-Akbar'' (God is the Greatest) and
respond with ``Alláh-u-A`zam'' (God is the Mightiest). Women should
greet with ``Alláh-u-Abhá'' (God is the Most Glorious) and respond with
``Alláh-u-Ajmal'' (God is the Most Beautiful).

The summary of this gate is that the essence of the Qur'án is manifested
in the declaration of magnification. God, the All-Knowing, revealed this
utterance in the precincts of the Bayán. It is the first tree that
proclaimed the greatness of God in the kingdoms of the heavens, the
earth, and what lies between them. God revealed in response that what
soars above is greater in majesty than what was described or mentioned.

From this mystery, it was decreed in the Bayán that greetings should
begin with the magnification of God and responses should affirm His
exaltation. Likewise, among those of the circles, the greeting
``Alláh-u-Abhá'' should be met with the response ``Alláh-u-Ajmal.'' All
should meet one another with this.

The result of this practice is that perhaps, on the Day of the
Manifestation of Him Whom God shall make manifest, they may acknowledge
that, after God, He alone is worthy of being adorned with the attributes
of greatness, mightiness, glory, and beauty in the realm of origination.
He is beyond being described, characterized, praised, or glorified.
Perhaps those who proclaim these utterances will be able to turn toward
that Sun of Reality. Likewise, in names and similitudes, this single
reality moves within its own essence.

If you reflect on the beginning of the Furqan, the essence of all the
manifestations of Islam came from Him. Similarly, if you reflect on its
return, you will observe that the essence of all is present in the final
letter, where nothing is seen in the return but the selfsame beginning.
Even though He may outwardly appear in the pillar of magnification, it
transforms into glorification in His heart. On the same day He revealed
His knowledge, its assistance came from the same point of the Qur'án,
for all that He uttered returned to it.

Even on that day, that Manifestation stood before God, yet He and all
others were sustained by Him through the prior Manifestation. In the
same manner, reflect on the Manifestation of the Bayán, considering that
everything originates from Him. Observe carefully: nothing comes to Him
but rather emanates from Him. And all that has been commanded---that no
soul should grieve another---is for the benefit of that soul itself.
Otherwise, who among the rest has the rank for the Point of Reality to
issue such a decree concerning them?

Yet, because in the depth of night eyes do not perceive Him, perhaps
this brings blessings so that no one is grieved by it. All, by means of
this, move in the ocean of grace until the day He reveals Himself to His
creation and proclaims: ``I am the master of the Bayán, which you now
act upon according to its ordinances.''

Yet, on the Day of Resurrection, none can endure in the shadows of the
fire of God unless they enter, except those whom God wills. This is
because they witness that the heavens, the earth, and all that is
between them are filled with the commands of the previous Manifestation.
Yet, when they look upon Him in that form without seeing His true
essence, they become veiled. But if they reflect on the origin of that
same form, they will not be veiled.

Consider the countless prayers offered today under His shadow; the
number of those sheltered by Him is beyond reckoning. Reflect on the
beginning of the Cause and examine the records to see what transpired,
deeds so unspeakable that the pen is ashamed to record them. Similarly,
observe the Bayán, and likewise the Manifestation of Him Whom God shall
make manifest, so that you are not veiled from the Sun of Reality or
lost in the ocean of multiplicity, veiled from the mystery of unity.

Look today at Islam: whatever the Muslims do in the name of the religion
of the Messenger of God, they claim it to be part of His faith.
Contemplate the Sun of Reality, whose words were the origin of Islam.
Reflect for a moment on what has transpired. Is it not true that the
deeds of the people are no proof? Rather, the people of Truth in that
time were limited to the believers in the Gospel, all of whom awaited
the name of the Prophet and His descriptions.

God knows how many proclaimed the religion of Christ in His name. And
yet, the Sun of Reality, whether for seven years according to one
account or nine years according to another, was unable to reveal the
Qur'án in its entirety as it truly was. Similarly, consider the
Manifestation of Him Whom God shall make manifest. When He appears, all
the believers in the Bayán will be steadfast in their faith and
expectation.

But the moment of His appearance will reveal the truth of their faith.
If not a single soul is veiled from belief in Him, their sincerity will
be evident. Otherwise, in an instant, everything they have done will be
scattered as dust, for all that they possess of their religion
originates from the previous Manifestation. Exercise the utmost care so
that, at the end of every Manifestation, you are not veiled from the
Source of that Manifestation by its elevation. This is the essence of
knowledge, should you be able to grasp it.

\textbf{``God sends blessings upon those who believe in Him and His
signs and are assured of their meeting with Him on the Day of
Resurrection.''}

\subsection*{Gate 6 (Erase All Books Except Those That Affirm the Cause
of
God)}\label{gate-6-erase-all-books-except-those-that-affirm-the-cause-of-god}
\addcontentsline{toc}{subsection}{Gate 6 (Erase All Books Except Those
That Affirm the Cause of God)}

The Sixth Gate of the Sixth Unity, concerning the ordinance to erase all
books except those created or to be created within that Dispensation.

The summary of this gate is that in every Manifestation, just as the
inner realities of hearts, spirits, souls, and bodies appear anew, so
too do the fruits that emerge from these trees of love. Their essence is
what was previously made manifest. If hearts, spirits, souls, or bodies
derive sustenance from them, they are immediately overtaken by death.
What death is more grievous than that which arises from a decree falling
short of faith?

Reflect on this: consider the two testimonies by which a Muslim becomes
a Muslim. These same testimonies existed in the time of Christ and
remain to this day, yet they are not regarded as binding upon others.
The same applies to this Manifestation and to the Manifestation of Him
Whom God shall make manifest. In the beginning of every Dispensation, a
new expression of submission to God is revealed.

How, then, can this apply to matters branching from it? For this reason,
it has been decreed to erase all books except those that affirm the
Cause of God and His religion. Consider, from the days of Adam until the
Manifestation of the Messenger of God, the heavenly scriptures even
though all were the truth and from God, at the appearance of the Furqan
(Qur'án), all prior scriptures were abrogated. A decree of invalidity
was revealed in the Furqan concerning the beliefs of those who held to
them. Similarly, at every Manifestation, this principle can be observed:
even the books attributed to God are ruled as abrogated in each new
Manifestation.

Now consider: if the divinely attributed scriptures are deemed abrogated
at every Manifestation, what of the books of human creation, which, in
comparison to those sacred texts, are but shadows in a mirror relative
to the sun? It is as though one sees Him Whom God shall make manifest
writing the testimony of faith, bearing His name in every word. He is
the essence of all religion, and should a soul not recognize Him the
moment this testimony is revealed, that soul remains in disbelief.

Everything possessed by the Bayán is erased in the same manner at the
appearance of each new Manifestation. From the Bayán until the Day of
Him Whom God shall make manifest, the signs of the Point and all that
serve as proofs of Him within His shadow are the fruits of the paradise
of oneness. What has been mentioned in traditions, that the followers of
the Shí`ah would endure great tribulations on the Day of the
Manifestation of Truth, reflects this reality.

As in the time of the appearance of the Messenger of God, the same
principle applied to those who did not believe in Him. Even though they
were sustained by that which was revealed by God to Jesus, I swear by
the sanctified Essence, which has no partner and will never have one,
that reciting a single verse of His verses on the Day of Him Whom God
shall make manifest is greater than all of the Bayán and what is
abrogated within it.

This is because, on that Day, faith in Him will be determined by
adherence to His words, even if it be to a single verse. For those who
do not turn to Him, even if they ascend to the highest heights of the
Bayán, it will avail them nothing unless they return to Him.

O people of the Bayán, do not remain veiled from the sustenance of His
new revelation at the Manifestation of Him Whom God shall make manifest.
Do not be veiled as the people of the Furqan were, sustained by the
provisions of the prior Manifestation while remaining veiled from the
provisions of His new revelation. This is the essence of all knowledge
and action, should you be able to comprehend it.

\textbf{``God guides whom He wills to the path of true certainty.''}

\subsection*{Gate 7 (Dowries and
Marriage)}\label{gate-7-dowries-and-marriage}
\addcontentsline{toc}{subsection}{Gate 7 (Dowries and Marriage)}

The Seventh Gate of the Sixth Unity, concerning marriage and the
prohibition of dowries exceeding ninety-five mithqáls of gold for the
people of the cities, and ninety-five mithqáls of silver for the people
of the villages. The dowry must not be less than nineteen in either
case. The increments or decrements should not vary except by units of
one according to the numerical value of ``one.'' The rule of separation
is lifted, and the law of union is established following mutual consent,
with the mention of the Word of God within it.

The summary of this gate is that God, in His bounty and grace, abolished
in the Bayán the boundaries of separation to ensure no soul experiences
abasement in seeking God's pleasure for themselves. Consent is granted
for both man and woman with a phrase signifying their devotion to God
and satisfaction with His decree. This has been elaborated in its
appropriate context, and its brief form is as follows:

If one declares this verse: \textbf{``Indeed, I am for God, the Lord of
the heavens and the Lord of the earth, the Lord of all things, the Lord
of what is seen and unseen, the Lord of the worlds''}---according to the
determined limits of the dowry---and similarly from the other side, both
parties affix their seals on a document. Witnesses from both sides,
numbering ten or more, should attest to it. This will establish the law
of union. God has decreed for the people of the cities---that is, the
inhabitants of urban areas---a dowry of ninety-five mithqáls of gold,
the numerical value of ``God.'' The maximum is set at this amount, while
the minimum is nineteen mithqáls, the numerical value of ``one.'' Any
increase or decrease must proceed step by step, unit by unit, without
exceeding five levels. The first level is one unit, the second is two
units, the third is three units, the fourth is four units, and the fifth
is five units.

For the villages, the same structure is decreed using silver instead of
gold, as determined by God. If the amount exceeds this limit, even by a
fraction of a qíráṭ (a small measure), it becomes invalid. Likewise, if
it falls short by even one-tenth of one-tenth of a qíráṭ, it is also
invalid.

This command has been given so that all those obligated to it may share
in the bounty and expansive mercy of God. It also ensures that they
allocate their wealth to other worthy causes in these contexts. If they
give thousands upon thousands to one another, there is no harm, or if
they allocate such wealth at the time of union, there is no restriction.
This decree has been made so that all may partake in the abundance and
mercy of God.

Let them focus on what constitutes the lawful foundation of union: that
it must be established in the name of God. Perhaps, on the Day of the
Manifestation of Him Whom God shall make manifest, they will not stray
beyond the bounds of this Word, which signifies God. For if, on that
day, someone fails to believe in Him---even if they claim to act ``for
God''---their act is rendered void, except for what is confirmed by His
command on that day.

Consider the religious practices: all communities believe they act ``for
God'' in what they do. But if their actions were truly for God, why are
they rejected? Similarly, in Islam, if all the deeds performed during
the time of the Imáms were truly for God, how could they have been
invalidated? Similarly, at the appearance of the Sun of Reality, if
someone truly acts ``for God,'' they would not turn away from His
Mirror. All those who claim they act ``for God'' or recite the verse,
\textbf{``Indeed, my prayer, my rites, my living, and my dying are for
God, the Lord of all worlds,''} are, in God's sight, false. Otherwise,
judgments contrary to God's will would not apply to them. They act
according to their own understanding of serving Him, but in reality,
they act against Him in what they do.

Thus, all deeds are accepted or nullified based on recognition. In the
Bayán, until the appearance of the new Manifestation, any act done ``for
God'' is indeed ``for God,'' as they remain under the shadow of His
Mirror and their deeds are accepted by God. However, at the appearance
of Him Whom God shall make manifest, only deeds performed for Him are
truly ``for God.'' Otherwise, \textbf{``God has not placed a third
between truth and falsehood.''}

Whoever acts for Him will be gathered under the shadow of
\textbf{``there is no God but God.''} Whoever acts for another will be
gathered under the shadow of negation. Yet all, in every case, revolve
around this Mirror. For instance, what the followers of Jesus perform,
they do under the belief that it is for God through Jesus, who was, in
His time, the Mirror signifying God. However, at the appearance of the
Messenger of God, only those among the Christians who believed in Him
were truly acting for God. Otherwise, all their claims were revealed as
false in God's sight, being neither true then nor now.

Testimony regarding anything must relate to the preceding Dispensation
and be described in such terms. The one who manifests this testimony
becomes the Mirror, which represents the primal Will that is recognized
in every Manifestation. What God bears witness to is true, and what
pertains to His essence, none but He knows. He has eternally been, and
will forever remain, a witness over all things, even before their
existence. His testimony over all things before their existence is like
the testimony of all things after their creation. \textbf{``None knows
how He is but He. Glorified is God above what they bear witness to.''}

Nothing ensures salvation except sincerity in intention. Similarly,
nothing causes veiling except when one, believing they act for God,
becomes veiled in their own mind. Otherwise, no soul desires to disobey
God in their innermost being; rather, they assume their actions are
``for God,'' while in reality, they are ``for something other than
God.'' This is why their deeds are rendered void.

Consider the Manifestation of Him Whom God shall make manifest: all
those who act for God within the Bayán base their actions on the proofs
of God's verses. On that day, those same proofs will remain valid. How,
then, could they fail to act for Him? Yet this is precisely what
happened during the appearance of the Bayán: the people of the Furqan
were told the same, but those who did not act for God were unable to
accept, while those who did, believed immediately.

At each Manifestation, God tests His servants to reveal to them their
own inner reality---whether their deeds were truly for Him or not. For
example, at the appearance of the Messenger of God, every soul who
claimed, in the Gospel, to act for God through Christ was tested. Only
those who embraced Islam were proven to have truly acted for God. Their
faith in Christ and their adherence to His religion became evident only
when they entered the new Dispensation of Jesus. Those apart from them
did not act for God nor were they adherents to the law of Jesus, the
Spirit of God. Otherwise, God, the All-Knowing and Most Exalted, is far
above allowing someone to act for Him and then enter the fire.
Similarly, observe the appearance of the Bayán: whoever among the people
of the Qur'án entered it acted for God. Those who did not were acting
for something other than God, believing in their own desires that they
acted for God. Yet in God's sight, they acted for something else.

The levels of that ``other'' and its names are apparent to the people
who believe in the Bayán, where any ``other than God'' is clearly
identified. From this, it becomes evident that acting for God is
intrinsically tied to acting for the Manifestations of His Cause. In the
Qur'án, if anyone acted for the Manifestations of Truth---Muhammad, the
Family of Muhammad, His Gates, and His Guidance, peace be upon
them---they acted for God. Otherwise, their actions did not return to
God.

The Mirror of God existed before the Messenger of God, such that the
eighteen Mirrors, by the radiance of His sun of bounty, became Mirrors
of God for Him. All those who acted for God, finding shelter under the
shadow of the Manifestations of His Cause, found their deeds accepted in
God's sight. Likewise, in the Bayán, if someone uttered the word of
unity without coupling it with the mention of the Manifestation of the
Cause, their deeds would still be accepted as acts for God, even without
explicitly acting for the Manifestation.

Consider this example: in the Gospel and the Furqan, or in the Bayán and
the Manifestation of Him Whom God shall make manifest, the matter
becomes clearer. For instance, the testimony of unity in the Gospel was
not accepted without acknowledging Jesus, the Spirit of God. Similarly,
the testimony of Jesus was not accepted without the acknowledgment of
the living letters of that time. Each was exclusively linked to God's
Truth.

This is why all point to this One, so that you see no duality.
Otherwise, all that you observe in the Gospel is the multiplicity of
that primal unity. Even if a single soul exists in the East or the West,
it is still part of that One. If someone acted ``for God'' in the
Gospel, they acted for the letters of unity of that Manifestation, as
whatever returned to them ultimately returned to God.

Thus, everyone who acted for those letters of unity did so in a way that
what returned to them also returned to God. Yet at the appearance of the
Messenger of God, those who did not believe in Him rendered void all
that they had done for God and the letters of their time, except for
those who turned to Him. Their actions then became truly for God and for
the letters of unity of that Manifestation. Otherwise, in God's sight,
they were not truthful. Had they been truthful, they would have believed
in the Messenger of God and in the living letters of His Dispensation.

Even though you see multitudes acting ``for God'' according to the
letters of unity in the Gospel, they remain in the fire, for they act
for something other than God. Similarly, consider the Qur'án: from its
beginning until the year 1270, whoever acted ``for God'' were those who
served Muhammad, the Family of Muhammad, and the Gates of Guidance. If
someone, in the primary reality, acted for Muhammad but, in the
secondary reality, did not act for the Commander of the Faithful (Imám
'Alí), they were not truthful in their claim of acting ``for God.''

This extends to all the Gates until the end. However, since the
appearance of the Bayán, if someone has truly acted ``for God,'' it
means they acted for Muhammad and, in the manifest reality, acted for
the Point of the Bayán and the Manifestations of His Cause. Likewise, on
the Day of Him Whom God shall make manifest, you will see that everyone
claims to act ``for God'' and to believe in the letters of unity. Yet
if, at the time of the Manifestation, their actions for God are true,
and they are sincere. Otherwise, all they do between themselves and God,
claiming it is for Him, is immediately rendered void. How could their
deeds for God or their actions for the letters of unity---or even for
the believers in the Bayán, who hold a station through their connection
to Him---hold value, when the decree of God does not apply to them?

This is because the verse to which a servant turns between themselves
and God, and through which they behold the beauty of God in their
hearts, and by which they intend their deeds for God, is a verse that
belongs to Him Whom God shall make manifest. This verse has existed
within the hearts of all before His appearance. That verse is like the
sun in a mirror when compared to the sun of the heavens. The tree of
truth reveals its outward form through servitude in the hearts of all,
but its inner reality becomes manifest within the verses of their
hearts, where nothing is seen but God alone, alone.

If this were not the case, how could a servant act ``for God'' between
themselves and God, yet the Manifestation of Truth not accept it? What
they do between themselves and God is but a shadow of the true verse
that emanates from that Sun of Reality within them. For instance, had
the Messenger of God not spoken the command to act ``for God,'' how
would anyone have known to act ``for God''? Even if expressed
differently, the same idea would have emerged from the Messenger sent in
that Dispensation. It always leads to the origin, which has no
beginning, or to the Manifestations that follow, which have no end.

For example, if a person between themselves and God brings sorrow to the
Point of the Bayán, the verse through which they act between themselves
and God, and by which they intend to act ``for God,'' is still a verse
that is manifested within them from the Sun of His bounty. Yet, in the
subsequent Manifestation, because they lack certainty, they become
veiled. However, in the previous Manifestation, if you say that this
verse comes through the Messenger of God, they accept it, for they have
neither heard nor seen anything else. In this Manifestation, it has not
yet become apparent.

Consider the Qur'án at the time of the union of two souls who acted
``for God.'' For example, Imam Husayn (Siyyid al-Shuhadá), peace be upon
him, acted for the Messenger of God, and thus his action was for God.
But the one who committed an injustice against him believed they acted
``for God,'' yet in reality, they acted for something other than God.
The verse that the wrongdoer relied upon between themselves and God
contained only the essence of God, yet at that time, Imam Husayn was the
living expression of that verse in its outward form.

In truth, if the veil were removed, they would have seen that what they
believed they were doing ``for God'' was actually directed toward Him,
for and against Him, even though they were veiled and acted against Him.
This elaboration is to emphasize that, on the Day of the Manifestation
of Him Whom God shall make manifest, one should not sit comfortably,
claiming, ``We act for God,'' when the verse they turn to is but a
shadow of the Sun of His bounty as reflected in the verse of their inner
realities, made manifest through the previous Manifestation.

In the subsequent Manifestation, if their actions align with Him, they
will be truly for Him, just as they were for the Point of the Bayán
before. Otherwise, between themselves and God, they will not have acted
``for God,'' because God has decreed that whatever is done ``for Him''
does not return to Him unless it is also done ``for Him'' in His
Manifestation.

This is because finite beings cannot perceive the essence of the
Eternal; rather, they recognize Him in His Manifestation within their
realm of possibility. For example, when a person commands a mountain to
ascend by God's command, they reflect inwardly between themselves and
God, make a judgment, and act. However, that very inward reflection and
judgment, which they think is directed toward God, is in fact a verse
rooted in the previous Manifestation. The essence of that verse remains
within them, but because they do not recognize it, they act as they do.

Had they known that this is the very Point of the Furqan (Qur'an) in its
subsequent Manifestation, the verse to which all Muslims turn to God,
they would never have allowed the thought to even cross their hearts,
let alone acted upon it. Just as they draw near to God day and night
through the Messenger of God, so too, on the Day of Him Whom God shall
make manifest, all the believers in the Bayán will face similar trials.
They will think they act ``for God,'' but their actions will be directed
against Him.

The Sun of Creation, which embodies those verses through which they
previously drew near to God, will shine upon them and manifest itself in
their souls, though they will not recognize it. If they understood this,
they would realize that no fire is greater than the one in which a
person does to their Beloved what they would permit against another.
From the beginning to the end of their lives, they act toward that
verse---which belongs to their Beloved---as if it were for another.

If their actions were truly for God, consider how profound the matter is
for those who genuinely act ``for God.'' In the world of existence, they
act for Muhammad and the Manifestations of His Cause; in their souls,
they act according to the verses that are made manifest within them
through the letters of the Furqan's unity. Yet, because they are veiled
from the mystery of the Cause, they act for something other than God in
this Manifestation.

If their actions were truly for God, they would never fail to recognize
the radiance of the Sun of Reality in the Manifestation of the new
Revelation. Instead, they act only upon what they have from the previous
Manifestation, failing to grasp the ruling as truly ``for God'' unless
they enter this Manifestation. Reflect on the span from the Day of Adam
to the appearance of the Bayán, as well as before and after it: observe
that no period of time, even as brief as a fraction of a moment, has
passed without a Book from God being revealed. People have adhered to
His religion through these Books, and in their respective
Manifestations, those who acted according to them acted ``for God,''
provided they acted in accordance with what was revealed in them.

Now, consider the appearance of the Furqan: everything that people
claimed to do ``for God'' through those previous Books became actions
for something other than God, for if their actions had truly been ``for
God,'' they would have necessarily entered the faith of the Furqan. The
same applies in the Manifestation of Him Whom God shall make manifest,
relative to the appearance of the Bayán. Observe that if someone claims
to act ``for God'' in the Bayán, their claim cannot be validated as
truly ``for God'' unless they also act for the letters of unity.

The multiplicity of appearances operates under the shadow of the
singular truth, and within its own station, the decree of the singular
applies to the multiplicity. On the Day of the Manifestation of Him Whom
God shall make manifest, the entirety of the Bayán becomes a single
unit, representing a numerical oneness that refers back to the One
beyond enumeration---that is, the essence of Him Whom God shall make
manifest.

Following this, the living letters are manifested through Him, and then
the primal unity becomes multiplied until the Day of the ultimate
Manifestation of Him Whom God shall make manifest. At that time, all
must become one single unit, within which nothing is seen except the One
beyond enumeration, who is His very essence.

For instance, today, all believers in the Qur'án are mere reflections of
that primal unity, which culminates in the living letters and ultimately
reaches the Messenger of God. Consider the mystery of existence: do not
equate the ruling of the Sun with the ruling of the Mirror that reflects
it, nor the mirrors that follow the initial Mirror to the Mirrors. Do
not equate the ruling of subsequent mirrors with that of the Mirrors
directly connected to the source. This is why, at the appearance of the
Sun of Reality, not all can be guided by it directly. The bounty of
existence has reached them through mirrors that preceded them, and
similarly, guidance operates in the same way.

Consider the lowest terrestrial soul: if it were to directly encounter
the Sun of Reality, it could not be guided by it, as it is positioned in
a lower station. However, if it descends into the origin of the Cause,
it may immediately be guided. Yet, because it lacks the capacity to
perceive this directly, the Manifestation's radiance appears difficult
for it to grasp unless it refers back to its own village or realm, and
from there to the realm above it. Gradually, this process culminates in
one who can comprehend the words of the Gates of Guidance.

Step by step, this ascent continues until reaching someone who can
understand the words of the Imáms. That individual then progresses
further, reaching the capacity to comprehend the words of the Messenger.
At this stage, the soul enters the realm of hearts, where it can grasp
the Word of God, which is the same Word revealed to the Messenger.
Immediately, such a soul becomes humble, and the exaltation of the
Messenger's station appears as nothing compared to the greatness of the
Word itself.

Yet, for the lower terrestrial soul, even if a new book is revealed each
day, it does not humble itself before the Sun of Reality. This is
because all these veils---intermediary layers within its heart---must be
torn asunder for understanding to be achieved.

This dynamic is evident in Mount Máku, where, despite the revelation of
God's verses in every matter, no fruits are borne for its inhabitants.
If the veils of intermediary souls were removed, allowing the heart to
pierce through, it would immediately become a believer in God within its
own station and His verses. Consider how the essence of all existence
becomes a believer upon hearing His verses, despite the loftiness and
elevation of its essence, which cannot be compared to anything else. How
could it reach anything greater than this? This is the meaning of the
saying, \textbf{``What is concealed in the Prophet is greater than the
worship of the two weighty worlds''} (humankind and jinn). This is
because such a soul, upon the descent of God's verses, becomes humble
and submissive, acknowledging the oneness of God.

However, a soul that has been guided through countless intermediaries
does not remember upon hearing His verses. Such a soul cannot understand
even with reason, let alone attain faith or act upon it after belief.
Consider the martyrs of the Bayán: they were not veiled at the time of
the Manifestation, for they, like you, recognized the Word of Truth
through infinite intermediaries.

At the time of His Manifestation, if you fail to remain steadfast, know
that all the essences and deeds of the people of the Bayán are like a
garden in His hand, which He turns as He wills, according to His desire.
He who removes the mirrors of your essences from the shadow of God and
turns them into something other than God---how do your deeds appear
before Him? Reflect on the creation of your hearts and derive your
understanding from this.

All this explanation serves to ensure that unions are made ``for God''
and not ``for something other than God.'' If any fruits arise in
creation, they should be ``for God.'' It is as if one can see that, at
the time of the Manifestation of Him Whom God shall make manifest, all
claim to act ``for God'' in their private deeds. Yet, in truth, He rules
that their actions are ``for something other than God,'' except for
those who act for Him. Their deeds are ``for God'' in God's sight. The
same applies in the Bayán if someone acts for the letters of unity, they
have acted ``for God.'' This extends to the decree concerning the
smallest atom: if it is for the elevation of the Bayán, it becomes ``for
God.'' In the same way, in the Furqan, as long as the explicit text from
the Manifestations was not severed, whatever was done with their
permission was considered ``for God.'' However, after that severance,
only what aligned with their teachings remained ``for God.''

For instance, the scholars of the Furqan acted according to the Book of
God, the traditions of the Messenger, the Imáms of guidance, and the
Gates of guidance. In this Manifestation, too, as long as the explicit
text remains unbroken, whatever is done in accordance with it is ``for
God.'' But after its severance, only what aligns with it is ``for God''
and does not exceed the letters of unity. Perhaps, on the Day of
Resurrection, they may be guided by them.

Similarly, nothing can truly be considered ``for God'' unless it
pertains to the Messenger of God. This principle applies from the very
beginning to everything that stems from His command. ``For God'' cannot
be truthful unless it pertains to the Manifestations of His Cause.
Anyone who acts according to the Bayán becomes a reflection of its
letters until they reach the culmination of existence.

Exercise the utmost vigilance so that, at the Manifestation of Him Whom
God shall make manifest, all that was done ``for God'' is not rendered
``for something other than God.'' If it pertains to Him, it remains
``for God'' and for the Point; otherwise, it is invalid. \textbf{This is
what God, your Lord, has enjoined upon you if you act according to it.}

\subsection*{Gate 8 (Seeking Proofs Outside the Verses of
God)}\label{gate-8-seeking-proofs-outside-the-verses-of-god}
\addcontentsline{toc}{subsection}{Gate 8 (Seeking Proofs Outside the
Verses of God)}

The Eighth Gate of the Sixth Unity, concerning those who seek proofs
outside the Book of God and the verses of the Bayán, as well as the
inability of all to produce anything like them. For such a person, there
is no proof. Whoever attributes the miracle of the verses to anything
else, their claim is invalid. Anyone who claims the verses as their own
should not be opposed, yet such claims must be recited within every nine
days. Once every ten days, they should reflect on what God has revealed
by night and day.

The summary of this gate is if someone uses anything other than the
verses of God as proof for the truth of the Point of the Bayán, they
remain veiled from the greatest proof and the highest path. In every
Manifestation, all aspects of the Tree of Truth are exalted above
comparison, equivalence, likeness, parity, and disbelief. Yet, because
most hearts fail to perceive its elevation and are heedless of the
verses, if everyone wished to understand, they could. For this reason,
the proof has been made singular, so that on the Day of Him Whom God
shall make manifest, no one might ask, ``Why?'' or ``How?''

What has been revealed in the Qur'án was not reliant on external proofs.
Had all understood this, their acknowledgment of truth today would have
been easier than relying on their interpretations of matters unsupported
by the Book of God. Indeed, if their interpretations conflict with the
command found in the Book of God, it is God's command that remains valid
and has always been so.

It has been ordained that once every nineteen days, reflection should
occur on this matter, so that in the Manifestation of Him Whom God shall
make manifest, people are not veiled by matters other than the station
of the verses, which are the greatest proofs and testimonies. However,
this reflection should not merely be routine, nor should it lead to
veiling.

This is akin to those who recite the Covenant prayer every morning,
repeatedly calling out, ``Hasten, hasten,'' until they have confused
themselves, mistaking their own desires for true love. They imagine they
love Him, while over three years have passed since the Cause of God has
become manifest, and yet today they have established their object of
worship on a mountain. Despite this, they fail to acknowledge the proof
upon which the religion of all Muslims stands upon. It has become
manifest that after the cessation of revelation until the appearance of
these verses, no command or sign has emerged that could produce verses
of this nature. Yet, they lack the insight to recognize that no one but
God can reveal verses. The moment you see verses of this nature manifest
as proof, be assured that this is the same primal reality upon which God
revealed the Qur'án at the dawn of Islam. Now, He has again willed to
reveal upon Him.

If you had true faith in the proof of your own religion, you would
understand this matter. The command is clear and without exception: God
revealed the incapacity of all in the Qur'án. When you observe this same
incapacity reflected in the entirety of creation, you doubt. This doubt
causes veiling, for just as God revealed in the beginning, so too does
He reveal in the return. If you observe with this understanding, you
would recognize and affirm it more swiftly than the blink of an eye.

When you look at creation, you say, ``It is impossible.'' Indeed, it is
impossible for creation, but not for what comes from God. By His
boundless power, He manifests whatever He wills. It is evident that
after the setting of the Tree of Reality, no one can produce verses like
His. Just as in the Qur'án, where twelve hundred and seventy years
passed, all became certain of the incapacity of others to match it.
Similarly, after the setting of the Sun of Reality, it is impossible for
verses to appear from anyone but Him.

These verses must conform to innate nature and divine power, without
learning or characteristics imagined by scholars. Given this
impossibility, no one other than Him Whom God shall make manifest can
claim this command. This is the final proof within the Bayán that if any
soul claims this station and verses appear from them, no one should
oppose them. Perhaps this will prevent any sorrow from afflicting the
Sun of Reality. If this principle had been observed in the Qur'án, the
actions of all would not have been invalidated at once. If a soul is not
for the Truth yet does not oppose it either, no decree will be issued
against them affecting all, but only concerning that specific soul.

Similarly, now, if you hear of such a claim and do not attain certainty
but refrain from outright denial, you commit no act that would cause
sorrow to Him, even if, in reality, it is someone other than Him.
Although such a scenario is nearly impossible, merely mentioning His
name makes it unlikely that those who love Him would cause grief to such
a claimant, out of respect for His names. For the matter lies within two
possibilities: either it is Him, or it is not. However, no one but Him
can reveal verses conforming to innate nature. Why, then, would someone
deny the Truth while they have spent their days and nights in
anticipation of Him?

If, hypothetically, someone falsely attributes such a station to
themselves, leave their matter to God. It is not for creation to judge
them, out of reverence for the name of their Beloved. Moreover, no such
soul exists who could claim such a station. If such a scenario occurred
during the Dispensation of the Qur'án, it will occur in this
Dispensation as well. His verses are, in themselves, proof of the
radiance of His existence, and the incapacity of all others is evidence
of their dependence upon Him.

The purpose of this directive is that, on the Day of the Manifestation
of Truth, their feet may not falter upon the path. They must not judge
the shadow within their hearts as the determinant of existence, applying
a verdict based on a verse to themselves, which would suddenly nullify
all their essence and actions without their awareness. If all adhere to
this single command, it is upon God to make the truth manifest to them
and instill the proof and evidence into their hearts with clear
arguments that radiate from Him. This will enable the believers to
express certainty about Him, while those less certain may claim
awareness, which is still acknowledgment short of full truth. It
suffices for all the people of the Bayán to act according to this
command to secure their salvation on the Day of Resurrection.

That is the day when all things upon the earth will proclaim through the
radiant verses in their hearts from the Sun of Reality:
\textbf{``Indeed, we act for God.''} However, the truthful servants are
those who turn their gaze toward the origin of those verses, recognizing
that they emanate from the Sun of Reality. That Sun is the source of its
verses, and its light connects all things.

\textbf{``This is the proof of God, made complete upon you. O servants
of God, fear Him.''}

\subsection*{Gate 9 (Permissible to Wear Silk Garments and Use Gold and
Silver)}\label{gate-9-permissible-to-wear-silk-garments-and-use-gold-and-silver}
\addcontentsline{toc}{subsection}{Gate 9 (Permissible to Wear Silk
Garments and Use Gold and Silver)}

The Ninth Gate of the Sixth Unity, concerning the permissibility of
wearing silk garments in all situations, as well as the use of gold and
silver.

The summary of this gate is that God has permitted the use of gold and
silver so that all may partake in this paradise, achieving what brings
tranquility to the hearts of His servants. Through faith, they may
manifest the utmost gratitude to God on the Day of the Manifestation of
Him Whom God shall make manifest.

If all that is upon the earth were to agree, they could not prohibit
this ordinance. Yet, through His infinite generosity and grace, God, the
Almighty and Glorious, has permitted it for those capable of its use.
This ensures that no soul feels sorrow for lacking access to these
things. Purely out of love for God, He doubles their reward and blesses
them with honor in the afterlife. If a soul seeks to take pride or
elevate themselves through anything from these Manifestations, they
become veiled from the good pleasure of their Beloved. Instead, in every
matter where God bestows His bounty upon a servant through the
appearances of His dominion, it is fitting for them to express greater
humility and submission to creation. This is the true gratitude to God
within their station.

There is no honor or exaltation in these Manifestations except through
faith in Him Whom God shall make manifest. If honor were tied to
material means, the Christians today would hold more honor than any
other community, yet their faith is not recognized as valid. How, then,
could they attain honor? However, when material means are coupled with
faith, they become an expression of God's blessings for His servant.

If a soul possesses wealth in gold or silver and uses it to revive
another soul, this is far better than merely enjoying what they own in
the pleasure of their Beloved. For the hearts of believers are the
dwelling places of God's good pleasure, and there is no doubt that their
happiness is nearer to God than the delight of a soul in its
possessions.

This holds true as long as one does not focus on the outward boundaries
of creation. When viewed within the hierarchy of existence, there is no
doubt that the pleasure of a higher station is greater in God's sight
than that of a lower one. For instance, if something pleases Him Whom
God shall make manifest, His joy surpasses that of anything else.
Likewise, the closer one is to the source of all names and attributes,
the greater the significance of their pleasure, until it reaches the
smallest particle of existence.

On the Day of Resurrection, this matter can be discerned clearly, but in
the night of veiling, it remains indistinguishable. This is because all
claim to be elevated and near to God, and none truly knows the station
of another except for the Tree of Reality, which does not reveal the
stations of creation. Therefore, it is fitting to act according to the
first decree, so that no soul on earth who believes in God and His
verses may suffer sorrow. \textbf{``Thus does God lighten your burden
and grant you permission in the Book for that which brings you
gratitude.''}

\subsection*{Gate 10 (Have a Ring of Red
Agate)}\label{gate-10-have-a-ring-of-red-agate}
\addcontentsline{toc}{subsection}{Gate 10 (Have a Ring of Red Agate)}

The Tenth Gate of the Sixth Unity, concerning the obligation for every
soul to inscribe on a red agate the verse: \textbf{``Say: God is Truth,
and all besides Him are His creation; all are devoted to Him.''}

The summary of this gate is there is no symbol more beloved in the Bayán
before God than for one to have a ring of red agate engraved with this
majestic verse: \textbf{``Say: God is Truth, and all besides Him are His
creation; all are devoted to Him.''}

The significance of this is that it serves as a testimony from the Point
of Reality that God has eternally been Truth and that all else has
eternally been His creation. Perhaps, on the Day of the Manifestation of
Truth, they will acknowledge that Truth which affirms all truths and
will confess that all besides God is His creation.

It is not merely the act of wearing the ring with the inscription that
matters; rather, the essence of this verse should manifest itself before
that Truth. If it is revealed and a soul does not acknowledge its truth,
then even the ring itself will testify against them, nullifying all they
have done within the Bayán. Acknowledgment of this truth is not
difficult; rather, whatever has been affirmed as true within existence
has only been so through the Truth of His Reality.

However, because the Manifestation of Truth is a paradise for the people
of truth and a fire for those outside it, all will be tested on that
day. He is the Mirror of Divinity and the Sun of Lordship, reflecting
God alone, alone. If a soul wishes to enter the protection of Truth,
they should command that a circular red agate be inscribed. The design
should include:

\begin{enumerate}
\def\labelenumi{\arabic{enumi}.}
\item
  In the first circle: \textbf{The verse of the Throne} (\emph{Ayat
  al-Kursi}).
\item
  In the second circle: \textbf{The names of the circle} (specific
  divine names).
\item
  In the third circle: \textbf{The letters of the \emph{Bismillah} (In
  the name of God, the Most Gracious, the Most Merciful)}.
\item
  In the fourth circle: \textbf{The six names} (perhaps referring to
  sacred attributes or divine titles).
\item
  In the fifth circle: \textbf{Whatever suits the intent of the wearer},
  provided it does not exceed nineteen letters.
\end{enumerate}

If the first and second circles also contain the nineteen letters, this
is beloved before God. However, all of this holds value only if, in the
Manifestation of Him Whom God shall make manifest, they comprehend the
meaning of what has been inscribed.

This is because the first circle represents the letters of unity, which
multiply around Him until they reach the fifth degree, appearing in the
form of the number of God. If you have faith in the living letters of
Him Whom God shall make manifest, once you enter the first unity, by
God's might and power, you will remain part of that unity even if it
multiplies infinitely.

If you were present in the Manifestation of the Point of the Bayán, you
would have witnessed that all were guided by those initial letters.
Guidance itself is but a reflection of that primal unity, which
multiplies infinitely, yet is nothing but that original unity.

This is the foundation of all created beings and all their subsequent
characteristics. Observe the nature of creation and its attributes
within the context of existence, and do not veil yourself from the Sun
of Manifestation. Love every mirror in which you see its reflection, as
it represents a name signifying Him. Conversely, anything that does not
signify Him---if it is as insignificant as a speck of dust in the domain
of a disbeliever---should be subject to denial.

On the Day of Resurrection, whatever belongs to Him Whom God shall make
manifest will belong to God, and whatever belongs to something other
than Him will belong to something other than God. Similarly in the Point
of the Bayán, observe this same principle as was evident before its
Manifestation in the Point of the Furqan (Qur'án), and before that in
the Point of the Gospel, continuing until it culminates in the primal
origin of creation. Similarly, the process of ascension from Him Whom
God shall make manifest extends infinitely. Reflect and observe what you
have witnessed in this Resurrection, and always proclaim:
\textbf{``There is nothing from God but God; indeed, we are all devoted
to Him.''}

\subsection*{Gate 11 (Prohibition of Excessive Punishment of Children by
Teachers)}\label{gate-11-prohibition-of-excessive-punishment-of-children-by-teachers}
\addcontentsline{toc}{subsection}{Gate 11 (Prohibition of Excessive
Punishment of Children by Teachers)}

The Eleventh Gate of the Sixth Unity, concerning the prohibition of
excessive punishment of children by their teachers.

The summary of this gate is that a teacher must not strike a child more
than five light blows. Before the child reaches the age of five,
striking them is absolutely prohibited. After five years of age, no more
than five light blows are permitted, and even then, the blows should not
be on bare skin but over clothing. If a teacher exceeds five blows or
strikes bare skin, they are forbidden from approaching their spouse for
nineteen days. Even if the act was unintentional, this prohibition
applies. If the teacher is unmarried, they must pay nineteen mithqáls of
gold to the one they struck.

God has permitted children to engage in playful activities during
festival days with whatever is in their hands. Additionally, it is
decreed that every soul is entitled to sit upon a chair or throne during
such times. The time spent sitting on a chair, bed, or throne does not
count toward their lifespan.

\textbf{Explanation of this Gate:} God does not, under any
circumstances, desire any soul to be saddened, let alone harmed by
physical punishment. It has been decreed that children under five years
of age are to be disciplined only through verbal instruction, and no
physical punishment is to be inflicted upon them. After the age of five,
no more than five light strikes may be given, and even these must be
through a barrier, such as clothing, rather than directly on the skin.
Furthermore, physical discipline must not reflect the harsh practices
common in this age.

If a teacher violates this decree, they are prohibited from marital
relations for nineteen days. If the teacher is not married, they are
required to pay nineteen mithqáls of gold as compensation for exceeding
the bounds of God's decree. This payment must be given to the soul that
was struck. God desires that, in all circumstances, the people of the
Bayán should sit upon thrones, chairs, or seats, for such time is not
counted as part of their lifespan.

The purpose of these decrees is to ensure that no sorrow afflicts the
soul from whom all are sustained by the ocean of His bounty. A soul,
unable to comprehend the station of its teacher, is thus safeguarded.
Similarly, in the Manifestation of the Furqan, the Sun of Reality was
not recognized until forty years had passed. In the Point of the Bayán,
recognition took twenty-five years. God alone knows what has been
decreed for Him in this Dispensation.

During those days, His true joy is found, though all await Him. Yet,
because the devotion of all is not sincere, sorrow inevitably reaches
Him. Reflect on the Messenger of God: before the revelation of the
Furqan, all bore witness to His beauty, perfection, and piety. But after
the revelation of the Furqan, consider the things they said about Him,
words that the pen is too ashamed to recount.

Likewise, consider the Point of the Bayán: the stations He held before
His Manifestation were well known to those who recognized Him. Yet,
after His appearance, despite the revelation of 500,000 verses across
various subjects, some continue to utter words about Him that the pen
cannot record out of shame.

However, if all were to act according to what God has commanded, no
sorrow would afflict the Tree of Reality. For if no one were to grieve
another, not even one soul among creation would bring sadness to
another. If they abide by what all have been created for, no one would
even approach anything less, as no bounty is greater than this, nor will
there ever be.

Even though the days of joy are observed as the days before His
Manifestation, His appearance will occur amidst the multitude of
creation, who have always been and continue to be enraptured by His
love. Yet, as today, people act in His name in ways they do and consent
to matters as they are. \textbf{``O servants of God, fear Him.''}

\subsection*{Gate 12 (Divorce)}\label{gate-12-divorce}
\addcontentsline{toc}{subsection}{Gate 12 (Divorce)}

The Twelfth Gate of the Sixth Unity, concerning divorce, which is only
permissible after the husband and wife endure a period of one year
together, in the hope of reconciliation. If reconciliation does not
occur, divorce is permitted. A man may remarry the same woman up to
nineteen times, with no requirement of patience following a reunion
except for a period of one month.

The summary of this Gate is once God unites two souls through the word
of God, it is unworthy for matters less than the Tree of Love to cause
separation. The union of souls is not to be undone except by dire
necessity. If separation becomes unavoidable and is declared justly,
they must endure a period of patience equaling one cycle of nineteen
months as defined in the Bayán.

If, during this time, the attributes of love reappear and unity is
restored, the separation is nullified. If not, separation becomes
lawful, and they may part with a word that signifies this. Afterward,
they may reunite up to a maximum of nineteen times. From the time of
union until any potential reunion, a period of patience of nineteen days
is required to purify them from the attributes of the fire.

After the cycle of nineteen is completed, reconciliation becomes
permissible, and this continues until they reach the number one. Once
they reach this point, further reconciliation is no longer allowed, as
it would then fall under the rule of duality. In paradise, the law of
duality does not exist and never will, for all are created from a single
soul, and once the station of unity is complete, a new beginning must
arise---not duality. Even if unity continues infinitely, it remains as
one. However, if a single element is added to the number one, it becomes
twenty, and a single addition to the soul of unity makes it dual. This
is the secret of wisdom for those who wish to understand.

The purpose of this Gate is to observe the origin of the decree, so that
on the Day of Manifestation, you may remain steadfast in His decree and
not become veiled from Him. For example, today, you see that under the
shadow of each decree, countless souls benefit from its blessings. Yet
on the Day of Manifestation, all will become nothing before His Word
unless they return to it. Otherwise, they will only receive what is
within the bounds of potential grace. How challenging is the matter for
one who remains veiled from the origin, and how easy it is for one who
turns back to it! Blessed are the righteous on that great day.

\subsection*{Gate 13 (Doors in the House of the Point and the Houses of
the
Letters)}\label{gate-13-doors-in-the-house-of-the-point-and-the-houses-of-the-letters}
\addcontentsline{toc}{subsection}{Gate 13 (Doors in the House of the
Point and the Houses of the Letters)}

The Thirteenth Gate of the Sixth Unity, concerning the number of doors
in the House of the Point, which may not exceed ninety-five, and the
number of doors in the houses of the letters, which may not exceed five.

The summary of this Gate is that during the night when people awaken
from the shock of the Day of Resurrection, they desire to draw nearer to
God through the first unity. Though on that very day, the Day of Origin
begins, and all return to nothingness, without the intermediary, the
test becomes apparent. Everyone claims proximity to the Beloved and His
satisfaction.

For this reason, it has been decreed that if they can, they should enter
the chambers of the letters of unity, for those are the realms beyond
which there is no paradise higher. It is decreed that the House of the
Point may not have more than ninety-five doors so that it may serve as
evidence that He has been and will always be the Mirror of God, both
before and after, in which the Sun of Reality is manifest. The Letters
of the Living are not permitted to exceed five doors, serving as a
testament to their all-encompassing form in His dominion. This reflects
what God has attributed to Himself, not worldly matters in which people
take pride.

Those who emulate such attributes to the extent of their capacity are
worthy of being regarded in their secondary reality. This continues
until it culminates in the ultimate existence.

The purpose of this decree is for servants who enter the chambers of
religion, on the Day of Resurrection---the Day of the Manifestation of
the Point---will see the resurrection of these Letters, alongside other
proofs and ranks, such as the Prophets, the Truthful, the Martyrs, and
the Believers. If they were truthful in their faith before, their
truthfulness will also be evident on that day before God and His Names.

For example, consider the Manifestation of the Messenger of God:
initially, no one approached Him in faith or visited Him. Now, however,
you see seventy thousand souls visiting each year. Yet today, during
this period of universal testing, it is evident that it resembles the
early days---none visit Him with sincerity. What you see now is
motivated by pride and status, which is why their deeds are reduced to
\textbf{``scattered dust''}---because they lack insight.

The same proof that established His prophethood in the early days of
Islam is present today as a divine proof. How, then, is it that all
remain veiled from it? Similarly, reflect on the Qur'án: at the time of
its revelation, during the height of eloquence, its detractors spoke
against it. Later, all believers who heard those detractors' words
became astonished, wondering how anyone could hear the Word of God and
speak such things. All professed faith, adorned themselves with the
Qur'án, and recited it fluently, yet they were tested---those same ones
who spoke as they did.

The essence of Islam lies within these five divisions, from the Day of
the Manifestation of the verses of God until today. If someone wished to
count the pure believers, they could not be enumerated. Yet, these very
souls, had they been present in those days, would have said the same
things their predecessors said. And just as they were absent in those
days, they are present today, witnessing the verses of God flow like an
ocean from their source, but labeling them madness.

Meanwhile, they place themselves in the highest ranks, claiming
understanding beyond others. This is the condition of people.

\textbf{An Example from the Point and Its Traces:} Even now, you see
thousands lamenting for the Fifth Letter (a reference to Imám Husayn)
and attributing their actions to him. Yet, the Word of the Fifth Letter
was distinct from the Word of the Messenger of God, as it is not
contained in the Qur'án in the same manner as its verses. And even if it
were, it is no longer in the hands of the people.

How is it, then, that not a single soul can traverse the path while
thousands act in His name, claiming righteousness? This is why all deeds
are reduced to \textbf{``scattered dust''}---because each day is like
that day of martyrdom. Had you been there, you would have heard what was
said then, but even more sharply. Similarly, just as the stations of
paradise have ascended, their stations of descent have also deepened.

This is why, in the night of testing, no true examination takes place;
all voices shout, \textbf{``I am! I am!''} Their claims ascend to the
throne. Yet, on the Day of Resurrection, when the time comes, they all
fall into the first shock of the blast and does not progress to the
second blast. The first blast is not an imagined event; it is real. For
example, the same Letters (of the Living) who began the mission conveyed
the message to every soul, and those who did not immediately believe
fell into the first blast. This is because they were presented with the
same proof upon which their religion was established, intending to make
them recognize the proof of the Manifestation of the verses.

Despite all the grace and mercy bestowed, the Point, in its generosity,
lowered itself to the station of the final Gate, hoping to save them
from the first blast and enable them to bear the truth. Yet, the result
bore no fruit, though the end of the Manifestation is, for the people of
truth, one with its beginning.

This describes the condition of visitors today: though 1,270 years have
passed since the mission began, no one has sought to meet God, the
purpose for which all were created, as stated explicitly in the third
verse of the first chapter of \emph{Ra`d}. This is because encountering
the essence of the Eternal is impossible within existence. What is meant
by ``meeting God'' in the Qur'án is encountering the Tree of Reality,
where one sees nothing but the verses of the Qur'án in His words.

In their primal reality, this applies to Him, for no one else can reveal
the verses of God in such a manner from the outset of the command---not
even the Letters of the Living, nor all the names, metaphors, and
attributes within creation, including the believers. The matter of
``meeting God,'' which is so emphasized in the Qur'án and declared the
purpose of creation, has gone unrecognized by all. Yet, the graves of
the dead, which have no relation to the Word of the Origin, are visited
day and night by multitudes. This shows that all move without awareness,
and their movements have always been without true understanding. For if
they were conscious, they would not take a tradition whose authority is
affirmed through the word of the Messenger, and whose prophethood is
established by the proof revealed through Him. Day and night, they
circumambulate around it or exert themselves in its study, while
remaining veiled from the Source to which all things return and from
which all are disseminated.

If you document, until the end of this Manifestation, what befalls the
Letters of Unity from creation, you will realize how many claim love for
these Letters and visit them, yet only a few are truthful among those
below this station. Most act according to their own desires, not for the
sake of God. Even if they imagine themselves acting for God, in the
sight of God, they are acting for something other than Him.

They are not commanded to concern themselves with the physical resting
places of these holy ones so that, on the Day of Resurrection, which is
referred to as the ``Return,'' they might act with sincerity. If their
devotion is genuine, how much effort would they spend to visit their
physical resting places? Yet today, the Day of Resurrection, when they
could attain the encounter with these souls, they fail to do so. Even
when they attempt to do so, the effort seems to them as insurmountable
as Mount Uhud, and they feel they are bestowing a favor upon the one
they visit. In truth, they themselves would endure the toil of the
journey and boast to others about visiting those graves.

Had they been sincere in their devotion, they would undoubtedly have
been just as truthful during the lifetime of those souls. Yet in the
darkest of nights, let them do what they are able, but ensure that the
fruit they reap is such that their actions on the Day of Resurrection
are not rendered void.

If someone visits the graves of the Letters in the Bayán but does not
attain the meeting with their souls on the Day of Resurrection, all
their actions will be nullified. Likewise, reflect upon the Qur'án,
derive its ruling, and do not veil yourself from the Source. Do not act
merely out of imitation or association, but with true understanding.

Do not act out of association or imitation, but act solely for God, even
if you have no companions in doing so. Consider how, in the early days
of Islam, for seven years, no one except Amír al-Mu'minín ('Alí) truly
believed in the Messenger of God with pure sincerity. Those who later
professed belief, if they were truly sincere, would not have deviated
after the ascension of the Messenger. At that time, only three remained
steadfast among His companions.

Always look to the essence of the matter, for faith is established by
its essence. Those individuals, on that day in Medina, practiced all the
ordinances of the Qur'án, yet the judgment of faith remained only for
those three who adhered to what established the religion. Had they not
adhered to it, they would have been merely practicing the ordinances of
the Qur'án, as others did at the time, but without the essence of faith.
Their actions would not have borne any true fruit. This is the essence
of knowledge and action, and the essence of visiting the Point and the
Letters of the Living in the subsequent Resurrection---if you are able
to grasp it. In the night, you may attempt to act, but in the Day, you
will see its fruits.

Imagine a gathering of a hundred people sitting together in friendship,
yet the Letters of the Living are alone and unrecognized. On that day,
you cannot truly visit them, and your past actions will be rendered void
without your awareness. You may not even realize that your religion was
established through their love, yet today you are veiled by secondary
matters derived from the origin.

If you reflect on that day, you will see that you have no proof before
God. The same proof that once established the unity of the Letters of
the Living is the proof of that day. It is the same proof that was
present in the Qur'án. Yet, failing to act with insight, you nullified
your deeds without realizing it. Your soul is seized, you are cast into
the fire, and it does not even occur to you that the Resurrection has
been established and the Letters of Unity returned, and the divine
decrees concerning creation, emanating from the Point, were carried out.
Yet, because you were deluded by your own station, you remained veiled
from all. This is why you must place your trust in God, ensuring you are
not veiled from the origin of the command. For if the origin is
affirmed, all other realities are established; but if it is lost, all
becomes void.

There is no doubt that the distinction between humans and animals lies
in the essence of knowledge. Knowledge, however, is only revealed
through speech or writing. If you examine it closely, you will see
endless degrees within the realm of knowledge. Yet all knowledge bears
no fruit unless it pertains to God.

For instance, among non-Shi`a Muslims, there are numerous scholars in
every field. Yet, as a believer in the truth of Islam, you do not
ascribe faith to them. How, then, could a just and perceptive person
judge otherwise? True knowledge of God should not be regarded as
speculative. It is knowledge of the Manifestation of the Divine in every
Dispensation, borne by the proofs in His hands.

Without such knowledge, no one is truly a believer in God. From the time
of Adam until today, all nations---regardless of their distance from
truth---believed in God and their respective Messengers. Yet, because
they did not enter the new Dispensation, their belief was rendered void.
Had they possessed the essence of true knowledge, they would not have
remained veiled from the Manifestation of God.

This is the meaning of the verse: \textbf{``My Lord, why have You raised
me blind while I was once seeing?''} Today, you hear people claiming,
``I was a believer before,'' but this refers to spiritual insight, not
physical sight. The answer given is: \textbf{``Thus did Our signs come
to you, but you ignored them; so today, you are forgotten.''}

Without those very Letters of Unity in whom you previously believed,
relying instead on the names you hold and the verses upon which your
religion was once affirmed, but because you ignored it and remained
veiled, you have now become blind. This blindness is not physical; for
you see that everyone, with their physical eyes, can perceive the world.
Animals, too, share this same external vision with humans. Rather, the
intended meaning is the eye of the heart, by which one sees and
recognizes their Beloved.

Today, understanding the Point of the Bayán may seem difficult to you.
However, reflect upon the Point of the Furqán (the Qur'án). There is no
doubt that the community of Jesus was entirely awaiting the Promised
Ahmad, just as you awaited the appearance of the last of the Imáms. If
you say not all were awaiting Him, know that even among the sects of
Muslims today, not all hold certainty.

Though the Christian community was awaiting Him, when He appeared, none
among them recognized Him. This is because the eyes of their hearts were
unable to perceive and recognize their Beloved. Otherwise, if they had
truly known Him, they would never have deviated from the word of Jesus.
Instead, 1,270 years have passed since the mission of the Promised
Ahmad, and the eyes of their hearts remain blind, unable to see.

Though their physical eyes perceive all things---indeed, their vision is
so sharp that they can, with telescopes, observe distant lands and even
details on the moon---they remain spiritually blind. Consider this
carefully: 1,270 years from His mission, at the time of the first
appearance of truth, the earth was filled with oppression and injustice,
from the hidden realm of wisdom in the hearts to the physical realm of
the earth.

The first appearance of the essence of faith was found in those souls
who moved in search of truth. Their hearts were directed toward it, and
they recognized the truth. In that land, none recognized the truth until
after His appearance. Because those souls possessed the eye of God, the
truth could not be comprehended except through His eye. For this reason,
they did not recognize the appearance of their Beloved but affirmed His
truth through His verses, even though they did not know that He was the
promised one. Recognition of Him is only possible through His eye, which
had not yet been created in them outwardly, though it existed within
them inwardly.

This is why one person becomes receptive and sees, while another,
blinded, turns away and cannot see. Consider the progression from the
appearance of the Point of the Furqán to the ultimate station of faith.
How is it that, in the plain of Karbala, one gives their life while
another acts as they have heard? One sees, and one does not, each
believing they act for God, but instead one comes against Him. If they
knew they were acting against the truth, they would prefer death to
committing such an act.

Step by step, this journey progresses through the stages of
manifestation until it reaches the final appearance within the Shí`a. As
you observed, the companions of the Master of Martyrs (Imám Husayn) were
wholly devoted, while others adhered to their own inclinations. This is
the secret of unity from the beginning, culminating in the last
manifestation, where those who were not receptive are negated in denial,
while those who were receptive are affirmed in truth.

Not all among the receptive had the eyes of their hearts opened. Had it
been so, they would not have deviated from His purpose after His
ascension. His intent was to educate all for the rising of the Sun of
Truth and to prepare them for meeting God on the Day of Resurrection.
Yet, it reverted to the final stages of existence, where the lands of
Islam and beyond were filled with believers in that same first unity.
Now they all believed, in whatever station they occupied, that the truth
was with them. However, among countless multitudes, only the Letters of
Unity arose, rooted in their primordial and true nature, to seek the
truth. Though all others exerted the utmost effort and piety, attaining
the highest degrees of certainty, they failed to recognize the truth.
Over 1,270 years, the celestial sphere revolved around them, yet the
essence of all these manifestations was to prepare them for clarity in
the Dispensation of the Bayán.

Perhaps, on the Day of the Manifestation of Truth, which is the fruit of
the Bayán, they could, with the eye of God, recognize Him. Through the
single eye of guidance, they might be guided, and with the eyes that
point to God, they might perceive the manifestations of divine names and
attributes.

The relationship of the Day of Resurrection to the preceding night is
like the planting of a tree. The Day of Resurrection is the time to
harvest its fruit, while before that time, it has not yet reached
maturity. For instance, in the Dispensation of Jesus, the tree of the
Gospel was planted, yet it did not reach perfection until the first
mission of the Messenger of God. If it had matured earlier, the day of
His mission would have come sooner---on the 26th of Rajab rather than
the 27th.

The trees planted in the Gospel bore fruit over the 23 years of His
appearance, during which divine revelation was present and God's decrees
unfolded. These rulings, rooted in the will of God, descended into
existence. After the planting of the tree of the Qur'án, its perfection
was reached 1,270 years later. If its maturity had come earlier, even by
two hours on the night of the 5th of Jumádá al-Awwal, it would have
manifested five minutes later.

This is because the tree of truth always observes from the heights of
its own throne, gazing upon the trees it has planted in the hearts,
souls, lives, and bodies of creation. As soon as the tree is seen and
its fruit plucked, the initial appearance of unity begins. From behind
the veil of proof, the statement ``I was a hidden treasure; I loved to
be known'' was revealed, so that through it creation might come into
existence, with the fruit being the recognition of Him.

This recognition, however, is deferred until the next Resurrection
because the foundation of religion is the knowledge of God. Since the
knowledge of God cannot be manifested except through what God has
described of Himself through the tongue of His Messenger, recognition
depends on perceiving the Manifestation during His appearance. This
process continues until it culminates in the ultimate stages of
existence.

The fruits of the tree of the Qur'án must manifest before the ascension
of the tree of the Bayán. If they do not appear, it is evident that no
fruits existed; otherwise, they would have been visible. The gatherers
of this garden are the praising angels, who observe all creation. If
they see the love of their Beloved on the tree of the garden, they
gather it as a sign of guidance toward recognizing the purpose---not
anything else.

Similarly, on the Day of the Manifestation of Him Whom God Shall Make
Manifest, all those who believed in the Bayán and in His Letters of the
Living represent the fruits of the Bayán. Yet, He does not accept that a
barren tree should be counted among them. If, in the knowledge of God,
such a tree is deemed worthy, it is granted a station appropriate to its
capacity. Even so, all are sheltered beneath the Sun of Truth, which
grants benefit and shelter to all who turn to it.

However, how many souls have sought refuge in God yet failed to
recognize His guides, as if they had never sought shelter in God at all!
Though all say the words, \textbf{``I take refuge in God,''} even those
addressed by the revelation itself, their actions often contradict the
spirit of those words. For instance, in the early days of Islam, people
recited these words but failed to recognize Amír al-Mu'minín ('Alí).

It is not that the essence of refuge is in God, that one is not granted
refuge except from the fire of recognition. Otherwise, every nation you
see speaks this word in their own language, though their limits are
apparent. Before every Manifestation, seeking refuge in God is seeking
refuge in Him, and in every concealment, seeking refuge lies in the
ordinances of that Manifestation, until the dawning of another radiance.
At that time, the previous Manifestation and its ordinances do not grant
refuge except through the next Manifestation and its ordinances. Always
keep watch at the beginning of the Manifestation, for if you endure even
a fraction of the ninth, you will attain renewal in that mentioned
truth. Just as today, you say of those servants who enter Islam from
among non-Muslims, this was the fruit of visiting the Letters of Unity
during the subsequent Resurrection. If you are capable of perceiving
this, then you will understand; and in this Resurrection, if you were
truthful in your faith and sincerity. And God bestows His mercy upon
whomever He wills, for God is the Possessor of great bounty.

\subsection*{Gate 14 (The Day of God -
Naw-Ruz)}\label{gate-14-the-day-of-god---naw-ruz}
\addcontentsline{toc}{subsection}{Gate 14 (The Day of God - Naw-Ruz)}

The fourteenth chapter of the sixth Unity, and God's ruling at the time
of the sun's transition:

The summary of this chapter is that God has assigned one of the days
among all days to Himself and has called it the ``Day of God.'' He has
guaranteed that whoever recognizes the truth and sanctity of that Day
and acts upon what God has commanded within it will be rewarded as if
for the entirety of the year. A single pure gold coin spent on that day
will be as if three hundred and sixty-one coins were spent in the path
of God. The same applies to all acts and aspects of goodness, as decreed
by the command of God. This day is when the sun transitions from the
constellation of Pisces to Aries, whether it occurs during the night or
the day. It is worthy that at the very least number of unity, no
blessings or favors are excluded, and above it lies the point of refuge.
During this time, whatever anyone is capable of, God's permission has
been and remains granted. For other days outside this Day, in the Bayán,
such permissions for delights and multiple favors were not given
simultaneously. This is so that the truth of the matter becomes complete
before the soul. Similarly, in the banquet of the Beloved, each blessing
becomes available, but the highest of them prevails. Likewise, in the
regular cycle of day and night, one enjoys a single blessing in one
gathering, while the multiplicity of favors and blessings is divided
across different gatherings. This approach is closer to piety in the
sight of God.

Permission has been granted to recite this verse in the night,
corresponding to the total number of days in the year: \emph{``God bears
witness that there is no God but Him, the Sovereign, the Sustainer.''}
During the day, one may recite the verse: \emph{``God bears witness that
there is no God but Him, the Mighty, the Beloved,''} or the verse of
\emph{``God bears witness''} that contains the mention of divine power,
which is more excellent for the reciters in the sight of God.

This Day is the Day of the Point, and eighteen days following it
correspond to the Days of the Letters of the Living. These are more
exalted than the eighteen months, where each day is attributed to one of
the Letters. The ordinances of all things are thus connected to these
existential realities, which serve as witnesses to the unity of God. In
outward terms, the mention of the Prophet and His Successor does not
occur in this cycle, and the term ``believers'' applies only until the
Day of Resurrection. On that Day, the tree of reality mentions everyone
by whatever name it wills, and no one knows it except Him who commands.

In the first unity, fasting is not permissible; rather, entering the
gates of unity and Paradise is obligatory. All these Manifestations
exist so that on the Day of Him Whom God shall make manifest---which is
the first Day, His Day---He may guide, and His likeness is like the sun
in the daytime.

It was not that this Manifestation would resemble another, for such a
Day has passed, and the one for whom this Day was created was in sorrow.
Yet this Day has become exalted in relation to Him, and on the Day of
Resurrection, every thing shall appear in the form of a human
being---even the minutes, hours, nights, days, months, and years, and
beyond that. Until it reaches the horizons of the eternal signs and the
ancient Manifestations, at which point it transcends the mention of
created limits. God has ever been knowing and ancient, and God has ever
been sovereign and powerful.

\subsection*{Gate 15 (Stand and Sit in Reverence of He Whom God Shall
Make
Manifest)}\label{gate-15-stand-and-sit-in-reverence-of-he-whom-god-shall-make-manifest}
\addcontentsline{toc}{subsection}{Gate 15 (Stand and Sit in Reverence of
He Whom God Shall Make Manifest)}

The fifteenth chapter of the sixth Unity concerns the command of God
that, upon hearing the name of Him Whom God shall make manifest, the
title of ``The Standing One,'' one should rise from one's place. The
decree also mandates the punishment of those who grieve Him on the
surface of the earth, as much as is possible.

The summary of this chapter is that God has permitted all people that,
upon hearing the mention of Him Whom God shall make manifest with this
name, every soul should rise from its place and then sit down in
reverence to Him as commanded in the Book of God and in exaltation of
Him as decreed by the Primal Point. This is so that, on the Day of His
Manifestation, no one exhibits arrogance in His presence. All hidden
deeds exist for the Day of Witness, so that if someone, throughout their
entire life, rises upon hearing His mention but fails to do so an hour
before their soul is taken---whether they hear of His appearance,
receive His Book stating ``I am He,'' or He personally meets them and
declares ``I am He,'' presenting proofs and verses upon which His
religion is founded---and yet they do not submit to the Revealer of the
Qur'an or prostrate before the Sender of the Bayán, then all their
hidden deeds are rendered void, as if they had done nothing.

Conversely, its fruit is granted to him, and if He wills, He forgives
him, provided love is witnessed in the days of His concealment. Yet true
love cannot deviate from the good pleasure of its Beloved. Consider this
with utmost care: it is inevitable that you will meet Him, for the sake
of whose name you show such reverence and act accordingly. But for the
reality signified by that name, you fail to do what you do for the name,
even though, if you knew, you would act. Yet you deceive yourselves. As
in the Manifestation of the Messenger of God, all awaited Him, but when
He appeared, you heard what they did to Him. And yet, if they had seen
Him in their dreams, they would have boasted about it.

Likewise, in the Manifestation of the Primal Point, everyone rose for
His name, offering supplications day and night for His appearance. If
they had seen Him in their dreams, they would have gloried in that
dream. But now, when He has appeared with the most sublime proof by
which their religion stands firm, and countless are those awaiting His
manifestation, they sit comfortably in their homes after hearing His
verses. Meanwhile, He is now in this Mount of Máku, alone.

O people of the Bayán, reflect on yourselves, so that this does not
happen: that you cry for Him night and day and rise for His name, but
when the Day of reaping the fruit arrives---where the rising for the
name should lead to the path toward the reality signified by it---you
remain veiled. Recall the conduct of those awaiting the Messenger of God
and what they did to Him. Do not say they were not within the religion.
Today you claim they were not within the religion, but at that time,
among themselves, they considered themselves at the highest degree of
virtue and practice of their faith.

They did not perceive any falsehood in themselves, just as you see today
that all act according to the highest degree of virtue and religion, and
no thought of falsehood crosses their minds. Yet you will hear what has
befallen the Primal Point, for the events of the Resurrection are such
that they should all be recorded and written down, so that perhaps in
the next Resurrection they may be guided.

You will perceive Him Whom God shall make manifest. If you resolve, O
people of the Bayán, never to accept for anyone what you do not accept
for yourselves, and not to make truth appear as falsehood or falsehood
as truth, for it is pure truth. If you do not take refuge under His
shade, your religion will turn to falsehood, and sorrow may come upon
it. All the deeds you have performed since the Day of the Manifestation
of the Primal Point until that day will become void. Rely upon God, so
that you do not deviate from this thread and perhaps attain your
purpose. Do not bring sorrow in the days of His concealment, and in the
days of His Manifestation, where He makes all known to themselves
through His verses and His words. Yet they do not attain certainty about
Him and fail to recognize Him, thus remaining veiled.

There is no excuse for one who does not attain certainty in Him after
hearing these verses. This is My path, through which all enter Paradise,
while those veiled from it enter the Fire. No Paradise greater than this
has been created; rather, it transcends description as Paradise, for it
is above such characterization. Paradise is created by His command, and
if you wish to witness Paradise, look to the beginning of the religion
of Islam, where whoever entered it entered Paradise.

Otherwise, they were in the Fire, and the levels of Paradise were
detailed, degree by degree, so that you may observe each member of the
Household. In the manifestation of each, their companions were
companions of Ridván and Paradise, while those who turned away from them
were in the Fire. This continued until the last of the Doors, when the
specific text (explicit guidance) ceased. Whoever aligned with the will
of God and the will of the Household was in Paradise, and whoever
deviated by even a hair's breadth was in the Fire.

With the Manifestation of the Name ``The Inmost Hidden,'' observe how
Paradise was created and how those who did not recognize Him and did not
know His truth were judged beneath Paradise. Then examine the origin of
Paradise as explained in the Bayán, and distinguish the followers from
the deviators up until the manifestation of Him Whom God shall make
manifest. Judge the Paradise and Fire by witnessing them clearly, for
this is the purpose of what was revealed in the Qur'an regarding
Paradise and Fire. Seek refuge at all times from the Fire caused by lack
of faith, for this is the very cause of the next Fire. Similarly, faith
itself is the cause of the Most Great Ridván, wherein is created
everything that the servant loves, free from sorrow. And whatever the
servant does not desire, the knowledge thereof is with God, who creates
whatever He wills by His command. Indeed, He has power over all things.

\subsection*{Gate 16 (Prohibition of Non-Obligatory
Travel)}\label{gate-16-prohibition-of-non-obligatory-travel}
\addcontentsline{toc}{subsection}{Gate 16 (Prohibition of Non-Obligatory
Travel)}

The sixteenth chapter of the sixth Unity concerns the prohibition of
travel for anyone unless they intend the House of God or the House of
the Point, provided they have the means. Or, if they intend to engage in
trade, visit the Letters of the Living, provided they can do so with
ease and comfort, or to assist someone in the path of God. It is
forbidden to compel anyone to travel or to enter another's house without
permission, or to expel someone from their home without their consent.
If one does so, their spouse is forbidden to them for nineteen months.
Likewise, it is forbidden for anyone to transgress\ldots{}

If someone transgresses this ruling and does not act upon it, the
Witnesses of the Bayán must take from them ninety-five mithqáls of gold,
without exception. If someone attempts to compel another against their
will, it is obligatory for anyone who is aware or becomes informed of it
to intervene and prevent it. If they know and do not act, their spouse
is forbidden to them for nineteen days. After the expiration of these
nineteen days, it will not be lawful for them unless they provide
nineteen mithqáls of gold if able, or, if not, the equivalent in silver.
If they cannot afford silver, they must seek forgiveness from God
nineteen times until they can afford to give it. The funds should be
directed to the Witnesses of the Bayán, who will distribute them
according to the guidance of the Most High to the poor and needy among
the people of faith and the Book, each according to their rank.

No one should undertake a journey except for obligatory travel, such as
pilgrimage or presenting themselves before the Point. If they wish to
visit or engage in trade, they must not prolong their journey
unnecessarily. Should they choose to extend it, they must resolve any
matters pertaining to their intrinsic obligations and should not extend
their journey beyond thirty-eight months, except for those trading by
sea, who are allowed up to ninety-five months. It is not permissible to
exceed this duration. If someone surpasses these limits, they must
provide two hundred and forty mithqáls of gold, or, if unable, two
hundred and forty mithqáls of silver.

The summary of this chapter is that travel is permitted toward the House
and the Seat of the Point if one has the means, as well as for visiting
the Seats of the Living, engaging in trade, or assisting another person
if they so desire. Beyond these purposes, no permission for travel is
given. In trade, if what was created for them is present with them,
there is no harm; otherwise, exceeding these limits is not permitted.

Permission is not granted for travel by land exceeding two years unless
the purpose aligns precisely with what has been specified. In such a
case, God's permission applies. For travel by sea, permission is not
granted for more than five years. The reckoning begins from the day of
departure from one's home until the day of return. If someone exceeds
these limits, and if they are capable, they must provide two hundred and
two mithqáls of gold, or, if unable, the equivalent amount in silver, as
prescribed. This is one of the bounds set by God.

The purpose of this ruling is that on the Day of the Manifestation of
Him Whom God shall make manifest, upon hearing of Him, one may travel to
Him, prioritizing this above what is mentioned in this chapter. This is
because the entirety of the Bayán exists for Him, and this chapter is
but one among the chapters of the Manifestation of the previous
religion. If it is not renewed in the subsequent Manifestation, it does
not constitute a ruling of faith. Travel is not permissible without
adequate means for comfort and ease, except on the Day of Resurrection,
when it becomes obligatory---even if it be on foot---since everything
was created for Him. How could one separate themselves from the fruit of
existence if they observe the creation of their own being?

It is also decreed that anyone who enters another's house without
permission or compels someone to take even a single step on a journey
against their will, or expels someone from their home without consent,
shall not be allowed to marry for nineteen months. If someone
transgresses this ruling, the Witnesses of the Bayán are obligated to
impose a penalty of ninety-five mithqáls of gold to resolve the
transgression.

Any soul who becomes aware of someone compelling another must intervene
to prevent it. If they knowingly ignore this obligation, they are
forbidden from marrying for nineteen days. After the completion of this
period, they must provide nineteen mithqáls of gold if capable;
otherwise, the equivalent amount in silver. If neither is possible, they
should seek forgiveness from God nineteen times, after which marriage
becomes lawful for them. After obtaining the means for one or the other,
they are obligated to direct the funds to the Witnesses of the Bayán,
who will distribute them to those in need. If they deem it necessary,
they may allocate it to themselves; otherwise, it should go to the
rightful recipients among the believers in need, wherever they may be.
This practice is praiseworthy.

The purpose of this is to ensure that no one imposes undue hardship or
injustice upon another soul, thereby cultivating a universal habit that
prevents sorrow from befalling the intended purpose of the Day of His
Manifestation. If not for Him, no ruling would apply to any soul with
merit. Indeed, it is through the ocean of His bounty that all emerge
from the realm of obligation, for most people incline toward negligence.
After receiving a directive, they do not accept the command of God. Yet
God remains, in all circumstances, independent of His creation, desiring
that all ascend with utmost love in His paradises.

No soul should cause another even the slightest sorrow, so that all may
rest in His cradle of safety and security until the Day of Resurrection,
which is the first Day of the Manifestation of Him Whom God shall make
manifest. God has never sent any prophet nor revealed any book without
obtaining a covenant from all regarding faith in the next Manifestation
and the subsequent Book. This is because His grace has neither cessation
nor limitation.

In travel, it has been forbidden to stop at unnecessary stations. The
closer and lighter the stages are in the sight of God, the more beloved
they become. If a stage that could be traversed in one day is extended
to two, God will double the sustenance of the traveler. However, if an
animal suffers hardship at a station, it invokes retribution from God.

The owner must, in all circumstances, consider the limits of each
animal, ensuring that, after its arrival, they do not impose a burden
greater than what the animal can bear. Any benefit derived from it will
not yield fruit for its owner if it surpasses the animal's endurance. In
travel, one must always prioritize the condition of the weakest and
avoid anything that causes excessive burden or hardship, except for
journeys aligned with spiritual ease and joy as determined by God's
decrees.

Caring for pedestrians is always praiseworthy. If one allows another to
ride a step in the path of God's good pleasure, the reward of a
pilgrimage is recorded in their book of deeds. What virtue is greater
than this for someone striving in God's way? Should all travel be
transformed into paths of ease and joy, it becomes a portion of Ridván
itself. The previous rulings on burdens and hardships arose due to the
general veil of self-interest, where partial benefits imposed suffering
upon oneself and others. However, if these journeys were aligned with
ease and joy, such rulings would not apply.

The travels of the Arabs today bear witness to those prior decrees. This
cycle progresses step by step until no one can travel more than a
farsakh (approximately three miles). God protects whom He wills in their
paths by His permission, for He is a guardian over all things.

\subsection*{Gate 17 (Cleanliness and Refinement Do Not Effect
Purity)}\label{gate-17-cleanliness-and-refinement-do-not-effect-purity}
\addcontentsline{toc}{subsection}{Gate 17 (Cleanliness and Refinement Do
Not Effect Purity)}

The seventeenth chapter of the sixth Unity addresses the purity of what
is expelled from mice and the absence of an obligation to avoid it, as
well as similar rulings concerning creatures that fly at night, such as
those called ``bābīl.''

The summary of this chapter is that what had become burdensome for all
believers---such as the impurity of what is expelled from mice or
nocturnal flying creatures and the like---is now without harm.

Cleanliness and refinement have always been beloved and remain so,
without negating purity. In all circumstances, look toward the Purifier,
so you do not remain veiled from the Origin. On the Day of the
Manifestation of Him Whom God shall make manifest, do not mention purity
in a manner beneath His station, for He is above such matters. His words
themselves are purifying, and in every condition, He abides in places of
purity and sanctity. From the Day without beginning to the Day without
end, He has been and will remain in the heaven of purity and the earth
of refinement. Nothing can alter Him. His lineage to Adam, as well as
His mothers, have been among the chosen of creation and the abode of
purity and sanctity. Blessed is the one who attains His presence on the
Day of Resurrection with purity from Him, for that is the supreme
bounty.

\subsection*{Gate 18 (Required Consent to View Another's
Books)}\label{gate-18-required-consent-to-view-anothers-books}
\addcontentsline{toc}{subsection}{Gate 18 (Required Consent to View
Another's Books)}

The eighteenth chapter of the sixth Unity concerns the prohibition
against anyone looking into the books of another unless they are learned
and have obtained permission or are certain of the owner's consent.

The summary of this chapter is that no one is permitted to look into
another's book under any circumstances---whether sealed or unsealed,
whether scientific books, accounts, or personal writings---except when
they are sure of the owner's consent or when permission is explicitly
granted. In such cases, it becomes lawful. This prohibition exists so
that all may clothe themselves in the garment of modesty. If you attain
the Day of Resurrection and the Tree of the Desired One, conduct
yourself in a manner that aligns with paths of modesty. God safeguards
whomever He wills among His servants from what He does not love, for He
is the Guardian over all things.

\subsection*{Gate 19 (Every Soul Must Respond to Written or Oral
Communication)}\label{gate-19-every-soul-must-respond-to-written-or-oral-communication}
\addcontentsline{toc}{subsection}{Gate 19 (Every Soul Must Respond to
Written or Oral Communication)}

The nineteenth chapter of the sixth Unity concerns the obligation of
every soul to respond when written to or when asked about themselves.

The summary of this chapter is that in this Manifestation, it has become
obligatory that if someone writes to another, the recipient must reply.
It is not required for the response to be in their own handwriting but
can be through someone they appoint. Similarly, if someone asks a
question, it is incumbent upon the listener to respond in a way that
provides clarity. This is so that on the Day of the Manifestation of
God, no one remains veiled from the Supreme Luminary when He proclaims
the words of God, \emph{``Am I not your Lord?''} and all respond,
\emph{``Yes!''} For the obligation of responding is ordained here but
extends to the furthest reaches of existence.

Undoubtedly, on the Day of Resurrection, His Books will be revealed to
all. No one should become veiled from the response to their Beloved due
to their own veiling. By answering, the essence of their being is
created: in the particles of their hearts, acknowledgment of God's
oneness; in the particles of their spirits, acknowledgment of
prophethood; in the particles of their selves, acknowledgment of
guardianship; and in the particles of their bodies, acknowledgment of
servitude. In every Manifestation, those who respond are distinguished
from the silent.

In the previous Manifestation, where all were lovers, it was as if
anyone in Islam could simply recite the two declarations of faith,
without which acknowledgment of the Qur'anic ordinances and guardianship
would not be conceived. But in the subsequent Manifestation, the
truthful will be distinguished from the less truthful through their
responses. A discerning servant is one who responds to the Truth in all
worlds and stations, whether through writing, speech, or action, the
latter being the strongest.

From the blessings of such responses, every soul is commanded to answer
one another. Even if a child cries, it becomes obligatory to respond to
them, addressing their need. Similarly, if someone's state speaks for
them, even without words, their unspoken plea must also be answered.

It is incumbent upon those who possess insight to respond to such needs.
Similarly, if the gatherings or circumstances of a place necessitate a
response through other manifestations or expressions that the discerning
soul perceives, it is obligatory to respond. This is so that no soul, in
any situation, observes a cause of sorrow. Perhaps on the Day of
Resurrection, when the eyes of hearts fail to recognize their Beloved
and Desired One except through the will of God, they may not encounter
sorrow unexpectedly. This could happen when their foundation of faith is
suddenly removed, and they remain veiled from the conditions and
realities that stem from faith. Such has been the case in every
Manifestation for those who remain veiled. But God guides whom He wills
by His grace, for He encompasses all things.

\section*{Vahid 7}\label{vahid-7}
\addcontentsline{toc}{section}{Vahid 7}

\markright{Vahid 7}

\subsection*{Gate 1 (Renewal of Books Every 202
Years)}\label{gate-1-renewal-of-books-every-202-years}
\addcontentsline{toc}{subsection}{Gate 1 (Renewal of Books Every 202
Years)}

The first chapter of the seventh Unity concerns the renewal of books
every two hundred and two years and the disposal or use of prior
writings.

The summary of this chapter is that in every Manifestation, God desires
all things to be renewed. Thus, He has commanded that every soul renew
its books once every two hundred and two years. This can be done by
immersing them in fresh water or obliterating them personally, so that
the gaze of any servant does not fall upon writings that cause sorrow or
repulsion. This is so that on the Day of Resurrection, the tree of
reality may not witness sorrow upon the form of what is not beloved.
Perhaps the soul, too, will not dwell under the shadow of that which is
not beloved. Every sorrowful thing written is overseen by seventy
thousand angels, both at the time of its writing and at its erasure,
ensuring its protection.

If you consider today, you will see on the earth how many souls preserve
even a single letter of the Qur'an---so many that they cannot be
counted. Every word inscribed in the form of the beloved is the angels
desire to gaze upon him; rather, such a condition can be observed in
everything. Likewise, consider the opposite. In the Bayán, reveal
nothing except what reflects the loftiness of craftsmanship and
perfection, so that on the Day of Resurrection, the gaze of the Beloved
might fall upon it, and it may not be seen by others except through the
lens of My love. The sight of all the angels is under the shadow of that
gaze. How few are the things that, on the Day of Resurrection, are
worthy of the gaze of the Tree of Truth. Yes, if a thing exists above
the earth that has no equal, it is that thing which has been rendered
worthy, for ``nothing is like unto Him.'' And everything will desire
that its expositions be renewed. However, be vigilant before the
appearance of the Tree of Truth from whichever land it may arise, for
upon it depend the sustenance and delight of that Day. What came before
is a precursor to what will follow, serving as a sign for attaining that
purpose.

No soul writes even a single word of the Words of the Bayán with faith
in it and with beautiful calligraphy except that what their Beloved
desires becomes incumbent for them in the presence of God. The degrees
of calligraphy correspond to the ranks of a single name: the first
degree is the most glorious, and the last degree is the most exalted,
with various levels mentioned in between. It is as though, in every
Manifestation, those endowed with understanding perceive the confluence
of the nineteen pens, writing sweetly. Yet perfection in one surpasses
mere association, and the completion of all rests in the loftiness of
perfection if it aligns with the good-pleasure of the Eternal Beloved
and the timeless Purpose. Otherwise, it is evident today that countless
books in Islam have been written with the finest calligraphy, but the
traces that confirm the truth and cause it to prevail reduce all prior
books to insignificance in proximity to even the first letter.

No acknowledgment is accepted from them unless accompanied by faith, for
without it, their being is not regarded as valid. How then can the
beauty of their writings or the excellence of their bookcraft be
recognized, when their traces resemble an abandoned well or an imposing
but desolate palace? The most excellent beauty is that which is
recognized by the people of hearts as the beauty of faith, and such a
beauty is known by the name ``Jamal'' (Beauty). Yet, it has not been
heard that the \textbf{Kitáb-i-Qayyúm al-Asmá'}, reflecting the essence
of ``Qayyúm,'' corresponding to the numerical value of the name Yusuf
(Joseph), has been written with its due merit. From the beginning of the
manifestation until today, numerous books have been written, yet without
faith in Him, they yield no fruit at the time of the manifestation of
Him Whom God shall make manifest. In such a way, all will remain veiled,
except for those whom God wills.

The fullness of time has decreed that after the manifestation, nothing
should be written except the traces of that Sun of Truth. Writing even a
single letter of it is greater than writing the \textbf{Bayán} or
anything composed under its shadow. Consider the beginning of the
Qur'an: if, during the time of the appearance of the Messenger of God,
someone had written even a single letter of it with faith, they would
have been judged as believers. However, if someone had written the
entirety of the Gospel or all the teachings established in the religion
of Christ under the shadow of the Gospel, it would not have benefited
them.

This is the essence of knowledge---if you can comprehend it. And you
will comprehend it if you remain discerning and do not remain veiled.
For nothing in that manifestation is more beloved than presenting the
traces of it in beautiful calligraphy. Indeed, it is enjoined upon all
that whatever emanates from that source of bounty should be owned by all
believers in the \textbf{Bayán}. For it is the \textbf{Bayán} of this
Day, expressed in the loftiest manner conceivable within the
\textbf{Bayán}. Whoever is able to, should complete it and present it to
that Sun of Truth, so that all\ldots{}

The subject in question has always been this: How excellent is the art
of printing for elevating His words and multiplying His traces. If, on
the Day of Manifestation, they can aid the religion of God, well and
good. But if they remain as they are today---each veiled in their own
home---then God imposes no obligation upon a soul beyond its capacity.
And let them rely upon God; on the Day of Resurrection, you will be
assured of the signs of God.

\subsection*{Gate 2 (All Actions Require
Sincerity)}\label{gate-2-all-actions-require-sincerity}
\addcontentsline{toc}{subsection}{Gate 2 (All Actions Require
Sincerity)}

The Second Gate seventh unity, concerning intention: It is necessary
that no one performs an act without saying with their tongue, ``\emph{I
arise or I sit for the sake of God, the Lord of the heavens and the
earth, the Lord of all things, the Lord of what is seen and what is
unseen, the Lord of the worlds.}'' And if one recites this with their
heart, it is acceptable on their behalf.

The summary of this gate is that no deed is to be performed unless it is
done with sincerity. For this reason, it is commanded that every doer of
an action should, at the time of acting, say, ``\emph{I perform this for
God, the Lord of the heavens and the earth, the Lord of all things, the
Lord of what is seen and what is unseen, the Lord of the worlds.}'' If
one recites this in their heart, it suffices. However, no action becomes
truly for God unless it is acknowledged by the Tree of Truth, as this
verse is one of His signs. And through His sign within themselves, they
do not become veiled from Him on the Day of Resurrection.

For instance, in the Qur'an, whoever acted on behalf of the Messenger of
God and His living letters acted on behalf of God. It is not beloved
that one performs an action for anyone except for God. And it does not
truly become for God unless it is for that Manifestation. Today, the
servants who act in accordance with the Gospel perform all deeds for God
by the command of Jesus, peace be upon Him. However, before His advent,
the Day belonged to the Messenger of God. Yet at the time of
manifestation, actions performed for anyone but God cease to be valid.
Rather, in that Manifestation, one must act on behalf of the Messenger
of God, as it is then that actions for God become established.

Thus, the servants who act according to the Bayán and recite this verse
are considered to act for God only if, on the Day of the Manifestation
of Him Whom God shall make manifest, they act for Him. Otherwise, their
actions become void, as though they had done nothing at all. This
applies to the essence of religion. As for its branches, discern them
yourself, as well as the affairs of the world that ought to be for God.
Reflect upon them, for example: when you consume food, you may intend it
to be for God, yet you remain veiled from the Tree that signifies God.
The verse within yourself emanates from that Tree and ultimately returns
to it at its final appearance. Sometimes, you even withhold yourself
from it, while what you do for Him within yourself remains hidden from
Him.

Observe all worldly matters in this way---those you claim to do for Him
while being veiled from Him---and consider likewise the affairs of
religion. Through this process, you may grasp the essence of the word of
unity. If, on the Day of the Manifestation of Him Whom God shall make
manifest, you act for Him, you act for God, whether it be through
proclaiming ``There is no God but God'' or drinking water. Otherwise, if
you do not act for Him, even if you proclaim ``There is no God but
God,'' you will be cast into the fire. Similarly, if you drink water, it
will be as though you have drunk a wine less than paradise.

This knowledge is a simple essence: during the Manifestation of the
Messenger of God, it was decreed that all religions acted for someone
other than God, even though they were believers in the Book of God and
His Messenger during their own time and remain so even now. The mystery
lies in this: the same One who has been obeyed since the Day of Adam is
the Messenger of God. All revealed books are but a Qur'an that was sent
down upon Him. In truth, they have remained veiled from the
Manifestation before and His Book because they have not recognized that
this same One is He who appears in the subsequent Manifestation.

Now, even if you act for God according to the Bayán and do not stray
from the Letters of Unity, seeing all things as upheld by the First
Letter and perceiving nothing within them except the manifestation of
His appearance---from night to night, you are acting for God. However,
at the time of the manifestation of Him Whom God shall make manifest, if
you do not direct all your actions to the Point, they become for someone
other than God. This is because the Point of the Bayán on that Day is
none other than Him Whom God shall make manifest, and not anyone else.
Likewise, the living letters are the same as His living letters, for
whom you have acted. How then, when they appear, can you not act for
them?

This is why, during every manifestation, countless souls, thinking they
act for God, become immersed in doing so for someone other than God
without realizing it, except for those whom God wills to guide. If a
soul guides another, it is better for the guide than owning the east or
the west. For the one guided, it is better than all that is upon the
earth, as through guidance they enter paradise after death. However,
with all that is upon the earth, they only receive what is due to them
after death.

This is why God loves that all should be guided through the words of Him
Whom God shall make manifest. Yet, arrogant souls do not guide
themselves. Some are veiled by the name of knowledge, others by pride,
and each soul is veiled by something that provides no benefit at the
time of death.

Be exceedingly vigilant, for the path of truth is sharper than a sword
and finer than a strand of hair. With the guidance of the True Guide,
all are led to be rightly guided. Perhaps, what has been for God from
the beginning of life to its end may, in an instant, turn into being for
someone other than God, and you remain unaware. Indeed, God guides whom
He wills to the true and certain path.

\subsection*{Gate 3 (Every Person Must Immediately Fulfill Their
Obligations)}\label{gate-3-every-person-must-immediately-fulfill-their-obligations}
\addcontentsline{toc}{subsection}{Gate 3 (Every Person Must Immediately
Fulfill Their Obligations)}

The Third Gate from the seventh unity: Concerning the fulfillment of
religious obligations, the obligation is immediate.

The summary of this gate is that lending to a believer is beloved in the
sight of God and has always been so, as is the repayment of loans, which
is an obligation before God above all else. If one has the means to
repay, delay is not permissible. The purpose of this ruling is to
establish that just as the verses of glorification, praise,
sanctification, unity, and magnification---as well as all religious
matters---are manifestations of Him, so too is this an act granted by
the Point of Truth to His creation at the time of His Manifestation.
Repayment is obligatory, from the word of unity to the most detailed
aspects of limitation.

If someone promptly repays their debt at the time of the Manifestation,
nothing diminishes from what is owed. Otherwise, they will observe
something greater than this both within themselves and in the horizons.
How distant is the soul that fails to fulfill the rights of another
soul! How, then, could it fulfill the rights of God, by which the
servant becomes a believer in Him? Yet, in the moment of repayment, they
may outwardly profess their faith while remaining veiled from the debtor
and the loan itself. This is the limit of the creation if you observe
with the eye of certainty. Truly, God judges with justice, and He is the
best of judges.

\subsection*{Gate 4 (Purification Once Per
Cycle)}\label{gate-4-purification-once-per-cycle}
\addcontentsline{toc}{subsection}{Gate 4 (Purification Once Per Cycle)}

The Fourth Gate from the seventh unity: Concerning purification.

The summary of this gate is that it is permitted for the servant to
purify themselves once every year. This purification begins with the
first night of a cycle and concludes at sunset on the last day of that
cycle. It neither diminishes nor increases. The intent is that the
servant may find delight in one of the names of God. However, during
this focus, they are not permitted to turn their attention to another
name. If they forget, there is no harm, as it may be by the blessing of
this act that on the Day of Resurrection, they may affirm the oneness of
a single name and remain unshielded from His guidance.

It appears as though the Tree of Truth has manifested, being the source
of all names and attributes. Yet, the navigators of the ocean of
purification remain veiled by their own purification. This is because
the act of purification is intended for reaching Him, yet it becomes a
veil to Him. It is similar to the revelation of the Furqan and the
establishment of the religion of Islam, whose foundation is the
knowledge of God and whose ultimate purpose culminates in the most
detailed ordinances. All of this was ordained for recognizing the Sun of
Truth at the time of its rising. However, observe how every soul has
become veiled by one aspect, just as you see today. They remain heedless
of the purpose, which is the fruit of all these matters, and fail to
reflect upon it.

Even in this state of heedlessness, the proof of God has always been
complete upon all, for if they were to reflect upon the very proofs by
which they were bound to the religion of Islam, they could affirm the
truth. What you observe---how people act according to their religion
from the beginning of their lives to their end, never considering
anything beyond it---is because they are not tested, and the proof has
not yet been explicitly revealed. Otherwise, the same words that were
spoken in the early days of Islam at the manifestation of the Sun of
Truth are now spoken by these very souls.

For instance, at the zenith of Islam and its perfection, the intended
purpose of those words was precisely what they themselves once marveled
at, questioning how anyone could hear the verses of God and respond as
they did. This demonstrates that those who act with insight in every
matter are few, while the majority express faith in the religion in
which they were raised. Yet God knows in which souls purity resides and
in which souls insight dwells. Truly, God purifies those who believe in
Him and His signs from the remembrance of anything other than Him.
Indeed, He is mighty and invincible.

\subsection*{Gate 5 (A Person Who Hears the Call Must Abandon Their
Prior
Religion)}\label{gate-5-a-person-who-hears-the-call-must-abandon-their-prior-religion}
\addcontentsline{toc}{subsection}{Gate 5 (A Person Who Hears the Call
Must Abandon Their Prior Religion)}

The Fifth Gate from the seventh unity, concerning on the Day of the
Manifestation of the Tree, it will not be permissible for anyone to
follow the religion they adhered to before its appearance. When they
hear the call, they must present themselves until they are commanded as
He wills. If they do not yet attend, they may continue their previous
actions, but once they have arrived, they must abandon everything else
except what is commanded.

The summary of this gate is that what leads to salvation is the
recognition of the Manifestation, and likewise, the opposite leads to
loss. The religion of God in every manifestation consists of His
commands, which appear with the Manifestation itself. Consider from Adam
to the Seal (Muḥammad): the believers in a prior manifestation were
considered faithful only if they believed in the succeeding one.
Otherwise, all they had was rendered void. Similarly, at the
Manifestation of Him Whom God shall make manifest, the essence of
religion will be following His commands. For the good-pleasure of God,
exalted and glorified be He, is revealed only through His good-pleasure.
After the Manifestation, adherence to what preceded brings no benefit.

Reflect deeply upon the matter of your religion, that perhaps on the Day
of Resurrection, you may find salvation from the terror of that Day,
which is the day when God's proof becomes manifest to His creation.
Strive, therefore, at the time of the Manifestation---not before nor
after---if you desire to succeed.

\subsection*{Gate 6 (Prohibited To Own Instruments and Tools of War
Except in Times of
Necessity)}\label{gate-6-prohibited-to-own-instruments-and-tools-of-war-except-in-times-of-necessity}
\addcontentsline{toc}{subsection}{Gate 6 (Prohibited To Own Instruments
and Tools of War Except in Times of Necessity)}

The Sixth Gate from the seventh unity: Concerning the prohibition of
acquiring instruments and tools of war except in times of necessity or
at the time of struggle, except for those who manufacture them.

The summary of this gate is that instruments which cause one soul to
fear another are not beloved in the sight of God, except on the Day when
permission is granted to servants who are occupied with their
production. Likewise, clothing that causes sorrow to a soul shall not
enter paradise. It is incumbent upon the servant to be vigilant so that
no action arises from them that might instill fear in another soul.
Perhaps, on the Day of Resurrection, all will be adorned with the form
of humanity and qualities befitting it. Perhaps the Sun of Truth will
not gaze upon anything except that which accords with His good pleasure,
for you do not know whether you may become veiled from the essence of
existence. How often does one consent to associate with that which is
beneath their rank, while no fire of the soul is greater or more severe
than the veiling of one's essence.

Consider the soul whose words cannot be understood except by those
firmly grounded in the knowledge of the Word---those who are the Imams
of Guidance. They dwell in a realm where even one word of the language
of paradise, the Arabic tongue, cannot be spoken by its inhabitants, let
alone understood. How, then, can such souls comprehend? Reflect on what
transpires regarding the essence of existence. By the holy essence of
God, if anyone truly contemplates, their heart would immediately break
apart, for this knowledge is of the highest weight and responsibility
from God.

Such matters are revealed to those who, in every aspect, have faith in
Him and turn toward Him, whose beginning is from Him by His command, and
whose return is unto Him by His command. This knowledge descends upon
them. However, believing souls observe and comprehend the divine justice
required for meeting the One for whom all have been created. When this
kind of revelation occurs, it may settle in a place where no person of
insight exists to perceive it except those whom God wills.

This is why, in the Bayán, it is forbidden for a soul to associate with
one not of their kind. In accordance with what is apparent to all, it is
incumbent upon everyone to observe that scholars remain within their own
ranks, rulers within their own stations, merchants within their trade,
and other workers within their own sphere. This ensures that no soul
associates with others outside their own category, for that is
befitting. None can perceive the Sun of Truth except for those eyes that
see nothing but God. How, then, can all creation comprehend it? Maintain
the utmost vigilance, for you will encounter the Day of Resurrection.
Let nothing befall the essence of existence that is unworthy of its
station.

Whoever hears from the tongues of all created things the word
``Glorified is God (Subḥán Alláh),'' which exalts and sanctifies Him,
knows that anything lesser than this is negated in the court of His
holiness. How, then, could it be appropriate to hear any word other than
love or to witness any seat other than one of glory? Is it not the case
that, on that day, you will not merely hear or know but will experience
the reality itself? Consider the Day of Resurrection as akin to this
present one: if you do not comprehend the reality, at least acknowledge
its outward manifestation.

Servants who have renounced all that is on earth and have reached His
recognition through His grace are worthy to dwell in such a station.
This pertains to the outward level, but if you reflect on the reality,
you will find that day and night, nothing but Him is mentioned in your
heart's loftiness, and everything you do is for Him and through Him. Yet
even so, this veil persists.

Take heed, then, of God, your Lord, the Merciful, and distance
yourselves from anything that grieves the soul. O servants of God, all
of you, without exception.

\subsection*{Gate 7 (Seek the Grace of Him Whom God Shall Make
Manifest)}\label{gate-7-seek-the-grace-of-him-whom-god-shall-make-manifest}
\addcontentsline{toc}{subsection}{Gate 7 (Seek the Grace of Him Whom God
Shall Make Manifest)}

The Seventh Gate from the seventh unity: It is befitting for anyone who
attains the presence of Him Whom God shall make manifest to seek His
grace, if He wills, and to honor their station by being ennobled with
the dust of His feet.

The summary of this gate is that just as all created realities in
relation to the Sun of Existence are like reflections in a mirror, so
too should you observe the limits of all things. Know that no soul, in
the presence of God and in the sight of those endowed with knowledge,
nothing is more exalted than the Tree of Truth, and there has never been
anything greater than it. From the station where it appears, manifesting
the essence of lordship, none can fully perceive it, for they cannot see
its reality. Therefore, on the Day of Resurrection, which is the Day of
His Manifestation, all are commanded to seek from His grace that which
exalts them, so that they may not be deprived of the fruit of existence,
which is meeting Him. For this path is not open to all, and it is not
appropriate to compare it to other unions.

If you were to gather all that is other than God---what has already been
created and what could possibly be created in the future---and measure
it against even a fraction of the essence of His existence, it would not
equate to one-tenth of one-tenth of one-tenth of a moment of Him. For
the being of all things is derived from Him. How, then, could one
compare His mention with the mention of anything else? Likewise, in all
matters, extend this truth of reality so that you do not become veiled
from the source of all good.

Although this station is the most exalted of all, if it is made manifest
in a way that can be comprehended by others, it is as if those servants
who are adorned with the garment of honor cannot perceive it. How, then,
could anyone grasp His reality? This is because if His manifestation
were to appear through anything other than Himself, no soul would remain
veiled from the fruit of its existence. Truly, God bestows His grace
upon whomsoever He wills among His servants, for He is the possessor of
great bounty.

\subsection*{Gate 8 (Write One Name of God Each Day For One
Month)}\label{gate-8-write-one-name-of-god-each-day-for-one-month}
\addcontentsline{toc}{subsection}{Gate 8 (Write One Name of God Each Day
For One Month)}

The Eighth Gate from the seventh unity: It is enjoined upon all that
from the beginning of one month to the end of the next, they should
write, each one for themselves, one of the names of God that they love,
such as \textbf{``Alláhu Akbar'' (God is the Greatest)},
\textbf{``Aʿẓam'' (The Most Great)}, \textbf{``Aẓhar'' (The Most
Manifest)}, or similar. Permission has been granted to record this from
the beginning of one's life to its end, and then to inscribe what has
passed from it. If the person dies, it becomes incumbent upon their
heirs to write for them according to the time that has passed in their
life.

The summary of this gate is that every soul is commanded to fill out one
unit within one unit each month, with its accounting beginning from the
time of the formation of their seed until the time of their soul's
departure. If they fail to complete it, their heirs are responsible. The
purpose of this practice is that, on the Day of the Manifestation of the
Tree of Truth, it may increase the number of believers in Him. For the
innate realities of hearts are supported by God through these names,
just as the mention of anything provides assistance to its essence,
gradually elevating it to the rank of the body. This, in turn,
facilitates its entry into the unity of the Resurrection.

During that cycle, the outcome is that the first unit may multiply until
the heavens, the earth, and all that lies between them are filled with
it. On the Day of Manifestation, all will see within that unity nothing
except that unity itself in its most exalted form, where not a single
step falters upon the path. If this same unity had been witnessed during
the Manifestation of the unity of the Qur'an, not one Muslim would have
deviated from the path.

This entire unity, through its various degrees, culminates in the first
unity. Unite the entire world, one by one, and place the support for
each lower degree in its higher degree until all reaches a single unity.
That unity is the first one, which all are bound to recognize. In that
unity, nothing is seen except the unity without number, which begins
with the first letter. Thus, in the mirrors, nothing is seen but the
rising of the Sun of Unity itself.

This is the essence of oneness and the mystery of detachment. It appears
that the secret of unity flows and merges until it permeates all things.
Even the number of pens within a pen case becomes a single number,
symbolizing the point among the pens, exalting all through its radiance.

If someone initiates an action but does not establish the radiance of
the first unity as the source of all splendor, they have not recognized
the first unity nor accorded it its due right in its station. For
instance, if the worth of a pen is nineteen mithqáls of silver, then the
worth of eighteen pens should each be eighteen mithqáls, and their value
should derive from the value of the whole, as all are sustained by it.

This is why in its arising, the arising of all is mentioned; in its
gathering, the gathering of all is included; in its presentation, the
presentation of all is encompassed; and in its reward, the reward of all
is delivered. Consider how, in Islam, everything derives its sustenance
from the religion through the name of Muḥammad, His manifestations, and
the gates of guidance. Likewise, in the world, all depend on them for
support. Thus, all multiplied numbers are upheld by this single unity,
and that unity is upheld by the first unity, which is without number.
That first unity, in turn, is sustained by God, exalted and glorified be
He.

Beyond the rank of the first unity lie infinite degrees of unity, whose
enumeration none but God can comprehend. God purifies whom He wills and
multiplies the first unity as He wills by His command. Truly, He is
powerful over all things.

\subsection*{Gate 9 (Every King or Noble Person Must Build Two
Houses)}\label{gate-9-every-king-or-noble-person-must-build-two-houses}
\addcontentsline{toc}{subsection}{Gate 9 (Every King or Noble Person
Must Build Two Houses)}

The Ninth Gate from the seventh unity: It is enjoined upon every king
raised up in that religion to construct for themselves one house with
five doors before reaching ninety, and another house with ninety doors.

The summary of this gate is that every noble person elevated in the
Bayán should build two houses in the name of Him Whom God shall make
manifest. One should serve as their dwelling place, with the number of
its doors not exceeding ninety-five. The second house should have doors
numbering from ninety onward, so that the secret of truth may extend
even to the rank of inanimate objects.

So that the tongue of its essence---appearing as it is---may proclaim:
\textbf{``\textbf{\emph{To God belongs the dominion of the heavens and
the earth and all that lies between them.}}''} Perhaps, on the Day of
Manifestation, the testimony of clay will not fall short in bearing
witness to Him and will not withhold what is due to Him. Undoubtedly,
death will encompass all, and if one departs in faith and in support of
Him, their good name will endure until the Day of Resurrection.

Until now, it has not been heard that a sovereign, in the truest sense
of the word, has acted in accordance with their religion in a
Manifestation of Truth. Otherwise, they would have been set as an
example. From the time of Adam until the Manifestation of the Bayán, the
rulers of every nation, manifest in the name of God in that nation,
acted as they did. Yet, in the Manifestation of Him Whom God shall make
manifest, let the people of His era take pride that their names are
remembered for good until the Day of Resurrection before God.

Otherwise, they will vanish like those who have passed from the time of
Adam until now. None among them has been found on the Day of
Resurrection to serve as a model. Even though all, in their respective
nations, have acted as they believed for God, what benefit does this
bring when, on the Day of the Manifestation of the Tree of Truth, the
signs of God---His verses---are veiled like leaves concealed from His
Tree of Love?

As in the era of the Bayán, separation reached such an extent that,
imagining the truth to be with someone, they entrusted it to them, doing
as they did in the name of God and their own selves. It is clear how, in
the mountains of Maku, they caused Him to dwell.

The purpose of this ruling is that, perhaps, on the Day of the
Manifestation of the Tree of Truth, such houses may have multiplied, and
perhaps His house may come to be realized. It is not the case that, in
the beginning the purity of His proof is not complete unless it is the
proof that God has bestowed upon the Point of the Bayán. Until today, it
has not appeared in the hands of anyone among the foremost believers
that someone writes the verses of God in their own book and sends them
forth such that even a single verse of it establishes a conclusive proof
upon the recipient. Nor has anyone revealed, by the language of divine
verses, an answer to every soul who might inquire, with words sent down
from God.

In the Manifestation of the Furqán, which was the essence of previous
manifestations, the addressee was none other than the Messenger of God.
No verse was revealed directly to anyone else nor in written form.
Instead, what was revealed to Him was conveyed orally in the spoken
Arabic of that time. Even with the manifestation of such a proof and the
perfection of His bounty, consider what has transpired.

Yet this is the very Word by which all believers act upon even a single
verse. If all those on earth were to act, they would be worthy; indeed,
if the Resurrection were established and all were to act, it would be
borne. However, it is revealed upon hearts that do not behold the
mystery of existence or the essence of evidence at the source of
witnessing. They do not reflect, nor do they immediately prostrate
themselves before God, though this is the very reality of that which, if
it were revealed upon a mountain, you would see it humbled and shattered
in awe of God.

This was revealed concerning Him before and later addressed to souls who
hear the verses of God and do not prostrate themselves, as expressed in
the noble verse: ``And when they hear the verses of God, they do not
prostrate'' (Qur'an, 84:21). Despite this, they do not even consider
anything less than faith for themselves, even though they do not attain,
in the sight of God, the station of a mountain at the moment of its
humility. And there is no doubt that the later verses are greater than
the earlier ones, in ways beyond measure, from them, through them, and
to them.

If all the believers in the Bayán were, upon hearing a single verse, to
prostrate themselves and say, ``Yes,'' with no thought of ``No''
crossing their hearts, then they would be worthy of being called
believers in it. For that verse is the same one that declared, ``Am I
not your Lord?'' (Qur'an 7:172) to all, and all actions were for His
good pleasure. At that time, the fruit was taken, and the test was true.

By the One who is God, alone and without partner, if a soul in the East
were to behold all the Bayán within their preserved tablet, and were to
reach the highest degree of grace and proximity through all that is
possible in existence, and if the Book of that Sun of Truth were
revealed to them in the form of verses that demonstrated their own
incapacity, yet they were to hesitate for even the blink of an eye and
fail to say in their heart and with their tongue, ``This is from God,
there is no doubt in it; truly, we are all certain in God and His
signs''---then, not even the weight of a mustard seed of faith would be
deemed for them in the sight of God.

Even the Bayán that they preserved and acted upon would offer them no
benefit, for the innate disposition of unity would not be present in
them. They would not recognize the Word of their Beloved. If their heart
were like that of a mountain, it should have shattered out of fear of
God. How, then, can it be that such a judgment is applicable to their
prior rank, as when it pertained to the Qur'an? How is it that the
verses of that later manifestation, at its station, would make such a
soul more distant from God than a mountain?

They do not find justice within themselves concerning their own piety,
imagining they can preserve the Bayán while failing to perceive that it
is unimaginable for such a soul to arise---one who could preserve all
the Bayán and act entirely upon it. This notion is presented as an
example of impossibility so that all creation might understand its
limits in the presence of that Manifestation.

Do not consider yourselves harder than a mountain. Though you imagine
that whenever you hear the mention of your Beloved, you become as soft
as a tree, in the moment when submission is required---the moment when
the essence of your being must respond---it is as if you had not heard
at all.

O people of the Bayán, be vigilant over yourselves, for there is no
escape for any on the Day of Resurrection, which dawns suddenly and
judges as it wills. It can elevate the lowest of beings to the highest
station or lower the highest to the lowest. This is as it has been
decreed in the Bayán, if you reflect. None but Him has this power, and
whatever He ordains becomes so.

Consider how, at the time of the Messenger of God, when He wished to
make `Alí His successor, though they did what they did, His will was
nevertheless established. Now, too, whatever He wills must inevitably be
established, for it is the same decree by which the prior religion was
established, and the same decree by which the succeeding one will be
established. None but God is the possessor of command, and all act by
His command---if they act in accordance with His will. Otherwise, they
remain outside the realm of His decree.

Truly, God exalts whom He wills among His servants, for He is powerful
over all things.

\subsection*{Gate 10 (Safeguard the Form of the Name
Al-Mustagháth)}\label{gate-10-safeguard-the-form-of-the-name-al-mustaghuxe1th}
\addcontentsline{toc}{subsection}{Gate 10 (Safeguard the Form of the
Name Al-Mustagháth)}

The Tenth Gate from the seventh unity: Let every soul safeguard the form
of the Name \textbf{Al-Mustagháth} (The One to Be Invoked) at the time
of its birth, and no one should neglect it.

The summary of this gate is that among the names of God, none equals the
numerical value of the Name Al-Mustagháth. It is the highest fruit of
the names, having reached the pinnacle of its manifestation. Within it,
nothing appears but the first unity, and within the first unity, nothing
appears but the first unity itself. In the Qur'an, it was the Messenger
of God; in the Bayán, it is the essence of the Seven Letters; before the
Qur'an, it was Jesus; and after the Bayán, it will be Him Whom God shall
make manifest.

The thrones manifest differently in various dispensations, yet the one
seated upon the thrones, who is beyond the limits of all bounds, remains
the Primordial Will. The thrones do not alter it. Among the names of
God, none is higher in numerical value than the name Al-Mustagháth in
the rank of the divine names.

When the numbers of Alláhumma are computed, if you count each unity as
one, the numerical value of the name Aḥad (The One) is incomplete.
However, if you calculate Al-Mustagháth with the definite article Alif
and Lám, its numerical value exceeds that of the name Ḥayy (The Living).
On the Day of Resurrection, the manifestation of that name becomes
evident, pointing directly to God. Therefore, it is enjoined that from
the time of the formation of the seed, all should safeguard its form in
an image that contains the numerical value of the name Al-Mustagháth.

From the beginning of manifestation to the final appearance, God knows
its extent, but it will never surpass the number of Al-Mustagháth unless
God wills otherwise. In the dispensation of the Qur'an, its beginning
and return were aligned with the name Aghfir (Forgive), which fell short
of the numerical value of the name Huwa (He). In the Bayán, God knows
its extent, but there is no fixed measure for it, as the interval
between the Gospel and the Furqán did not even reach a thousand years.

This is because the Tree of Truth, at all times, observes its creation.
Whenever it sees the capacity for its manifestation reflected in the
mirrors of hearts, it makes itself known to all with the permission of
God, exalted and glorified be He. For it has no motion or stillness
except by God, the Mighty and the Glorious.

The result of this is that all the names revolve around the name of God,
and the perfection of all names is in reaching this name. Perhaps all
souls in the Bayán may attain as much as possible so that, at the time
of the manifestation of the Truth, they may be able to perceive the Sun
of Truth.

And let them revolve around it, remaining vigilant not to exceed that
number. If a soul in the Bayán reaches that number and hears the
Manifestation of the Truth, the Tree of Truth revealed to them, they
should return to Him. Even if they do not fully believe, perhaps through
this grace, they may be saved from the fire. There is no bounty greater
in the Bayán than this, if only they would recognize its worth and
deliver themselves from the fire of Him Whom God shall make manifest,
dwelling in the shade of His light.

For His Manifestation is the beginning of the creation of inner
realities within the atoms of hearts, following the completion of the
atoms of bodies before His appearance. Just as writing a thousand and
one names on a form suffices for protection, so too might this safeguard
them from being veiled by the outward appearances of the names and
enable them to see nothing but God. They will witness only the
good-pleasure of their Beloved.

So, let your souls be adorned with God, your Lord, and with all His most
excellent names. Truly, to Him belongs the creation and the command in
the dominion of the heavens, the earth, and all that lies between them.
\textbf{``There is no God but He, the Mighty, the Beloved.''}

\subsection*{Gate 11 (Ascending Pulpits is
Forbidden)}\label{gate-11-ascending-pulpits-is-forbidden}
\addcontentsline{toc}{subsection}{Gate 11 (Ascending Pulpits is
Forbidden)}

The Eleventh Gate from the seventh unit of the first unity: Concerning
the prohibition of ascending pulpits and the command to sit on thrones
or chairs.

The summary of this gate is that ascending pulpits has been forbidden,
and sitting on thrones, platforms, or chairs has been commanded. This
ensures that no one exceeds the bounds of their station or dignity. If
the gathering place requires it, a chair should be placed on a platform
so that all can hear the words of truth.

The purpose of this command is that, on the Day of the Manifestation of
Truth, someone might, with humility, learn from the source of knowledge.
How great is this matter, for His knowledge is not separate from His
essence, and those firmly grounded in knowledge, the Imams of Guidance,
cannot comprehend His Word; how then could all, who have chosen
learning, do so? Indeed, every form of knowledge that can bear the name
of a thing was created for the recognition of His Word. And in His
sight, it is nothing but the essence of existence. Even this Word that
you witness is brought into being within its own realm, by its own self.
Otherwise, His essential station is far above being associated with
letters. No greater joy has been created in the realm of possibility
than for someone to hear His verses, understand their intent, and
refrain from questioning the reasons or comparisons regarding His Words.
They should not measure them against the words of others. Just as His
Being is the manifestation of Divinity and Lordship over all things, so
too is His Word the manifestation of Divinity and Lordship over all.

If that Being were human, He would declare, ``Indeed, I am God; there is
no god but Me. All beneath Me is My creation, feeding upon My words. So
fear Me.'' Yet, He states, with the tongue of His very essence, what He
has uttered, and all things hear it. If this were not so, how could it
be that at the advent of every Manifestation, all previous heavenly
Books must show faith in Him? Similarly, the believers in those Books
must believe in this Manifestation. Thus, with a single verse, His proof
over all on earth is established for every soul that possesses a tongue,
in their own language.

For instance, if one of the people of Adam's community were addressed
with a single verse, it would be as if addressing the first believer in
the Bayán. To them, it would be said that all on earth are powerless to
produce anything like it. If they immediately acknowledge this, they
have believed in God, for none is more truthful than He in speech. By
God, reflect deeply! If one does not attain certainty in the Word of the
Lord, it is incumbent upon them to present their case to all on earth.
Once they witness the incapacity of all and observe the evident truth,
they must turn to Him and affirm His truth in His Word. From the moment
of hearing until they attain certainty, they remain in the fire of
veils.

It is said by all, ``O Truest of the True!'' But at the time of the
Manifestation, if they acknowledge the Tree of Reality and, during the
recitation of His verses, hesitate not in affirming Him, then they truly
have said, ``O Truest of the True!'' Otherwise, their deeds belie their
words. For this is but one of His Names and one of His Lights, which
indicates His truthfulness. How, then, can His Essence be described by
this description? For if truth exists within the realm of possibility,
it is through His truthfulness. And yet, He proclaims with the loudest
voice to all creation, ``Why do you hesitate in affirming that which is
revealed?''

This is why the Pen encircles the world in an instant, save for those
who are willing, yet none perceive. If they claim, ``We have not heard
His verses,'' they have indeed heard them. If they argue that the
faithful servants are truthful only through adherence to the Qur'an,
this is untrue. If they say, ``This verse is not the verse of God, and
none are powerless before it,'' where is the one who has brought forth
its like, born of innate disposition? And yet, like an ocean, abundant
waves flow from this Sea of Bounty.

Thus, all who believe in Him are truthful, affirming His Word. Yet, they
withhold the Name He has bestowed upon one of the Manifestations of His
Cause---a Name that establishes truthfulness and affirms His first Word.
If they had not withheld, no Manifestation would have been rejected at
the time of His appearance. Thus, all are truthful in His Name, but they
deny Him who bears that Name. It is as though one were to say, ``The sun
in the mirror is truthful in its light to its degree,'' yet denies the
sun of the heavens. How veiled they are!

Consider the monks during the time of the Messenger of God (Muhammad):
they were deemed truthful by adhering to the religion of Jesus. Yet,
they did not affirm the Sun of Reality, around whom all religions
revolve as their Seal. They sought to affirm the sun of the heavens
through the truth of its reflection in the mirror, were it to be true.
But by failing to affirm the Messenger of God, the radiance of the Sun
was withheld from them.

So too, consider the Manifestation of the Bayán, and similarly, the
Manifestation of Him Whom God shall make manifest. Reflect deeply, lest
you affirm Him based on the acknowledgment of the followers of the
Bayán, for this would be akin to the example given. Rather, affirm Him
by His own self. This is the essence of the verse, ``Know God through
God.'' Upon this foundation, all branches are established: ``Believe in
God through God,'' ``Love God through God,'' ``Obey God through God,''
``Follow God through God.''

Thus, in that Day, behold all names and attributes through Him. Consider
how intricate this matter is: if, at the time of the Manifestation of
the Tree of the Bayán, all on earth had been believers in the Qur'an,
yet failed to affirm Him at the first verse, they would all, in the
sight of God, have been false, while He alone was truthful. Reflect on
how subtle this matter is! Today, you desire to affirm individuals whose
understanding of His Words originates in the interpretations of those
who believe in Him.

You have not recognized your Beloved and have wandered in the night.
Otherwise, if you truly know Him, you would say ``no'' if He says
``no,'' and ``yes'' if He says ``yes,'' because all that you previously
understood as truth is confirmed by adhering to His Word. This is why,
at every Manifestation, the people of that age become veiled by the
scholars of their own Dispensation, while remaining heedless of the
matter itself. With one ``no'' from Him, all other proofs become
invalid.

Be vigilant regarding the Manifestation of Truth. At the time of His
appearance, do not affirm Him based on the acknowledgment of the
followers of the Bayán. For in His sight, all their affirmations amount
to a single ``yes,'' and all their denials to a single ``no.'' Consider
the Qur'an: if the Messenger of God had once affirmed a group, today
everyone would affirm that group by affirming the Messenger of God. And
if the opposite had been said, the opposite would hold true.

This is because truth is determined by the testimony of God, not by the
testimony of the people. Affirmation is based on His Word, not the word
of creation. All the promised ones of Ahmad who failed to acknowledge
the Messenger of God through His Word were rendered false, even though
there was no doubt that some among them were truthful in their time.
These individuals had not deviated from the religion of Jesus, yet in
the sight of God, they were not truthful, for if they had been, they
would have believed in the Messenger of God.

Likewise, consider the Manifestation of Him Whom God shall make
manifest. All are false, except for those servants who affirm
Him---whether they are the highest of those on earth or the lowest. For
affirmation of truth lies in His Word, and all become truthful by
following Him. There is no doubt that in His presence is every
Manifestation, God tests His creation by that which their religion has
been established upon and by that which has drawn them nearer to God.
God guides whomever He wills to the path of truth.

\subsection*{Gate 12 (One Who Acts For God Cannot Claim Anything For
Themselves)}\label{gate-12-one-who-acts-for-god-cannot-claim-anything-for-themselves}
\addcontentsline{toc}{subsection}{Gate 12 (One Who Acts For God Cannot
Claim Anything For Themselves)}

The twelfth gate of the seventh unity concerning: It is unbefitting for
one who acts for God to associate anything with Him.

The essence of this gate is that in every Manifestation, whoever acts
for Him has acted for God, and whoever acts for God has acted for Him.
On the Day of Resurrection, which is His Manifestation, along with His
Letters of the Living and all the believers in Him, no one is permitted
to claim anything for themselves, whether it be a grain of dust or the
smallest particle. This is because their existence was created for Him.
How then can anything branch forth from that existence, even though He
is, and always has been, exalted beyond such limitations?

This has been the pattern in the Manifestations of Reality within the
Qur'an and the Bayán. However, this pertains to the station of the
servants and the fruit of their existence on the Day of Return. Beyond
this, there is no doubt that the Tree of Reality on the Day of
Resurrection is unknown to anyone before His appearance, nor can anyone
establish this principle prior to His Manifestation. After His
appearance, disagreements arise to such an extent that it becomes
impossible for this Cause to be obeyed except by those who comprehend.
Unless in His Manifestation all advance together such that no division
arises in faith, only then will the fruits of the Bayán be confirmed for
them.

In such a time, even in limited circumstances, how much can one
enumerate for a single soul's actions? This is nothing but the
recognition of all beings in their respective stations. Beyond this is
the Reality, within whose dominion countless realms of wealth and
self-sufficiency, the Manifestation is independent of these acts, yet
because they carry a fragrance other than the oneness of God, they are
prohibited. And God bestows grace upon whomever of His servants He
wills; indeed, He is the All-Bountiful, the All-Knowing.

\subsection*{Gate 13 (All Servants Must Possess Nineteen Verses Written
In His
Hand)}\label{gate-13-all-servants-must-possess-nineteen-verses-written-in-his-hand}
\addcontentsline{toc}{subsection}{Gate 13 (All Servants Must Possess
Nineteen Verses Written In His Hand)}

The thirteenth gate of the seventh unity: God has decreed for all His
servants to possess nineteen verses, written in His hand, during the
days of His Manifestation.

The essence of this gate is that nothing is, or ever has been, greater
in the sight of God than His verses. Should these be adorned by the pen
of the Tree of Reality, upon which the name of God is invoked---such
that when you say ``verses of God,'' you are also saying ``the station
of God''---then they become the most precious of all things in the sight
of God and the people of knowledge.

It is incumbent upon every follower of the Bayán to possess a single
tablet containing nineteen verses written in His own hand. On the Day of
Resurrection, nothing can compare to its value. Even if someone were to
own the entirety of the earth and exchange it for such a tablet, its
worth would still be greater in the sight of God and the people of
knowledge. For it is a certificate of salvation from God for that soul.
However, if, God forbid, they turn away, it becomes a writ of fire for
them until the Day of Resurrection.

Similarly, just as anything revealed to the believers during this
Manifestation became the highest fruit of their existence and their
source of honor in Paradise, anything revealed to those who do not
believe becomes, in turn, a divine proof against them, condemning them
to the fire eternally---unless God wills otherwise. Whether it is
written in His own hand or includes His traces, the value remains
unparalleled.

Should it ever be possible for this state to change during the
Manifestation---where the rejected ones become accepted---all His traces
would then be preserved with the utmost beauty in a script that
surpasses all that is conceivable within the realm of possibility---but
since this is unattainable, others will partake of this grace. If
someone writes a single verse from Him, it is better than transcribing
the entirety of the Bayán and all the books composed within its
dispensation. For all else is elevated and vanishes, while that verse
endures until the next Manifestation.

In that Manifestation, if someone writes even one letter from the new
Revelation with faith in Him, its reward is greater than writing all the
sacred writings of past Truths and those composed under their shadow.
Thus, ascend from one Manifestation to the next, where there is no limit
to your ascent within the knowledge of God, just as there was no
beginning to it.

It is as though the writings of that Sun of Truth are descending upon
the believers, and they receive the one who bears them with an honor
greater than they would give to their dearest beloved. They rise to
serve that Revelation, for they are the pillars of the Faith and the
witnesses of certitude. Not like what you observe among the believers in
the Qur'an during this Manifestation, where they stand in the presence
of the Qur'an yet remain veiled from the Tree of Reality, which is its
Source. Such is the way of those outside the Twelve Imams, or perhaps
this is done to exalt the Qur'an such that it becomes elevated above its
Source.

The essence of knowledge is evident to those endowed with understanding:
if someone recognizes the Manifestation, these deeds are but branches
arising from Him, and they will comprehend all goodness. For instance,
if a soul in the early days of Islam had written the chapter of Tawḥīd
(Oneness of God) from the Book of God, it would have been better than
transcribing the Gospel and all the books composed under its shadow.

Similarly, in the present, from Manifestation to Manifestation, let the
Point of Reality flow, lest you remain veiled from the true purpose.
Then you shall possess the best of all that God has created, if God
permits, and give thanks to Him. Know that this command arises from His
grace being elevated; otherwise, no one is worthy of receiving His
bounty. It is all through His generosity that creation seeks His grace,
while He bestows only what His will decrees. He gives to whomever He
wills and withholds from whomever He wills. However, when the people
truly believe in God and His verses, God bestows His favors upon them
collectively.

\subsection*{Gate 14 (Repent to the Manifestation or in Private With
God)}\label{gate-14-repent-to-the-manifestation-or-in-private-with-god}
\addcontentsline{toc}{subsection}{Gate 14 (Repent to the Manifestation
or in Private With God)}

The fourteenth gate of the seventh unity: Repentance is not permissible
except through the Manifestation of God during His appearance.
Otherwise, seek forgiveness from God privately within yourselves.

The essence of this gate is that seeking forgiveness from one soul to
another is not permitted during the night or in the Day of Resurrection,
except with the permission of Him Whom God shall make manifest.
Otherwise, one must seek forgiveness from God in all states, privately
between oneself and Him. If a servant seeks forgiveness from God to the
fullest extent possible, that servant deserves to be forgiven according
to the essence of their reality, insofar as they align with His decrees.

One should not seek forgiveness while remaining veiled from the One from
whom they are seeking forgiveness. For true forgiveness from God is not
realized except through seeking forgiveness from the Manifestation of
His Cause, who is the Tree of Reality, as well as through His Letters of
the Living during His Day of Manifestation. Otherwise, even if one
performs acts of repentance as numerous as all created things, it will
yield no fruit for them. This is akin to seeking forgiveness day and
night while remaining unjust toward the One through whom forgiveness is
truly sought, for repentance before Him is repentance before God.

Within the realm of possibility, there is no path for creation to reach
God except through the gates of unity, which culminate in the One
without number, the First, who is the Originator of all infinite numbers
before and after. Similarly, in the letters of unity, the letter
\emph{Sīn} is not worthy of forgiveness except through \emph{Bā}, and
likewise, \emph{Mīm} except through \emph{Sīn}. This continues, letter
by letter, until the First Unity is completed, at which point its
multiplicity can be enumerated. Thus, the matter of seeking forgiveness
is cut off except through the Manifestation of the Resurrection.

However, if a soul seeks forgiveness privately between itself and God
without exceeding the bounds of the Bayán, God will accept that
repentance until the time of the Manifestation. At that time, repentance
will no longer be accepted except through the Manifestation of the
Cause. Similarly, consider all deeds and behaviors, and recognize the
station of the Day of Manifestation, for all revolve around it.
Therefore, seek forgiveness from God at all times---before, during, and
after His appearance.

\subsection*{Gate 15 (Prostrate at the Gate of the City Where the Point
is
Revealed)}\label{gate-15-prostrate-at-the-gate-of-the-city-where-the-point-is-revealed}
\addcontentsline{toc}{subsection}{Gate 15 (Prostrate at the Gate of the
City Where the Point is Revealed)}

The fifteenth gate of the seventh unity: On the obligation of
prostration at the Gate of the City wherein the Point of Divinity is
revealed, as an exaltation from God for Him, for He is the Mighty, the
Beloved.

The essence of this gate is that since all souls are created under the
shadow of the signs of Divinity and Lordship, they are ever moving
toward loftiness and elevation. Yet, as they lack the insight of true
vision to recognize their Beloved, they remain veiled from humility
toward Him. While throughout their lives they may have bowed in
submission to the commands of prior dispensations, worshipped God, shown
reverence for that Truth, and displayed humility toward it in its
reality, they remain veiled in the moment of the Manifestation. People
look upon themselves and remain veiled from Him, for they perceive Him
as a form like their own. Yet, God is exalted beyond association. The
reality of His form is like the sun of the heavens, and His verses are
its radiance. The believers, if they are truly faithful, are like
mirrors in which the sun is reflected; the radiance they show is
according to their capacity.

Thus, it has been decreed that in the city from which He arises, all
must prostrate themselves at its gate. Likewise, the land that becomes
the site of His Manifestation, such as the city of His rising or the
fortress of His recognition, is sanctified. It is incumbent upon all
souls, when entering that city or that land, to bow in prostration. When
the Manifestation appears, the decree of the prior dispensation is
nullified, and the ruling of that day is established by the permission
of the One who is made manifest.

Although everything is preserved under the Tree of Reality, at each
Manifestation, the obedience of the preceding dispensation is revealed
during the new appearance. For instance, on the Day of the Manifestation
of Him Whom God shall make manifest, the humility of creation before Him
is revealed through the teachings of the Bayán. The beginning of a
Manifestation is likened to the seed of the preceding one, even though
the seed of the new Manifestation is greater than the fullness of the
prior one.

However, since humanity cannot fully comprehend this progression, such
explanations are given. Otherwise, there is no soul that does not, in
some way, obey its Creator. This obedience is inherently submission to
the Tree of Reality, even if veiled. Yet, this obedience becomes
rebellion in the next Manifestation, similar to how those who acted
according to the Gospel and humbled themselves before the words of Jesus
failed to humble themselves before Muhammad, the Messenger of God.

In the preceding Manifestation, the deeds of that time are not accepted
in the subsequent Manifestation. Similarly, the actions performed today
according to the Qur'an, even with humility and devotion to God, are in
essence acts of submission to the Point of the Bayán. However, because
they do not recognize the Sun of Reality, they remain veiled from faith
in Him. This is not due to His unwillingness for all to believe in Him;
rather, the salvation of all depends on their faith in Him. He, Himself,
is independent of their belief, just as if all had believed in Muhammad,
the Messenger of God, it would have been their own salvation. Yet, the
Tree of Reality remained, and remains, in Paradise by itself.

O people of the Bayán, if you believe in Him Whom God shall make
manifest, you become true believers yourselves. Otherwise, He is
independent of all and will remain so. For example, if an infinite
number of mirrors were placed before the sun, they would reflect its
light and bear witness to it, while the sun itself remains independent
of the mirrors and the sunlight reflected within them. Such is the limit
of possibility in the face of the Manifestation of the Eternal.

Take heed, then, to reflect upon yourselves, making your essential
qualities, realities, and souls mirrors of love for the One. Perhaps on
the Day of the Manifestation of Reality, they may reflect the First
Unity and not be veiled by the Second Unity or those that follow unto
infinity. This is the supreme grace and great victory if you comprehend
it. Otherwise, you will nullify the fruits of your existence with your
own hands.

Today, seventy thousand souls journey annually to visit the House of God
in accordance with the command of Muhammad, the Messenger of God. Yet,
He Himself, the one who commanded this, resided for seven years in the
mountain of Mecca. Meanwhile, the one who commands is stronger than the
command itself. This is why all these people who now go on pilgrimage do
so without insight. For if they had insight, they would have believed in
His command during the time of His return, which is greater than His
previous Manifestation. Yet you see how they remain bound to the command
of His prior Manifestation, worshiping night and day, prostrating to God
in its name. Meanwhile, He resides in solitude, and they are veiled from
the true honor, which lies in faith in Him.

Consider how today, all take pride in their belief in Him from a prior
Manifestation, while remaining veiled from His truth in His current and
subsequent Manifestations. The faith of the past is judged as incomplete
in the light of His new Manifestation, just as the faith of the
believers in the Gospel was judged deficient at the time of Muhammad,
the Messenger of God.

Observe this pattern in every Manifestation: the teachings of the prior
one appear, but only to a degree. Each subsequent Manifestation reveals
the essence of the previous ones in greater clarity, to the point where
what was merely peripheral in a prior Manifestation becomes the core of
the current one. Even if this occurs at a single degree, it is a
continuous progression. Reflect on all such instances and avoid being
veiled so that you may attain the fruits of your existence and remain
eternally in your own paradise.

This is from the grace of God upon you, that you may prostrate before
Him. It is not like today, where people prostrate at a distance of two
leagues from Najaf or in the sea at great distances to honor the
purified dome, despite such acts not being decreed in Islam. Yet, at the
time of the Manifestation, they do not prostrate even once before Him in
His presence. If they surpass this and no injustice befalls them.

On the Day of Manifestation, all are tested. If they reflect and
recognize God's Manifestation by knowing themselves, all will attain
salvation. However, since they fail to do so, veils obstruct them, and
they remain ensnared within their own coverings. God guides whomever He
wills to the path of true certainty.

\subsection*{Gate 16 (Every Bayani Ruler Must Not Allow Non-Bayani's In
Their
Land)}\label{gate-16-every-bayani-ruler-must-not-allow-non-bayanis-in-their-land}
\addcontentsline{toc}{subsection}{Gate 16 (Every Bayani Ruler Must Not
Allow Non-Bayani's In Their Land)}

The sixteenth gate of the seventh unity: God has decreed that every
ruler sent forth under the religion of the Bayán must not allow anyone
who does not adhere to that religion to remain in their land. Similarly,
this obligation is placed upon all people, except for those who engage
in universal commerce that benefits humanity, such as the Letters of the
Gospel.

The essence of this gate is that it is decreed by God for every
sovereign, past and present, to prevent non-believers in the Bayán from
remaining in their land. In the Manifestation of Him Whom God shall make
manifest, the same applies to those who do not believe in Him. The
result of this is that on the Day of Resurrection, the Tree of Reality
will not be observed in lands where it appears, except by those who
believe in it. Likewise, in the lands of paradise, no fiery soul shall
dwell.

To expel the people of the Bayán from the bounds of paradise is contrary
to God's pleasure. Be vigilant, lest in the Manifestation of Him Whom
God shall make manifest, this same treatment is applied to its
believers. Consider how, in this Manifestation, the Letters of the
Living were regarded. Despite all demonstrating their knowledge through
the traces of previous dispensations, the foundation of their faith was
firmly established in their love for Him. Yet, they were not subjected
to what others were subjected to. Reflect on how many remain veiled,
while the Letters of the Living are so enlightened.

Be cautious, so that no soul is subjected to similar treatment. For
nothing has been prohibited as strongly as this. Perhaps on the Day of
Resurrection, by adhering to this command, you may attain salvation,
following the Letters of Unity, the foundation of the Faith.

Do not bring sorrow upon those whom you have affirmed as the foundation
of your faith, for you do not truly know them. In moments of
uncertainty, they bear the signs of truth that can lead you to certainty
that they are indeed of God. If you are among the people of insight, you
will recognize by hearing their verses that the beginning of every
Manifestation is the rising sun of the Day of Resurrection, and the
Letters of the Name \emph{One} convey the Cause of God to all.
Permission is granted to the Letters of the Book and to any being whose
existence benefits the believers through commerce. In such cases, their
trade is permitted; otherwise, it is forbidden.

This prohibition exists so that, on the Day of Resurrection, the Tree of
Reality is not observed in the land of faith except by its believers.
Should there exist a soul within a land that is not in alignment with
this belief, it is equivalent to that soul being in the fire---except
for those granted permission due to universal trade. Otherwise, it is
entirely prohibited, particularly if it pertains to matters that do not
honor the divine station. In such cases, no permission has ever been
granted. Therefore, fear God in this ruling and remain mindful.

Moreover, if one soul associates with another, such association is not
lawful unless it aligns with the purity of faith. Any interaction that
falls short of this standard remains impermissible. The condition of
association is the purity of religion, not anything less. Fear God with
true certitude, O people, all of you collectively.

\subsection*{Gate 17 (Friday is for Purification, Refinement, and
Tranquility)}\label{gate-17-friday-is-for-purification-refinement-and-tranquility}
\addcontentsline{toc}{subsection}{Gate 17 (Friday is for Purification,
Refinement, and Tranquility)}

The seventeenth gate of the seventh unit: On the recitation of the
following verse on Friday, facing the sun: \emph{``The radiance is from
God upon your countenance, O rising sun. Bear witness to what God has
borne witness to concerning Himself: that there is no god but Him, the
Mighty, the Beloved.''}

Know that God created Friday for purification, refinement, and the
tranquility of the servant from the burdens borne during the six
preceding days. Any deed performed on the night or day of Friday is
rewarded as if it were performed across the entire week. Since the
spirit of all things is connected to humanity, and the testimony of all
things is through the testimony of humanity, it has been decreed that on
Friday, one should face the sun and take it as a witness over the verse
that affirms the oneness of God, faith in the Point of the Bayán, and
all that has been revealed therein.

Perhaps, on the Day of Resurrection, this act may cause the sun of
reality to testify and proclaim the oneness of God before Him and the
truth of all who follow Him. This is the fruit of this command if one
can grasp its significance. Otherwise, there is no doubt that after the
Manifestation of the Cause, every soul will, on Friday, proclaim the
truth. However, on the Day of Resurrection, their statement will vanish
if they fail to proclaim it in the presence of God. It is incumbent upon
all to utter this word on the Day of Manifestation before Him Whom God
shall make manifest, on every Friday, whether they are present before
Him or permitted to do so in accordance with His will.

``He does as He wills and decrees as He desires. He is not to be
questioned about what He does, while all shall be questioned about all
things.''

\subsection*{Gate 18 (Penalty for Bringing Intentional Sorrow to
Another)}\label{gate-18-penalty-for-bringing-intentional-sorrow-to-another}
\addcontentsline{toc}{subsection}{Gate 18 (Penalty for Bringing
Intentional Sorrow to Another)}

The eighth gate of the seventh unit: Whoever intentionally brings sorrow
to a soul must pay nineteen mithqáls of gold, if able; otherwise,
nineteen mithqáls of silver. If neither is possible, they must seek
forgiveness from God nineteen times, unless they obtain explicit pardon,
in which case nothing further is required.

Whoever imprisons a soul is prohibited from partaking in anything lawful
for them until the imprisonment ends, and all their deeds are nullified
of the believers. If someone returns to an act prohibited to them, they
are required to pay nineteen mithqáls of gold every month. Anything
contrary to this has no basis in the Bayán.

The essence of this gate is that God, in His grace and bounty, has
commanded His servants not to bring sorrow upon another soul, for the
Tree of Reality should not be subjected to grief, especially on the Day
of Concealment, when its nature remains unknown to others. If anyone
transgresses the divine bounds, the prescribed penalty applies to them.
If they persist in their transgression, they will not be regarded as
believers. Permission has been granted in such matters, and the
prescribed measures have been outlined, applying to gold and silver in
the numerical value of the Unity (\emph{One}). Should they remain veiled
from the divine bounds, they are to seek forgiveness from God by
repeating the numerical count of Unity (\emph{19}) and request pardon
from the one they have wronged.

This is because the heart of a believer is the locus of God's
Manifestation. To bring grief upon such a heart is akin to grieving the
Letters of the Living. To grieve the Letters of the Living is akin to
grieving the Tree of Reality, and to grieve the Tree of Reality is as
though one has grieved God Himself, exalted and glorified. Therefore, no
act of worship in the Bayán is more meritorious than bringing joy to the
hearts of the believers, and nothing is more abhorred than causing them
sorrow.

The same applies doubly to those with higher stations of responsibility
(\emph{ulu'l-dawá'ir})---whether in bringing joy or in causing grief.
Thus, every individual must remain vigilant in all circumstances,
ensuring that, even if they cannot bring joy to another, they should at
least refrain from causing sorrow. If someone obstructs another soul in
any manner, they have transgressed the divine bounds because no action
has been as severe in its distance from God, nor is any other act
regarded with such gravity. The perpetrator of such deeds is not
considered a believer, and belief cannot be ascribed to them. If anyone
obstructs the connection (\emph{maqám-i-iqtirán}) of another, the means
of lawful provision are cut off from them. So long as the obstruction
remains, the state of lawful association cannot be applied to them, and
all their deeds are nullified---even if they were performed with the
highest degree of piety and godliness.

Should they return to what is impermissible, they are liable for the
divine penalty, which is the payment of the numerical value of Unity
(\emph{19}) in gold every month. If the obstruction results in a state
of harmful consequences (\emph{ẓuhúr-i-ḥayátí}), the ruling will reflect
this state under the Bayán. One must always remain vigilant in all
circumstances, prostrating day and night before God, and avoid creating
obstacles that nullify all deeds unknowingly.

\emph{``Fear God with true certitude so that you may attain success.''}

The outcome of this teaching is that the people of the Bayán are to be
trained in such a manner that they may not bring sorrow upon Him Whom
God shall make manifest on the Day of Reality. Nor should they act in
accordance with their customary behavior, for every servant, from the
beginning of their life to its end, is indebted to His Cause. Their
works and efforts are for the sake of attaining His presence.

It is possible that the Manifestation may appear, yet such a soul would
grieve Him or pass judgment against Him, even while outwardly appearing
to be among the most exalted, possessing all the means of guidance.
Their spirit might depart without recognizing the Manifestation, despite
their days and nights being spent in yearning and supplication for that
meeting.

Had those in positions of authority during the time of Músá ibn Ja`far,
peace be upon him, altered their behavior after hearing his plight, it
is possible that on the Day of Reality, they would not have brought
sorrow upon the Tree of Reality. For lack of such behavior, such sorrow
was not inflicted. Yet how many structures have been erected by rulers
during their reigns, which bore no fruit for them on the Day of
Resurrection? If they had eradicated such innovations, it is possible
that no soul would have caused sorrow. These seemingly trivial matters,
which may appear insignificant at first glance, become far more
significant in their station than all that exists on earth. Their
effects ripple outward, even if their origin seems slight.

There is no doubt that the leaders of the Bayán will address the
instances of sorrow inflicted upon the Tree of Reality and work to
eliminate them. This is so that, on the Day of the Manifestation of
Truth, such events will not recur during His return. Nevertheless,
people may still fear the Manifestation itself, for all are entrapped by
the specter of their own limited understanding. If one resolves not to
grieve another soul, it is possible they may impose a veil upon their
own soul and prevent it from attaining faith in God.

Yet, if sorrow is inflicted upon the Tree of Reality, this is far graver
than any sorrow that could be imagined within the realm of possibility.
The sorrow of the Tree of Reality outweighs all, from the smallest
particle to the greatest being, for all things are connected to that
Reality. It remains exalted beyond association with anything.

Next come the Letters of the Living, one after another, in their degrees
of proximity, followed by the First Unity, the Second Unity, and onward
to infinity. For instance, if the lowest of beings believes on the Day
of Manifestation, they ascend higher than the highest who fail to
believe. This is why, in every Manifestation, the exalted become abased,
the lowly are elevated, and the reverse also occurs: the exalted are
made even more exalted, while the lowly are further abased if they do
not affirm the Truth. \emph{``God bestows by His command upon whomsoever
of His servants He wills; indeed, He is All-Knowing of all things.''}

\subsection*{Gate 19 (Prayer)}\label{gate-19-prayer}
\addcontentsline{toc}{subsection}{Gate 19 (Prayer)}

The nineteenth gate of the seventh unit: On prayer.

The essence of this gate is that the first prayer established was the
noonday (\emph{ẓuhr}) prayer, and all subsequent prayers were ordained
with the numerical value of Unity (\emph{19}). Each prayer serves as a
gateway to paradise, provided it is offered in obedience to the Truth.
In this paradise, nothing is seen except the Manifest One in each
letter, and through all, the One without number is perceived. Each
prayer serves as an emblem of humility and submission before Him Whom
God shall make manifest.

If anyone outwardly disdains obedience to Him but inwardly worships God
through Him, such worship is still accepted. After recognizing the
Truth, no act is greater than prayer, and there never will be. The
prayer of every soul corresponds to their station of existence. For
example, the prayer of the Point in relation to the Letters of the
Living is analogous to the prayer of the Letters of the Living in
relation to the Second Unity, and so on. The prayer of the Second Unity
stands in relation to the Third Unity, continuing until it reaches the
last created being.

Before the appearance of the Tree of Reality, prayers are offered
according to the apparent ordinances. However, the prayers of all souls,
even if they encompass numerous cycles of the Letters of the Living, are
as nothing compared to a single prayer of the Letters of the Living. How
could such prayers approach their station? Consider this for all deeds:
a single utterance of \emph{There is no God but God} by Him Whom God
shall make manifest is incomparable to the proclamation of divine unity
by all things---whether uttered previously, afterward, in secret, or in
public.

This is because He is the one of whom God testifies concerning Himself.
Whatever others affirm of divine unity---whether the Letters of the
Living or the countless multiplicities derived from the First
Unity---all affirm unity through Him. If you find it difficult to
conceive of this, consider the case of the Point of the Furqán
(Muhammad), whose prayer could not be equated with the prayer of any
other, even that of the noblest of creation, such as Amír al-Mu'minín
(Imám 'Alí), to the last being in existence. All prayers are created by
His command. How, then, could they be compared to His own prayer?

This is because everything revolves around its own self and cannot move
beyond its origin. In the prayers commanded by Him, the prayer of all
creation cannot equal a single unit of prayer offered by Amír
al-Mu'minín. The same applies to those nearest to the First Unity,
progressing closer and closer until reaching that Unity itself. The
glory (\emph{bahá}) of all is measured against His prayer, much like
stripping away the multiplicity of things to arrive at a single grain
containing the essence of all.

This essence exists in its \emph{bahá} (glory), not in its \emph{dhát}
(essence). Similarly, the prayer of Amír al-Mu'minín contains within it
the \emph{bahá} of all prayers, though not their \emph{kaynūniyyah}
(existential reality). Reflect on this in all matters: the prayer that,
at its inception, elicited mockery and derision from the Arabs, is now
the subject of extensive discourse among scholars. Consider how much
discussion and debate occurs regarding a single minor issue related to
prayer, with scholars composing thousands of verses to address it.

Do not let these circumstances veil you from the Source. For example,
today, all believers in the Qur'an perform seventeen obligatory units of
prayer each day, and in every unit, they perform two prostrations before
God. Yet, they remain veiled from its reality and impose upon it
whatever interpretations they wish.

On the day when the Messenger of God ordained prayer, it was established
solely to instill humility and submission in the hearts of the people in
preparation for the Day of His Return. Since no act of worship is
acceptable except through the recognition of divine unity in their
station of worship. Consider how all, in their acts of worship, direct
themselves toward God, yet within themselves they neglect the outer
bounds of prayer. Those who devote themselves to prayer day and night,
but fail to turn toward God, are judged below the station of truth.

Reflect on prayer as one aspect of their religion: at the time of the
Manifestation, even the twin testimonies of faith, which are the
foundation of religion, are renewed. For instance, during the
Dispensation of the Messenger of God (Muhammad), if someone proclaimed
\emph{``There is no God but God, and Jesus is the Spirit of God,''} it
was not accepted as faith in Islam unless they also proclaimed
\emph{``There is no God but God''} accompanied by \emph{``Muhammad is
the Messenger of God.''}

Likewise, in the Dispensation of Him Whom God shall make manifest, the
testimony \emph{``There is no God but God''} is only accepted if it is
accompanied by the testimony regarding Him. This is essentially the same
testimony as \emph{``There is no God but God''} and the testimony of the
Point of the Bayán. After the advent of the new Manifestation, other
expressions are no longer acceptable, just as in the Dispensation of the
Qur'an, previous expressions were only accepted when they were replaced
with the Arabic language and an acknowledgment of Muhammad's
prophethood.

If such a transformation is decreed in the essence of religion, how
should one understand the rulings pertaining to it? Today, you see that
every soul is veiled from its reality by one of its external practices.
Were they to focus on the essence of the mirror, they would, in the
blink of an eye, surpass the path and not remain veiled by such matters.

Do not let these matters veil you at the setting of the sun. If you
deviate even slightly, the witnesses of the Bayán will judge your faith
to be beneath its loftiest station. Nevertheless, keep your vision
steadfast. Always direct your attention to the Source, for all of these
matters are as insignificant as a ring in your hand, which you move
however you wish. Indeed, one is called a ``worshiper'' (\emph{muṣallī})
when they believe in the Letters of Unity and perceive honor in the
Cause of God rather than in the object itself. They must not be veiled
by focusing on the object, for if it were unworthy, it would not serve
as the locus of the Cause.

In prayer, one must intend God alone, considering this point: on the Day
of Resurrection, they must be humble before even a single Letter of the
First Unity. Only then can the term \emph{muṣallī} be applied to them.
If they pass the test, they are deemed truthful on the Day of
Resurrection. Otherwise, in the night of concealment, their worship is
accepted only to the extent that they observe this perspective.

The worshiper, during prayer, should see none but their Beloved and
focus solely on God, \emph{One without Partner}. If they bring to mind
any name or attribute in their worship, they become veiled, and their
worship is not accepted. Instead, they must turn their attention to the
Most Sacred Essence of God, who ``neither begets nor is begotten.''
Everything apart from Him is His creation, and He cannot be comprehended
in His essence, nor can He be described by the sanctity of His being.
None is worthy of worship but His Essence.

However, within all these matters, the worshiper does not stray from the
gates of guidance. Each unit of prayer opens a gate to paradise, which
is the recognition of the Letters of Unity on the Day of their
Manifestation. This paradise is beyond the bounds of material forms. The
paradise containing forms and limits---of silk garments, gold ornaments,
fresh delights, pure beverages, and companions described as resembling
pieces of ruby, along with other descriptions. These are safeguarded by
what is below the First Unity, and they all derive their sustenance from
the First Unity. However, one must not be veiled by these descriptions
when standing before the First Unity, which is the locus of love and
manifestation. This station has not yet descended into the realm of
material and formal boundaries. When it does manifest, it does so
through the assistance of multiplicities, not independently.

For instance, if today forty thousand crystal lamps were illuminated in
one place as a likeness of the purity of paradise, consider how such an
illumination reflects divine truths. A saying of old declares that
\emph{there is no extravagance in lamps}. Thus, even these material
boundaries are sustained by the First Unity. Although, in the early days
of Islam, there was but a single lamp before Him, this sustenance
applies to all dispensations. Reflect upon these manifestations so that
you may not remain veiled on the Day of Resurrection.

Even before the First Unity, it is fitting that whatever type of lamp is
possible within creation should be present before Him. However, all
these formal boundaries derive their sustenance from the multiplicities
within His realm, not from the essence of the First Unity itself.
Consider the visible world and reason upon it: if all creation were
taken individually and considered one by one, they would ultimately
converge upon the First Unity. This is Muhammad, the Messenger of God,
with the Letters of the Living as His companions.

At every level of multiplicity, they are sustained by that Unity, upon
which all depend. Likewise, the prayers of all creation do not exceed
their own bounds. All prayers are, in essence, prostrations before the
prayers of the Letters of the Living. The prayers of the Letters of the
Living are themselves in prostration before the prayer of the Point. And
the Point's prayer is in prostration before God, alone and without
partner, mirroring the reality of prayer itself.

I have expressed this so that you may perceive the realities of
existence as akin to what you observe externally: actions are like
existential realities (\emph{kaynūniyyāt}). For instance, though the
prayer of the last created being is essentially like the prayer of the
first created being, just as the essence of the last created being
cannot be mentioned alongside the first created being except within its
own station as the last, so too should all actions be observed.

In three units of your first prayer, affirm the oneness of the Divine
Essence. In the four units that follow, affirm the oneness of His
Attributes. In six subsequent units, affirm the oneness of His Acts. And
in the final six units, affirm the oneness of worship. Proclaim
throughout: \emph{``There is none other God but God, alone, without
partner.''}

Similarly, view all multiplicities as reflections of the First Unity. If
a soul worships in the final creation, whether in the oneness of
actions, attributes, or essence, it is nothing but the reflection of
what is manifested in the First Unity. It demonstrates that
\emph{``There is no creator but God, no provider but God, and no giver
of life but God.''}

In all mirrors, it is the single Sun that is revealed, which is the
essence of the primal Will. This Will, in itself, points to God.
Whatever is done in their mirrors is, in reality, directed toward Him.
He is God, exalted and glorified.

If you observe with this perspective, you will see with certitude that
the Will governing all things is the apparent Will within creation. It
is not that the essence of the Will becomes visible in the realities of
other beings; rather, in each Manifestation, whatever is possible within
the realm of existence is revealed. This is the meaning of the verse:
\emph{``Is it you who sow the seed, or are We the Sowers?''}

Thus, I have likened this from the lowest individual to the highest, so
that you may comprehend these truths for yourself. Observe all these
ranks as clearly as you see your own palm, and pass through them with
understanding, so that on the Day of Resurrection you may reap their
fruits. Otherwise, countless philosophers have written thousands upon
thousands of books on the knowledge of God.

If they had observed to whom their praises were directed, where they
originated, and in what Manifestation they were grounded, their
knowledge would have benefited them on the Day of Resurrection. Thus,
only that understanding which is witnessed and realized bears fruit on
the Day of Resurrection. Otherwise, it remains entangled in the whims of
the soul.

In worship, direct your focus solely toward the Essence of the Unseen,
the Eternal, who alone is worthy of worship. However, since this
attention is linked to acknowledgment of what the Letters of Unity
uphold in knowledge and submission to Him, one enters the city of divine
unity by seeing nothing other than God. If a soul, in the act of
worship, focuses on a name, they have not truly worshiped God and are
veiled from the divine purpose, exalted and glorified be He.

All names point to the reality that there is no God but Him, no deity
other than Him. Everything that is called into being is His creation,
and He alone deserves worship and adoration, not His creation. Worship
God in such a way that even if He were to consign you to the fire for
your worship, your devotion to Him would remain unchanged. And if He
were to grant you paradise, your worship would still be unaffected by
that reward.

This is the essence of worthiness for worship: God alone. If you worship
out of fear, you are not fit for the divine realm of holiness, and the
oneness of God cannot be affirmed in your case. Similarly, if your
attention is focused on paradise or seek His approval through worship,
you are associating creation with Him---even though creation, such as
paradise, is beloved by Him. Both paradise and the fire worship and
prostrate themselves before God. What is truly befitting His Essence is
worship that is rendered solely for His sake, without fear of the fire
or hope of paradise.

While it is true that after authentic worship, the worshiper is
preserved from the fire and attains the paradise of His pleasure, these
should not be the cause or motivation for worship. Worship in its
essence flows from the grace and existence of God as determined by His
divine wisdom.

The most beloved prayers are those offered with spirit and joy, and
lengthy prayers are not favored. The more refined and pure the prayer,
the more it is beloved in the sight of God. Apart from the prayer of
Unity, all other forms have been transcended. What has been decreed in
this Manifestation is the remembrance of God, performed with spirit and
joy, which is the highest form of worship and the noblest of ranks.

If a person remains veiled from the significance of a single unit of
prayer due to attachment to all that is on earth, they are at a loss in
the sight of God. The grace of that prayer surpasses all else, and its
essence transcends all creation. All units of prayer revolve around the
Point of Unity, which is the origin of motion and the essence of prayer.

Know that when you stand in prayer, you are in the presence of the One
in whose grasp lie your beginning and return. Nothing is hidden from His
knowledge, and nothing renders Him powerless. He is capable of all
things and knows all things, both before their existence and after their
creation, just as He knows their essence and reality.

This understanding has been conveyed in the places of divine decree:
that all who worship God should worship Him as He truly deserves and
witness the manifestations of divine decrees, taking pleasure in their
sweetness. \emph{``So perform your prayers by the permission of your
Lord, then fear Him, and worship none but God. Truly, you are among
those who are certain in the signs of God.''}

\section*{Vahid 8}\label{vahid-8}
\addcontentsline{toc}{section}{Vahid 8}

\markright{Vahid 8}

\subsection*{Gate 1 (The Deeds of Him Whom God Shall Make
Manifest)}\label{gate-1-the-deeds-of-him-whom-god-shall-make-manifest}
\addcontentsline{toc}{subsection}{Gate 1 (The Deeds of Him Whom God
Shall Make Manifest)}

\textbf{The first gate of the eighth unit:} On how the deeds of Him Whom
God shall make manifest compare to others, as the sun compares to the
stars.

The essence of this gate is to understand that the deeds of Him Whom God
shall make manifest are like the sun, while the deeds of all other
beings, even if they align with the pleasure of God, are like stars and
the moon. The first to recognize the Sun of Reality through
understanding achieves its fruits.

On the Day of the Manifestation of the Sun of Reality, if all existence
testifies to a truth in alignment with His pleasure, His testimony among
them is like the sun. In the presence of the sun, no other light can be
mentioned. His word is light yet simple in expression, but acting upon
it at the beginning of each Manifestation is extremely difficult.

For instance, if someone had adopted this perspective during the early
days of the Qur'an, even if all the words of the Gospel were true, they
would not equate to the words of the Messenger of God. This is because
the words of the Messenger are like the sun, while others are mentioned
in the night, not in the day. Similarly, if someone were to hold this
view during the Bayán, at its advent, they would see that the word of
the Point of the Bayán is like the sun, while the words of all
others---even if true---are like the light of stars in the night.

In the Manifestation of Him Whom God shall make manifest, if all the
people of the Bayán act according to His word and perceive their
existence and deeds as stars in the presence of the sun, they will reap
the fruits of their existence. Otherwise, even the status of
``star-like'' will not apply to them, except for those who believe in
Him, for in the light of the day, stars are utterly effaced.

And in the night, with light---this is the fruit of this decree if one
should grasp it on the Day of Resurrection. And all knowledge and action
are contained within this, if one attains success. If all had viewed
matters from this perspective, then in no advent of manifestation would
there have been an appearance with the decree of annihilation rather
than subsistence in relation to creation. This is because all, in their
night, perceive themselves as possessing a light within their own
limits, yet they remain veiled from the reality that, when the dawn of
another daybreaks forth, nothing of the self remains for them; rather,
they are obliterated in the presence of the radiance of the sun.

And consider that the light of all represents their knowledge and
speech, just as the radiance represents the actions of ``He Whom God
shall make manifest''; his words are such that they fold up all
existence and establish it within the shade of a single letter of
relation. And he proclaims with the tongue of his theophany---which is
none other than God, the Mighty, the Glorious---``Indeed, I am God;
there is no god but Me, and all else is My creation. Say: O My creation,
fear Me!''

And his deeds, too, utter the same, if you should hear. And the essence
of knowledge lies in the recognition of the origin of the command until
its return. Look upon the stars of ignorance, which, after the ascent of
the sun of reality, had shone forth in their own manifestation. But once
that same sun arose, no light remained for them. And in every
manifestation, perceive the same: though they still assume themselves to
be luminous and continue to act, before you---who have attained faith in
Islam and know---they are devoid of light, veiled from the radiance of
the sun of truth, which is the Messenger of God.

And so too, reflect upon the origin of every manifestation, so that you
may not remain veiled from the fruit of your own existence when you
reach the presence of your Beloved. And so, ponder that which appears
from God. Then, just as God has taught you in the Book, you deduce from
it.

\subsection*{Gate 2 (Upon Death, Give 19 Sheets of Parchment and 19
Rings)}\label{gate-2-upon-death-give-19-sheets-of-parchment-and-19-rings}
\addcontentsline{toc}{subsection}{Gate 2 (Upon Death, Give 19 Sheets of
Parchment and 19 Rings)}

The Second Gate of the Eighth Unity: Every soul must bequeath to its
heir nineteen sheets of fine parchment and nineteen rings inscribed with
a name from among the names of God. None shall inherit from the deceased
except his father, mother, wife, son, brother, and sister. And whatever
knowledge he has gained after utilizing it for himself from his own
possessions shall be according to what brings him honor.

The essence of this decree is that since the degrees of divine oneness
are completed within seven letters---those being the letters of
affirmation---it has been ordained that none shall inherit from the
deceased in the reality of truth except seven souls. Just as in the
station of every attribute, God may be invoked through seven ranks
within that attribute, such as Aḥad, Daḥḥād, Wāḥid, Waḥīd, Mutawaḥḥid,
Muwaḥḥid, and Mumahhid. It is from this mystery that this decree has
been set forth from the origin of the command.

No soul that seeks to will a matter and that which aligns with its
intent does so without invoking God through seven names, each of which
may be recited the number of Qāf, unless the means of that matter become
manifest for him. The intent is that if it is for God and in God, it
shall be decreed to come to pass.

The fruit of this is that on the Day of Resurrection, when all decrees
are established from the presence of God, if the Tree of Reality issues
a command, no one shall ask ``why'' or ``how.'' Just as the laws of
inheritance are now made clear in the Furqān---had the origin of
manifestation decreed upon the soul beyond these limits, that decree
would still be the decree of the Messenger of God. The difference is
that on that day when he issued his decree, today, we remain infinitely
within its shade.

And they act accordingly. But on that day, when it is brought upon the
soul alone, it will become difficult for it unless it turns its sight
toward the origin of the command. Just as on the day when the laws of
inheritance were revealed in the Qur'an, the matter was made
manifest---so too, the origin of this decree appears in the same manner
to that soul.

And yet, on the Day of Resurrection, until all are tested, the Tree of
Reality will not make itself known by the name of the First
Manifestation. All must attain such a degree of certainty and insight
that if a hundred thousand souls were to gather for the circumambulation
of the House, having left their homelands for the day in which
permission for circumambulation has been granted, and on that very day
the Tree of Reality appears and commands, ``Do not circumambulate,'' if
all immediately comply, then their circumambulation is realized.
Otherwise, all their deeds become void.

This is because their circumambulation, as it is now performed, was
originally undertaken at his command---not for any other reason. The
proof of this lies in the Book that was before and the Book that will
come after, which none but God can reveal in its like. And if, at the
moment of manifestation, one discerning soul remains, he will turn his
gaze to the origin of the command, attain certainty, and realize that he
was circumambulating for the sake of God alone, purely for Him.
Meanwhile, the rest will become like scattered mobs and ignorant crowds.

This is the path which, for one person, becomes wider than the heavens
and the earth, while for another---who does not attain certainty---it is
sharper than a sword and finer than a hair. This is why, at the origin
of every manifestation, those endowed with insight are rarer than red
sulfur, for the majority follow one another and act upon the decrees of
God's command through mere imitation. Even if they act in truth and for
His sake, since they do so without insight, they are found wanting at
the moment of the manifestation of its bearer.

They remain veiled from the new command of His decree, even though the
previous command was but a manifestation to ensure that the servants,
obedient to the Truth, would remain obedient to the Truth. This was so
that, when the new manifestation occurred, they would obey it as well.
However, when the manifestation takes place, those who believe
themselves to be obedient still follow in their assumed obedience---yet,
in that moment, it is disobedience.

It is like the believers in the Gospel: until the appearance of the
Messenger of God, they were all obedient to God within their religion
and were praiseworthy in their deeds---so long as they upheld the law of
Christ as it had been revealed. Yet, when the Messenger of God appeared,
the very foundation of their faith, which was the word of testimony, was
made manifest in a new decree. Where before their declaration of divine
unity was ``There is no god but God,'' upon the new manifestation it
became, ``There is no god but God, and Muhammad is the Messenger of
God.''

And concerning the station of the successors, the decree of 'Ali and the
Imams as the proofs of God was revealed. And in the mention of the
pillars of His House, the names of the Gates of Guidance were
established. If even the fundamental principles of their faith underwent
a transformation at the moment of revelation, how then could the
manifestations of the ordinances and laws of that religion remain
unchanged?

This is the mystery behind the statement of the late Shaykh---may mercy
be upon him---when someone asked him about the saying that on that day,
even the three hundred and thirteen chiefs of that manifestation would
be unable to bear it. The Imám Ṣādiq, after much admonition that they
would not be able to endure, spoke of the letter Kāf in reference to
them.

And when that man was asked, ``If the Manifestation were to appear and
command you to renounce the guardianship of the Commander of the
Faithful, would you do so?''---immediately, he refused and protested,
saying, ``Never! By no means!'' Yet it is evident to the people of
reality that the word which flows from the tongue of the Manifestation
was conveyed to him, he could not bear it, and so he became an
unbeliever---yet he was unaware of it. This was only because he did not
turn his gaze toward the origin of the command and perceived the
appearance of the Manifestation as something distinct from the
appearance of the Messenger of God. But if he had viewed the
Manifestation in the same manner as the Messenger of God, relative to
the appearance of Christ, he would have been able to bear that word,
which is vaster than the heavens of accepted beliefs and the earth of
receptivity.

Yet, because he failed to perceive it, the path became sharper than a
hair for him and keener than a sword. The intent of the Speaker was not
that one should abandon the guardianship of the Commander of the
Faithful, for this is an impossible matter. His light has ever been and
shall always be manifest in His chosen ones. Rather, the meaning was
that in that Manifestation, he appeared under the name of the Commander
of the Faithful ('Alí), while in this Manifestation, he appears under
another name---so let none be veiled by this.

Just as the Commander of the Faithful ('Alí) in the time of the
Messenger of God was the very same as the successor of Christ in his own
time, after his ascension---so too, in every manifestation, if one looks
toward the origin of the command, the path becomes vaster than anything
that can possibly be conceived. But if one remains veiled, it becomes
narrower than any strand of hair that knowledge can encompass.

If all believers were to attain such insight that they would stand as
one in their station---just as in the example of circumambulation--- and
if that Tree of Reality were to summon a single soul among them and
command him to recite its verses before them, then, if they immediately
acknowledge and confirm it, the decree of faith is established for them.
But if not, then the decree of faith is lifted from them. And how, then,
could they attain to circumambulation, which is but one aspect of their
religion?

And know that whatever all possess is but from the grace of the Tree of
Reality is what all possess, and nothing lesser than that. For example,
if it had been decreed that these Manifestations should not inherit, who
would have had the right to question why or how? This is because all
comes from Him, and yet all remain veiled from Him---from the origin of
existence to its final destination. He bestows His remembrance upon them
so that, on the day of His Manifestation, they might attain the fruit of
their existence, which is faith in Him. Yet, they show no shame and, in
every manifestation, those who are veiled continue to veil themselves.

Even though it has been stated in the Qur'an that the fruit of the
creation of all things is mentioned in the verse:

``It is God who raises the heavens without pillars that you see, then He
established Himself upon the Throne, and He subjected the sun and the
moon---each running for an appointed term. He directs the affair and
details the signs so that you may attain certainty in the meeting with
your Lord.''

If, in this manifestation, one attains certainty that this Manifestation
is the same as the Manifestation of the Point of the Furqān (Muhammad),
then perhaps he has already attained to the meeting with the Messenger
of God, which is the meeting with God. And yet, the cause of certainty
in both instances is the same---rather, in this manifestation, it is
even stronger.

The Qur'an, which was revealed over twenty-three years in that
manifestation, has in this one appeared in less than a single week. And
yet, despite witnessing the fruit of the creation of all things, all
continue to act according to the laws of the Qur'an, while remaining
veiled from the very fruit of their own existence.

Had, on the Day of Manifestation, all believers in the Qur'an attained
certainty through a single verse from the Bayán, they would have
realized that its reality is the very same as that upon which the Qur'an
was revealed in the early days of Islam. Perhaps then, they would have
attained the will of God as expressed in that verse.

Yet, it is clear that all recite the Qur'an without comprehension,
heedless of the divine intent. While the Manifestation of Reality is
utterly detached from the sight of those who do not recognize Him, He
does not recognize his meeting with Him as the meeting with God. If all
believers in the Bayán were to attain certainty in the manifestation of
``He Whom God shall make manifest,'' recognizing that He is the very
Point of the Bayán, then perhaps they would have attained the will of
God in the Bayán.

And since, on that day, all believe in the Point of the Bayán, for this
reason, He is mentioned as His likeness. Otherwise, exalted is His
station---the form of the manifestation of the Hereafter is not
befitting to be identified with the form of the manifestation of this
world. Even though the One who appears in both is the same, the
realities of the Hereafter are more exalted and greater. Therefore, He
makes mention of Himself in this way.

However, since all remain dependent on the previous manifestation, and
their own perception is not sharp enough to recognize the next
manifestation with certainty, He refers to Himself by the name of the
previous manifestation. This is so that even a soul covered in the
densest veils might recognize Him. Just as on the Day of the
Manifestation of the Messenger of God, had all believers in Christ
attained certainty that He was the same, they would have indeed
fulfilled the divine intent of the Gospel. Yet He was, in truth, greater
than that manifestation.

Nevertheless, to establish certainty among the people of the previous
manifestation, He mentions the name of the origin of the previous
appearance so that they might pass along that path and attain the fruit
of their existence.

\emph{``So reflect upon what God has created for you, and let all your
deeds be for God, your Lord, so that on the Day of His Manifestation,
you may believe in His signs. That is the Day of `He Whom God shall make
manifest'---if you believe in Him, then indeed you have believed in God
and in what God has revealed in the Bayán. But if not, then you have
become veiled from the meeting with God and from that which God revealed
before in the Bayán. And do not show patience in this, for your patience
would be upon the fire itself, while on that day they will not perceive
it.''}

They do not perceive it, nor do they show patience, yet you do not
understand. Why then do you not purify yourselves for God? Why do you
not attain certainty? So purify yourselves for God, your Lord, so that
you may render yourselves receptive to He Whom God shall make manifest,
who has appeared with truth over all the worlds.

Be watchful at the first moment of His manifestation---do not hesitate
in the cause of God, and be among those who embrace the new command. For
this has been revealed upon you, if you were already certain in God and
His signs beforehand. You were created for this alone, and no command
was ever given to you except for this purpose. So do not veil yourselves
from the meeting with God---neither before nor after---so that you may
be among those who are certain in His signs.

\subsection*{Gate 3 (Seek Forgiveness From the Tree, Not Through
Another)}\label{gate-3-seek-forgiveness-from-the-tree-not-through-another}
\addcontentsline{toc}{subsection}{Gate 3 (Seek Forgiveness From the
Tree, Not Through Another)}

The Third Gate of the Eighth Unity: On the Day of Resurrection, when
after the manifestation of all things, nothing remains except His face,
it is ordained that every soul must seek forgiveness from the Tree of
Divinity personally and not through another---unless there exists a true
excuse preventing one from being physically present. In such a case, one
may seek forgiveness either by sending a message through speech or
writing. If this is not possible, then let him not seek forgiveness
through any other means, whether through written supplications or
otherwise.

The essence of this decree is that the Day of Resurrection shall surely
come, and the Tree of Reality shall be made manifest. Although it may
not be pleasing to the hearts of the believers in the Bayán to hear this
decree, the statement that was previously revealed---\emph{``All things
perish except His face''}---was uttered so that all might be vigilant.

On that day, let none allow veils to come between them and their
Beloved, such that their affirmation turns to negation. Rather, if there
is any possibility of negation, let it be transformed into affirmation.
Yet, since these people remain bewildered and forever heedless, their
movement has never been guided by insight.

Thus, if He appears and establishes the Resurrection, and if He reveals
such a decree---more intense than any fire---for the people of the
Bayán, it is only because they were performing deeds in the hope of
salvation, but once God has decreed destruction, what salvation remains?
It is incumbent upon them, therefore, to be present after having
recognized the revelation of this verse and its decree, to seek
forgiveness before Him and turn toward Him, so that He may transform
destruction into salvation. This is the true fruit of engaging in deeds.

If, after hearing this decree, one performs all manner of righteous
acts, he will still be counted among those who perish---unless he
returns to that same decree by which he was deemed to be among the
destroyed, and thereby attains salvation. If you look under the shadow
of this word ``perishment,'' you will see that all beings and all deeds
are brought to an end by God's command. If, on that day, someone had
wished to act according to this verse, not a single soul would have
remained upon the earth, for what benefit is there in the continued
existence of one whom God has decreed to be among the perished?

And yet, the severity of this decree before God and the people of
knowledge is far greater than any visible consequence. However, within
this world, no one can truly comprehend such severity. You might spend
your entire life performing deeds for salvation, yet in a single
instant, you may find yourself cast into the shadow of
destruction---something that cannot be rectified except by turning back
to the origin of the command.

If, after the revelation of this verse, you were to live as long as the
world endures and seek forgiveness continuously, it would bear no
fruit---unless you return and take a single word from the origin, even
if it is nothing more than `We have saved you.' That alone would grant
you salvation. Yet, no amount of seeking forgiveness would avail you
without it.

It is therefore obligatory that, upon hearing this decree, you sever
yourself from all deeds, for all were performed in pursuit of salvation.
But once the decree of God has changed to something else, what purpose
remains? Return, then, and take hold of the word of salvation.

Even if it be but a single word, it is still the path. If you were to
give away all that is upon the earth in exchange for obtaining such a
word, it would be more beneficial for you than spending all your wealth
in the path of God. This is because, if you give in charity, it will not
grant you salvation, but if you obtain this word, it will save you until
the next Day of Resurrection.

And if you are unable to obtain it immediately, then hasten with all
that you are capable of, seek it, and take hold of it. For if you delay
even for a single moment after hearing it, and in that moment death
overtakes you, you will be counted among those who perish. But if you
hasten and take the necessary steps to acquire that word, then should
death come upon you, perhaps the decree will flow from the Origin,
declaring that after your passing, you shall be in paradise and that
your fire shall be transformed into light.

This is closer to every matter that might bring one into acceptance, for
with this, all deeds become accepted, and evil deeds are turned into
good. If the fragrance of a new revelation is breathed forth from that
source, then becoming a new believer is better than perishing among
those who are lost.

No decree has been imposed upon the people of the Bayán greater than
this, for their salvation lies not in their deeds, but in this alone.
Let not laws, places, or limits veil you from the Origin---just as it
may happen that you might be among the most learned in the Bayán,
sitting comfortably in your house, while the One who bears the command
has already appeared, and yet you remain heedless and deceived, failing
to recognize Him until His station has changed.

Consider how, in Máku---a place of humiliation---He may outwardly
manifest, yet this does not alter the decree of God. Reflect upon His
station in this land, just as you would consider the stance of the
Messenger of God upon the mountain of Mecca, when the verse was revealed
in the Qur'an, and all the people of that time were counted among those
who perished. They entered into destruction before God, before His
Messenger, before those endowed with knowledge, and before every soul
that recited the Qur'an. They were mentioned as those who perished, and
the decree of being outside of faith was passed upon them. Yet, to this
day, they continue acting in their religion, believing themselves to be
righteous.

In the same way, the people of the Bayán, after the appearance of He
Whom God shall make manifest, will continue to act with the utmost
piety. However, after the revelation of this verse, what value remains
in their deeds? A servant may strive in the path of God, may even be
willing to sacrifice his life for that path in the hope of attaining
salvation---but once the decree of destruction has been issued from the
Origin of the Command, what benefit remains?

At best, their actions resemble those of the servants who, after the
revelation of the Bayán, continued acting upon the religion that
preceded it. And before that, they were like those today who still act
according to the Gospel, even though the Qur'an was revealed and its
decree replaced it.

Act with insight, so that you may not perish on the Day of
Resurrection---or if you do, that you may yet be saved. For as long as
the Tree of Reality remains ascendant, all things are possible. But once
it has set, no change or alteration can take place except through
another rising.

For example, in the manifestation of the Furqān (the Qur'an), if
someone, after the revelation of this verse, had sought salvation from
the Messenger of God, there is no doubt that the Source of Generosity
would have, by God's permission, revealed the decree:

``Indeed, We have saved you after having destroyed you---by grace from
Us, for We are ever-giving.''

And by this decree, they would have remained in salvation until
today---until this very day, when the origin of the manifestation of the
Bayán stands before God, His Messenger, the Imáms of guidance, the
Gates, and all believers. Indeed, even before the angels of the heavens,
the earth, and all that is between them---indeed, before every thing.
Yet even then, the grace of the Origin was not diminished in the
slightest, nor was any thing reduced. Rather, it was that very soul who
was deprived.

In the same way, reflect upon the manifestation of He Whom God shall
make manifest and be vigilant in observing the minutes and hours of the
Day of Resurrection. Keep account from the moment of His appearance
until His setting, with even greater precision than you count the years
of your own life among the people of this world. The fruit of this is
that you may spend in the path of God and attain salvation. But if you
are veiled from this reckoning and all things are rendered as
nothingness, then what benefit is there in religious calculations
concerning yourself or worldly calculations concerning the people of
God?

By the sacred Essence that has ever been---there is no soul among
creation who, upon hearing this decree, would not have melted away in
awe, such that the very thought of life would never again enter their
heart. For what greater test is there than to strive one's entire life
for salvation, only for the decree to descend at the end from the Origin
of the Command, proclaiming a ruling that is absolute in its issuance
and comes from none but God? This decree makes manifest the utter
incapacity of all.

And know that the number of ``Face'' (وجه) is the same as the number of
the name ``One'' (واحد). Whoever, in the Qur'an, believed in Muhammad,
the Messenger of God, and in the letters of ``Living'' (حيّ), was not
among those who perished---up until the manifestation of the Bayán. And
whoever entered beneath the shadow of the letters of ``One'' (واحد) in
the Bayán was not among those who perished---up until the appearance of
He Whom God shall make manifest.

And so it continues, from one manifestation to another, infinitely unto
the infinite. The servants who truly dwelled beneath the shadow of the
letters of ``Face'' in the Qur'an---there is no doubt that on the Day of
Resurrection, they attained salvation by virtue of their love. And that
love is the same as the love for the letters of ``One'' in the Bayán.

Likewise, if someone in the Bayán truly finds themselves under the
shadow of the letters of ``Face'' in that Dispensation, then on the Day
of Resurrection, he shall attain salvation. This is because, in that
manifestation, he does not deviate from the Manifestation Himself nor
from the letters of ``Living'' (حيّ) that proceed from Him. If he were to
deviate, it would be a sign that he had never truly been steadfast in
the letters of ``Face'' (وجه) of the Bayán.

This is the mystery that has been flowing from the origin of creation
until today and continues from today into eternity, without end.

Be vigilant in every manifestation, for the Manifestation of God is not
like the manifestation of creation. And the Proof of God is such that
all who are upon the earth are powerless before even a likeness of
Him---until the blind ascend and attain the essence of reality, at which
point people are able to witness the appearance of the Manifestation. At
that moment, the radiance of that Sun of Reality becomes, in itself, the
proof of His manifestation.

It is then that the decree becomes manifest: ``Know God by God.'' For
until this day, the decree has always been ``Know God by His Proof.''
This does not mean that at the moment when ``Know God by God'' is
revealed, the Proof ceases to exist. Rather, one should not become
veiled from the Origin---for in that time, the blind ascend to such a
degree that the exalters of the Supreme Concourse recognize their
Beloved in every manifestation by His very being. Rather, they know the
Proof by means of Him---not Him by means of the Proof.

And know that the knowledge of God in the station of ``Know God by God''
is not established except through the decree ``Know the letters of `One'
(واحد) through the ranks of their mention, by what is revealed from the
First One.'' For all things have come into being only through the
multiplication of that First One.

``Will you not then be mindful?''

And the meaning of multiplication is His manifestation within Himself,
through Himself, in His own realm. It does not mean that the letters of
``Living'' (حيّ) become something other than themselves, or that the
letters of ``Living'' multiply into separate numbers. Rather, by the
guidance of the First One, all are guided. If you look at the final
existence, you will see nothing but the First Manifestation of existence
itself within His own station. Just as it is evident today that the
lowest of creation adheres to the religion of the Messenger of God, who
is the highest of creation, this is the intended meaning of the
multiplication of all numbers from the First One.

``And place your trust in God on the Day of Resurrection, so that you
may prosper.''

The meaning of this trust is not that one merely recites this verse or
weeps upon their prayer mat, saying, ``O God, I have placed my trust in
You---grant me salvation on the Day of Resurrection.'' Rather, on that
day, true trust is when the Tree of Reality appears, and one believes in
Him, attaining certainty in His signs. Only then has one truly placed
their trust in God, and only then does their supplication bear fruit.

For no prophet has ever been sent except that he has commanded his
people to place their trust in God. And there is no doubt that God is
true to His promise---whoever truly places their trust in Him shall be
saved from all that grieves them. Yet, what has caused these many
diverse nations upon the earth to remain outside of the truth, even
while each one believes themselves to be one who trusts in God?

So entrenched are they in this notion that within their own communities,
each person believes they have grasped the power of trusting in
God---whether by what is in the Book of God, or by the command of the
Messenger, or by the decree of the letters of ``Living'' (حيّ), or by
following the letters of ``One'' (واحد).

But observe how, in every age, all are ultimately referred back to the
Manifestation Himself. Just as today, in Islam, all truths originate
from the Messenger of God. And yet, for every station, there are
infinite stations beyond it, and for every explanation of trust,
thousands upon thousands of meanings can be given.

``So place your trust in God, and then in God's signs, so that you may
attain certainty.''

\subsection*{Gate 4 (Highest Station Belongs to the
Point)}\label{gate-4-highest-station-belongs-to-the-point}
\addcontentsline{toc}{subsection}{Gate 4 (Highest Station Belongs to the
Point)}

The Fourth Gate of the Eighth Unity: All things---its highest aspect
belongs to the Point, its middle to the letters of ``Living'' (حيّ), and
its lowest to creation.

The essence of this decree is if all were raised upon the nature of the
signs and the greatness of the Proof before God and before those endowed
with understanding, then, on the Day of Manifestation, there would be no
separation between their hearing and their faith in the Truth. This is
the essence of all knowledge, for without it, even if one were to
possess all knowledge, they would become nothing, and the decree of
being outside of faith would be passed upon them. Yet, if they possessed
no knowledge except this essence of all knowledge, they would attain
salvation.

This is because, in every manifestation, it is witnessed that the
highest of creation becomes the lowest, and the lowest becomes the
highest---or that the highest becomes even higher, and the lowest
becomes even lower. Whoever comprehends the Day of Resurrection will be
able to enumerate the ranks of the believers according to the precedence
of their faith.

By those manifestations, the station of each believing soul can be
understood---in what chain of ``One'' they are situated. For example, if
a soul attains faith in He Whom God shall make manifest after three
hundred and sixty people, then that soul is the last ``One'' in that
specific sequence of unity. This chain of decree continues in the same
manner. Thus, comprehend the boundaries of all numbers, for every thing
that cannot be equitably paired belongs to the First One. And so it
continues, degree by degree, until it reaches all numbers. This is the
merit of every being at the origin of existence.

If, on the Day of the Manifestation of He Whom God shall make manifest,
there were three hundred and sixty-one pieces of diamond in His
possession, each increasing in value over the other by ninety-five
mithqáls of gold, and if three hundred and sixty-one people attained
faith in Him on the same day, with only the span of saying ``Yes''
separating their acceptance, and if He so willed, He could grant those
numbers in exact proportion, each according to their rank. He bestows
these numbers accordingly, and in the same manner, let the mystery of
God be enacted in all matters. Make the decree of God manifest so that
on the Day of Resurrection, you may take precedence by affirming the
truth at the moment of ``Am I not your Lord?'' with the response,
``Yes!''---and in every moment of existence, by responding to its
appearance.

Be attentive, for everything has its own moment of affirmation. For
instance, if a single particle of clay is lifted and it is declared that
this clay is the existence of the first creation in its original form or
any other name is given to it---even if it is simply referred to as clay
and nothing beyond that---and if you fail to respond to it in that
instant, then you have failed to respond to God in that moment of its
affirmation, even if in your original essence you had already said
``Yes'' to ``Am I not your Lord?'' But in that specific instance, you
fall short of your true existence.

What difference is there between being commanded to prostrate before
Adam or to prostrate before a mere particle of clay? The true purpose is
obedience to His command, not the distinction between one or the other.
If, at that moment, you become veiled from the decree concerning a
single particle of clay, then you have remained veiled from the mystery
of existence itself.

And if you perform all acts of goodness, yet fail in that one command,
you will still be counted among those who did not prostrate before Adam
at the first moment. And if you say, ``I obey in all things''---even if
you have indeed obeyed in many instances---the decree of God still
descends upon you as it was revealed:

``Worship Me as I command, not as you desire.''

Even though it is impossible for the Tree of Reality to issue a decree
that the intellects cannot comprehend, or to command something whose
virtue cannot be understood by all, this serves only to make the servant
aware of the stations of divine command. For the same decree that once
commanded ``Believe in God and His signs'' could, at another moment,
decree ``Let no soul grieve''---even if it concerns the furthest extent
of existence.

If you truly observe the decree of God, why, then, does one become
veiled from a single command while remaining steadfast in another? This
is why vigilance over all divine commands has always been among the
aspects of piety---provided that one does not become veiled from the
origin of the command in every manifestation. Otherwise, there is no
doubt that in whatever manifestation you find yourself, you act
according to its laws if you are truly devoted to your religion. If not,
then your own disobedience serves as its own witness against you, and
``Sufficient is your own soul today as a reckoner against you.''

The fruit of this is that, on the Day of Resurrection, no soul should
remain veiled from the commands of He Whom God shall make manifest. For
if He commands over all existence, His command is the command of
God---yesterday, today, and forever. And whoever questions ``why'' or
``how'' in the matter of God's decree has indeed questioned God
Himself---whether in this moment or in nine times nine, or nineteen
times nineteen ages, when the worth of a thousand pieces of gold is no
more than a single mithqāl of silver in that manifestation.

``And you shall surely come to know the limits of your own selves. Then,
on the Day of Resurrection, you shall recognize what God has decreed for
you.''

\subsection*{Gate 5 (A Gift of Jewels for He Whom God Shall Make
Manifest)}\label{gate-5-a-gift-of-jewels-for-he-whom-god-shall-make-manifest}
\addcontentsline{toc}{subsection}{Gate 5 (A Gift of Jewels for He Whom
God Shall Make Manifest)}

\subsection*{The Fifth Gate of the Eighth
Unity}\label{the-fifth-gate-of-the-eighth-unity}
\addcontentsline{toc}{subsection}{The Fifth Gate of the Eighth Unity}

It has been ordained that whoever has the means should take three pieces
of diamond---a number corresponding to the basmala, and four pieces of
yellow topaz---a number corresponding to the name of God and His law,
and eight pieces of emerald---a number corresponding to the Most
Forbidden, and six pieces of ruby---a number corresponding to the Most
Holy. He should take them and present them to He Whom God shall make
manifest and to the letters of ``Living'' (ḥayy) on the day of their
appearance.

The essence of this decree is that it has been established in its
rightful place that all existence is encompassed within the Bayán, the
entirety of the Bayán is contained within the First One (Wāḥid), and the
First One is drawn from the First Point (Nuqṭa). Since, on the Day of
Resurrection, all beings are gathered according to the degrees of
``One,'' and every stage of this ``One'' is determined by the decree of
the First One, it follows that within each rank, the water of life is
present---the divine decree itself, which sustains all things. Moreover,
within each realm, nothing reaches completion unless it conforms to the
station of this ``One.'' Only then does it become a true sign of God,
manifesting His decree and purpose.

Thus, it has been ordained that, on the Day of Manifestation---until the
appearance of He Whom God shall make manifest---every soul who possesses
the means should present before Him three pieces of diamond, four pieces
of yellow topaz, six pieces of green emerald, and six pieces of red
ruby, thereby attaining a resemblance to the First One (Wāḥid).

And if they are able, they should bring these gifts under the dominion
of the First One, but if not, then in the manifestation of He Whom God
shall make manifest, they should offer them by His command to His
letters of ``Living'' (ḥayy). This is a gift granted from God for the
First One in that manifestation, and the value of all numbers should
correspond to the value of the First One, so that those who seek
understanding may not become veiled from the mystery of divine unity.

And if in that manifestation, all take pride in obeying this command,
the multiplied ``One'' shall grant them possession over all existence.
Whoever attains His presence, and if the Tree of Reality shines forth
upon him and decrees a matter, then his patience in bearing that decree
shall be in proportion to the same fire.

It is therefore necessary for those who reflect upon the unity of
essence, attributes, actions, worship, creation, sustenance, death,
life, glorification, praise, divine unity, exaltation, fire, air, water,
earth, heart, spirit, soul, body, and the lights of white, yellow,
green, and red---to consider all of these within the letters of ``In the
Name of God, the Most Forbidden, the Most Holy.''

And also the four supplications that have been mentioned for the four
lights, which are greater than all other supplications. Every soul who
is enabled to recite them shall attain the good of this world and the
next.

\emph{``God bestows His grace upon whom He wills among His servants, and
He shall surely grant whom He wills from His dominion what has been
decreed from His presence, for He is ever-generous, encompassing all
things.''}

\subsection*{Gate 6 (Refinement and
Cleanliness)}\label{gate-6-refinement-and-cleanliness}
\addcontentsline{toc}{subsection}{Gate 6 (Refinement and Cleanliness)}

The Sixth Gate of the Eighth Unity: It has been permitted that, every
four days, purification should be observed by entering a heated bath and
removing all bodily hair using depilatory substances. This should be
done every eight or fourteen days. Trimming the nails, applying henna
over the entire body, and inscribing the word ``ar-Raḥmān'' (The Most
Merciful) upon the chests of men and ``Allāhumma'' (O God) upon the
chests of women has also been allowed. Additionally, it is prescribed to
look into a mirror every day and night.

The essence of this decree is that in the Bayán, permission has been
granted for refinement and cleanliness to the highest degree possible.
It is beloved before God that every four days, one should trim their
nails, remove excess hair, and engage in any act of purification
suitable to them. The more garments are cleansed and refined, the more
beloved they become. The application of henna, whether on the entire
body or partially, has also been permitted.

If, upon the chest---the seat of the love of God---one inscribes the
word ``Allāhumma'' in the case of the possessors of circles and
``ar-Raḥmān'' in the case of the possessors of temples, in the finest
script and by means that allow it to leave an imprint, this is beloved
and has been permitted. Additional inscriptions beyond these two words
have also been allowed.

Regarding depilatory substances, if one has the habit of washing the
entire body, the letter ``ḥāʾ'' (ḥ) has been mentioned as a sign for
purification. However, if one is not accustomed to using them
specifically for the hands and feet, it is not beloved for them. Yet,
its use over the entire body has been permitted, for no hair is removed
except that seventy types of ailments are lifted from the body by the
permission of God.

It is also fitting for a servant to gaze into a mirror and witness their
own form, giving thanks to their Beloved for the beauty they have been
given. Otherwise, let them seek forgiveness from their Beloved, for all
beings were originally created in the most excellent form. If any veil
should arise to obscure this beauty, then it is the image of contingent
existence that becomes imprinted upon their outward form. And this is
why spending the night in negligence has been forbidden. The origin of
this decree has not been explicitly mentioned in the Book of God, yet in
every condition, a servant must remain in such a state that, should the
Tree of Reality appear at any moment and should they stand before the
presence of God, they would not feel even the slightest aversion in
their own soul from one thing to another.

For nothing has been ordained except for the sake of the Day of the
Manifestation of He Whom God shall make manifest, so that if a soul
stands before God, there should be no condition within them except
refinement---that nothing may be witnessed in them except the love of
God.

In the Bayán, it has been forbidden to engage in anything that would
serve as a veil to refinement. Even if a soul comes to recognize a
single speck of impurity in their own body or detects a flaw in their
garments, and if they find it repugnant, then it is unworthy for them to
stand before God in that state.

Thus, all have been raised in the Bayán with this discipline, so that on
the Day of the Manifestation of Reality, nothing that is displeasing may
be found among the believers themselves. The blind ascend, step by step,
gradually attaining higher stations. If one wears a garment and it
becomes soaked in sweat, they replace it---how then could they bear to
remain in it? Yet still, the people of this manifestation have not
reached that degree.

If the refinement of bodies also extends to refinement in relation to
God, then the education of souls in refinement will be perfected, and
the sooner it is cultivated, the more beloved it is in the sight of God.
But if a soul is unable to attain this, let them not be sorrowful, for
their love of refinement itself is accepted in place of the act.
However, it is an obligation upon every soul to be vigilant in
manifesting whatever is within their ability in themselves.

For the Tree of Reality, within its hidden depths, witnesses all
creation, knows all things, and perceives all. They see Him, yet none
recognize Him and yet, with their very eyes, they do not see Him, for
that eye itself is created only after His manifestation. This applies to
all things, including the house in which He dwells---where neither
father nor mother recognizes Him, nor do His family, nor any among
creation.

Consider the example of the Manifestation of the Messenger of God before
His mission and the appearance of the Point of the Bayán before His
declaration. Yet, even then, His knowledge encompassed His own soul and
the creation of God, all of whom move under His shadow. From the moment
of His manifestation, His knowledge of Himself is complete, extending to
the farthest degree, where the Spirit becomes attached to another
Throne.

He perceives all, and all perform their deeds in His name, from the
beginning of existence until its end. All things stand by His essence,
yet none recognize Him---except those whom He enables to recognize Him,
through His own Self and His signs.

He does not make Himself known except at the moment of His manifestation
when the decree of knowledge is made binding upon all. Then, the signs
of majesty shine forth from the sacred horizon of His being.

\emph{``Blessed is the one who attains the meeting with God on the day
of His manifestation, standing in a state wherein nothing is witnessed
upon him except what is beloved by God. Indeed, He is ever subtle,
infinitely subtle.''}

\emph{``Say: God is the most subtle above all that is subtle. If He
willed to withhold the manifestation of His grace, He would not be able
to, whether in the heavens, the earth, or between them. Indeed, He is
the most gracious, the most mercifully subtle.''}

\subsection*{Gate 7 (Printing The Bayan Is
Permitted)}\label{gate-7-printing-the-bayan-is-permitted}
\addcontentsline{toc}{subsection}{Gate 7 (Printing The Bayan Is
Permitted)}

The Seventh Gate of the Eighth Unity: Regarding printing and the decree
concerning it.

The essence of this decree is that printing has been permitted in the
Bayán, and all that is composed within its shadow must conform to it,
until the manifestation of He Whom God shall make manifest.

At that time, if all matters have been decreed in a certain manner, they
may preserve the words of God in the finest script, according to His
command. Otherwise, all shall be subject to the outpouring of His
generosity and grace. He shall grant permission, and after this decree,
no soul shall have any excuse before God for not possessing a copy of
the Bayán by which they may be reminded of He Whom God shall make
manifest.

It is enjoined that the Bayán be transcribed in the finest script---not
in the manner common in this age, where anything written is hastily
printed, such that the value of a gifted Qur'an has diminished to
twenty-eight grains of silver. If it were not for consideration of the
inability of all believers to afford otherwise, printing would not have
been permitted.

Yet, now that all dwell within the grace of the existence of Truth by
His permission, whoever is able to transcribe the Bayán in the most
elegant script should do so, for this is preferable---until such a time
that proper printing of it may be obtained.

\emph{``This is from the grace of God; He bestows it upon whomsoever He
wills among His servants, and God is the possessor of infinite grace.''}

Know that the respect due to the Qur'an is solely by virtue of its
attribution to God Himself and on account of the spirits contained
within its words. All spirits ultimately return to the spirits of the
letters of ``Living'' (ḥayy)---just as their beginning was from those
very letters.

For instance, today, if a believer acts according to the words of the
People of the House and the Four Gates during the Minor Occultation,
then after that time, whatever has been transmitted from them could not
be altered or changed. In truth, all the letters of ``Living'' (ḥayy)
return to the Point of Furqān---which is the Messenger of God---and He,
in turn, returns to God Himself. And His return to God is as that which
returns unto its own self, for existence cannot transcend beyond its own
limits.

Likewise, in the Bayán, all exalted spirits return to the First Gate,
which is He Whom God shall make manifest. All spirits, except for the
exalted ones, return to the first of those beneath the Truth---that is,
to the one who refuses to prostrate before Him. Likewise, observe all
heavenly scriptures, for their sanctity is solely due to their
attribution to God. The Word has ever been alive, even as past
manifestations fade and are linked to those that follow.

There is no doubt that the Gospel was once the Book of God, yet after
the revelation of the Qur'an, its true spirits returned to the Qur'an.
What remained were those spirits that had not ascended to the exalted
realms, leaving the lower aspects of the Gospel to persist.

Similarly, in the Bayán, those believers in the Qur'an who enter into
the Bayán have their spirits raised to the exalted realms, while those
who do not remain among the lower ones. Likewise, in the Bayán, those
who believe in He Whom God shall make manifest will have their spirits
counted among the exalted ones.

Yet, if a soul is found that does not prostrate before Him, then it
belongs entirely to the lower realms, beneath the exalted ones.

\emph{``So act upon that which will guide you in the finest script you
are able to attain.''}

\subsection*{Gate 8 (Styling of Hair, Praying in a
Cloak)}\label{gate-8-styling-of-hair-praying-in-a-cloak}
\addcontentsline{toc}{subsection}{Gate 8 (Styling of Hair, Praying in a
Cloak)}

The Eighth Gate of the Eighth Unity: Concerning the permissibility of
styling the hair on the head for boys and removing facial hair for
strength, as well as praying in a cloak but not in a robe, for God does
not love it.

The essence of this decree is that it has been permitted to trim and
style the hair on the head and to remove facial hair to enhance
strength, so that one's appearance may be refined.

The removal of the mustache has been commanded in all circumstances.
Those who remain veiled from this command are among the servants who
have veiled themselves from the decree of God.

Permission has also been granted for prayer in a cloak, provided that
both hands remain concealed, except for the fingertips, which should be
visible---this is considered more dignified.

Praying in a robe is neither beloved nor permitted, except in cases of
necessity.

All of these decrees have been permitted so that, on the Day of
Resurrection, nothing may be present within or without a soul except the
love of God, allowing those who seek proof to recognize that, if even an
outward body should not bear anything apart from His love, then how much
greater is the decree concerning essential bodies, souls, spirits, and
hearts, which are the very places of the manifestation of the First One.

\emph{``So be mindful of God, O you who are most mindful, on the Day of
Resurrection, so that you may prosper.''}

\subsection*{Gate 9 (Record All of Your
Deeds)}\label{gate-9-record-all-of-your-deeds}
\addcontentsline{toc}{subsection}{Gate 9 (Record All of Your Deeds)}

The Ninth Gate of the Eighth Unity: Every soul must write down their
name and all they have done, whether good or otherwise, from the
beginning of the manifestation until the day of its setting. This record
should be preserved by the successors until the day when God causes the
Tree to appear.

The essence of this decree is that, from the beginning of a
manifestation until the next manifestation, every soul has been given
permission to record in their own book---either by their own hand or
through another---their deeds in the Bayán, including anything they have
acquired therein. Similarly, if they have engaged in anything other than
good, they should record it, so that, on the Day of Resurrection, those
who have acted in accordance with the previous manifestation may be
judged---provided they do not become veiled from the Sun of Reality.

Otherwise, it is possible that the Manifestation will appear while they
continue to act as they did before, just as the Qur'an was revealed
twelve hundred and seventy years ago, yet to this day, there are those
who still act according to the Gospel.

Thus, in every manifestation, observe the decree of God, and do not be
veiled, for the Day of Resurrection is a day like today---the Sun rises
and sets.

It has often been the case that, when the Resurrection is established
upon a land, even the people of that very land are unaware. For even if
they hear of it, they do not believe---and so, it is not declared to
them during the Manifestation of the Messenger of God, since the people
were unable to bear it, the declaration of the Resurrection was not made
to anyone except the believers. Yet, it is a day of immense magnitude,
for the Tree of Reality, whose voice has eternally proclaimed, ``Indeed,
I am God; there is no god but Me,'' appears once more.

And yet, all those who are veiled assume that He is nothing more than an
ordinary soul like themselves. The name of a believer, which in His
dominion extends infinitely even to the least of the believers from the
previous manifestation, is now used to reject Him.

Just as in the Manifestation of the Messenger of God, had they regarded
Him merely as one of the believers of their time, how could they have
confined Him for seven years in the mountains, placing a barrier between
Him and His house? Likewise, in the Manifestation of the Point of the
Bayán, had they not sought to deny that Name, how could they have
confined Him to the mountain?

Yet, faith itself is created by His very word. This is why those who
lack the eyes of the heart do not see, while those who possess them
circle the Lamp of Reality like moths until they are consumed. For this
reason, the Day of Resurrection is spoken of as greater than any other
day---even though it is, in essence, like all other days, except that on
this day, the Manifestation of God is revealed.

The essence of this decree is that, on that day, may the eyes be open to
behold their Beloved. And on that day, one may record whatever they
acquire, for it shall remain binding until the next Resurrection.

As for what the Point of the Bayán must write, it is this: ``That which
came before has been raised unto this new revelation, and God has sent
it down upon Me.'' This is the record of what He has acquired in the
Resurrection. And all shall be judged by the essence of that decree.
They should write down whatever they acquire, so that their mention may
be recorded on the Day of Resurrection before God. If they are able, on
that day, they should enter the fire of divine unity. Otherwise, they
will weep day and night, for their Beloved will appear, and yet they
will not recognize Him.

This is like those among the Christians who have long awaited the
Promised Ahmad, and who continue to supplicate for His appearance, even
though twelve hundred and seventy years have passed since His
manifestation.

Be attentive with the eyes of your heart so that you do not remain
veiled on that day, lest the Resurrection be established and you remain
unaware. It is upon God to inform you, but should you hear, then it is
indeed He Whom God shall make manifest who informs you. If you accept,
you will have placed your trust in God, your Lord, the Most Merciful.

\emph{``Then, whatever good you acquire or otherwise, from one
Resurrection to the next, write it down.''}

\subsection*{Gate 10 (Men and Women May Have
Conversations)}\label{gate-10-men-and-women-may-have-conversations}
\addcontentsline{toc}{subsection}{Gate 10 (Men and Women May Have
Conversations)}

The Tenth Gate of the Eighth Unity: Whoever has been raised within the
community is permitted to look and to speak, whether they are male or
female. Permission has also been granted for men to converse with women,
and for women to converse with men, to the extent that their words bear
fruit between them. However, they should not exceed twenty-eight words,
for that is better in piety.

The essence of this decree is that every soul raised within the
community is permitted to look and to speak---whether concerning the
form of the body or the structure of the circle.

In times of necessity, permission has been granted for a man to speak
with a woman to the extent that suffices to achieve a meaningful
outcome. If their conversation does not exceed twenty-eight words, it is
closer to piety. However, if more words are required to convey benefit,
then an increase has been allowed.

By ``community,'' what is meant is the recognized inner circle, not
merely the outward designation---just as a hundred thousand tents of a
single tribe may be referred to as one community so that on the Day of
Resurrection, the vastness of this decree may ensure that no soul is
veiled from receiving grace from the Origin.

And if the Tree of Love commands either an increase or a decrease, after
its appearance, no one should question ``why'' or ``how.''

\emph{``So be mindful of God, with the mindfulness due to Him, that you
may attain success.''}

\subsection*{Gate 11 (Washing of the
Deceased)}\label{gate-11-washing-of-the-deceased}
\addcontentsline{toc}{subsection}{Gate 11 (Washing of the Deceased)}

The Eleventh Gate of the Eighth Unity: Concerning the washing of the
deceased, it is to be performed three times in the following order:

First, the head is washed while saying, ``O Singular One.''\\
Then, the abdomen while saying, ``O Living One.''\\
Then, the right side while saying, ``O Self-Subsisting One.''\\
Then, the left side while saying, ``O Wise One.''\\
Then, the right foot while saying, ``O Just One.''\\
Then, the left foot while saying, ``O Powerful One.''

This should be done using water, or water mixed with camphor and lotus
leaves, as desired.

The body should then be shrouded in five layers of cloth, and a ring
should be placed on the right hand.

For men, the inscription on the ring should be:\\
\emph{``To God belongs whatever is in the heavens and the earth and what
is between them, and God is All-Knowing of all things.''}

For women, the inscription should be:\\
\emph{``To God belongs the dominion of the heavens and the earth and
what is between them, and God is over all things Powerful.''}

The essence of this decree is that all the laws of the Bayán have been
established upon the mysteries of divine unity and knowledge. Whoever
looks from the beginning to the end of this revelation will see the
water of divine oneness flowing in a single course throughout all
things.

It has been made obligatory to wash the deceased once, though permission
has been granted for up to three or even five times, as the degrees of
divine unity are mentioned in five stages:

\emph{``There is no god but He.''}\\
\emph{``There is no god but I.''}\\
\emph{``There is no god but God.''}\\
\emph{``There is no god but You.''}\\
\emph{``There is no god but the One Who\ldots{}''}

A soul who, in the first year of the manifestation, attains faith in the
Revelation of Divine Oneness, will by the fifth year have reached the
final stage of divine unity.

For this reason it has been made obligatory to perform the washing once,
but if there is no difficulty, the four additional stages may be
observed. It has also been permitted to wash the head, abdomen, hands,
and feet while offering praise and glorification to God. The body should
be washed according to what was suitable for it in life, whether with
warm or cool water, as required.

The washing should be performed by the hands of the righteous. After it
is completed, if possible, the body should be bathed in pure water or
perfumed with fresh and fragrant scents.

For the shroud, five layers of cloth are permitted, ranging from silk to
the finest cotton. No more than nineteen names may be inscribed upon it.
Whoever wishes may choose which names to inscribe.

A small amount of earth from the first and last graves should be buried
with the deceased so that they may not experience sorrow after death,
and they may find delight in the paradise of God with that which they
have loved and continue to love.

A ring is permitted to be placed on the right hand, engraved with the
following:

For men:\\
\emph{``To God belongs whatever is in the heavens and the earth and what
is between them, and God is All-Knowing of all things.''}

For women:\\
\emph{``To God belongs the dominion of the heavens and the earth and
what is between them, and God is over all things Powerful.''}

The handling of the deceased should always be conducted with dignity and
stillness, so that no action contrary to reverence may take place. For
the respect of the body of a believer is the respect of the believer
themselves.

The six names of God or simply the name of God alone should be repeated
constantly, either in the heart or upon the tongue, from the moment of
transition until its end.

And know that death is like life. If a soul departs this world with
faith in the Manifestation, they shall experience delight in the
paradise of God. Otherwise, they shall abide in the fire.

So be mindful, lest, on the Day of Resurrection, a soul should die and
enter the fire without ever realizing it---just as from the Day of the
Manifestation of the Messenger of God until today, every soul that has
departed without faith in Him has not entered paradise. Likewise, from
the beginning of the Bayán, every soul that has passed away has been
taken by God Himself, through the angels entrusted with this task.

If they have obeyed the decree revealed in the Bayán, they are granted
entry into paradise. But if a soul departs without faith in the Bayán,
then even if they had performed the deeds of both worlds combined, it
would not benefit them. And if, after their death, all manner of
charitable acts were performed on their behalf, it would still not avail
them--- unless they had already believed in God and His signs and obeyed
their Beloved in what had been revealed in the Bayán.

Only then might they be encompassed by God's mercy and abide eternally
in the paradise of creation.

Be vigilant concerning the Manifestation of He Whom God shall make
manifest. For if you hear of His appearance and hesitate even for the
time it takes to say ``Yes,'' then whether you are alive or dead, you
are in the fire. This is the intent of God in every manifestation for
every soul.

Be watchful, lest you be swept away by the blasts of the Day of
Resurrection. These blasts are the waves of sorrow that descend and the
unfathomable signs that appear concerning those who manifest the divine
revelation. Or they may be the breezes of paradise that blow from the
dawn of eternity upon the hearts and forms of all created things.

For if you look within this manifestation, you will perceive all things
and witness how the dead are visited in every cycle of unity. Additional
decrees concerning this matter have been revealed, which you will find
recorded in the Tablet:

\emph{``Say: It is God who takes your souls by His command, and He shall
command the angels to seize His servants.''}

Just as the angels seize the souls of the believers by the command of
God, so too do the demons seize the souls of those who do not believe.

Whoever ascends under the shadow of the Bayán has their soul taken by
the exalted angels, while whoever dies outside of the Bayán or beneath
it is not approached by those angels, for a believer does not touch
them--- how then could an angel come near?

And if the veil were lifted from the eyes of those who serve the demons,
they too would be unable to endure the intense heat of the fire. Yet,
because they remain veiled, they seize souls and deliver them into the
keepers of the fire.

Meanwhile, the souls of the believers are brought before God. As for the
souls of those beneath belief, they remain confined within their own
station---God does not look upon them. Rather, He commands the fire to
seize them.

So place your trust in God, lest death overtake you and you find that
all your worship has been in vain because it was not upon the right
path.

\emph{``God enjoins you concerning your own souls, and then concerning
all His servants, that you may be mindful.''}

\subsection*{Gate 12 (Visiting the Sacred Place of
Visitation)}\label{gate-12-visiting-the-sacred-place-of-visitation}
\addcontentsline{toc}{subsection}{Gate 12 (Visiting the Sacred Place of
Visitation)}

The Twelfth Gate of the Eighth Unity: Concerning the laws of the sacred
place of visitation.

The essence of this decree is that whoever resides within that land or
within sixty-six farsakhs (198 miles) of it, once they have passed the
age of twenty-nine, is required to enter that sacred place once every
year and remain there for nineteen days to purify themselves.

In that place, five cycles of prayer have been ordained.

Whoever does not have the means to travel there should perform the same
acts in their own home. This decree does not apply to those who dwell
outside this boundary. Had this command been made obligatory for all,
who could have turned away from the decree of God?

So behold the grace of the All-Merciful, and observe the limit placed
upon creation. God is aware of how much is spent on this path, and in
that day, there was not a single soul who would take even one step for
the sake of God. In the same way, when the Manifestation of He Whom God
shall make manifest appears, you will see the same condition---all will
act, yet remain veiled from the Origin of the Command.

All will prostrate before Him, yet they will be veiled from His very
self. Had you been content with remaining veiled, these decrees would
never have been imposed upon all. But because one became veiled, all of
creation was then bound by the divine limits.

However, if on the Day of the Manifestation of He Whom God shall make
manifest you resolve to have faith in God---which is faith in Him, to
obey Him---which is obedience to God, to love Him---which is love for
God, and to be content with Him---which is contentment with God, then no
decree shall be revealed that would bind all until the next
Resurrection. This is the power of God over whatever He wills and His
irresistible decree over whatever He desires.

Therefore, on the Day of Resurrection, be vigilant over your own selves,
lest you reject those veiled from this manifestation, while you
yourselves become even more veiled than they. Just as you now reject
those veiled from the Messenger of God, yet you have become more veiled
than they.

\emph{``So be mindful of God, with the mindfulness due to Him, and do
not be pleased for another what you would not be pleased with for
yourselves, lest on the Day of Resurrection you pronounce judgment
against God.''}

\subsection*{Gate 13 (95 Exaltations Over the
Point)}\label{gate-13-95-exaltations-over-the-point}
\addcontentsline{toc}{subsection}{Gate 13 (95 Exaltations Over the
Point)}

The Thirteenth Gate of the Eighth Unity: It has been permitted to recite
ninety-five exaltations over the Point, both at the beginning and at the
end.

The essence of this decree is that since the Tree of Reality is the
mirror of God and within Him nothing is seen except God, it has been
ordained that, upon ascending to His throne and transitioning from the
first throne, ninety-five words of exaltation should be spoken.

Beyond this, only five additional exaltations have been permitted for
the letters of ``Living'' (ḥayy) manifest from that One who is beyond
number, and all multiplications of numbers emerge one by one from the
First One.

So that on the Day of Resurrection, at the moment of the appearance of
the Tree of Reality, if you have acted for the sake of God, you will see
all the letters of ``Living'' (ḥayy) in that mirror---not as you see
them within themselves, but rather as a pure manifestation, through
which those manifestations reflect His essence.

Similarly, you may say with your tongue, ``I act for God.'' Yet, on that
day, it shall become clear whether you were truthful---if, at the moment
of the appearance of He Whom God shall make manifest, your deeds were
truly for Him, then indeed, they were for God. Otherwise, your claim was
false.

Every believer who is truly faithful on that day, and who has acted for
God, will, in turn, find their actions attributed to Him. But if not,
their deeds were not for God and will not be raised on their behalf.
Just as in the Manifestation of the Messenger of God, if a soul had been
a true scholar for God, they would have proclaimed faith in the
Messenger based on the letters of the Gospel. The fact that they did not
is a sign that they were never truly sincere.

Likewise, in the Manifestation of the Tree of the Bayán, if one acts for
God, they are the ones who follow Him, for their actions are for God.
Otherwise, they were never pure in their intention---for if they had
been, their deeds would not have been for anything other than God.

\emph{``So be mindful of God on the day of His manifestation, that you
may attain success.''}

\subsection*{Gate 14 (Recite 700 Verses From the Bayan Every Day and
Night)}\label{gate-14-recite-700-verses-from-the-bayan-every-day-and-night}
\addcontentsline{toc}{subsection}{Gate 14 (Recite 700 Verses From the
Bayan Every Day and Night)}

The Fourteenth Gate of the Eighth Unity: It has been permitted for every
soul to recite seven hundred verses from the Bayán every day and night.
If one is unable to do so, they should instead mention God seven hundred
times.

The essence of this decree is that since divine unity reaches its height
in the letters of ``Dhāl'' (ذ) and its ascent culminates there, its
mystery is that the number seven of ``Allāhumma'' (اللّهمّ) passes through
the third rank of divine unity in the letters and becomes manifest in
the fifth rank.

It has been permitted that whoever is able should recite seven hundred
verses from the Bayán every day and night. If they are unable, they
should instead say ``Allāh-u Aẓhar'' seven hundred times.

The fruit of this practice is that if it is the Day of Resurrection,
such a soul shall attain faith in He Whom God shall make manifest with
the very essence of their being. They shall become worthy to be a
reflection of the letters of ``Dhāl'' (ذ) and counted among their ranks.
And if they should surpass the limits of number, they shall perceive
nothing but the One beyond number.

This is not an easy matter---rather, it is the easiest of all things, if
you believe. Yet because that day is so great, it is also immensely
difficult to remain among the believers. For on that day, the believers
are the companions of paradise, while those beneath belief are the
companions of the fire.

Know for certain that paradise is the knowledge of He Whom God shall
make manifest and obedience to Him, while the fire is the existence of
those who did not prostrate before Him. For on that day, you may assume
you are among the companions of paradise, believing yourself to be a
believer in Him, yet you may be veiled---and in reality, your dwelling
is in the fire, without knowing it.

Consider His manifestation as the appearance of the Point of
Furqān---how many among the People of the Gospel had awaited Him! Yet,
after His manifestation, there were no companions of paradise for the
first five years except for Amír al-Mu'minín ('Alí), and whoever
attained faith in Him during that time.

Other than him, all were companions of the fire---while believing
themselves to be companions of paradise. Likewise, in this
manifestation, observe carefully---for until today, through the divine
decrees, the essence of creation has been set in motion, so that in the
end, three hundred and thirteen souls have been gathered in the land of
Ṣād---which outwardly appears to be the greatest of all lands, and in
every corner of its schools, there are countless servants renowned for
their knowledge and jurisprudence.

Yet when the pure essence of wheat is gathered, the one wearing the
cloak of leadership will be separated. This is the mystery behind the
words of the People of the House concerning the Manifestation:

\emph{``The lowest of creation shall become the highest, and the highest
shall become the lowest.''}

And so it shall be in the Manifestation of He Whom God shall make
manifest. Among the people, there will be those whose hearts are filled
with nothing but the good pleasure of God, and yet all others will
follow their example in outward piety. And many among the people of the
fire would have been saved, had they believed in Him.

Meanwhile, there will be souls whom no one regards, yet they shall be
granted the robe of faith from the Origin of Existence. For by His Word,
all things in creation come into being---from the highest point of
existence to its lowest. Just as in the Manifestation of the Messenger
of God, the successors were appointed by His Word alone.

So reflect: the One who bestows the robe of divine grace and
guardianship is the very One whose own creation denies Him even the
title of a believer, placing Him among the lowest of His own creation.
By the Sacred Essence, eternally beyond time, if all the people of the
Bayán were to believe in the Sun of Reality in the same way that the
first to believe in Him had believed, then He would clothe them in His
very Name---such that nothing could be seen in their inmost reality
except Him.

And if His Name is great, He would make it even greater and attribute it
to His own Self. And in His Book, He would reveal:

\emph{``God, there is no god but Him, the Most Great, the Most Great.''}

So behold, this is the ocean of His bounty, whereby He raises absolute
nothingness from the depths of annihilation to the sanctuary of His
presence, where nothing remains within the essence of the heart except
His Name. And if, outwardly, nothing is seen in His will except His own
Manifestation, this is the eternal outpouring of divine grace, the
ceaseless bounty of the Ever-Bestower. He clothes whomever He wills in
the robe of annihilation, and they see themselves as worshiping Him, yet
they remain veiled from Him---just as they are veiled from His
Manifestation.

Consider how the Messenger of God saw that all the believers in the
Gospel worshiped God and had faith in what He had revealed. Yet, because
He saw them veiled from His very Self, and because to be veiled from Him
is to be veiled from God, they were judged as belonging to that which is
beneath God. The same was true in the Manifestation of the Point of the
Bayán, and so it shall be in the Manifestation of He Whom God shall make
manifest. If, on that day, all were to turn toward the Origin of Proof,
toward the simple reality of the Divine Essence, not a single soul from
among the people of the Bayán would remain without accepting Him.

Just as in the Manifestation of the Point of Furqān, had all the
believers in the Qur'an truly believed from the beginning, not one soul
would have hesitated. At the moment of hearing the verses of God, they
would have passed over the Path faster than the blink of an eye. It is
not that any merit belongs to you, O people of the Bayán, for believing
in Him. Rather, if you fail to believe, you become as those beneath God.
So strive with all your being to believe in Him, so that you may become
for God alone and be transformed from fire into light. Otherwise, He is
independent of all that exists.

For just as if today, all on earth were to believe in the Bayán, they
would be saved from the fire and enter paradise---so too would they be
delivered from the station of being beneath God, which is more severe
than any fire. And they would enter the paradise of God Himself, which
is greater than any paradise. And they would be saved from the mention
of rejection and would dwell in the shadow of faith, they will be
admitted. Otherwise, the Point of Reality has ever been and shall ever
remain independent of all things, while all else has forever been in
need of Him, by virtue of their own contingent existence. If all on
earth had believed on the Day of the Manifestation of the Messenger of
God, they would have saved themselves from the fire. But since they did
not believe, they have remained in eternal fire. In every manifestation,
it is for souls to strive to deliver themselves from the fire of the
previous revelation. Otherwise, the Manifestation Himself is
self-sufficient.

There is nothing in existence except that, by its very nature, it is
prostrate before Him---for God, the Almighty and Glorious--- even if it
remains veiled and fails to believe on the Day of His Manifestation. For
if the veil were lifted, such a soul would recognize its faith in Him,
just as it had faith in His previous manifestation. O people of the
Bayán, do not do as the people of the Qur'an have done---worshiping God
but rejecting His Manifestation. For when this occurs, in a single
instant, all deeds become as deeds beneath God, and the one who performs
them does not realize it. This is how all nations have been veiled, and
in every manifestation, it is fitting that all should believe in the
Manifestation, for they are sustained by Him.

\emph{``Recite the Bayán upon the letters in the watches of the night
and the edges of the day, so that you may be drawn by the Name of God,
and then grieve over the Names of God.''}

\subsection*{Gate 15 (Obligatory to
Marry)}\label{gate-15-obligatory-to-marry}
\addcontentsline{toc}{subsection}{Gate 15 (Obligatory to Marry)}

The Fifteenth Gate of the Eighth Unity: It has been made obligatory upon
everyone to marry so that a letter of their own being may remain to
proclaim the unity of God, their Lord. They must strive to fulfill this
duty, and if anything should arise from either party that prevents this
purpose from being realized, then it is permitted for each to separate
with permission, so that the fruit may be brought forth elsewhere.

It is not permitted to marry one who has not entered into the religion.
And if one is already married to such a soul, it is incumbent upon them
to separate if they witness a lack of faith in the Bayán from their
spouse. However, no fault shall be attributed to them unless they refuse
to return to the Bayán before the decree of God is lifted. On the Day of
the Manifestation of He Whom God shall make manifest, permission shall
be given to the believing men and women so that they may return.

The essence of this decree is that, in this world, among the greatest
fruits granted by God---after faith in Him, in the letters of ``One''
(Wāḥid), and in what has been revealed in the Bayán--- is that one may
bring forth a child from their own existence who, after their death,
will continue their mention in goodness.

Marriage has been enjoined in the Bayán, and it is a firm obligation.
Even to the extent that, if an obstacle preventing procreation is
observed in one of the partners, separation has been permitted so that
the fruit of existence may be manifested elsewhere. For if a child is
born and attains faith in He Whom God shall make manifest, they become a
leaf from the leaves of paradise. Otherwise, they become a leaf from the
leaves of the fire. And if no child is conceived, it is better for them
not to exist than to exist without faith.

It is not lawful to marry except one who has attained faith in every
manifestation, according to the outward decree of that manifestation. If
either spouse refuses to accept faith, their marriage is not permitted.
Furthermore, one who does not believe is denied the rights of that
union, for God is the true possessor of all things, and He has not given
permission for any property to be transferred to one who does not
believe. Whatever possessions you see in the hands of non-believers
exist without true right. If one with divine authority were to arise,
they would be prevented from possessing even their own souls---unless
they attain faith. How then could they claim ownership over anything
else?

Until the Word of God is lifted, which marks the beginning of a new
manifestation, permission has been given to protect the lives of the
believers. However, once that decree is lifted, no permission shall be
granted, for paradise cannot be joined with the leaves of the fire, for
the very essence of one is negated by the other, just as the essence of
the fire seeks to negate existence, while the essence of paradise
affirms it. The one is absolute nothingness, while the other is pure
being, by the decree of God.

It is therefore fitting for all souls illuminated by the Bayán to bring
forth a child from their own existence, so that the ranks of numbers may
increase until they are immersed in the ocean of infinity. For at the
beginning of every manifestation, the numbers are limited, but they
gradually increase, rank by rank, until they reach the limitless.

Consider the case of twelve hundred and seventy years ago, when Muhammad
and Amír al-Mu'minín ('Alí) were the only believers in the Qur'an. But
look at today---can you even begin to count the numbers?Thus, existence
ascends from one to infinity, and there is no end to its progress.
Likewise, observe the beginning of the Manifestation of the Bayán. For
the first forty days, apart from the Letters of ``S'' (Sīn), there was
not a single believer in the Bayán. But gradually, the forms of the
Letters of ``Bismillah'' donned the robe of faith, until the First
``One'' (Wāḥid) was completed. Then observe how, until today, that One
has multiplied beyond measure.

By the Sacred Essence beyond time, had the means of manifestation been
revealed at the beginning of this revelation, then today, there would
not be a single soul on earth outside of faith. For the essence of His
Manifestation is the Fire of God. If all were to enter under His shadow,
they would be kindled in the fire of His love, reciting His praises,
affirming His oneness, and proclaiming His greatness--- without
diminishing His dominion in any way, nor adding anything to it.

For whatever is in the heavens, the earth, and all that lies between
them has always belonged to God---whether He chooses to manifest His
Reality openly, or whether all things proclaim His Name in His absence.
Step by step, you will behold until you witness, from a boundless limit
and an endless boundary, the creation of the Paradise of Origination.

Place your trust in God, and then, in the days of God, remain patient.

\subsection*{Gate 16 (19 of 100 Mithqals of Gold Belong to
God)}\label{gate-16-19-of-100-mithqals-of-gold-belong-to-god}
\addcontentsline{toc}{subsection}{Gate 16 (19 of 100 Mithqals of Gold
Belong to God)}

The sixteenth gate of the eighth Unity, concerning what is prescribed
for every soul from all that it possesses of one hundred mithqals of
gold: from the splendor of all things, nineteen and one belong to God.
If the sun is shining, entrust it to Him so that He may distribute it
among the Letters of the One, each receiving one mithqal, if He so
wills; otherwise, the matter is in His hand---He is not questioned about
what He does, while they shall be questioned. And if the sun is
concealed, and the Letters of the One have descendants, it is conveyed
to them; it may also be directed toward uniting two souls, even if a
person should give it to their child or their household. As for the
mithqal of Fire, it is kept for the One whom God shall make manifest, or
used in the Bayán, and he himself recites it and preserves it as though
it were his own eyes, so that it may be beheld by its companion.

The essence of this gate is that after something attains the splendor of
one hundred mithqals of gold, its owner must give nineteen mithqals to
the Letters of the One and one mithqal for the sake of Fire. If the Tree
of Reality has appeared, let him follow the command of the Lord; but if
the night has arisen, everything is conveyed to the descendants of those
Letters. If none remain, it is allocated to unite two souls, and the
mithqal of Fire is preserved so that it may be returned to the One whom
God shall make manifest. Upon His appearance, the ruling of uniting and
granting to descendants ceases, except by His leave. The outcome is that
if on that day He ordains a command similar to what is ordained today,
then all must obey it, just as they obey the Messenger of God.

In all commandments, it is the same. Obedience to the Tree of Reality in
every Dispensation, on the Day of its Manifestation, is more
potent---whether openly, or veiled in the night for those who recognize
Him---because that Day is the meeting with God. And if someone cannot
comprehend, then let it be until another Resurrection. It is proper that
after each prayer, a servant should ask mercy and forgiveness from God
for his parents; at that moment, a summons reaches out from God's
presence: ``For you is twice a thousandfold of what you asked for your
parents.'' Blessed is the one who remembers his parents with the
remembrance of his Lord---indeed, there is no God except Him, the
Mighty, the Beloved.

\subsection*{Gate 17 (95 of 6,005 Mithqals of Silver and God Belong to
the
Point)}\label{gate-17-95-of-6005-mithqals-of-silver-and-god-belong-to-the-point}
\addcontentsline{toc}{subsection}{Gate 17 (95 of 6,005 Mithqals of
Silver and God Belong to the Point)}

The seventeenth gate of the eighth Unity: concerning silver and
gold---if, by the measure in which they are weighed, they reach six
thousand and five mithqals, then ninety-five mithqals belong to the
Point. Let God take from you; all will be questioned about Him. And
return it to the One whom God shall make manifest, preserving it as you
would your very eyes.

The essence of this gate is that there is no true might save in
obedience to God. As in every Dispensation, among the believers in that
Manifestation, some pride themselves over others through obedience to
God, rather than through other matters---because other matters,
according to the people of each Dispensation, never receive the command
of Truth nor endure. If you wish to witness this, look at the end of
every Dispensation. Sometimes, from the beginning of one's life to its
end, a person may never leave off performing ablution, which is
recommended, and then boast, ``I do not lift my gaze to the sky except
in a state of ablution.'' Indeed, such a thing is honor if it is joined
with that which upholds religion---namely, knowledge of God and
recognition of the Manifest One so that it may be according to His
command and in His presence; otherwise, essential realities transform
from the realm of light into the realm of fire. How would such a thing
find expression in deeds? Know that when the amount of gold and silver
reaches the total number of all letters, ascending to the Hidden Ten, it
becomes six thousand and five. And if you reduce six, it becomes six,
and then the first letters become an allusion to ``Huwa'' (``He''). For
this reason, after these two have reached this limit, ninety-five
mithqals from each are to be set aside for God.

In the appearance of the Point---whether the first or the last---this is
to be acted upon by His leave. And in between these two, nineteen souls
of those who obey might be permitted to have their share distributed
among them in that same numerical measure, with details to be recorded
in their proper places. Thus it remains until the Day of Resurrection,
and believers will carry it out. It has been, and still is, greater than
every other form of trade, for in it there shall be no change or
alteration until another Resurrection.

Now observe the Point of the Bayán itself: if after tribulation, its
justice had decreed such a ruling, you could have asked why or how---had
you been able and among the people of Paradise who obey God. In that
moment you would have witnessed how, by a single command, so much
generosity has been bestowed upon all created beings. If all on earth
believed and wished to enter Paradise by obeying the divine
commandments, surely you would see how many ordinances would apply to
everyone. Observe, then, how abundant the generosity has been. Thus,
everything proceeds from the Origin, and all are agents.

For instance, if the Messenger of God had commanded to perform the
pilgrimage once a year if you have the means. Could a believer with the
means have turned aside from it? Rather, he would seek closeness to God
through obedience and would take pride among the people in his
observance of it. In the same way, observe every commandment, for they
all lie in His grasp. If He wishes to enrich someone, He enriches them
until the Day of Resurrection, and that in truth---not without right.
Likewise, if He wishes to grant rulership, He does so until the Day of
Resurrection, and if He wishes to confer honor upon someone, He makes
them honored until the Day of Resurrection.

This happens as follows: if the Messenger of God had said that the
descendants of a certain believing person must be enriched by everyone
(this being one of the ordinances), then consider how much would have
been given to them so that true wealth would come to pass. And if He had
said that rulership on His behalf belonged to the descendants of a
certain person, the believers could not have opposed it, and it would
have remained in their hands until the Day of Resurrection. Similarly,
if He had said that the descendants of a certain believer must remain
honored until the Day of Resurrection, reflect on how that honor would
appear today.

Yet you see that He has said, ``Pilgrimage to the House is God's due
upon the people,'' and every year, seventy thousand souls depart,
circling mere clay. This is the exaltation of God's command and its
independence from all else. Likewise, if He wills that someone be poor,
that person remains poor until the Day of Reckoning. Reflect on how a
single mention of Abu Lahab was revealed unequivocally. Travel the East
and the West to see whether his name is even mentioned. Even if he
subsisted in a lower station beneath the Truth, there is no poverty.

It surpasses that to the point that not even the mention of a single
name remains. In the same way, observe other conditions: the command of
Reality extends from the Resurrection to the valued manifestations of
authority, who suppose that for a month they possess leadership and
thereby step out from the shelter of obedience. Yet if you look into
those who follow them, it is upheld under the name of Truth, for they
say it stems from Islam. In reality, in the presence of God and those
endowed with knowledge, it is judged to be beneath God. Behold then the
later creation: from the origin of that command, which is from the
Resurrection until the Resurrection, it remains hidden. And for the sake
of imagining one day's enjoyment---even if it is beneath God---see how
they stake their lives. This is only due to their lack of insight and
understanding. Otherwise, a perceptive soul, a precise believer, would
never pass from one Resurrection to another in Paradise, then seize one
day to dwell in the Fire---equal to that very day in what comes
after---until God wills to rescue that person from the Fire and thus
judges them as beneath God, dwelling in the Fire.

Know that God's deliverance of the people of the Fire from the Fire is
that He calls them toward Himself. If they turn toward Him, they attain
salvation; otherwise, they remain in the Fire. Since God's claim never
becomes manifest except through a manifest summons by His appearance, so
too God's acceptance is not made evident except through answering that
summons. For this reason, in each Dispensation, when people fail to
respond openly to God's call in that Dispensation, in the next
Dispensation they remain without deliverance. Those manifestations who
summon everyone to Him---who are guided by the ``letters of the
Living,'' each one pointing to the First Point---cannot rescue them from
the Fire.

For example, if, in the very hour of the appearance of the Messenger of
God, all on earth had responded to him and to his commandments, they all
would have been saved from the Fire and admitted into Paradise, for
whatever is decreed in the hereafter regarding Fire and Paradise
revolves around the decree of this realm. Thus, in the religion of
Islam, force was commanded in the hope of compelling the people of the
Fire to enter Paradise, and there was also a command for intense mutual
love, so that perhaps they might don the garment of the people of
Paradise. If, in the Bayán, a single individual were to clothe all who
dwell on earth in a single garment and bring them into the Bayán, they
would all be delivered from the Fire and enter Paradise. This is the
grace conferred upon them. By the sacred Essence of God, if in the
appearance of Him Whom God shall make manifest everyone were to obey
him, not a single soul would remain in the Fire; rather, they would all
enter Paradise, and all that is upon the earth would be assured of that
Paradise.

My sorrow, however, is for the believers in him, not for those below
that rank, who, in the darkest of nights, supplicate and pour out their
laments in his name, proclaiming religion and worldly matters in his
name, day and night weeping for his meeting, humbly entreating. Yet when
he makes himself known to his servants---this being the greatest
Paradise, beyond which none can be imagined, since the first principle
of religion is the knowledge of God, and the knowledge of God is not
possible except through knowledge of him---those servants who, by the
verse deposited within them from his previous appearance, were acting
for God's sake, arise and endure what is unworthy. If any thought less
than the recognition of his truth should cross their minds, it is
considered before God to be graver than any disobedience, and in a
single moment erases all their deeds, as though they had never been.
This is just as you heard in the appearance of the Point of the
Criterion: all the believers were awaiting, through the Gospel, the
promised Ahmad. You have heard what befell the Sun of Reality from them
over twenty-three years of His appearance, to the point that He said,
``No prophet has been made to suffer as I have been made to suffer.''
All of them were beseeching for His appearance, so they might act in
accordance with the words of Jesus concerning Him. Praise be to God that
you were not in those days, but rather in the appearance of the Point of
the Bayán. You saw how all the believers in the Messenger of God were
awaiting the appearance of the promised Mahdí, for this hadith is from
the Messenger of God, and both the general and the particular agree upon
it without doubt, so that faith would be confined to the Twelver
denomination. Indeed, the manifest Islam consists of five clear
divisions, whose adherents refer to themselves as Twelvers; outwardly,
they call the land of Persia the abode of knowledge, even though the
Tree of Reality arose and not a single one among them recognized Him.
Once they did recognize Him, the extent of their remoteness was
apparent, which in itself is enough to show their abasement. Yet day and
night they cry out ``al-ʿajala, al-ʿajala'' (hurry, hurry) for His
advent.

In the Bayán, you will see the same. Do not be deceived by everyone's
claim that they believe in it, for this is the very dawning of the
universal Sun that was at the beginning of the Point of the Bayán,
exactly as it was at the beginning of the Point of the Criterion. Yet
the Point of the Bayán has appeared in such a way that no child could
deny it, while they were all affirming---indeed, they were
certain---that the Qur'an is the Book of God, that the Prophet's
prophethood is true, that the Guardian's guardianship is true, that the
love for the Gates is true, and that every ordinance of the religion of
Islam is established by it. They called it the greatest miracle of the
Messenger of God, believing with certainty that none besides God can
produce anything like it, and throughout one thousand two hundred and
seventy years, no one has brought forth even a single verse comparable
to it. As soon as it appeared, it stood as proof, and like an ocean
without interruption, it poured out from the sea of its bounty so that
all might be certain it is from God and not from anyone else, in
accordance with the text of the Qur'an and by their own faith.
Nevertheless, they went on to claim it was from someone other than God,
and they acted as they did. O people of the Bayán, do not do what the
people of the Qur'an did, for you do whatever you do in His name while
remaining veiled from Him. Should you remain veiled, you wrong only your
own selves. If you do not cause Him any sorrow, and---God forbid---if
something less than the truth should befall Him, it would have befallen
God. Yet day and night you prostrate yourselves before Him, and from
your earliest moments to your last you wish to act for His pleasure.
This is a most immense matter, one you cannot endure---not on account of
His lowliness but because you suddenly see a soul you never knew before,
and perhaps you were his father, mother, or nearest kin. He appears,
uttering the words ``I am God's; there is no God but I.'' At that, all
at once, you are torn apart and bewildered by the height of that Sun of
Reality's transcendence and the loftiness of that Divine Manifestation.

Yet if you do not turn away from the essence of proof---which is the
signs of God---and do not speak as those did before in the Qur'an, nor
as those who came after in the Bayán, perhaps even if you do not
believe, you will not have passed judgment against God. Otherwise, that
judgment falls upon yourselves. If anyone judges against God, what
boundary could there be for that?

By the sacred Essence of God, those servants who pass judgment against
Him are worse than those who, in this very Dispensation, have passed
judgment against Him. Consider now how distant are those who stood in
opposition to the Messenger of God, and know that you would be in the
same condition if you did not believe, in the eyes of those who come
afterward. Today, there are indeed people of true insight; likewise, in
the appearance of Him Whom God shall make manifest, those who believe in
Him with discernment will regard the condition of those who remain
veiled as even more remote than those veiled in this Dispensation. You
may now mention some well-known figures from Mecca and Medina, or recall
their names, yet this is exactly how, for those who come after, the
figures of this Dispensation will be mentioned. So too in the appearance
of Him Whom God shall make manifest: if some empty mention remains, it
is only for the sake of recalling the truth, not because that mention
itself is worthy of remembrance---just like the name that appears in the
Qur'an. Reflect a little, and do not turn away from the essence of
proof; perhaps on that day you may be saved. Otherwise, in every
Dispensation, people perform all their deeds without knowing, imagining
that they do so for God.

``\emph{Fear God as He should be feared; then, by God's command, attain
certitude.}''

\subsection*{Gate 18 (Fasting)}\label{gate-18-fasting}
\addcontentsline{toc}{subsection}{Gate 18 (Fasting)}

The eighteenth gate of the eighth unity deals with fasting. ``Remember
God over nineteen days of each year's end, while you are fasting.'' The
essence of this gate is first to understand God's intention in fasting
and what its outcome should be. If you had been present in the
Dispensation of the Qur'an and had asked the Messenger of God about the
reason for its prescription, He would have answered as follows: fasting
is ordained so that the one who fasts\ldots{}

If you did not love anyone who does not love Him, and were not for
anyone who is not for Him, and you were fasting, the reward of fasting
would have been granted to you. In the same way, carry this letter by
letter throughout the Letters of the One in the Qur'an until you reach
the final letter, which encompasses all appearances of the entire One.
If you were fasting beneath it, then on that day you would have been
fasting for God. Likewise, consider the Point of the Bayán: if you heard
of the appearance and the thought of something less than its truth
crossed your heart, the essence of your religion would vanish. How,
then, would fasting---which is one of your religious branches---endure?
At the moment of hearing, the proof had already reached you, because the
one who informed you presented evidence through signs.

The very moment you remained veiled, that was veiling yourself from
God's encompassing in the fourth atom, for He had manifested through
that appearance and identified Himself with the appearance of one of the
appointed deputies. Thus, in the fourth atom, the first atom was made
manifest, because that same atom revealed the words ``Indeed, I belong
to God; there is no God but Me.'' If someone has discernment within
possibility, they could traverse and be certain that the end is the very
same as the beginning, and that the outer is the same as the inner at
the first rank, not the second, for the names of each rank pertain to
that rank and do not exceed it. For example, look at the first dominion
of that Existence, mentioned until its final existence, yet the first
mentioned there cannot be compared with the first mentioned at the end
of existence. The same applies to all the ranks of names and likenesses.

Observe and judge. If in this manifestation you have fasted out of love
for anything less than the First Letter, then among all letters nothing
is seen except that single letter. Since the decree pertains to the
First One and not to the multitude of numbers, anyone who fasts for them
is truly fasting, and anyone who fasts for anything less than them is
only fasting beneath that station. In their being, behold the gates of
Paradise, and in the number of the One, under the shadow of the gates of
Fire---because the reality of fasting comes from them.

For example, on the Day of the appearance of the Commander of the
Faithful, everyone was fasting, yet the decree of fasting applied only
to those servants who were in love with him, and beneath that love it
was no true fast. In every manifestation, all the people of that
manifestation act according to its laws, but at the beginning of the
next manifestation, the original decree is annulled. How then could its
lesser aspects continue? When it is said that it is annulled, this means
it appears again in the new manifestation; otherwise, it would not be
considered annulled at all. If in this manifestation someone were in
love with the Last One, such a person would be fasting in the City of
God; otherwise, everyone remains obedient in whatever station they
occupy, but to what end?

If all who believed in the Qur'an had not fasted---so that what happened
would not have happened---it would have been more pleasing before God
than what did happen through their fasting, for then such events would
never have come to pass. His testimony of what those souls upheld would
not have been nullified; but now, although they have acted according to
the laws of religion, that decree is set aside. During the time of
fasting, it is incumbent upon the one who fasts to be mindful of God's
good pleasure, so as not to be veiled from it. For if, while one is
fasting, the Tree of Reality should arise and issue a command, one would
obey Him instantly, for this fast that one now undertakes was
established by His command in the previous manifestation. Observe
likewise all deeds---drinking, eating, marriage, disputation (even if by
the learned), and oppression (even if only a qirát in measure). Let
God's decree preserve your soul, and carefully reflect on the final
third of the ruling: from the start of the manifestation until the
beginning of the next manifestation, anyone who pronounces judgment
against the Point has passed that judgment against God, and it is
invalid.

Similarly, at the time of the appearance of Him Whom God shall make
manifest, there is no doubt that all the people of the Bayán are
fasting. Yet if they pass judgment on Him, it is nullified. How could
the origin of their religion compare to even one facet of His dominion?
Be watchful from sunrise to sunset, and gaze upon the Name of the One.
Before reaching the number of the Name ``Huwa,'' there is no ruling upon
believer or believeress except until midday---if it goes beyond that,
then it is not considered fasting. After that, up to one year is
acceptable, but beyond it, there is no fasting.

Reflect on each numerical part of God's commandments. If all who dwell
on earth were to gather together, they could not even bring forth the
smallest fraction of nine times nine or nineteen times nineteen truly,
without presuming to pass judgment. Now behold the ocean of God's bounty
surging toward His servants without their deserving it. If it were based
on merit alone, they would have remained in the veils in which they
were. The origin of their recognition of His grace lies in this: had He
not revealed Himself until another Resurrection, they all would have
continued to act as before.

``\emph{And you shall indeed fast for the sake of God, your Lord, that
perhaps on the Day of Resurrection you may distance yourselves from
those who did not believe in Him Whom God shall make manifest.}''

\subsection*{Gate 19 (Invoke Blessings Upon the
Tree)}\label{gate-19-invoke-blessings-upon-the-tree}
\addcontentsline{toc}{subsection}{Gate 19 (Invoke Blessings Upon the
Tree)}

The nineteenth gate of the eighth unity: when the name of the Tree is
mentioned, invoke blessings upon it.

Whenever the One whom God shall make manifest is mentioned, send
blessings upon Him. Whenever the Letters of His living truth are
mentioned, offer salutations to them. Remember God and Muhammad and the
manifestations of His command on every Friday night and day, two hundred
and two times; then call upon God on those two days four thousand times.

At any moment when the name of Him Whom God shall make manifest is
spoken, direct blessings toward Him. Whenever His living, true Letters
are mentioned, recall the divine splendor upon them, whether in what has
already appeared or what shall appear. If you recognize the value of
each Friday night and day as a time in which deeds are multiplied, then
on those nights and days mention Him Whom God shall make manifest and
the living Letters associated with Him two hundred and two times, and
invoke God with the measure of four thousand, not in the sense of bowing
down and calling upon Him while remaining veiled from the One whose
mention is God's mention and whose knowledge is the knowledge of God.

Consider how, in the appearance of the Messenger of God, many Friday
nights and days passed while the believers in the Gospel continued to
supplicate God with their own tongues. Did it bear fruit for them? In
the same way, look now at those servants who, in the era of the Point of
the Bayán, spend every night until dawn immersed in the remembrance of
God. Meanwhile, the Sun of Reality has nearly ascended in the sky of its
manifestation, yet they have not stirred from their prayer rugs. If
someone should recite newly revealed verses to them, they say, ``Do not
distract me from the remembrance of God.'' O you who are veiled, you
invoke the remembrance of God, yet how can you remain veiled from the
One who has caused that very remembrance to arise within you? If it were
not previously revealed, where did you learn to mention Him, and where
are you turning now?

Know that if you mention Him Whom God shall make manifest, then you have
mentioned God. Likewise, if you hear the verses of the Bayán and
acknowledge them, then the verses of God will benefit you. Otherwise,
what profit do you gain if, from the beginning of your life to its end,
you perform a single prostration and fill every moment with the mention
of God, yet you do not believe in the Manifestation of that
Dispensation? See whether it benefits you. But if you recognize Him,
become aware of His truth, and say, ``I accept,'' then your entire life
spent in His remembrance is truly counted as remembering Him to the
utmost degree---because your actions are performed so that God may
accept them, and the acceptance of God is made manifest only through the
acceptance of His Manifestation when He appears. If the Messenger of God
accepted a matter, it was accepted by God; otherwise, it remained as an
action based on personal desire and did not return unto God. In the same
way, if the Point of the Bayán accepted an action, God accepted it.
There is no pathway for anything in creation to the Ancient Essence
except that whatever is revealed should be revealed by the
Manifestation, and whatever is deemed genuine should be referred to the
Manifestation.

Praise be to God that no one has been seen whose acceptance is
evident---despite the fact that, from the start of their life to its
end, they may act with the utmost rigor and effort. And if you ask them,
``Why do you do this?'' they reply, ``Because we want God to accept
it.'' O you who are heedless, God's acceptance is not made manifest
except through the acceptance of His Proof. Do you possess any word from
His Proof stating, ``I accept''? Thus do all people act without
awareness, remaining veiled from the fruit of their labor---except for
the one who acts in accordance with the Bayán.

On the Day of the appearance of Him Whom God shall make manifest, if
someone possesses a direct statement from Him about His acceptance, that
person is worthy for it to be said that they acted for God and that God
accepted their deed. Otherwise, what benefit is there, when everyone on
earth is acting in accordance with their religion, yet observe the
source from which God's acceptance becomes manifest. It is as though, in
the appearance of the Messenger of God, there was not a single
discerning soul---apart from those who recognized Him---who asked Him
for acceptance of their deeds; had anyone done so, it would have been
revealed in the Qur'an by the tongue of revelation. For the acceptance
of God cannot be conveyed in the language of mortal beings. If it were
through the Messenger of God's own words, that would be His acceptance,
not God's acceptance.

They hide themselves in a mountain, performing deeds night and day for
the sake of God, while the verses of God rise from that place like an
ocean. See whether even the fragrance of awareness has reached them,
though the fruit of all their deeds is that God should accept them.
God's acceptance is not made manifest except through the tongue of the
verses that prove the impotence of all else. Today, the Qur'an shows the
powerlessness of all existence. Now act as you wish and see if there is
even the slightest mention of acceptance in your favor, like a mere
straw. It is as if one walks in the darkness of night without witnessing
any fruit from a lifetime of deeds supposedly done for God, never once
turning them toward the Manifestation to whom those deeds must return.
Had they done so, they would not face such affliction on the Day of
Resurrection. Observe how mighty this affair is and how veiled all
remain. By the sacred Essence of God, all remembrance of God and all
deeds done for His sake are the remembrance of Him Whom God shall make
manifest, and deeds performed for His sake.

You nearly give up your own selves, saying that you act for God, yet you
act for something lesser than God. If you truly acted for God, you would
act for Him Whom God shall make manifest, and you would remember Him.
Otherwise, what benefit is there for those living on this mountain---who
know nothing and say ``There is no God but God'' day and night? Reflect
a little so that you do not remain veiled from the origin of the matter,
for all your worldly deeds trace back to your very being and thus to
your religion, and all your religious deeds bring forth as their fruit
the acceptance of God.

God's acceptance does not become evident except through the acceptance
of the one who has appeared from Him, Him Whom God shall make manifest,
from whom the tongue of the verses emerges. If acceptance occurs through
anyone or anything else, it is not God's acceptance, because God's
acceptance is His Word, which is unlike the word of creation, and they
cannot be two. If you say, ``Acceptance by the Gates is acceptance by
the Imams, and acceptance by the Imams is acceptance by the Messenger of
God,'' it is only because that Tree has decreed it so. Likewise, that He
has counted the sorrow of a believer's soul as His own sorrow and the
believer's joy as His own joy is because He revealed it that way. If it
were not so, there would be no resulting fruit. Always fix your gaze
upon the origin of the command, for all conditions appear under its
shade; it is not two, but one, and that one is not confined by number
but transcends all numbering. And that one or that number becomes one
through His command; without that, no decree could be carried out.

``\emph{That they may single out God, your Lord, the All-Merciful, in
purity, then act in truth for His sake.}''

\section*{Vahid 9}\label{vahid-9}
\addcontentsline{toc}{section}{Vahid 9}

\markright{Vahid 9}

\subsection*{Gate 1 (Exaltation of Every Land Belongs to
God)}\label{gate-1-exaltation-of-every-land-belongs-to-god}
\addcontentsline{toc}{subsection}{Gate 1 (Exaltation of Every Land
Belongs to God)}

The first gate of the ninth unity is that the exaltation of every land
belongs to God, and the singular distinction of every city belongs to
God, and Houses that on that day are attributed to kings---if someone
from among the people of the Bayán prays in any of them, let that person
give one mithqāl of silver. Moreover, only those referred to as the
Martyrs of the Bayán or the manifestations of the One may dwell in them.
In every seat of honor, leave vacant, according to the number of the
One, a place for a single soul if the land is vast. Otherwise, leave
simply the One without any number, which will suffice for those who act.
Whatever is taken from the Point must be allocated to one of the two
sanctuaries, and on that seat a structure should be built for
remembering God and performing prayers.

The essence here is that the glory of every land has belonged to God and
returns, on the Day of Appearance, to Him Whom God shall make manifest,
or to what He permits. Likewise, the glory of cities and places once
held by former sovereigns: if someone from among the people of the Bayán
prays there, they must give one mithqāl of silver, so that none may
reside there except the Martyrs of the Bayán and the manifestations of
the One. In every noble gathering convened without justice, it is
appropriate to leave open a space corresponding to the number of the
One, so that if at that moment Him Whom God shall make manifest, or the
living Letters, appear, no one may occupy that spot. If the gathering
place is not spacious, then leave room for one soul only. The same
applies to every seat, which should have space for a single soul left
vacant.

It is observed that even in the house belonging to Him Whom God shall
make manifest, this principle holds: people may refuse Him a seat
because they do not know Him, except for His mother, who alone
recognizes Him by His name---yet He knows them all and smiles at those
who show such reverence and respect for His name while failing to
recognize His reality.

They take note, yet on the day of His appearance they remain veiled from
Him by their own inventions. The clay taken from the Point has been
commanded to be placed in one of the two sanctuaries, where a chamber of
mirrors will be raised so that those who pray may perform their prayers
there. This serves as a sign before the people that the Point of the
Bayán was a servant---created, sustained, born, and an heir---and that
whatever was spoken by God came from Him, not from the Point itself.
Thus, no one should exaggerate or go beyond the bounds of servitude.
Those who observe the Bayán's movement and ascent cannot compare it with
others in the Qur'an, let alone with all upon the earth; yet all these
wayfarers circle around those wayfarers in the appearance of Him Whom
God shall make manifest, because the truth is confirmed through them on
that day, not apart from them. In all that proceeds from Him, glorify
God, for God's command is exalted, just as it was in the beginning---so,
O servants of God, fear Him.

\subsection*{Gate 2 (Write 1,000 Verses for God if You Hold a Pen and
Lack
Justice)}\label{gate-2-write-1000-verses-for-god-if-you-hold-a-pen-and-lack-justice}
\addcontentsline{toc}{subsection}{Gate 2 (Write 1,000 Verses for God if
You Hold a Pen and Lack Justice)}

The second gate of the ninth unity: whoever holds a pen yet lacks
justice in his days must write a thousand verses for God and stipulate
in his will that they be delivered to the Point, so that God may, on the
Day of Resurrection, reward him through His mercy, for He is indeed
all-knowing.

The essence here is that if, in the era of the Bayán, someone acquires a
pen yet has no justice in his time, it is pleasing that every manner of
script---ranging from its most splendid to its most exalted
form---should be brought to completion, reaching the rank of oneness, so
that a thousand verses would be written on a parchment likewise devoid
of justice, and all other aspects should match this form. A will must
then be made whenever the Day of the appearance of Him Whom God shall
make manifest arrives, present that Tree of Reality with it, so that He
may reward the person with what lies with Him of His signs, and so that
the person may be remembered thereby before the Beloved. If someone
cannot manage to write out a full thousand because they lack the means
at the time of writing, the Martyrs of the Bayán must provide them with
the splendor of the alif---unless that person is capable. If one is
capable, it is not fitting that someone acting for themselves should lay
any price upon their own act.

Everything stemming from the proof of the Point---be they verses,
prayers, commentaries, scholarly matters, or Persian writings---whatever
one writes is accepted. The outcome is that if, on the Day of
Appearance, such a person exists, they will not turn their pen to
anything except the writings of Him Whom God shall make manifest,
because it is forbidden for them to write anything but His words.
Perhaps in that Day one soul will act purely for God, which is better
than all that is written in the night.

No script in this manifestation is more cherished by the One revealed at
His appearance than the ``broken'' script of a living being---not a
lifeless one---because most people write with learned skill and are, in
effect, lifeless rather than living. Its beauty lies in its being alive;
in relation to the naskh style, it is like a living being compared to
something perfected in its every aspect, and each thing, in its own
measure, is beloved before God. Teach your children the most splendid
and guarded forms of script you possess, that perhaps on the Day of
Resurrection you may take pride in it before your Lord.

\subsection*{Gate 3 (A House Where the Verses of God Are
Inscribed)}\label{gate-3-a-house-where-the-verses-of-god-are-inscribed}
\addcontentsline{toc}{subsection}{Gate 3 (A House Where the Verses of
God Are Inscribed)}

The third gate of the ninth unity: To God belongs, from every dominion
or sovereignty, the right to build for Himself, in those multi-layered
dwellings, a house where the verses of God would be inscribed, and
before its gaze would stand the verse spoken of in that religion.

The essence of this gate is that all created things have been brought
into being for the Day of God's appearance, which, in the terminology of
the Bayán, is called the Day of Resurrection, running from the first
moment of the appearance of the Tree of Reality until its setting. For
instance, in the Point of the Criterion, this spanned twenty-three
years, for that Day---on account of which all were created---is like the
sun in relation to the stars. The same holds for the people of that
Dispensation compared to the Manifest One. For this reason, those who
truly know in that Dispensation refrain from letting their pens flow in
compositions, writings, and expositions out of modesty, because if a
star rises in daytime, it is outshone by the sun. Likewise, if the most
learned among those living after the appearance of that Word were to
compose anything, it would be akin to that very same condition.

Because the Point is the Sun of Reality, their works, compared to the
Sun's own works, are mere reflections; even so, if they become mirrors
until the end of existence, they will continue to reflect rays of the
Sun's works, and all remain independent of anything beneath it. But once
it sets, permission is granted for all under its shade to ascend in
whatever measure they can. Even if they all were to become learned, they
would still be unable to attain the knowledge contained in even a single
letter of that Sun. It is mentioned regarding the first potency, derived
from M and S in the Bayán, that there is a lofty station among the
mirrors for He Whom God Shall Make Manifest, and between his two eyes
should be inscribed a reminder that indicates that if Him Whom God shall
make manifest appears and he believes and helps Him, he will become
higher than all creation; otherwise, he will be lower. Perhaps on that
day he will watch over himself so that he is not veiled for several
mornings from meeting his Beloved, for whom he has labored from the
beginning of his life until its end. Nothing bears fruit except acting
for Him and supporting His religion in what pleases Him.

Otherwise, he will depart as those departed before, leaving no trace
behind except a mention beneath the truth and a veil before the Beloved
for whom he was doing whatever he was doing, the One who was honored in
the Bayán by His name. If any sorrow befalls Him Whom God shall make
manifest, He exacts from all things whatever retribution is possible;
and if anyone aids Him, then the grace that flows to all things will
descend upon that person.

If no one takes any step to grieve Him, none will be subject to such
retribution. In this era, the learned have no authority except through
that assistance. God alone knows how He will nurture humanity and define
their limits in that age. If any learned one in that appearance sets out
on something less than His good pleasure, it is as though he has taken
on the burden of all existence, and every fire created for what lies
beneath God will be his. The gaze of everyone, high and low alike,
returns to the learned of each manifestation. If they are sincere in
their faith, they will not stray from the truth.

If hardship befalls the Truth, it comes about through their
estrangement, for everyone assumes they are right, while in reality they
belong to what lies beneath God in His sight. Thus the fire that reaches
all first touches them, and then others---just as, if they turn toward
God, the divine grace first reaches them, then extends to others. How
excellent is the station of knowledge if it is knowledge of Him Whom God
shall make manifest and of what pleases Him; otherwise, it is the worst
of stations in the sight of God and before all things. If someone knew
nothing at all, that would be better for them than possessing knowledge
of everything but lacking knowledge of Him Whom God shall make manifest,
since all things are negated by His command until they assume the
garment of ``thingness.'' Should one who knows nothing at all turn
toward Him, ``Blessed is he''; but if he turns away, he has cast himself
into the fire. Likewise, anyone who followed him into knowledge would
enter the fire. If, however, he were to guide people with his knowledge
toward the Truth, by means of that guidance they would enter Paradise.

Because following any soul leads to veiling oneself from the Truth, it
is more beneficial for such a person to be without knowledge than to
possess it---unless that knowledge is solely for the sake of God, such
that he might use it to aid the Truth on the Day of its appearance. By
it, whoever comes to faith will do so when they behold their own
powerlessness compared to the verses of God, and at once they will
prostrate and acknowledge, ``These are indeed the verses of Him Whom God
shall make manifest, the One promised by all.'' They will praise God for
having made them a knower of Him on the Day of Resurrection, that they
might attain the fruit of their own being and not be deprived of meeting
God.

If only we would not remain veiled from His presence, for we were
created according to His will and have done all these deeds for nothing
else. This is a favor from God upon us, for He is the most bountiful,
the most generous. Know that if you had certainty, you would act
accordingly, but because you cannot attain certainty---veiled by your
own self---you remain in the fire without even noticing. On the Day of
His appearance, nothing but believing in Him will save you from the
fire, no matter how many good works you perform. Yet if you believe in
the Truth, every good thing will be recorded in God's Book in your
favor, and with that, you will delight in Paradise until another
Resurrection.

Give your utmost attention, for the matter is extremely subtle, even
while it is more expansive than the heavens and the earth and all that
lies between them. For instance, if all those who awaited the promised
one in the words of Jesus, peace be upon Him, had possessed certainty
regarding the appearance of Ahmad, the Messenger of God---may blessings
be upon Him---not one of them would have turned away from Jesus's word.
Likewise, if everyone in the era of the Bayán were certain that He
{[}the Báb{]} is indeed the promised Mahdí foretold by the Messenger of
God, no believer in the Qur'an would turn away from the word of the
Messenger of God. So too in the appearance of Him Whom God shall make
manifest: observe that if all had certainty this is indeed the One whom
the Point of the Bayán foretold, then no one would turn away. There is
no valid proof for failing to have certainty in Him. If there was any
proof for the Christian monks and the learned of the Criterion to reject
the Bayán after its appearance, then to that same degree such proof
would exist for them as well. Pay the utmost attention, so that you are
not veiled by your own subtlety, but rather recognize Him with His own
eye.

So that you may attain His knowledge. These words are given so that on
that day, all---high and low---will watch over themselves, lest they
enter into the mention of annihilation and remain veiled from the summit
of meeting. ``Watch over God, and then fear Him alone.''

\subsection*{Gate 4 (True Knowledge is Knowledge of Character and
Attributes)}\label{gate-4-true-knowledge-is-knowledge-of-character-and-attributes}
\addcontentsline{toc}{subsection}{Gate 4 (True Knowledge is Knowledge of
Character and Attributes)}

The fourth gate of the ninth unity: God has prescribed for humanity the
mention of the secret. ``Say: All shall be questioned about it.''

The essence of this gate is that all true knowledge is knowledge of
character and attributes, which a person practices so that by means of
that knowledge, they neither behold any sorrow in themselves nor cause
sorrow to anyone else. The commands regarding piety, chastity, and other
such matters all lead back to this. For example, if a person is
afflicted with poverty and remains content and patient, their honor
remains intact in their own sight, and they do not grieve. Once the days
of poverty pass, it is as though nothing happened. But if they lament,
they might find someone who can ease the cause of their sorrow, yet
afterward, when they reflect upon themselves, that remedy proves unequal
to the self-abasement they endured.

In the same way, observe every attribute and condition at every level.
The command to mention the secret is so that you may remain mindful of
the remembrance of God, keeping your heart ever alive and never veiled
from your Beloved---not by reciting with your tongue while your heart
fails to turn toward the lofty summit and realm of communion, in the
hope that, come the Day of Resurrection, the mirror of your heart may
face the Sun of Reality. If it rises, it will instantly cast its
reflection, for it is the Source.

If it appears and you remain ever immersed in your own remembrance, it
will not benefit you unless you remember Him with His remembrance, for
He is the remembrance of God in that manifestation. The remembrance you
practice now is by virtue of the command of the Point of the Bayán, but
that appearance is the essential reality of the Point of the Bayán in
its latter stage, which is infinitely more powerful than its first
appearance. If you mention Him secretly and believe in Him, your secret
remembrance is multiplied by ninety-five times more reward than open
remembrance. Yet on the Day of His appearance, call upon the Truth
openly, for on that day, open remembrance surpasses ninety-five
instances of hidden remembrance. This is the very essence of remembrance
in the presence of the one who remembers and the One remembered, should
you be able to comprehend it. ``Therefore let them remember God in
secret,'' while they remain empowered in that state, and be watchful
over the attachments of your heart, for its degrees are without limit.

If you are among those who are vigilant, you will recognize that in the
intensity of secret remembrance, its ruling can match that of the open,
until in both sleep and wakefulness the remembrance in the heart takes
on a single form. But even if you reach that level, should you fail to
recognize the Sun of Reality, it will not benefit you; if you do reach
it, absent that Sun, it may still benefit you. Multiplying your
remembrance is not what is beloved, whether in secret or openly. Rather,
if you offer one remembrance with spirit and fragrance, it is better
than a thousand devoid of spirit and fragrance. Each person knows within
themselves the measure of sincere remembrance.

What concerns us is the remembrance of Him Whom God shall make manifest.
In this Dispensation, there are souls claiming to practice the
disciplines of watchfulness---yet before God, their station is null and
void. The divine limits are those revealed in the Bayán. Observe these
names of God.

They have not reached the reality, for those who claimed such a thing in
this manifestation remained veiled, while those who paid no attention to
these notions, and indeed had never even heard the word
``watchfulness,'' were saved by their very turning. So watch over
yourselves and remember God in secret and in public; through God's
remembrance do hearts find rest. Do not occupy yourselves with what will
not benefit you on the Day of Resurrection before your Lord, unless you
stand in prostration before God. This takes place before Him Whom God
shall make manifest---O people of remembrance, be heedful.

\subsection*{Gate 5 (19 Days to Serve the Point at It's
Return)}\label{gate-5-19-days-to-serve-the-point-at-its-return}
\addcontentsline{toc}{subsection}{Gate 5 (19 Days to Serve the Point at
It's Return)}

The fifth gate of the ninth unity: God has ordained for every soul
nineteen days to serve the Point at its return, acting by its permission
whenever it grants leave, and remaining bound to its master, for God is
most bountiful.

The essence of this gate is that everything called a ``thing'' comes
from God through His Will, yet whatever the embodiment of that Will has
confirmed in each form of existence---taken from among His
verses---comprises nineteen verses; beyond that, they are multiplied
infinitely and cannot be numbered. Therefore, at His appearance, it is
enjoined that on each day, corresponding to the verse that resides in
oneself, one should stand before God by His command, so that the fruit
of His religion may be revealed in God's presence. In its branches, the
first day belongs to the Point, and the days of ``ḥay'' belong to the
Letters of the Living.

From the first Day of Resurrection to its last, this ordinance applies
to all believers in the Bayán, high and low alike, and it will not be
lifted from them until permission for its removal is granted---only then
is it annulled. Yet, to the fullest extent possible, on each day for
every verse, see how you stand in relation to one letter of the
Criterion in the place of its dust, so that you may stand likewise
before the Point.

Observe how many people today visit the shrine of the Commander of the
Faithful---peace be upon him---yet consider how, in the appearance of
the Messenger of God, he stood in relation to that Messenger, and
everything he possessed of honor and loftiness was from him. Now see how
you could possibly stand in the presence of such a sacred court, when
you witness these same Letters of the One---both the highest and the
lowest among creation---circling around their resting places today and
expending their wealth so that prayers may be recited for them in
connection with those resting places, even though all this grandeur in
them comes from the command that began with the Messenger of God.
Reflect on whether you can appear before that court, when, if all
existence were but one single soul, and that soul prostrated itself from
the primordial beginning until the day of His appearance, then on the
day of that appearance---should one truly reflect on His
worthiness---one's inner being would never permit lifting one's head
from that prostration.

The Sun of Reality possesses such an essence that the greatness of all
things is smaller than a speck before It. Yet behold these words of His
and see how He trains His creation, so that all might turn toward the
One who created Him, who is the Creator of all things, and toward the
One who provides for Him, who is the Provider of all things, the One who
brings about His death and the death of all things, and the One who
grants Him life and grants life to all things. If you perceive the
innermost knowledge and act upon it with certainty, then by His
grace---not by your own worthiness---you may become fit to enter the
presence of the Illumination of Light. Otherwise, even if you were to
stand before Him but did not enter the station of ``I do not ascribe any
partner to God,'' you would be unable to conceive so much as a fleeting
thought of what truly pleases Him.

How can one act, when if all people were to carry out on that day the
ways they act among themselves, their Beloved would assuredly be pleased
with them? You see how they spend thousands upon thousands on the road
to a House that is attributed to Him; yet on the day of His appearance,
when by each of His words a House like that is brought into being, if
any soul soars toward Him, it becomes evident. So be truly mindful of
God, then purify your deeds for His sake. If you stand before Him
without His due worthiness, you will surely remain veiled from His
command; and if you seek Him without knowledge of Him, you will not be
worthy of entering the court of His majesty. Look at yourself: they
might place you upon the parapet of the Throne as an envoy---indeed,
consider it even greater than that, for His utterance makes the Throne a
throne, while He is exalted beyond being described by such an attribute
or defined by such a mention.

Meanwhile, you observe no lower station of dignity for His House, which
emerges from those Manifestations. You heard in the final letters, as
recounted from the prior appearance to His own in the Point of the
Bayán, concerning a Letter that became the locus of His appearance,
where it was revealed: ``Glory be to the One who is the Most High; there
is none above Him to compare. And glory be to the One who is the Most
Low; there is none beneath Him as His like.'' How could it be fitting,
in the sanctified presence of One so exalted, to speak of such modes of
appearance? These belong to the paths of ascent and the essences of
oneness, which are mentioned. Yet in the path of outward limits,
whatever He possessed, He made manifest in the highest form within His
own domain of possibility, so that He might accept it---provided you do
not transgress His outward bounds. But if you regard Him with any other
vision than the vision of oneness, He will judge you to be in the
station of an animal. And if you dare say ``why'' or ``how''---God
forbid---it is as though you have spoken it about His own Self, for He
has no likeness.

Nothing is meant by it to serve as a mere analogy. If I say the Point of
the Bayán is more manifest in His appearance, or if I say the later
appearance within His very Self is more forceful, how can I possibly
speak of Him or extol His loftiest height or most unassailable
exaltation? Permission to mention Him is something He has granted to His
creation; otherwise, He is sanctified above all mention and exalted
above all praise. With this towering grandeur, this immeasurable
loftiness, this height of majesty, and this inaccessibility before God,
He has always been and will ever remain. Exalted is God beyond what any
describer might describe, immeasurably exalted.

\subsection*{Gate 6 (Exalting the Group From Among Whom the Point Shall
Appear)}\label{gate-6-exalting-the-group-from-among-whom-the-point-shall-appear}
\addcontentsline{toc}{subsection}{Gate 6 (Exalting the Group From Among
Whom the Point Shall Appear)}

The sixth gate of the ninth Unity concerns what people ought to do in
exalting that group from among whom the Point of Reality shall appear,
provided they believe in them.

Know that there is no land more temperate than one that becomes the
scene of His appearance. Likewise, there are no names closer to God,
after the Letters of the One, than those that become the intimate abode
of the Tree of Reality. However, should people believe in Him only after
His appearance, they become the closest of creation, just as you may
observe among the Letters of the One. If you look earlier, consider the
nearness found in the Criterion---since that is the pivot of faith. Yet
if one already stands as ``near,'' one may draw even nearer, as was the
case of the Commander of the Faithful---peace be upon him---in the
Qur'an. Indeed, proximity is based upon faith, as you have witnessed
among the Letters of the One.

Know that God looks upon His dominion and chooses for Him the most
excellent father one could possibly have; likewise, He selects the most
excellent mother that can be found, and so it goes with every other
condition. It is one existence, and all are enlivened by the ocean of
His bounty, wholly sustained by it.

Even if the highest or the lowest of creation, any soul that fails to
turn toward Him remains lifeless. Whatever that soul's relation to all
other beings, God has sent down all His splendor and approval upon His
parents before their creation, and bestowed blessings upon them before
the creation of the heavens and the earth and all that is between.
Indeed, it is the very blessings that proceed from that Child upon them,
not otherwise, for the Ancient Essence is exalted beyond mention of
union, and whatever appears in the realm of possibility arises from the
First Will through Its manifestations, not through Its essence. That
Will's reality is such that nothing is seen therein but God---glorious
and mighty. The Eternal Beloved---glorious and mighty---claims as His
own whoever is for Him, and whoever is not for Him belongs to that which
lies beneath God. Recognize that all existence is, before that ocean of
bounty, but a single drop of its waters. Perceive Him and speak with
loving devotion, for perception has become manifest. If His appearance
does not come to pass, yet you hear even a fraction of a fraction of one
out of nineteen-tenths, and still you do not say ``yes,'' you will not
comprehend Him after death.

Know that obedience to Him is obedience to God Himself, and love for Him
is love for God Himself. Do not remain veiled in books and words on the
day of His appearance, for all these are means of reaching Him. If He is
present, even were all else absent, He will continue to be, and all will
abide under His shade. But if He is absent, even if all else should be
present, they amount to nothing and will remain as utter nothingness.
That is the meaning of the sacred verse, if you look upon it with
illumined recognition: ``Say: God suffices over all things, and nothing
in the heavens or on the earth or between them suffices in His stead,
for indeed He is ever-knowing, all-sufficient, all-powerful.''

Do not recite this verse, in the number of the name ``Qadír,'' unless
you behold the answer from the Origin of the command, for God has ever
been nearer to you than your own self and has power over all things, and
He has known and continues to know every thing to which the name
``thing'' applies. After you have called upon Him with the tongue of His
essence, the letters of this verse abide in the realm of existence,
causing the means of response to be manifested from Him.

\emph{Yet look only to God, for everything below Him is His creation,
and God is independent, exalted.}

\subsection*{Gate 7 (Prohibitions on Selling Anqúzah, Waraq
Zakúm)}\label{gate-7-prohibitions-on-selling-anquxfazah-waraq-zakuxfam}
\addcontentsline{toc}{subsection}{Gate 7 (Prohibitions on Selling
Anqúzah, Waraq Zakúm)}

The seventh gate of the ninth unity concerns the prohibitions on selling
anqúzah, waraq zakúm. Know that the essence of all prohibitions first
lies within the Letters of Negation, and likewise everything that enters
their shadow is included in negation. Know that anyone who is not for
Him Whom God shall make manifest is beneath God, and anyone who is for
Him is for God. Observe the same in the Point of the Bayán, and before
that in the Qur'án, and before that in the Gospel, in every appearance.
If those from previous appearances do not enter this one, regard them as
negated. Thus, the use of tobacco and similar things, including what is
carried from beneath Khurásán with its unpleasant fragrance and the
like---however it may be transformed---has been prohibited.

If you wish to see the first and the second, look at these two, for
whatever is not pure reverts to these two, and whatever is pure reverts
to Muhammad and 'Alí---may God's blessings be upon them both---and all
things connected to them. These have been forbidden on account of them.
In all circumstances seek refuge in the one God, exalted and mighty, and
in His names and semblances. For an example of one who is beneath
belief, it is like this very thing.

Know that every good thing that has been and is within God's knowledge
resides in the whole Tree of Affirmation, namely Him Whom God shall make
manifest; what lies beneath is in the shadow of Negation. Bring this
meaning into being through ``Be,'' and observe it plainly in His
appearance, so you may not remain veiled from His countenance. Prostrate
yourself before God for His sake, and acknowledge what He reveals from
God's presence. Gather His words in fine script and adorn them; for what
descends from Him consists of expressions of the everlasting Paradise,
dawning upon the forms of the hearts of the dominion in every
manifestation. Take the prime blossoms of the gardens of each appearance
and, through whatever means possible, offer support and obedience, that
you may leave Negation and enter Affirmation. This is God's
all-encompassing mercy that touches all things, His all-embracing favor
that surrounds all things. Say: God has forbidden you Negation and all
that pertains to it, and has guided you, together with all things, unto
Affirmation and what pertains to it, so that on the Day of Resurrection
you may follow God and by His command find guidance. Say: all are
prostrate before Him.

Know, too, that Negation by itself is not mentioned except alongside
Affirmation. If, on the Day of Resurrection, the Tree of Reality does
not perceive in a lesser believer the capacity for mention, He does not
bestow it. Whatever is revealed is for the sake of the ascendancy of
Affirmation and the vanishing of Negation, and not otherwise. For
instance, these two things were forbidden to preserve the believing
soul, not for their own sake. In all levels of existence, behold the
ocean of bounty and rely upon your Lord in every matter, for He is the
best of guardians.

\subsection*{Gate 8 (Prohibition on Intoxicants and
Medicine)}\label{gate-8-prohibition-on-intoxicants-and-medicine}
\addcontentsline{toc}{subsection}{Gate 8 (Prohibition on Intoxicants and
Medicine)}

The eighth gate of the ninth Unity: All things beneath love are beneath
the Truth, and all things of love are of the Truth. A prohibition is
thus placed on intoxicants and anything classified as medicine, in every
form, so that you may purify yourself of whatever is associated with
something below God. In cases of necessity, replace them with gentle,
wholesome provisions that reflect the attributes of the Tree of Love,
for it always was and remains so. Its ruling in general is the same as
in specifics. For example, to avoid even a single soul who is a lesser
believer is akin to the broader principle. Yet God---exalted and
glorified---reveals Himself in every state in such a manner that all
creation prostrates before Him, all turn to Him with love, and all
ascend through obedience to Him. Not a single atom exists that, in its
inmost reality, fails to worship Him or speak with its own tongue.
Nevertheless, in this Dispensation, whatever does not arise from the
Tree of Love is not and never has been beloved, while all that does
arise from it is and always has been beloved.

In the appearance of certain arts and crafts requiring some of these
elements, permission is granted for precisely as much of these items as
the people of each Dispensation are allowed, for the sake of dealing
gracefully with those beneath the manifestations of Truth, that a
benefit might be drawn from their existence through their belief in the
Truth. Within every soul lies the possibility of such faith, provided
one does not shroud oneself within one's own being. Seek God's refuge
from whatever God does not love, if you long to prosper.

\subsection*{Gate 9 (Collective Prayer is
Forbidden)}\label{gate-9-collective-prayer-is-forbidden}
\addcontentsline{toc}{subsection}{Gate 9 (Collective Prayer is
Forbidden)}

The ninth gate of the ninth unity: collective prayer is forbidden,
except for the prayer for the deceased, for you gather together but then
proceed individually.

The essence of this gate is that when congregational prayer is
performed, an actual and certain Imám is presumed to exist who
represents those Letters of Affirmation. Yet at the end of every
manifestation, all present themselves in such a way that they appear as
the Manifestation of Affirmation, while at the beginning of the new
manifestation it becomes clear they were from the Manifestation of
Negation.

Therefore, this was forbidden so that none might worship God in a
station beneath God. Today, if someone has believed in God and His
signs, and in the Tree of Reality and its appearances, and previously
prayed behind a person who has not now professed faith, it is incumbent
upon that believer to repeat the prayer. This is one of the true and
essential rulings, because that individual was---on that day---beneath
God, otherwise it would not have been so, and the one who performed the
prayer was for God, or else they would not be a believer now. This is
one of the Davidic rulings applied inwardly, not outwardly, for had it
been outward, permission would have been granted; yet even within that
permission, one must ask why there was no discernment used to avoid
following someone who, in reality, stood beneath God.

For the funeral prayer, permission is granted, as it is part of honoring
the believer's loss; the greater the number participating, the more
pleasing it is before God. However, no one should stand ahead of the
rest. All should form their rows and pray over the deceased with the
intention of individual worship, even if it appears outwardly to be a
congregational prayer. Look at what took place from the dawn of Islam
until the appearance of the Point of the Bayán, when congregational
prayers were held so often that no one could count them. Praise be to
God that no one would be allowed to stand in prayer behind the
manifestation of His own Self in its final appearance, who would be in
the station beneath God.

So it came to pass that everyone prayed in His name and by His
utterance, yet look at the limits of the people. Despite all their
displays of love and anticipation of deliverance, when His appearance
takes place, few believe. They persist in praying five times a day
behind the lowly ones among creation until the divine decree is lifted
from that observation. Watch carefully for the Day of the appearance of
Him Whom God shall make manifest, so that you are not veiled in this
same way---spending day and night longing for His mention, practicing
the laws that preceded Him, but then, when He appears, your heart
remains unmoved and He suddenly rises upon you without warning, or else
He arrives at a moment when He has permitted the abrogation of the
previous laws, leaving you all deprived, missing the fruit of your own
existence.

This ocean of bounty is patient; even if there is but a single soul in
His world who will keep God's covenant on the Day of Resurrection, He
will endure so that this one might fulfill that covenant. However, it
may happen that when it comes to pass, you remain asleep and awaken only
after it's over, thus veiling yourself. Consider the Messenger of
God---He appeared and roused the servants who followed the Gospel, but
they would not awaken and remain asleep to this day, for His duty was to
declare, ``I am that promised Ahmad,'' and to establish His proof by way
of the signs God had revealed to Him. It was not His place to fulfill
everyone's personal wishes; if it had been, no Manifestation's proof
would ever have been rejected. Look at the Criterion, in which you
firmly believe, and note how severely those servants are rebuked who
demanded certain things from the Messenger of God---things they demanded
to such an extent that they even said, ``Bring us God and the angels
face to face,'' which was an even greater demand than anything else they
could have asked for, because God is sanctified beyond such descriptions
and nothing within creation can be deemed worthy to mention in that
sacred realm, let alone to be fashioned there. Yet they remain waiting,
and it may be that the Resurrection of Him Whom God shall make manifest
comes to pass while they are still awaiting.

If God does not send forth some able and resolute figure over His
creation from among the believers in the Bayán, then otherwise He would
realize the full extent of all existence's grace. Had the means of His
appearance been different, God would have revealed it for the Messenger
of God. Rather, it is for creation itself to become the embodiments of
His command, for in the Origin there is nothing but the command of God.

For example, ``Pilgrimage to the House is a duty God has placed upon
humanity, for those who can find a way to do so'' was ``from God,'' but
its outward cessation appeared when people obeyed His command;
otherwise, that same honor contained in God's command ever abides for
one with discerning vision, whether or not anyone carries it out.

``\emph{So pray to God, your Lord, the All-Merciful, that you may be
certain of the signs of God on the Day of Resurrection.}''

\subsection*{Gate 10 (Purification of the Ground of
Souls)}\label{gate-10-purification-of-the-ground-of-souls}
\addcontentsline{toc}{subsection}{Gate 10 (Purification of the Ground of
Souls)}

The tenth gate of the ninth unity deals with the purification of the
ground of souls.

The essence of this gate is that, in the knowledge of God, there is a
purification for every thing. All are purified by the remembrance of God
if they become believers in Him Whom God shall make manifest, and the
hearts cannot be purified except through faith in the threefold letters
and the fourfold spirits, nor can souls be purified except through the
sixfold, nor can innate bodies be purified except through the like
thereof. All purification lies in the word of oneness, so that you may
purify these verses of the One.

From the verses under that canopy, away from the fire---so too, observe
affirmation in each thing, and that which lies beneath it, so you may
bring about purification. For example, if a spot as small as a speck
turns black on the handkerchief you hold, it cannot be cleansed except
by what has been decreed for its degree. Likewise, from the summit of
being down to the furthest mention of each thing, reflect so that you do
not remain veiled from the remedy for every ailment. Know that in the
Bayán, purification has always been the nearest approach to the Divine
and the finest act of devotion.

For instance, purify your hearing so it does not listen to any mention
beneath God. Purify your sight so it does not behold it. Purify your
heart so it does not witness it. Purify your tongue so it does not speak
it. Purify your hand so it does not inscribe it. Purify your knowledge
so it does not encompass it. Purify your mind so it does not let it
enter. Likewise, cleanse every facet of yourself so that you may be
nurtured entirely within the garden of love. Perhaps then you will
perceive Him Whom God shall make manifest with a purity beloved to Him,
remaining free of what lies beneath the one who fails to believe in Him
or the one who simply does not exist. At that moment, you will be
purified with a purity that truly benefits you.

Be aware that any ear hearing His words in faith will not enter the
fire---meaning that once it perceives the loftiness of His words, it
chooses to recognize Him and does not enter the love of any soul denying
Him. That is the fruit that appears in the hereafter. Any eye that
beholds His words with faith in them is bound to paradise. Any heart
witnessing His words in faith is and shall remain in paradise before
God. Any tongue that speaks forth His words with faith shall abide in
Paradise, exalting and glorifying the everlasting One whose
manifestations of majesty and breezes of sanctity have never known
cessation or end. Any hand that records His words in faith shall be
filled by God with whatever He loves, in this world and the next. Any
breast that preserves His words shall be filled by God with His love,
provided it believes in Him. Any heart that loves His words, such that
at the mention of Him the sign of faith appears therein---like the
saying of God, ``Whenever God is mentioned, their hearts
tremble''---that heart will ever be a locus of divine vision; God will
mention it in the Day of Resurrection with the most excellent
remembrance.

Know that the purification of souls does not mean reciting the words God
has sent down. The beginning of your religion is established by the
phrase ``There is no God but God,'' through the mention of the first
manifestations of the One and the mention of the Bayán; whereas the
words belonging beneath ʿilliyyīn remain in the Letters of the First
within negation, and in their time of appearance, they may regard
themselves as among the purifiers of souls. Rather, by purifying your
soul is meant cleansing yourself from all that lies beneath God, and
adorning yourself with all that belongs to God. Yet if the Bayán were to
be manifested in such a way that no mention beneath God remains, then,
at that time, if you recite words less than those of ʿilliyyīn, you
would be permitted. Should you wish to sail your entire life upon the
ocean of pure love and ʿilliyyīn, you possess provisions for that
journey, for if all existence were to traverse the single word ``God the
Most Great,'' all of them reach their destination, which is Him Whom God
shall make manifest, for He appears through the appearance of this Word.

Know that the negation of one Dispensation enters into the next
Dispensation, not into the Dispensation itself. For example, the Gospel
was negated in the Qur'an, and the Qur'an was universally negated in the
Bayán---not in a partial sense. Likewise, the negation of the Bayán will
not be made manifest except by the appearance of Him Whom God shall make
manifest. On that day, all will claim to affirm Him and will disavow
negation. Yet it would be pleasing if the touchstone of experience were
brought forth, for then all would become so purified that the heavens
and the earth and whatever is between them would bow down before its
greatness, just as you witnessed in the appearance of the Bayán. And if
you live in that day, you will see how all the fish in the ocean of the
Bayán live by that water, yet remain veiled from it. So purify
yourselves, with all that you possess of true capacity.

The End.

\newpage{}

\chapter*{Arabic Bayan (Starting Where Persian Left
Off)}\label{sec-arabic-bayan}
\addcontentsline{toc}{chapter}{Arabic Bayan (Starting Where Persian Left
Off)}

\markboth{Arabic Bayan (Starting Where Persian Left Off)}{Arabic Bayan
(Starting Where Persian Left Off)}

\section*{Vahid 9}\label{vahid-9-1}
\addcontentsline{toc}{section}{Vahid 9}

\markright{Vahid 9}

\textbf{Then the First after the Tenth}\\
Do not sell or buy the four elements.

\textbf{Then the Second after the Tenth}\\
The hair of animals, and whatever has no spirit breathed into it, does
not invalidate your prayer. You give thanks in the religion of God.

\textbf{Then the Third after the Tenth}\\
Never tear any book at all.

\textbf{Then the Fourth after the Tenth}\\
Once nineteen years have passed, if you are able, renew all your means
and possessions.

\textbf{Then the Fifth after the Tenth}\\
Write down the mention of the Bayán upon all your crafts, so that
perhaps, in the appearance of truth, you would be wary in your religion,
and would not mention anything without right before the First Tree.

\textbf{Then the Sixth after the Tenth}\\
Never strike anyone at all.

\textbf{Then the Seventh after the Tenth}\\
Over the course of nineteen days, host nineteen souls, even if you
provide them only with a single measure of water; and if you cannot
manage that, attain to the number of one.

\textbf{Then the Eighth after the Tenth}\\
Do not tear your garments, nor strike your bodies when someone among you
or anyone else dies---never, ever.

\textbf{Then the Ninth after the Tenth}\\
When you deem permissible the fish of the sea and the river, say: ``In
the name of God, the All-Preserving, the Self-Subsisting.'' Then you may
eat all that bears scales.

\section*{Vahid 10}\label{vahid-10}
\addcontentsline{toc}{section}{Vahid 10}

\markright{Vahid 10}

\textbf{The Single Tenth}\\
In the name of God, the Most Restrained, the Most Holy. Indeed, I am
God---there is no God but Me---the Most Perfect, the Most Perfect. It
has been sent down in the Single Tenth that I bear witness there is no
God but Me, the All-Preserving, the Self-Subsisting.

\textbf{Say, the First}\\
Do not be cautious about the dog or others; and if its moist hair
touches you, there is nothing but that you may wish to cleanse
yourselves.

\textbf{Say, the Second}\\
Truly God has permitted those who believe in the Bayán, from among the
letters and letterings, to look upon them; and they, in turn, to look
upon those believers, whenever they so desire.

\textbf{Then the Third}\\
That which you possess from the dominion of God, you shall inherit.
Therefore, apportion among yourselves what We have allotted to you, so
that perhaps you may, through the enumerations thereof, enter into your
own selves on the Day of God's appearance and thereby believe in Him
Whom God will make manifest, and be certain through His signs. Say:
Indeed, your descendants inherit from the Book of Ṭāʾ; among them,
divide with justice. Say: Whatever God has prescribed for them is by the
measure of ``maqt,'' so that they might be thankful. Say: That which God
has ordained for your spouses from the Book of Ḥāʾ is by the measure of
ṭāʾ and fāʾ; among them, divide with fairness. Say: That which God has
decreed in the Book from the Book of Zāʾ for your father is by the
measure of ṭāʾ and kāf; judge therein by what God has appointed for you.
Say: That which your mothers inherit from the Book of Wāw is by the
measure of the exalted in the Book; decree it as God has determined. And
that which God has laid down for your brothers is by the measure of shīn
from the Book of Hāʾ; fulfill that for them as God has prescribed. And
that which God has prescribed for your sisters is by the measure of rāʾ
and mīm from the Book of dāl; deal justly with them in accordance with
what God has ordained. And that which God has decreed for those who
teach you the knowledge of the Bayān is drawn from the Book of jīm by
the measure of qāf and fāʾ; you must distribute it among them with
justice.

Say: God has apportioned your inheritance according to four degrees
after one-third, in keeping with what is determined in the letters. Such
degrees, before the four---one-third thereof---remain from the treasury
of knowledge in the Book of God, unchanging and unalterable; you behold
it within your own forms. Then, on the Day of Resurrection, by the
measure of ``hāʾ,'' you will believe in Him Whom God will make manifest
and be certain.

\textbf{Say: The Fourth}\\
The essence of the religion, in your beginning and in your return, is
that you believe in God---there is no God but He---then in Him Whom God
will make manifest on the Day of Resurrection in your return; then in
that which God sends down upon Him from the Book; and then in him whom
God made manifest under the name 'Alí before Muḥammad, according to what
was revealed in the Bayān, of which all are powerless to grasp. Should
you, at your return, attain unto Him Whom God will make manifest, then
indeed you reach your beginning.

\textbf{Say: The Fifth}\\
All that may be called a ``thing'' has been immersed in the ocean of
permissibility and purity for its own self by its own self, except for
those who do not believe in the Bayān, and you are not forbidden from
that in the Book. For that is what you have been enjoined; it does not
change from its own state, and you shall be questioned concerning that
which your Lord, your God, has commanded you. Therefore, avoid all that
you yourselves find repugnant.

\textbf{Say: The Sixth}\\
God has forbidden you in the Bayān to cause harm, even if it be merely a
hand striking a shoulder. ``O servants of God, fear Him!'' And indeed,
when you wish to contend with the utmost modesty, record your proofs,
and with the highest courtesy, speak. For you shall meet your Lord on
the Day of Resurrection by meeting Him Whom God will make manifest, and
the one who is the Gate for all the worlds, so that perhaps you may not
meet your Lord and perform a deed that grieves your Lord by causing
sadness to Him Whom God will make manifest, while you remain unaware and
fail to remember.

\textbf{Indeed, the Seventh}\\
Each of you must convey on behalf of yourselves to Him Whom God will
make manifest a crystal container of a most rare and invulnerable
perfume from the Point of the Bayán. Then bow down before God with your
own hands, not by the hands of others---unless you are entirely unable.

\textbf{Indeed, the Eighth}\\
Do not prostrate yourselves except upon crystal containing specks of the
first and the last dust, as a mention from God in the Book, so that you
do not bear witness to anything displeasing.

\textbf{Indeed, the Ninth}\\
Every soul must acquire a lofty, impregnable crystal in the measure of
the One, according to what is possible. If someone is able but does not
own it, it is prescribed for him to give nineteen mithqáls of gold as a
specified limit in the Book of God, so that you may act with awareness.

\textbf{Indeed, the Tenth}\\
The letters must not remain beyond ninety days once their counterparts
have been taken, nor may the counterparts remain beyond ninety-five days
once their letters have been taken---this is a fixed decree in the Book
of God, so that you may be conscious and bear witness that all dominion
belongs to God and that all return unto Him. If they remain beyond what
God has allotted for them, after they have the capacity and ability,
they must spend ninety-five mithqáls of gold; likewise, for women,
ninety-five mithqáls of gold---if they can, or if those responsible can.
Otherwise, they are absolved. And God has desired nothing but love and
contentment for everyone, that you may be grateful in the paradise of
the Bayán.

\textbf{Indeed, the Eleventh}\\
Those who produce a written work shall inscribe at its beginning,
``There is no God but God,'' and at its end, ``There is no proof but
upon the one before Muḥammad,'' so that on the day of Him Whom God will
make manifest you may reason with something akin to this and thus be
guided.

\textbf{Indeed, the Second after the Ten}\\
Your offspring are not bound by the ordinances of your death before the
spirit is breathed into them. After the spirit is breathed into them so
that they might be born alive, watch over your bounds of life within
them. If they emerge stillborn, the ordinances and prayers concerning
them are lifted from you. Neither fathers nor mothers shall approach
them, lest this cause sorrow---except if there is no one else, as a
mercy and favor from God in the Book, so that you may be steadfast in
the days of God.

\textbf{Indeed, the Third after the Ten}\\
In the Bayán, it is permitted for you to become single, one by one, by
choosing for yourselves the number of Ḥayy (the Living), so that on the
Day of Resurrection, you may present yourselves in that station before
God, your Lord. Say: the Point is the sign of the First Tree, and the
Living are the signs of the First Living. You must watch over yourselves
in that matter so that, on the Day of Resurrection, you may be with Him
Whom God will make manifest and with the Living Letters, do not conceal
yourselves. For if He Whom God shall make manifest appears in the
station of the Point or of the Living, He is certainly the truth from
God, with no doubt therein. Truly, we believe in Him altogether.

\textbf{Indeed, the Fourth after the Ten}\\
God has prescribed for your fathers and mothers to provide for you from
the very start of your creation until you complete nineteen full years.
And it is incumbent upon you to provide for them until the end of their
lives if they are not self-sufficient. Likewise, if they can provide for
you, they must do so. You are able to fulfill this as long as you remain
on earth, provided that each remains within the bounds of their
religion. Should any of them withdraw from these bounds, you are
absolved from their obligations. Whoever withdraws from God's limits in
this matter must, every year, spend nineteen mithqáls of gold in the
path of God, as a decree in the Book of God, so that you may remain
aware.

\textbf{Indeed, the Fifth after the Ten}\\
Do not ride upon cattle or place any burden upon them, if you believe in
God and His signs. Do not drink the milk of donkeys, nor load them---or
any other beast---beyond their capacity, as God has prescribed for you,
so that you may be conscious. Do not ride an animal except with bridle
and stirrups, and do not mount anything on which you cannot safeguard
yourselves, for God has sternly prohibited you from that. And do not
break an egg upon anything that would spoil its contents before it is
cooked. This is what God has designated as the provision for the First
Point in the days of Resurrection from His own presence, that you may be
thankful.

Whatever appears in the egg as blood is pardoned for you, and it is
pure; yet do not eat it, so that you will not bear witness to something
displeasing. Do not board the ship except in accordance with your
capacity and ownership, and do not dispute or contend while aboard. You
must journey among yourselves in the utmost spirit of harmony and
fragrance. It is prescribed for those with authority on the ship to
place foremost the well-being of those who travel therein whenever it
grows turbulent among its passengers, at which time you are unable to
stand firm.

Designate the place of your purification in a seat that does not alarm
whoever enters it; act just as you do with your privies in other
locations. Likewise, do not become overly watchful of purification in
the ship beyond what you are able. It has been lifted from those behind
the sea, what God ordained for them of obligatory travel if they lack
the means for land travel. It is permitted for them to appoint others on
their behalf to perform the pilgrimage and deliver their expenses from
their location to wherever they must return---if they can do so.
Otherwise, they are excused, along with whatever they undertake.

\textbf{Indeed, the Sixth after the Ten}\\
God has enjoined upon every king of a land one hundred and forty
mithqáls of gold each year, and upon the greatest vizier two hundred and
ninety mithqáls to be stored for Him Whom God shall make manifest, and
at the time of His appearance, they shall deliver it to Him with their
own hands. This is so that, in that Resurrection, they will not grieve
the Manifestation of their Lord. Perhaps, in return for the deeds of
those who came before them, those who remain behind in the Bayán, seated
in their places, will act with justice. O you who hear these words, if
you do not believe in Him Whom God shall make manifest, do not sadden
Him; for in that Resurrection, if these people had believed in the First
Point, no one in the Bayán would have been grieved, and all would have
journeyed toward the next Resurrection in peace and delight. But they
drew a veil over themselves and acquired what God does not love in the
Bayán. And you---do not distance yourselves from the mercy of your Lord
by acting as they did. If you fail to deliver to Him Whom God shall make
manifest what God has decreed for you in the Book, then at least do not
grieve Him and do not harbor doubt when you hear the matter.

Make yourselves the arbiters between Him and those who possess the
Bayán: submit His verses to those who have the Bayán if you acknowledge
your own inability, as well as theirs. Then you will believe; but if you
do not admit your own inability or theirs, then do not sadden Him. Even
if He were to appear as a judge in that Resurrection to clarify the
truth for everyone on earth, nevertheless each group would still return
to its own ruling in matters of religion and worldly concerns, making
judgments among themselves. Yet they will not manifest a decree by which
their faith could be established, something that would testify to their
incapacity before the signs of their Lord.

Thus they glorify themselves by that ruling day and night, wearying
themselves and exhausting their works, all the while imagining that they
are doing good. You who have the Bayán---do not be veiled in the same
way.

\textbf{Indeed, the Seventh after the Ten}\\
O possessors of authority, command those who follow you never to take
the clothing or possessions of anyone; if someone is found doing so,
then both you and they are forbidden from your spouses for nineteen
days. If, despite this, you still unite, you shall owe nineteen mithqáls
of gold, to be delivered to the witnesses of the Bayán, who will then
give it to the person from whom the clothing or property was taken, so
that you may be conscious. Also command those who follow you that no one
should ever oppose or confront anyone else---so that on the Day of
Resurrection, you will not stand against the companions of Him Whom God
shall make manifest. In every land, order that its houses, markets, and
places be organized; each type of goods should be kept apart from
others, so as not to commingle, unless it is in a place prepared for it.
Let every type of goods occupy a single area in a most pleasing
arrangement. Command as well that each type be placed in a single
caravansary, for that is more conducive to benefit and piety, if you
perceive it so.

\textbf{Indeed, the Eighth after the Ten}\\
Do not command that anything be taken from anyone---not even a hair's
worth---nor that anything be reduced from a person after God has
perfected the outward formation of that individual, according to the
divine decree recorded in the Book, so that none among you will grieve
anyone else. Whoever takes something from another's body---or alters its
color to any degree---or changes their clothing, or intends to degrade
them---God has forbidden him from his spouses for nineteen months in the
Book of God; and whoever does so must pay, within God's bounds,
ninety-five plus one mithqáls of gold, so that perhaps you may be
conscious. Do not command, nor commit, nor approve any such thing; do
not oppress anyone by even a mustard seed, if you believe in God and His
signs. If you do not believe in God and His signs, then act in a way
that does not strip you of your life; for before you were created, you
were in God's sight a drop of water after clay, and you shall return to
the palmful of clay. Therefore, be ashamed and never accept for anyone
that which you do not accept for yourselves. You are at the height of
managing your lives---reflect well, and do not ruin anyone's existence
after God has perfected their creation, simply for the sake of brief
worldly prestige or fleeting prosperity. Both will be cut off from you,
and afterward you shall enter the Fire, longing as though you had never
been created or acquired such deeds that cause sorrow to another. If
only you would reason during your lifetime, you would wish you had
scarcely perceived.

Say: \textbf{The Ninth after the Ten}---No command has God ordained, nor
any prohibition revealed, except for the exaltation of Him Whom God
shall make manifest. Should He present you with a command or a
prohibition, it is His honor, which is God's honor, that you must
regard, and from the totality of both (the prior command and
prohibition) you must withdraw.

\section*{Vahid 11}\label{vahid-11}
\addcontentsline{toc}{section}{Vahid 11}

\markright{Vahid 11}

\textbf{The Single Eleventh after the Ten}

In the Name of God, the Most Restrained, the Most Holy

Truly, I am God---there is no God but Me---the Most Firm, the Most Firm.
I have sent down the measures of all things in the numerical value of
``yá'' within the One, so that you may be thankful.

\textbf{Say: Indeed, in the Single Eleventh after the Ten}, you first
bear witness---if you swear by God and then by Him Whom God shall make
manifest, and you are truthful between yourselves and God---then no
obligation is upon you; and that which you swore regarding shall be
returned to you, unless they withhold it. If they do withhold it, they
must pay nineteen mithqáls of gold as a fixed decree in the Book of God,
so that you may be conscious. But if between yourselves and God your
Lord you swear and are not truthful, then you must, as prescribed in the
Book of God, pay nineteen mithqáls of gold and return it to whoever was
the object of your oath, as a fixed decree in the Book of God, so that
you do not swear without right.

\textbf{Say: The second}---any king who receives the Bayán must select,
from among the inhabitants of his realm, the number of ``káf'' and
``há'' from among the learned---those who should be the dawning-places
of the letters in the Book of God---so that on the Day of Resurrection
they may believe with certainty in Him Whom God shall make manifest,
give victory to God's religion, and make known to all people throughout
his realm what is required, so that they may help the weak and show them
mercy. Then they shall not hide themselves from God their Lord
concerning the limits of their faith.

\textbf{Say: The third}---whoever mocks a believing man or believing
woman must pay the measure of the One in gold, and then in silver. Then
one must utter the word of forgiveness ninety-five times, so that you
may be conscious and not mock others. One should return it to the one
who was mocked, if able. If one is unable, then the gold and silver are
lifted from him, but the obligation to seek forgiveness remains. And if
he has no tongue and mocks by his gestures, let him choose someone to
seek forgiveness on his behalf. O My servants, be conscious of God.

\textbf{Say: The fourth}---Truly, the Bayán and whomever is within it
are alive, whether through its light or its fire, until the day of Him
Whom God shall make manifest, during which you have power to bring them
both to life, then you shall proclaim, and then you shall establish the
judgment. Say: Indeed, the fire is for those who draw a veil over the
bounds revealed in the Bayán; and the light is for those who observe
God's limits. This is within the Bayán itself, not in the religion into
which they have entered. O every thing, be conscious.

\textbf{Say: The fifth}---Whoever enters the Bayán, do not drive him
away from his faith; and if you do drive him away, you must pay nineteen
mithqáls of gold, delivering it to the one you have driven out, as a
fixed decree in the Book of God, so that you will not expel anyone from
the Bayán. If you bear witness against someone to something for which
God has not granted permission in the Bayán, that person has disobeyed
God, his Lord, yet has not departed from the foundation of his religion.
In proportion to the extent of his veiling, the fire shall reach him.
With pleasing and gracious speech, you must make this known to them and
remind them.

\textbf{Say: The sixth}---Whoever awaits the appearance of Him Whom God
shall make manifest without knowing God or seeking His pleasure in
understanding His essence and His good pleasure. Such people have not
recovered even a single letter of the Bayán, and in the sight of God,
they have not been believers. Convey the Book of all things to every
soul, even if it be anyone at all, from what remains of the First
Wondrous Name, which is a remembrance from God to all the worlds. Then
seek forgiveness from God---there is no God but He---the All-Preserving,
the Ever-Living---and turn to Him in repentance.

\textbf{Say: The seventh}---In the Bayán, you are forbidden to possess
more than the measure of the One from any book. If you do possess more,
you must pay nineteen mithqáls of gold as a fixed decree in the Book of
God, that you may be conscious.

Say: The first is the Bayán itself. Then the Living are the sciences
brought forth within the Bayán which are incumbent upon you in your
religion, such as grammar, morphology, letters, the enumeration of
letters, and whatever you compose in the religion of God through the
ways of order, so that you may bring order. Therefore, produce only the
essences of knowledge and wisdom, and remain veiled from their
embellishments---this is so that, before Him Whom God shall make
manifest, nothing will be present except the Bayán itself and whatever
has come into being within the Bayán by the measure of the Living, who
have attained the pinnacle of knowledge and piety and have been sincere
in the religion of God.

\textbf{Say: The eighth}---Do not scatter the letters, but gather them
in delicate containers or in a gentle cloth. If you guard them
otherwise, you are still arranging all letters upon raised places; act
so that you may watch over their spirits, so that perhaps, through their
spirits, you will attain that which lies in the sublime realms, and
remain veiled from all else. Gather together the spirits that are
connected thereto within yourselves, so that you will not be dispersed
into things that cause you sorrow, but only into what brings you
satisfaction and gratitude. And whoever owns any letter must keep it in
a cherished station of dignity. If it is in a servant's chamber, it is
incumbent upon each individual to preserve what belongs to them of each
written letter, whether placed all in one place or in different seats.
God has permitted this for you, so that you do not find the matter
difficult.

\textbf{Say: The ninth}---Do not sit in the seats of majesty except
during the appointed cycle. If you do sit there otherwise, you must pay
nineteen mithqáls of gold, unless you are compelled. But whoever compels
you must likewise, according to the Book of God, pay what is due, so
that you do not depart from the bounds of your proper courtesies. You
are permitted to be seated in your houses when your families sit with
you, for you cannot always sit around the chambers unless you come
together in one place with mutual affection. The requirement regarding
seats of mourning has been lifted from you, so that you might grieve for
the trusted ones of God.

And if anyone arrives as a guest to another, the host must honor him
with inviolable esteem and offer him a place with his own self and those
around him. If they fail to do so, then it is incumbent upon all of them
to say: ``Truly we seek forgiveness from God, Who has the most beautiful
names, for all things, and to Him we do indeed return in repentance.''

\textbf{Say: The tenth}---Permission has been granted in the Bayán, that
all which has been revealed therein be in Arabic, for those who can
understand. And if anyone wishes to explain something in Persian, the
Book has given leave for those who cannot grasp the words of the Bayán;
yet interpret it only with truth. Do not transform the Persian into
Arabic unless it is done correctly. And let all of you, collectively,
own a beloved Arabic Bayán and a Persian Bayán for those who are unable
to comprehend what God has revealed. And you must guard what has been
revealed to the Witnesses as you would the apple of your eye, then
convey it to Him Whom God shall make manifest.

It is permitted for you to arrange from the ``Books of the One'' that
third portion in a single volume, first all in Arabic, then all in
Persian---by God's mention---so that you may fully encompass the outward
sense of all God has revealed in the Book, and then act in accordance
with it.

\textbf{Then the Eleventh after the Ten}---Do not put yourselves before
Him Whom God shall make manifest, nor before the First Living, whether
they appear in the highest station of creation or in the lowest, for in
God's sight they are exalted. Whoever steps ahead of them must pay
nineteen mithqáls of gold, as a decree in the Book of God, so that you
may be conscious.

\textbf{Say: The second after the Ten}---You, O people, are the guides
of the Cause of God. Whenever you testify against someone that they
should be turned away from something, if you can respond, then you must
do so; for God answers them according to what He has commanded you, once
He has made you aware of someone's request. It is prescribed that you
fulfill it, and if you veil yourselves, then seek forgiveness from God,
your Lord, nineteen times. And if you refrain from seeking your own
forgiveness, you must pay nineteen mithqáls of gold, as a bound set down
in the Book of God, so that you may watch over yourselves. So that
whenever any soul appeals to you concerning your religion, you may
respond; and concerning the bounds of your world, you may fulfill them
as a favor from God upon them, that perhaps you may become the
manifestations of how God responds to His servants.

\textbf{Say: The third after the Ten}---If a king is raised up in the
Bayán, it is ordained for him that he may possess for himself a
particular item to place upon his head, one that bears the number
ninety-five, which has no equal, no likeness, no match, nor any
companion or counterpart, and does not exceed the measure of ``hāʾ,''
signifying the appearances of His names as a divine honor unto him until
the Day of Resurrection. On that day, everything he fashioned by this
means in the Bayán shall be ransomed at the feet of Him Whom God shall
make manifest, and then you shall prostrate before God. Should you boast
of these things, O you who hold dominion---otherwise, God is indeed
independent of all the worlds.

\textbf{Say: The fourth after the Ten}---Assign from the very start of
your night until the end of your day five divisions. At each division,
you must call out the adhán. Begin with the start of night, then in the
first instance repeat nineteen times ``There is no God but God,''
followed by the measure of the One, ``God is the Most Independent,''
then speak it aloud. In the second instance, nineteen times ``There is
no God but God,'' then the measure of the One, ``God is the
All-Knowing,'' say this. In the third instance, nineteen times ``There
is no God but God,'' then the measure of the One, ``God is the Most
Just,'' say this. In the fourth instance, nineteen times ``There is no
God but God,'' then the measure of the One, ``God is the Possessor of
all,'' say this. In the fifth instance, nineteen times ``There is no God
but God,'' then the measure of the One, ``God is the Most Powerful,''
say this.

And it is prescribed for you that you must give the call to prayer in a
place such that those around you can hear. If the sound does not reach
someone, that person is required to pay nineteen mithqáls of the purest
white sugar each day and night, so that you may watch over yourselves
and not veil yourselves from the remembrance of God. Whoever is sleeping
is not held accountable; but whoever is short of true sleep must be in a
location where they can hear the sound. You are not required to leave
your rooms to hear the call; rather, it is sufficient in the Book of God
that you have a means for the caller's voice to reach your dwelling.

If the adhán becomes difficult for the one calling it, let him say once,
``God has borne witness that there is no God but He, and that the one
whom God shall make manifest is indeed the truth from God. By God's
command, creation proceeds from His presence; indeed, we believe in
everything God reveals unto Him.'' This is from the bounty of God upon
them in their days of cold and times when they cannot endure for long.

\textbf{Say: The fifth after the Ten}---If you forget something in your
prayer, then make up only that portion which has been missed, not all
your actions. The same applies beyond your prayer, for you are absolved
in that which came beforehand, and then you need not dwell on it. And by
that very same matter already settled, you observe and fulfill. It is
decreed for those who have been given the Bayán to let the knowledge of
their own selves regarding all that is on earth be reduced to nothing in
relation to every king, his daughter, his book, the limits of his
dominion, the number of his troops, and the splendor of what he
possesses or might possess without parallel---until the day when all
shall be presented before God, their Lord.

\textbf{Say: The sixth after the Ten}---Do not kill any soul, nor sever
anything from a soul, ever, if you believe in God and His signs. Whoever
commands or perpetrates such an act, or can prevent it but does not, or
is pleased with it, must, according to the Book of God, pay eleven
thousand mithqáls of gold to be delivered to the heirs of the victim.
Furthermore, all spouses are forbidden to him for nineteen years. This
testifies in the Book of God that his very being was created against
God's love and pleasure, and he shall enter the Fire after his death,
nor shall God ever forgive him. However, if he abides by these limits
and lightens that which has been decreed for him, let him be conscious
of God, and then remain so.

If anyone kills another unintentionally, nothing is binding upon him
except that the person and the heirs of the one slain must be satisfied,
and must grant him pardon, that he might be among those who seek
forgiveness before God, his Lord. Such is the case with any circumstance
that befalls a person. Therefore, be mindful of God, O every soul, and
remain aware.

As for those who committed murders in the land of Ṣād, if they believe
in God and His signs, they must take the blood-money for what they have
done from the heirs of those killed, according to the limits previously
determined, so that you may act with awareness in God's religion. After
this, do not draw near to such acts again.

\textbf{Say: The seventh after the Ten}---Whoever orders that someone be
driven out of his home, or city, or village, or realm of sovereignty, is
forbidden for nineteen months, and must pay nineteen mithqáls of gold,
to be returned to him who was driven out, as a fixed decree in the Book
of God, so that you may be conscious.

\textbf{Say: The eighth after the Ten}---Whoever drinks an intoxicant
that removes his senses must pay, in accordance with the Book of God,
ninety-five mithqáls of gold. And never treat your sick with any
intoxicant if you believe in God and His signs.

\textbf{Say: The ninth after the Ten}---Whoever writes even a single
letter about Him Whom God shall make manifest, or concerning that which
was not revealed in the Bayán before His appearance, must pay nineteen
mithqáls of gold according to the Book of God. God has granted no one
permission to seize that from him or to question him about it. And
whoever does question him concerning that ordinance must incur upon
himself the same penalty for having asked about what God did not allow
him to ask. Thus be mindful of God, and do not set down so much as a
single letter regarding Him Whom God shall make manifest, nor about
anything beyond the bounds of what God has revealed prior to the
appearance of truth. And do not judge after the appearance as you did
before it, imagining you consider yourselves as acting rightly. Yet if
you are not writing on behalf of the truth, then do not write anything
against the truth. This is what God has counseled you, so that you might
be mindful. If you do not support Him Whom God shall make manifest by
what you write for Him, then do not be distressed by what is written
against Him. Therefore, fear God as He should be feared, so that on the
Day of Resurrection, in God's presence, you may be saved.


\backmatter

\end{document}
